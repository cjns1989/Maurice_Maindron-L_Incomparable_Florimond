\PassOptionsToPackage{unicode=true}{hyperref} % options for packages loaded elsewhere
\PassOptionsToPackage{hyphens}{url}
%
\documentclass[oneside,11pt,french,]{extbook} % cjns1989 - 27112019 - added the oneside option: so that the text jumps left & right when reading on a tablet/ereader
\usepackage{lmodern}
\usepackage{amssymb,amsmath}
\usepackage{ifxetex,ifluatex}
\usepackage{fixltx2e} % provides \textsubscript
\ifnum 0\ifxetex 1\fi\ifluatex 1\fi=0 % if pdftex
  \usepackage[T1]{fontenc}
  \usepackage[utf8]{inputenc}
  \usepackage{textcomp} % provides euro and other symbols
\else % if luatex or xelatex
  \usepackage{unicode-math}
  \defaultfontfeatures{Ligatures=TeX,Scale=MatchLowercase}
%   \setmainfont[]{EBGaramond-Regular}
    \setmainfont[Numbers={OldStyle,Proportional}]{EBGaramond-Regular}      % cjns1989 - 20191129 - old style numbers 
\fi
% use upquote if available, for straight quotes in verbatim environments
\IfFileExists{upquote.sty}{\usepackage{upquote}}{}
% use microtype if available
\IfFileExists{microtype.sty}{%
\usepackage[]{microtype}
\UseMicrotypeSet[protrusion]{basicmath} % disable protrusion for tt fonts
}{}
\usepackage{hyperref}
\hypersetup{
            pdftitle={FLORIMOND},
            pdfauthor={Maurice Maindron},
            pdfborder={0 0 0},
            breaklinks=true}
\urlstyle{same}  % don't use monospace font for urls
\usepackage[papersize={4.80 in, 6.40  in},left=.5 in,right=.5 in]{geometry}
\setlength{\emergencystretch}{3em}  % prevent overfull lines
\providecommand{\tightlist}{%
  \setlength{\itemsep}{0pt}\setlength{\parskip}{0pt}}
\setcounter{secnumdepth}{0}

% set default figure placement to htbp
\makeatletter
\def\fps@figure{htbp}
\makeatother

\usepackage{ragged2e}
\usepackage{epigraph}
\renewcommand{\textflush}{flushepinormal}

\usepackage{fancyhdr}
\pagestyle{fancy}
\fancyhf{}
\fancyhead[R]{\thepage}
\renewcommand{\headrulewidth}{0pt}
\usepackage{quoting}
\usepackage{ragged2e}

\newlength\mylen
\settowidth\mylen{C’est où l’on fait ballet. On y voit face d'anges}

\newlength\mlena
\settowidth\mlena{C’est où l’on fait ballet. On y voit face d'anges}

\newlength\mlenb
\settowidth\mlenb{Avecque plus de grâce, ou celle qui fait mal.}

\usepackage{stackengine}
\usepackage{graphicx}
\def\asterism{\par\vspace{1em}{\centering\scalebox{.9}{%
  \stackon[-0.6pt]{\bfseries*~*}{\bfseries*}}\par}\vspace{.8em}\par}

 \usepackage{titlesec}
 \titleformat{\chapter}[display]
  {\normalfont\bfseries\filcenter}{}{0pt}{\Large}
 \titleformat{\section}[display]
  {\normalfont\bfseries\filcenter}{}{0pt}{\Large}
 \titleformat{\subsection}[display]
  {\normalfont\bfseries\filcenter}{}{0pt}{\Large}
\ifnum 0\ifxetex 1\fi\ifluatex 1\fi=0 % if pdftex
  \usepackage[shorthands=off,main=french]{babel}
\else
  % load polyglossia as late as possible as it *could* call bidi if RTL lang (e.g. Hebrew or Arabic)
%   \usepackage{polyglossia}
%   \setmainlanguage[]{french}
%   \usepackage[french]{babel} % cjns1989 - 1.43 version of polyglossia on this system does not allow disabling the autospacing feature
\fi

\title{FLORIMOND}
\providecommand{\subtitle}[1]{}
\subtitle{Mœurs du temps de Louis XIII}
\author{Maurice Maindron}
\date{}

\begin{document}
\maketitle

\clearpage

\hypertarget{chapitre-premier}{%
\chapter{CHAPITRE PREMIER}\label{chapitre-premier}}

Quand il eut bien musé dans le potager où, sous le grand soleil de midi,
les papillons jaunes ou blancs voltigeaient entre les ombelles des
fenouils et des carottes, Louis-Antoine passa le pont de bois jeté sur
le ruisseau et s'en fut par la campagne. Sous son bras gauche il avait
un \emph{De civitale Dei}, relié en peau de truie, avec des fermoirs de
cuivre, dont un manquait, et de la main droite il tenait un croûton de
pain où ses dents mordaient avec l'audace d'une jeunesse de seize ans.
Car ce pain de ménage, pétri et cuit chez sa mère, était plus dur et
compact que les cailloux du ruisseau.

Sans autrement s'occuper du \emph{Saint Augustin} in-folio dont il
s'était chargé par devoir, Louis-Antoine hésita un moment avant de
s'engager dans le petit chemin creux qui menait au boqueteau de Tonlieu.
Si d'une part les écrevisses dont l'onde transparente lui permettait de
compter, à travers les feuilles d'eau et les nymphées, les pattes et
jusqu'aux moindres accidents de leur carapace sombre, aussi exactement
jointées que le halecret d'un carabin, flattaient son instinct de
pêcheur, de l'autre les lapins de son ami Montenay parlaient à son
apathie buissonnière et à ses goûts de maraude.

C'est que, dans la garenne de M. de Montenay, Louis-Antoine avait droit
de chasse, l'an tout entier. Et ce droit ne l'empêchait pas, d'ailleurs,
de braconner sur les confins du parc de Bannes, quitte à débouler
vivement dans la garenne de Tonlieu si quelque sergent blavier ou
quelque garde-chasse à la livrée du marquis apparaissait avec des
intentions mauvaises. Pour ces raisons, Louis-Antoine décida de négliger
les profits certains de la pêche pour les succès aléatoires de la
chasse.

«\,J'irai donc, se dit-il, prendre des collets chez Symphorien. Si la
vieille Jeannette n'en a point, ce qui serait un cas jusqu'ici sans
exemple, j'emprunterai l'arquebuse de Marin. Elle est mal montée, mais
tue à trente pas comme à dix, à condition de tirer un peu bas. Je
payerai la poudre et la dragée de plomb avec un lièvre, quand j'en
abattrai\ldots{} Et puis, pendant que je quêterai, rien n'empêchera les
lapins de la Drapière de se prendre aux collets que j'aurai semés
derrière le vieux mur de sa chênaie\ldots{} Et, enfin, pour que le curé
trouve aussi sa part dans ces divertissements honnêtes, je me coucherai
dans l'herbe et apprendrai par cœur, pour les lui réciter couramment,
les vingt-cinq lignes du chapitre neuvième qui abondent, à ce qu'il dit,
en tours ingénieux, ce dont je me moque. Et ce que j'en ferai ne sera
pas pour obéir à la volonté dudit curé, mais pour le plaisir de ma mère,
tant je crains de la désobliger\ldots\,»

Ainsi disposé à se conformer aux commandements de Dieu, sans attacher la
même importance aux conseils de l'Église, représentée dans l'affaire par
le curé de Primelles, Louis-Antoine descendit à grandes enjambées le
chemin pierreux, raviné par les pluies d'hiver, défoncé par le pied des
bestiaux, et dont la neige des arbrisseaux en fleurs ne parvenait pas à
cacher le mauvais entretien sous son tapis de mai, velouté, odorant,
couleur de pêche et de crème.

Si l'entretien du chemin était mauvais, les chaussures de Louis-Antoine
ne lui cédaient rien sous ce rapport. Chaussures et chemin semblaient
avoir été créés l'un pour l'autre. Et d'abord ces souliers appartenaient
sans conteste à deux paroisses. Si la mode antique commandait bien que
le pied gauche ne différât en rien du droit par les courbes chantournées
de la semelle et le galbe de l'empeigne à bout carré, elle n'ordonnait
pas cependant que le pied droit fût de vache grasse et le gauche de veau
ciré, ni que le talon très haut de l'un fût un bel ouvrage de tourneur
en cormier, à l'image d'un tronc de cône, tandis que l'autre, beaucoup
plus bas et galamment ébrasé, plus large, se composait de croissants de
cuir vert assemblés avec des clous à ferrer les bœufs. Ces deux
souliers, du reste, pour dissemblables qu'ils parussent, présentaient ce
caractère commun d'être crevés en dix endroits et rapetassés en huit.
Tous deux s'ornaient de ficelles en guise de cordons. Les languettes
brillaient par leur absence, et les contreforts éculés s'écrasaient en
plis en demi-cercle au-dessus des talons, à peu près comme les
bourrelets charnus débordant le col d'un gros homme.

Mais, des bas drapés dont le ton demeurait indéfinissable, les plis
avaient adopté la disposition spirale. Par défaut de jarretières,
vraisemblablement, ils descendaient en serpentant. Et, grâce à la
maigreur héronnière des jambes de Louis-Antoine, aussi minces aux
mollets qu'aux chevilles, ils donnaient l'idée, avec leurs rides
étagées, de deux vis de pressoir sur quoi le jeune garçon avançait par
un artifice mécanique.

Parler de son haut-de-chausses ne serait point séant, d'autant que ce
vêtement contenait moins de pièces et de reprises que de trous. Jadis
taillé dans un vieux manteau de pluie en bouracan qui avait gagné, à
servir, la teinte indécise des arbres perdus dans le brouillard, le
haut-de-chausses de Louis-Antoine était devenu mince, luisant et
verdâtre. Aux endroits où il ne montrait point la corde, de larges
taches miroitantes indiquaient que les poils de chèvre de ce tissu
bourru, entre tous impropre à fabriquer des culottes, demeuraient collés
par les sucs des herbes champêtres, l'encre de l'étude et le sang du
gibier que le chasseur rapportait pendu à sa ceinture. Enfin ce
haut-de-chausses, fabriqué originairement sur le modèle de celui du
seigneur Pantalon, et destiné donc à descendre à mi-jambes, s'arrêtait
maintenant au jarret, car le tailleur rustique n'avait point fait entrer
en ligne de compte la croissance du baron Louis-Antoine.

Pour la même raison, les basques du pourpoint, dont la taille commençait
sous les épaules, battaient ce qu'on voyait de la chemise au-dessus de
la ceinture. Ce vêtement de bureau violacé, ou pour mieux dire ardoisé,
avait été pris dans une casaque dont les coutures blanchâtres furent
jadis recouvertes de passementerie. Deux boutons dépareillés pris dans
un lacs d'aiguillettes, et deux aiguillettes, passées dans une
boutonnière, retenaient, vaille que vaille, ce remarquable pourpoint,
fermé sur la poitrine. Les défauts en étaient d'ailleurs cachés par un
baudrier à demi neuf, mais veuf de ses récamures d'argent, dont la place
se lisait, en arabesques bleu de roi, sur le fond passé au bleu turquin
du damas. L'épée de cuisse, prise dans les pendants aux boucles et aux
mordants de bronze, avait sa garde de fer dédorée au point que c'était à
la seule rouille qu'elle devait sa chaude couleur. Le fourreau de cette
épée, garni de maroquin brun, laissait voir son corps de hêtre en plus
d'un endroit, et la peau était partout ailleurs tant éraillée qu'on la
pouvait croire remplacée par de la peluche roussâtre.

De la chemise, proprement blanchie sans apprêt, le col largement ouvert
et rabattu sur les épaules dégageait un cou hâlé qui supportait une
petite tête ronde à cheveux châtains, embroussaillés, dont les boucles,
pourtant soyeuses, longues, recourbées en crosse, s'échappaient, avec
une allure de révolte, du feutre, délavé et lustré par les rais du
soleil et l'eau du ciel, dont Louis-Antoine se coiffait crânement
jusqu'à la racine du nez. Ses yeux bruns n'en brillaient que mieux dans
cette ombre des bords mordillés, dentés, bossués, mais, malgré tous ces
accidents de contour et de surface, invariablement déclives. Et ces yeux
éclairaient toute la face juvénile, à peine estompée d'un duvet de fruit
et dorée telle une prune qui sèche au soleil. Et la mine de ce garçon
était douce et sauvage comme le parfum des fleurs des champs, le cri des
oiseaux de marais, la plainte du vent dans les saules. On devait
évoquer, à le voir, ces divinités élémentaires, vénérées par les
anciens, et qui empruntaient aux pâtres des monts Sabins leur figure
étonnée et joyeuse, cependant que leurs yeux profonds laissaient lire
les mystères bienfaisants de la mère nature qui chérit ceux-là surtout
qui vivent dans la paix inviolée des forêts sombres et moussues, où les
flèches d'or du soleil se trempent dans les eaux glacées et silencieuses
des lacs.

Louis-Antoine, sans penser aux faunes et aux sylvains, non plus qu'au
\emph{Saint Augustin} sous peau de truie dont sa jeune vigueur méprisait
le poids, marchait sans se presser, mordant à belles dents dans le
croûton qui diminuait à vue d'œil, car son dîner de dix heures était
déjà dans les talons disparates de ses souliers. Le croûton ne dura pas
plus longtemps que la longueur du chemin creux qui mourait devant la
chaumière de Symphorien. Un chien aboya, puis un autre, et enfin un tout
petit qui, flageolant sur ses jambes arquées, galopait de travers et
s'en vint rouler aux pieds de Louis-Antoine tout en cherchant les
mamelles de sa mère. Et la Baude s'accroupit, regardant le jeune garçon,
avec ses yeux jaunes qui brillaient ainsi que des écus d'or sous les
touffes gris de fer de sa tête terminée par une truffe humide et
luisante.

La vieille Jeannette apparut, la quenouille à la ceinture, chassant
devant elle six couples d'oies qui revenaient de tondre l'herbe du
fossé. Le jars allongea le cou, ouvrit largement son bec jaune, siffla,
s'enfuit de côté. Toute la bande le suivit du côté du toit à porcs pour
se réfugier dans la mare où les poules riveraines se chamaillaient avec
les canards qui accostaient en nageant.

Jeannette, sans cesser de tourner sa quenouille, se crut autorisée par
son âge à prêcher Louis-Antoine de Primelles, dont elle avait vu le père
quand il n'avait, lui aussi, que seize ans. Mais, suivant en cela
l'usage observé dans tout le pays, elle ne parla pas du défunt baron à
son fils. Elle le critiqua sur le mauvais état de ses hardes, en tout
indigne d'un gentilhomme sachant lire dans les plus gros livres, proposa
de battre le pourpoint, qui n'avait pas vu les vergettes depuis un grand
mois, et assura Louis-Antoine que dans ce bas monde on ne jugeait pas
les gens autrement que sur la mine\,: «\,On se devait à son état\,!\,»

Quand il en vint à l'arquebuse de chasse, elle s'excusa de ne pouvoir la
prêter. «\,Son fils Marin l'avait justement prise, ce matin même.\,»

Et, comme Louis-Antoine insistait, la vieille, continuant de rouler son
fil de laine, entra dans des explications où elle s'embrouilla, mêlant
le parler français à son jargon berrichon, jusqu'à ce qu'elle avouât que
son fils était parti pour tirer des lapins, sinon mieux, du côté de
Conquoy, «\,parce qu'en ce jour la terre, de ce côté, était sans
surveillance\,».

Tout en regrettant le contre-temps, Louis-Antoine s'intéressa à
l'entreprise de Marin. Celui-là était un braconnier réputé, dont le
marquis de Bannes avait dit qu'il le ferait pendre si jamais il le
prenait sur son bien. Mais narguant les gardes du marquis, le plus grand
maître de terres à dix lieues à la ronde, Marin continuait de braconner.
Et son commerce de gibier s'étendait si prospère qu'il y employait des
courtiers. D'Issoudun, de Bourges, de Vatan, les commandes affluaient.
Tout cela, pour se passer discrètement, n'empêchait pas le scandale
d'être considérable. Le curé de Lunery, pressenti par le marquis, refusa
«\,à son profond regret\,» de signaler Marin au prône parce qu'il
n'appartenait pas à sa paroisse. Quant au curé de Primelles, M. de la
Bassaie, vieillard humble, ratatiné et d'une probité qui touchait à la
simplicité d'esprit, il répondit au marquis, le jour où celui-ci s'en
ouvrit à lui\,: «\,C'est bien, monsieur, j'en parlerai à
M\textsuperscript{me} de Primelles, si vous ne préférez que j'en parle
au baron en personne.\,» Et, sans même remercier le marquis de l'argent
qu'il lui offrait pour sa très pauvre paroisse, il rompit l'entretien\,:
«\,Gardez vos écus, monsieur. Il est vrai que nos saints sont mal logés
et couverts, mais cela est de peu de prix au regard de Notre-Seigneur,
qui naquit dans une étable, entre un âne, une vache et son veau, pour
vous servir.\,»

Cela se passait en 1627, et aujourd'hui, en l'an de grâce 1633, les gens
de Bannes n'étaient pas plus avancés contre le braconnier Marin
Labrande, fils légitime de Symphorien Labrande, berger en titre du baron
de Primelles et de Jeannette Baudel, son épouse devant Dieu et les
hommes. Marin d'ailleurs se gardait à carreau. Il avait bien soin de se
poser d'un pied sur le domaine de Primelles, tandis que de l'autre il
s'assurait chez le marquis de Bannes pour disposer ses collets.
Tirait-il à l'arquebuse, c'était son chien Souillaud qui allait ramasser
le gibier chez l'ennemi\,: lui, Marin, demeurait sur le bien du baron
son maître.

Et, comme les terres du baron de Primelles se trouvaient entourées de
tous côtés par celles du marquis de Bannes, l'enchevêtrement des pièces
était tel qu'on ne pouvait longtemps marcher par les prés, les
emblavures ou les brandes de Primelles, sans fouler par instants celles
de Bannes. Ce pays semblait avoir été ainsi loti pour le plus grand
avantage des braconniers, qui abondaient en facilités pour passer de
l'un chez l'autre en cas de danger. D'autant qu'une haine sauvage
divisait les deux familles, et cette haine, loin d'avoir pris sa fin par
la mort du baron de Primelles et l'exil du marquis de Bannes, son
meurtrier, était devenue plus atroce.

Partout la guerre\,; à chaque croisée de routes, coin de champ, corne de
bois, un homme de Bannes était embusqué pour trouver le tenancier de
Primelles en faute. Les sergents de justice, les porteurs de
contraintes, entre lesquels Andoche Cottebleue, un Normand de Bayeux, se
faisait remarquer par son activité autant que par son brassard d'argent
et son bâton blanc, tous les recors, les praticiens, couraient entre les
deux châteaux et Issoudun, siège royal. D'autres s'empressaient entre
Primelles, Lunery et Bourges. Le papier timbré pleuvait.

Si les justices n'eussent pas été depuis longtemps rachetées, on se fût
disputé les armes à la main pour pendre. La violence s'en tenait aux
formes légales. On se vexait à coups de papiers, de parchemins, de
citations, d'ajournements. On se poignardait avec les grimoires, on s'en
gardait aussi comme avec un bouclier. Les seuls procureurs d'Issoudun et
de Bourges se réjouissaient de ces choses, pareils aux corbeaux qui
croassent d'allégresse à voir les armées aux prises. La noblesse du
Berry se désintéressait de la lutte, parce que les uns, hostiles au
cardinal ministre, blâmaient sa volonté de tenir le marquis de Bannes
exilé en Autriche, et parce que les autres ne voulaient pas paraître
gens à fouler un ennemi à terre. Toutefois, les sympathies allaient
s'affirmant pour la baronne de Primelles et son jeune fils
Louis-Antoine\,; on reprochait à la maison de Bannes d'exploiter leur
pauvreté en les ruinant à plat par l'obligation de soutenir des procès
iniques\,; on reprochait encore à cette orgueilleuse maison de se
continuer par une mésalliance et un bâtard légitimé.

Et voilà pourquoi Louis-Antoine prenait grand plaisir à entendre la
vieille femme du berger Symphorien lui parler de l'expédition de son
fils Marin. La crainte le tenait cependant que ce braconnier de mérite,
son professeur de chasse, ne tombât aux mains des gens de Bannes. On
disait dans le pays que M. Florimond, le fils du marquis exilé, ce jour
même arrivait au château. Si, par malheur, Marin était pris la main dans
le sac, les choses se gâteraient, car on connaissait M. Florimond et ses
façons d'agir avec le pauvre monde.

La crainte qu'inspirait la vieille Jeannette compensait, à la rigueur,
celle qu'on ressentait quand on parlait du beau Florimond. La femme du
berger Symphorien jouissait dans tout le pays d'une autorité que
personne ne songeait à contester, tant on redoutait ses sorts. Elle
savait en jeter aussi bien sur les bêtes que sur les chrétiens, sur les
enfants nés ou à naître, les poulets, voire les œufs, et les vaches
surtout, puisque chacun savait qu'elle les empêchait à volonté de vêler.
Nul n'ignorait que si Macée La Tuilière ne l'avait pas apaisée par des
excuses publiques et un cadeau de six chapons, ladite Macée n'aurait pu
nourrir son nouveau-né, puisque, pour un imprudent propos sur Jeannette,
elle avait vu tarir son lait subitement.

Quant au vieux Symphorien, son autorité n'était pas moindre, et son
infaillible science de berger ne faisait pas tort à sa réputation de
découvreur de sources. Symphorien était le sourcier de la région
d'Issoudun. Quand il tenait entre ses doigts sa baguette de coudrier
fourchue, une branche dans chaque main, et qu'il marchait prudemment,
suivant la direction de ce rameau indicateur, sans le quitter du regard,
les cheveux se dressaient sous les coiffes et les chapeaux, la sueur
perlait aux fronts. Nul n'eût osé troubler le silence, et l'on
s'essayait à marcher sans bruit. Tout à coup, obéissant à des forces
mystérieuses, la baguette se courbait, s'abaissait jusqu'à toucher
terre. L'on creusait, et la source jaillissait aussitôt. Cela était bien
connu. Aussi chacun cherchait à s'assurer le bon vouloir de Symphorien
et protégeait Marin, à l'occasion. Et l'on n'eût trouvé d'Issoudun à
Bourges, non plus que de Condé à Venesmes, homme ni femme pour témoigner
contre lui.

Posant sa quenouille sur le banc de la porte, où une chatte formait de
ses quatre pattes corbeille à ses petits, la vieille s'en fut chercher
des collets\,:

--- Prenez-les, mon mignon, je les ai tressés de mes mains, et je m'y
connais, je puis le dire. Voyez, tous sont de crin gris ou roux qui ne
brille pas au soleil, à six crins tordus par collet. Avec cela on
prendrait même un chevreuil. Tenez, en voilà six\,! Il y a plus d'une
belle coulée sous les vieux murs du parc de Bannes. Je les regardais
encore hier. Ce serait péché de ne pas reprendre sur ces mécréants un
peu de ces beaux biens dont ils firent tort au défunt baron votre
père\ldots{} que Dieu reçoive en sa grâce\ldots{} et vous aussi.

Furtivement, elle se signa, en marmonnant des mots sans suite, étendit
sa main sèche sur l'héritier de Primelles, comme si elle appelait sur
lui la protection de ces puissances du mal dont on la croyait l'esclave,
reprit sa quenouille et rentra sous son pauvre toit\,: sans arrêter son
attention sur ces choses, car il n'aimait point se fatiguer inutilement
l'entendement, Louis-Antoine tira vers la garenne de Tonlieu, avec ses
collets et son \emph{Saint Augustin}.

Bientôt, couché à plat ventre dans l'herbe épaisse où il disparaissait
tout entier, il put surveiller les lapins qui, en contre-bas, prenaient
leurs ébats dans le voisinage des lacets de crin que sa main déjà
experte avait disposés aux bons endroits.

Le poste d'observation de Louis-Antoine répondait à toutes les exigences
de cet art de la fortification, qui serait le premier de tous si
l'expérience ne prouvait qu'il n'y a point de place si forte qui ne
puisse être réduite quand elle est bien attaquée. Le fort de
Louis-Antoine était un plateau en façon d'éperon, ou de cavalier, pour
qui préfère les termes de la poliorcétique, un plateau saillant, enfin,
couvert par quelques ouvrages avancés. Ces ouvrages avancés n'étaient
autres que des blocs de roches massés sur la pointe où ils se
disposaient en parapet crénelé. Puis ils descendaient à pic jusqu'à la
plaine doucement ondulée qui dévalait insensiblement vers les bords
marécageux du Cher. Encore que l'eau fût éloignée d'une bonne
demi-lieue, on la voyait de cet éperon comme si elle eût été à cent
toises, avec ses aulnes, ses peupliers, ses saules et ses sables çà et
là coupés de roseaux, car rien n'interrompait l'uniforme tapis de la
plaine. À peine quelques pièces labourées montraient-elles leurs arbres
fruitiers clairsemés\,: tout le reste était brandes ou pacages.

Louis-Antoine, en regardant sur sa droite, put voir, au pied des coteaux
du Rillé, le vieux Symphorien debout, ravaudant un bas, et autour de lui
ses moutons pressés en masses floconneuses, roussâtres, que ses chiens,
galopant en cercle, resserraient ou dispersaient tour à tour. Tout cela
était à lui, Louis-Antoine, baron de Primelles, seigneur de
Saint-Godolphin. Mais, à gauche, les prairies de son bien étaient
séparées en deux pièces par une enclave du domaine de Bannes, enclave
broussailleuse, pierreuse, laissée inculte pour que le gibier y trouvât
de bonnes remises d'où il sortirait la nuit pour gâter la terre de
Primelles. Marin était là, heureusement, pour veiller au grain. Avec
quelques garçons de charrue, assistés du porcher et d'autres paysans qui
n'avaient pas froid aux yeux, il faisait des battues de nuit, raflait le
gibier de tout poil, bêtes noires et bêtes rousses. En une saison il
avait pris un cerf et deux biches, attirés du parc jusqu'à cette remise
par plusieurs moyens dont un seul était bon pour envoyer le dit Marin
aux galères.

Si les gardes du marquis voulaient s'en mêler, les coups de bâton
pleuvaient sur les épées de chasse, cassant les lames et fêlant les
têtes\,: les gardes devaient rentrer dans leur friche. Qu'ils
s'avançassent à découvert, Symphorien, son garçon Honorin et les valets
bergers criaient haro sur cette canaille de vert vêtue qui venait
évidemment pour voler les moutons du baron\,: et l'on s'envoyait des
amitiés, le couteau ou l'arquebuse au poing. De ce côté les gens de
Primelles tenaient ceux de Bannes en échec. Mais il n'en était pas ainsi
partout.

C'est pourquoi Louis-Antoine, les coudes écartés posant à terre, le
menton dans les deux mains, regardait sans amitié la masse régulière et
imposante du château de Bannes qui dressait, en face de lui, à cinq
cents pas de distance, en contre-bas, ses murs de pierres blanches,
appareillées à refends, avec leurs cordons de briques, ses pavillons à
l'italienne savamment opposés, ses longues fenêtres décroissant en
hauteur suivant les trois étages, ses toits de tuiles avec les
monumentales cheminées de briques à ancres fleuronnées, et sa ceinture
d'eau où les cygnes voguaient à l'exemple des galères, en prêtant leurs
ailes, doucement soulevées, comme voiles au vent léger qui soufflait de
l'est et sous quoi ondulait la claire feuillée des saules.

S'il se fût retourné, Louis-Antoine aurait vu son château à lui, ses
tours noirâtres, marbrées de vert par la mousse et le lierre, son donjon
démantelé, ses murs en ruines, ses douves bourbeuses où barbotaient des
canards. Mais peu enclin à l'envie, d'intelligence trop paresseuse pour
s'arrêter à des comparaisons attristantes, habitué dès sa première
enfance à cette inégalité que sa sagesse un peu lourde lui indiquait
comme la première condition de tout ce qui vit ici-bas, Louis-Antoine
préféra s'intéresser aux lapins. Il crut en voir un cabrioler et
demeurer pris par la tête dans le fouillis de ronces et d'ajoncs où
étaient tendus les collets. Rampant avec mille précautions, il avança
vers la pointe extrême du plateau.

A ce moment, il sentit quelque chose de piquant et de raide qui lui
offensait le mollet, et il se retourna, comprenant que son bas, ayant
fini de descendre, laissait sa jambe exposée sans défense aux injures
des chardons. Mais ce n'était pas un chardon, puisque maintenant on le
pinçait. Louis-Antoine crut distinguer une main\,; mais cette main,
qu'il avait vue rouge, avait disparu aussitôt. Déjà il se soulevait sur
ses poignets, quand il entendit un rire frais, argentin, un rire de
cristal, puis une voix qui l'appelait par son nom.

Et un visage espiègle apparut, entre les légers chaumes et les fleurs de
mai, coiffé d'un merveilleux chapeau en pain de sucre qu'anoblissait un
bouquet de plumes de héron\,:

--- C'est toi, Catherine\,!

--- Oui, c'est moi\,!

--- Tu m'as piqué avec une herbe, et pincé avec un gant\,!

--- J'aurais dû te pincer jusqu'au sang, Louis-Antoine\,! Car je t'y
prends, à marauder avec des collets\ldots{}

--- Catherine, tu ne me vendras pas\,?\ldots{} D'ailleurs M. de Montenay
me l'a permis\ldots{}

--- De poser des lacets à lapins\,?\ldots{} J'en doute\ldots{} Mon
pauvre Louis-Antoine\,!

Leurs rires se mêlèrent. Progressant à plat ventre dans l'herbe épaisse
où n'apparaissaient que leurs têtes, Catherine et Louis-Antoine se
trouvèrent ainsi face à face. Allongeant les bras, ils s'embrassèrent
franchement, puis demeurèrent tous deux, le menton dans les mains, se
regardant avec une satisfaction muette et joyeuse.

La mine de Catherine de Lépinière était aussi éveillée et ouverte que
celle de Louis-Antoine de Primelles était timide et sauvage. Brune de
cheveux, très blanche, avec des yeux bleu de mer, le nez droit, les
sourcils franchement arqués et la bouche moqueuse, cette jeune fille de
quinze ans avait une figure à la fois sérieuse et badine qui faisait
qu'on ne la pouvait regarder sans se sentir pris par l'envie de lui
vouer une amitié sans limites. C'était une figure franche et honnête,
mais dont le petit menton, quelque peu saillant, bien pointu,
contrastait avec la largeur du front bombé et poli, et annonçait un
courage de bonne étoffe et une volonté sûre de soi.

Les grandes boucles ondulées qui encadraient la face se massaient de
chaque côté en un écheveau bouffant que retenait, à son extrémité, une
rosette de rubans couleur de feu, dont l'éclat s'augmentait par celui de
ces cheveux si noirs que leurs reflets en paraissaient bleus au soleil.
Le chapeau de castor blanc, avec sa forme haute en pain de sucre et son
aile relevée sur la gauche par une agrafe émaillée, s'enfonçait
crânement sur cette tête fière et mignonne. Un gant de fauconnier en
cuir blanc couvert de velours vermeil habillait la main droite, et sa
garde à houppes montait jusqu'au coude.

--- Écoute, Louis-Antoine, dit Catherine, je vais te donner deux
perdrix. Mais tu me promettras de mieux travailler avec le curé. Il
n'est pas content, sais-tu, et l'on répète partout que tu sais à peine
écrire. Quoique cela ne soit pas vrai, la chose m'offense.

--- Je t'assure, répondit Louis-Antoine d'une voix dolente qui indiquait
plus la paresse que la contrition, que je fais tout le possible pour
m'instruire. Vois\,: j'ai apporté jusqu'ici ce gros livre. Il est fort
lourd. C'est un \emph{Saint Augustin}, et je me préparais, au moment où
tu m'as appelé, à apprendre ma leçon.

--- Oui, en regardant sauter les lapins. Je gage que tu as encore tendu
des collets, là, en bas\ldots{} chez nous\ldots{} C'est-à-dire chez la
Drapière.

La figure de Catherine tenta de prendre un air sévère. Elle n'y réussit
qu'à moitié, en vérité\,: ce fut un air emprunté. Avec la gravité
nécessaire, la jeune prêcheuse menaça Louis-Antoine de son index habillé
de cuir blanc et de velours rouge. Ainsi ganté, rouge en dessus, blanc
en dessous, ce doigt ressemblait à un petit biscuit glacé de confiture.
Et Catherine reprocha à Louis-Antoine diverses inobservances de son état
de noblesse\,:

--- Est-ce une besogne digne d'un gentilhomme de tendre des collets sur
la terre du voisin\,? Si encore tu tirais avec une arquebuse. Car tu as
une arquebuse, je la connais, avec une crosse de néflier et des rinceaux
de cuivre. Une vieille arquebuse, elle a quinze ans, pour le
moins\ldots{} comme moi.

Baissant la tête, Louis-Antoine rougit et avoua qu'il n'avait ni poudre,
ni plomb, ni argent pour en acheter\,: «\,Si sa mère ne lui donnait pas
d'argent, c'est qu'elle en manquait elle-même.\,»

Catherine regrettait déjà ses paroles. Prenant à deux mains la tête de
l'enfant, elle l'embrassa gentiment. Et, retenant les larmes qui
étouffaient sa voix, elle murmura très vite\,:

--- N'aie pas de chagrin, mon pauvre Louis-Antoine, cela ne peut
longtemps durer. De la poudre, du plomb, je t'en donnerai dès
demain\ldots{} Allons, courage, ne t'attriste pas ainsi\,!\ldots{}
N'es-tu pas un homme\,?\ldots{} Moi, qui ne suis qu'une fille, si je te
disais\ldots{}

Mais Louis-Antoine ne l'écoutait déjà plus, tant ce garçon était
distrait et bizarre. Il demandait à voir les perdrix promises, comme
s'il craignait déjà de les perdre. Sa nature timide, défiante et
inquiète flairait partout un mécompte.

Catherine, se redressant sur ses genoux, entreprit de dégager les
perdrix pendues entre ses grandes poches de ceinture, sous sa robe de
velours gris. C'était une robe fendue devant et derrière avec un
devantier assorti, du modèle dont usent femmes et filles qui montent à
cheval suivant la manière des hommes. Les mouvements de
M\textsuperscript{lle} Catherine dérangèrent l'autour qui sommeillait,
la panse pleine, à son côté, dans l'herbe. Et, gauchement, l'oiseau
s'avança sur ses jambes hautes et grêles, déployant à demi ses cerceaux
arrondis, avec cette allure dégingandée et boiteuse qui montre combien
agirent sagement les fauconniers en donnant le nom de mains aux pieds de
ces créatures aériennes qui n'ont que faire de marcher.

Hochant de sa tête plate éclairée par deux larges yeux dont la pupille
semblait taillée dans une pierre de jayet enchâssée dans le cercle de
topaze de l'iris, l'autour grinça de son bec festonné, sautilla et,
courbant son dos bossu, parut saluer cérémonieusement Louis-Antoine.
Puis il se posa sur le \emph{Saint Augustin}, envoya sournoisement et
vivement une belle traînée blanchâtre sur l'épée reposant auprès, ce qui
n'augmenta pas le lustre de la gaine éraflée, et enfonçant son cou entre
ses épaules, gonflant son ventre étoilé, s'endormit noblement.

Catherine, cependant, ayant relevé un pan de sa lourde jupe, s'égaya à
voir que le sang des perdrix tachait son haut-de-chausses de revêche et
jusqu'à sa botte de chamois où tintait un éperon d'argent.

À tout prendre, dans son riche habit de cheval où s'opposaient le
velours de Gênes, la lucquoise et autres draps de soie, avec ses
broderies et ses plumes, M\textsuperscript{lle} de Lépinière n'était pas
beaucoup plus soignée de mise que son ami Louis-Antoine. Si la qualité
des tissus était plus riche, la nature des vêtements demeurait également
singulière.

Sur la robe, passablement fanée et qui plaidait, non sans succès, en
séparation de corps avec sa doublure en bouracan de Bourgogne,
retombaient les pans d'une hongreline en drap blanc, surtout militaire,
dont les dames, ce semble, n'avaient jamais usé jusque-là. De cette
hongreline, au reste bien coupée, l'allure originale s'augmentait par la
brièveté des manches retroussées, à parements vastes et boutonnés qui
remontaient jusqu'aux épaules lorsque la manche elle-même ne descendait
qu'à la moitié de l'arrière-bras. Là elle découvrait une autre manche,
celle du corsage, manche de brocart à la grande chiquetade, montrant la
doublure de damas zinzolin qui s'échappait par larges bouillons, au
mépris des boutons et des agrafes. Un vaste col de point coupé, bordé de
picots en fleurons, complétait cet habillement cavalier.

Mais la hongreline, galonnée de soie grise sur toutes ses tailles,
richement garnie d'effilés et d'autres ouvrages de brodeur, joignait à
ces agréments des à-jours, c'est-à-dire de nobles accrocs, nobles parce
qu'ils donnaient la preuve que M\textsuperscript{lle} Catherine de
Lépinière, dans les actes de sa vie comme dans les actions de la chasse,
allait toujours de l'avant, sans s'occuper des épines, des rocailles et
divers empêchements.

Dans sa chevelure, les débris de folles herbes voisinaient avec les
rubans. Une chenille se promenait sur la forme de son chapeau.
M\textsuperscript{lle} Catherine l'y avait posée parce qu'elle la
trouvait remarquable avec sa livrée vert gai, tranchée de noir, et ses
caroncules orangées. Un peu plus haut, deux papillons blanc de farine,
avec bout des ailes couleur d'aurore étaient épinglés. Quant aux épis
piquants des graminées agrestes, ils lardaient la hongreline et la jupe
en tous endroits. Le brocart des manches en gardait sa part, comme aussi
des traces de l'autour. Mais ce dernier inconvénient est un honneur pour
qui sait porter son oiseau. Enfin c'était, chez M\textsuperscript{lle}
de Lépinière, un de ces beaux désordres que les grimauds de lettres nous
voudraient donner pour un effet de l'art.

M\textsuperscript{lle} Catherine se souciait peu de tout cela. Sortie
pour voler à l'oiseau, elle avait pris trois perdrix avec son tiercelet
d'autour. Alors elle avait pensé à Louis-Antoine et au plaisir que lui
ferait un joli cadeau de gibier. Connaissant son habituelle remise, elle
avait tôt fait de le découvrir enfoncé dans son gazon. Elle avait donc
attaché son cheval dans le chemin creux de Bannes, et, grimpant à
travers les broussailles, à hongreline perdue, rejoint son ami et
protégé.

Au vrai, les choses ne se passaient pas de façon aussi simple que le
racontait M\textsuperscript{lle} Catherine de Lépinière, belle-fille du
marquis de Bannes par sa mère, Anne de Cuzance, qu'il avait épousée,
veuve, en premières noces, avant de contracter avec la drapière, Julie
Péréal, cette union qui l'avait fait mettre au ban de toute la noblesse
du Berry. Les rencontres des deux enfants étaient toujours concertées,
et il fallait que Louis-Antoine possédât une forte dose d'inconsciente
apathie pour ne point s'en apercevoir. En tous cas, il ne disait rien.

Par Jeannette, Symphorien, Marin et les autres bergers, Catherine était
tenue au courant de tout ce qui concernait Louis-Antoine. Car ces deux
enfants ne se voyaient qu'en cachette, par la muette complicité du vieux
Symphorien, le berger marieur dont on chantait partout que ceux qu'il
avait unis devant Dieu dans la solitude des champs étaient mieux mariés
que par le curé et son registre. Ainsi Catherine et Louis-Antoine
vivaient entourés de cette protection occulte. Par ses émissaires, la
vieille Jeannette avertissait la jeune fille, lui indiquait l'endroit où
l'indolent Louis-Antoine faisait l'école buissonnière. Et Louis-Antoine
ne s'étonnait jamais de voir son amie apparaître subitement dans tout
lieu désert où il vivait dans le silence des prés ou des bois.

Mais Louis-Antoine ne s'étonnait de rien.

Si donc, ce quinzième de mai de l'an 1633, Catherine tenait fidèle
compagnie à Louis-Antoine, celui-ci ne se doutait guère que la vieille
Jeannette avait tout fait pour l'obliger à choisir pour théâtre de ses
exploits le petit plateau de Tonlieu. Pour le tenir en place, l'empêcher
de courir, elle avait inventé l'histoire de Marin et ainsi pu refuser
l'arquebuse, parce que les collets obligent le braconnier à ne point
s'écarter de l'endroit où il les a tendus.

Jeannette, le matin même, avait vu M\textsuperscript{lle} Catherine et
lui avait annoncé, avec cet air mystérieux que sa profession de jeteuse
de sorts l'obligeait à garder en toutes circonstances, que le jeune
baron serait sans faute, un peu après midi, sur le plateau de Tonlieu.
Et M\textsuperscript{lle} Catherine, qui n'avait point la tête faible,
savait que la vieille Jeannette empruntait cet accent augural à la ferme
volonté de sauvegarder son autorité de devineresse. Obligée à jouer sans
cesse son rôle pour ne point l'oublier, Jeannette n'ignorait point non
plus ce que pensait M\textsuperscript{lle} Catherine. Et l'accord
demeurait parfait.

A supposer même, car l'on doit tout prévoir quand on s'est établie
prophétesse, que Louis-Antoine et son \emph{Saint Augustin} se fussent
attardés aux écrevisses du ruisseau, Jeannette aurait trouvé un moyen de
l'attirer vers Tonlieu, où M\textsuperscript{lle} Catherine comptait
voir son protégé.

Car Louis-Antoine était bien le protégé de Catherine de Lépinière. Dans
cette amitié chaste et naïve de deux enfants du même âge, que tout était
pour séparer sur la terre, dans cette union morale, la jeune fille était
l'homme, et le garçon était la femme. Mais, au contraire de la plupart
des filles qui n'usent de leur ascendant que pour molester les garçons
et les asservir à leurs frivoles et effrénés caprices, Catherine de
Lépinière tentait l'impossible pour façonner à son exemple ce caractère
gauche, timide, sauvage et par-dessus tout indécis.

Les trois perdrix réussirent à rasséréner Louis-Antoine. Il les examina,
les soupesa, les estima en connaisseur. Comme il y en avait une grise et
deux rouges, il choisit les deux rouges, et Catherine garda la grise
pour ne point rentrer sans rien avec du sang frais sur son habit. La
chose eût paru singulière. Et Catherine, se sachant espionnée, montrait
une prudence extrême. Puis ils parlèrent de l'autour, et la jeune fille
fit l'éloge de son oiseau, et ils en revinrent, tout naturellement, au
sujet habituel de leurs entretiens, à Julie la Drapière. Ainsi
appelait-on effrontément la marquise de Bannes dans le pays. Pour la
centième fois, peut-être, Catherine allait esquisser un portrait peu
flatté de sa marâtre, quand elle s'écria brusquement, laissant là Julie
la Drapière\,:

--- À propos, tu sais la nouvelle\,? Florimond, son affreux fils, est
arrivé hier, au milieu de la nuit.

De son air le plus indifférent, Louis-Antoine répondit qu'il le savait.
La mère Jeannette le lui avait appris, sans qu'il en eût demandé
davantage. Mais le front jusque-là si pur de la jeune fille se coupait
d'un pli profond. Tout entière à sa préoccupation douloureuse, elle
murmura d'un ton suppliant\,:

--- Surtout, ne le cherche pas\,!

Louis-Antoine, toujours distrait et somnolent, ne parut pas entendre.
Mais Catherine insista\,:

--- Jure-moi, Louis-Antoine, que tu ne chercheras pas à le rencontrer.

Alors il répondit distraitement\,:

--- Pourquoi voudrais-tu que je le rencontrasse\,? Qu'ai-je besoin de le
chercher\,? Tu sais bien que nous ne voulons rien avoir à traiter avec
lui. C'est un méchant bâtard, et sa mère, Julie la Drapière, une
coquine. Qu'y pouvons-nous\,?

Il hésita, s'arrêta, puis reprit, fronçant les sourcils\,:

--- Aurais tu à te plaindre de lui\ldots{} Raconte-moi\ldots{}

Vivement Catherine l'interrompit\,:

--- Certainement non\,! Que pourrait-il me faire\,?\ldots{} D'ailleurs,
maintenant je les tiens. Écoute\,!

Allongés à nouveau dans la prairie, en face l'un de l'autre, ils
tendaient leurs faces pleines de santé et de jeunesse au-dessus des
folles herbes, des coquelicots, des bleuets, des chardons et des reines
des prés où butinaient les bourdons vêtus de peluche rousse, orange et
noire. Et les deux enfants des maisons ennemies étaient pareils à ces
jeunes faunes aux yeux clairs qui se regardent curieusement dans les
sous-bois herbeux, s'opposant leurs fronts cornus pour cosser à
l'imitation des béliers dont ils ont les armes.

Le menton dans ses paumes, les coudes dans la terre, son autour posé sur
son épaule, Catherine narrait les actions de sa belle-mère Julie, dite
la Drapière, marquise de Bannes\,:

--- Imagine-toi que, depuis que le marquis est en exil, la coquine
s'ingénie à me dépouiller de mon argent. Elle avait pensé d'abord
m'obliger à épouser son Florimond\ldots{} C'était mal me
connaître\,!\ldots{} Et tu sais bien, Louis-Antoine, que c'est toi, toi
et pas un autre, qui seras jamais mon mari.

Louis-Antoine ne crut pas devoir interrompre Catherine. Elle lui avait
si souvent dit cela que lui le trouvait maintenant tout simple.
D'ailleurs, ses idées sur le mariage étaient naturellement vagues et
obscures, comme toutes celles, du reste, qui ne se rapportaient pas à la
terre, aux plantes et aux remises du gibier.

--- Tu sais que mon beau-père le marquis a juré qu'il me laisserait
maîtresse de mon sort. Il n'est pas homme à enfreindre son serment.
Depuis qu'il a dû partir en exil, Julie a repris du poil de la bête.
Mais, pour grande que soit son audace, --- car je suis convaincue
qu'elle ignore tout des engagements de son mari envers moi, --- elle ne
prévaudra pas contre ma volonté.

Louis-Antoine regardait Catherine avec une admiration non feinte. Il lui
savait gré de s'exprimer d'une manière aussi ferme, s'extasiait sur sa
facilité de parole, lui qui devait réfléchir longtemps et s'y prendre
souvent à trois fois pour demander les choses les plus simples.

Sans se douter de cette admiration qui valait par sa simplicité et sa
franchise, Catherine de Lépinière continuait de charger la marquise
Julie\,:

--- J'ai déjoué ses artifices, percé à jour ses intrigues, rompu les
filets où elle me comptait prendre. Je connais ses complots avec le
procureur de Bourges, qui embrouille à plaisir la succession de mon
tuteur. Longtemps je m'étonnai de voir les lettres que j'écrivais à mon
beau-père demeurer toutes ou presque toutes sans réponse. Je découvris
d'où venait le coup\,: Julie dérobait mes lettres avec la complicité
d'une chambrière\,; et le marquis ne recevait que celles où je ne lui
mandais rien d'important. Je crois même qu'elle en a fabriqué de
fausses\,; mais je n'ai pas de preuves. Dès lors, grâce à M. de
Montenay, je pus tenir mon beau-père au courant des entreprises de sa
femme. Si j'en juge d'après la mine que fait depuis quelques jours notre
Julie, la réponse du marquis fut terrible. Julie a beau mettre un pouce
de rouge, son teint est couleur de navet. Florimond ne semble guère plus
à son aise. Le marquis leur a fait connaître ses volontés, et mes deux
voleurs ont baissé le nez, pris la main dans le sac. Ce matin, le
procureur de Bourges, le vieux Duvau, est arrivé au château. Quand il
est reparti, avant midi, il était encore plus semblable à un navet que
Julie.

Louis Antoine hochait gravement le menton, approuvant sans trop
comprendre, cependant que Catherine se soulageait le cœur à raconter la
déconfiture de ses ennemis\,:

--- Ils me volaient tous, te dis-je. Jamais je ne pouvais disposer d'un
sou. Tandis qu'aujourd'hui la mielleuse Maroie, la fille de chambre
favorite de Julie, est arrivée avec Nicole Deleuze, sa trésorière, un
gros sac d'argent, et des explications sur des termes en retard\ldots{}
C'est un os que l'on me donne à ronger\ldots{} Tiens, vois comme je suis
riche\,!

À genoux, Catherine s'occupait d'atteindre ses poches, quand elle
s'arrêta, tressaillit. Sa mine devint attentive. Du pied du plateau où
il se tenait depuis des heures, le vieux Symphorien s'éloignait
maintenant tirant vers les prés du Cher, et ses moutons trottinaient
massés en un grand triangle, avec les chiens en queue. Et le berger
soufflait, sans repos, dans son cornet à bouquin, qui rendait des sons
aigres, monotones et traînants.

Catherine se dressa sur ses pieds et tendit l'oreille. Alors elle
entendit un hennissement aigu, puis la plainte irritée d'un cheval.
Vivement, elle regarda par-dessus un bloc, recula, sauta sur l'épée de
Louis-Antoine et se lança à corps perdu dans la descente embroussaillée,
sans s'occuper de l'autour, qui, décollé de son épaule par un rameau,
plana, puis l'accompagna en volant au-dessus de sa tête.

Le geste impérieux par lequel elle cloua en partant Louis-Antoine à
terre n'aurait pas empêché le jeune garçon de la suivre si le voisinage
de Symphorien et de sa houlette en cornouiller bien ferré ne l'eût
rassuré sur les risques possibles de l'aventure où se lançait son
amie\,:

«\,Quelque croquant aura battu la jument pie de Catherine, et elle va me
corriger d'importance le drôle à coups de fourreau d'épée\ldots{} Voyons
cela\ldots\,»

Et, rampant, comme à son habitude, il avança, sans se découvrir,
jusqu'entre deux blocs de roches d'où il pouvait tout surveiller.

Catherine, toujours courant, était arrivée dans le chemin creux, limité
par deux haies, qui séparait le domaine de Bannes de celui de Primelles.
Source d'éternels conflits, ce chemin, pour lequel, bon an mal an, se
cassaient une douzaine de têtes, partait des prés du Cher pour aboutir à
la grande avenue du château de Bannes, où il se bifurquait dans la
direction de Primelles. Dans ce chemin, donc, Catherine se trouva nez à
nez, si l'on peut dire, avec un grand cheval dont le cavalier, armé d'un
fouet de chasse, maltraitait Mahaut, sa jument guilledine.

Or le chemin était si étroit que le cavalier ne put s'échapper, ce qu'il
aurait bien voulu faire quand il eut reconnu M\textsuperscript{lle} de
Lépinière qui se dressait devant lui. Mais il aurait dû la renverser
pour passer outre\,; et, pour revenir en arrière, c'est à peine s'il
avait la place de tourner. Car, à cet endroit même, s'avançaient de
chaque côté quelque six sauvageons de poiriers, de merisiers et de
pruniers, tortus, étêtés, noueux, vrais culs-de-jatte parmi les arbres
fruitiers, et qui allongeaient leurs branches basses, horizontales, en
zigzag, à moins de sept pieds du sol.

Ainsi M. Clément Malompret, premier valet de chambre de M. Florimond,
dont il portait la livrée chamarrée et galonnée sur toutes les coutures,
vert et or, avec des chevrons bleus et des ancolies d'argent alternés,
demeura-t-il mal à son aise, sans oser ni fuir ni marcher, et tâchant de
réduire sa monture, un fort étalon barbe que les ruades de Mahaut
mettaient en désordre.

Blanche de colère, les yeux plus brillants que les boutons dorés qui
couraient en deux lignes sur les manches de la mandille du laquais,
Catherine de Lépinière dit à M. Clément\,:

--- Viens ici\,!

M. Clément, sans être absolument flatté de cette invitation péremptoire,
n'osa pas pourtant y désobéir. Il essaya de pousser sa bête qui
renâclait et pointait. Mais Catherine cria\,:

--- Entends-tu ou n'entends tu pas\,? Et qui t'a rendu si hardi de
rester à cheval devant moi\,?

M. Clément mit pied à terre, et, gardant les rênes à la main, s'avança
avec toute la gracieuse confiance d'un écolier qui devine la férule
cachée derrière le dos de son régent. Ce premier domestique était si
troublé, outre qu'il avait à la main gauche la bride de son barbet et à
la droite son grand fouet de chasse, qu'il en négligea de se découvrir.

Du bout de l'épée engainée de Louis-Antoine, M\textsuperscript{lle} de
Lépinière fit sauter le chapeau du drôle, qui recula de deux pas. Alors
elle marcha sur lui et lui cria à la face\,:

--- Vassal, qui t'a permis de toucher mon cheval\,?

Très rouge, décidé à défendre sa dignité contre quiconque, M. Clément
répondit d'un ton rogue, encore que peu assuré, «\,qu'il en avait agi
ainsi par les ordres de M. le marquis mandant défense à toutes gens de
laisser les bêtes vagabonder dans le chemin de la grande avenue\,».

Et il insista\,:

--- C'est si vrai, mademoiselle, que Cottebleue vient de porter huit
exploits à la bicoque de ces mendiants de Primelles, toujours en défaut
avec les bœufs et les moutons errants. Si l'on m'en croyait\ldots{}

Il n'acheva pas. De l'épée toujours engainée, Catherine lui avait
souffleté le visage. Et, pour le malheur de M. Clément Malompret, la
gaine de l'épée de Louis-Antoine était tant fatiguée qu'un des
tranchants de la lame, passant entre les attelles de hêtre, coupa le
maroquin et aussi la belle, fade et insolente face du blond Clément,
gâtant ainsi pour jamais le bellâtre qui était revenu de Paris à Bannes
afin d'y raconter les ravages que sa galanterie avait exercés en haut
lieu.

Aveuglé par son sang, car la balafre commençait au sourcil gauche et
rejoignait le coin droit de la bouche, M. Clément trébucha, lâcha bride
et fouet, chut à terre, pleurant, jurant et, d'une même voix, criant
merci et vengeance. Et cela sans que M\textsuperscript{lle} de Lépinière
se retournât même pour le regarder.

Elle était déjà loin, regagnant à pas vifs et pressés le coteau. Elle
rendit l'épée à Louis-Antoine et prit congé avec un «\,Nous nous
reverrons demain\,» qui, s'il laissait son ami dans l'imprécision quant
à l'heure et au lieu de la rencontre, lui laissait cependant la
certitude de revoir Catherine au jour dit.

Et M\textsuperscript{lle} Catherine, descendue dans le chemin, avait
déjà enfourché sa jument\,; car cette jeune personne, en tout hardie
dans ses allures, montait à la façon des hommes. Elle rendit la main, et
la bête bien dressée prenant cette allure guilledine, chère aux Écossais
qui, dit-on, l'inventèrent au vieux temps, et qui tient du traquenard et
de l'amble, déboula d'un pied sûr vers l'avenue du château de Bannes, où
aboutit ce chemin creux qui avait vu la déconvenue de M. Clément. De
celui-ci il ne restait plus de traces.

Quand Catherine se trouva à l'entrée de cette avenue, longue de cinq
cents pas, remarquable par la triple rangée de gros ormes qui la
flanquaient des deux côtés, ombrageant de larges chaussées gazonnées,
elle aperçut un groupe de cavaliers, arrêtés au beau milieu, et qui
causaient avec animation, à en juger par leurs gestes. M. Clément, sur
son barbe, occupait le centre du groupe. Sous son chapeau de castor
gris, en forme de pot à beurre, le valet de chambre portait, comme
l'amour, un bandeau, bandeau fait d'un mouchoir de soie blanc et bleu,
mais taché de sang au point qu'il rappelait les trois couleurs de la
vieille maison du roi.

Trois cavaliers écoutaient son rapport. L'un magnifiquement vêtu, blond
ainsi que Phœbus Apollon, qui réchauffe le monde de ses rayons,
tourmentait les rênes d'un grand étalon isabelle qui s'égayait en
courbettes et secouait les glands de son poitrail et de sa culière en
damas bleu turquin. Des deux autres, plus modestes dans leur habit et
leur mine, l'un, avec le chapeau plat enfoncé jusqu'à son museau
chafouin, se balançait au caprice d'une mule harnachée de maroquin noir.
Et telle était l'attention peureuse avec laquelle il surveillait les
larges oreilles de sa paisible monture, plus occupée des mouches que de
son cavalier novice, que ce personnage à cheveux graisseux et plats,
vêtu de ratine noire, remplissait par la force des choses le rôle de
comparse.

Par contre, l'autre cavalier, dans son vêtement aux trois quarts
militaire, avec collet de buffle, casaque de drap fleur de seigle et
bottes de vache grasse montant à mi-cuisse, bombait le torse plein
d'assurance et de superbe, sur son roussin de trois poils, plus
avantageux encore que le duc de Toscane, propre grand-père de Sa
Majesté, qui se dresse en statue équestre à Florence, comme chacun sait.
Son chapeau de lièvre couleur pain bis, évasé, cambré, avec le bord
relevé au droit du front et l'aigrette, mesurait deux bons pieds en
hauteur. Le col droit de sa casaque sortait du buffle usé par le
hausse-col et supportait un de ces cols en rotonde, plats, empesés,
cartonnés, une vraie golille à l'espagnole. Et sur ce plateau reposait
le menton rasé à l'exception de la royale brune dont les poils raides se
terminaient en pinceau pointu.

Bien qu'ils fussent encore à trente pas d'elle, Catherine avait reconnu
son monde\,: Florimond, son soi-disant demi-frère, le poète parasite
Aimeri d'Olivier, précepteur perpétuel du jeune homme, et M. Pierre
Acresin de Tourouvre, sieur de la Fère-en-Combrailles, attaché par
l'amitié et l'intérêt au beau Florimond, et chevalier servant, à
l'occasion, de Julie la Drapière, marquise de Bannes.

«\,Le voilà bien, se dit Catherine, vaniteux comme un paon, faisant la
roue devant ses courtisans à gages. À droite le spadassin, à gauche le
coupe-jarret de lettres. Que le marquis n'est-il ici avec sa canne\,? Il
me dénicherait lestement cette engeance qui pique avec la plume et
assassine avec l'épée\ldots{} Belle bande de vautours\,! À qui en
ont-ils\,?\ldots{} Pourvu que ce ne soit pas contre Louis-Antoine qu'ils
dirigent leurs traits empoisonnés\,!\,»

Mais la peur qui rend blancs les foies des maroufles n'avait point de
prise sur Catherine de Lépinière, dame héritière de Paudy,
Saint-Lizaigne, Meunet, Rebourcin et Brion, fille de ce Sébastien
Paumier de Sallanches, marquis de Lépinière, capitaine aux gardes, qui
fut tué, la rondache au col et l'esponton au poing, le 9 novembre 1621,
à l'attaque des Sables-d'Olonne, où il combattait en volontaire. De sa
mère Anne de Cuzance, qui fut une sainte sur la terre même aux yeux de
son second mari, le marquis de Bannes, dont la vertu ne fut pas le
principal objet, Catherine avait hérité la sagesse, la force et la
constance. Voilà pourquoi cette charmante fille, livrée par l'aveugle
hasard à des gens peu scrupuleux et d'une ambition qui n'obéissait à
aucun frein, demeurait parmi eux aussi blanche que la létice qui court
sur la boue sans se salir. Mais de l'hermine elle avait aussi l'audace,
la prudence et la souplesse.

Sans ralentir l'allure pressée de sa jument Mahaut qui tricotait sous la
jupe de velours gris descendant de chaque côté ainsi que le panneau
d'une housse, Catherine, son autour sur le poing, donna dans le groupe
et arrêta net sa bête quand elle se trouva au milieu. On la salua très
bas. Un des résultats de ces saluts fut que le poète Aimeri d'Olivier
faillit mourir de peur. Car, devant ces chapeaux et ces plumes se
balançant sous son nez, la mule Alibourne, imparfaitement aveuglée par
ses œillères, marqua son émoi par un écart et une pétarade dont l'effet
fut de placer le favori des muses sur son cou.

Désespérément agrippé par une main à la crinière, par l'autre à l'arçon
de son bât, M. Aimeri perdit ses rênes, son chapeau et un de ses étriers
en façon de chaussons. Il roula d'un côté, se guinda de l'autre, et
reprit son aplomb, mais non son chapeau, que piétinait le roussin de M.
de Tourouvre. Le compliment dont il se promettait beaucoup d'honneur en
fut beaucoup écourté. On l'entendit balbutier\,: «\,Divine
apparition\ldots{} nec teneras\ldots{} cursu\ldots{} læsisset
aristas\ldots{} Camille\ldots{} Diane plutôt\ldots{} Artémise des
monts\ldots{} divine chasseresse\ldots{} Thalestris\ldots{} La clarté de
ses yeux, aux astres\ldots\,»

Lors M. Aimeri d'Olivier, laissant l'avantage à sa mule, tomba assez
gracieusement du côté hors montoir pour qu'on pût croire qu'il avait mis
pied à terre à cette seule fin de ramasser son chapeau. La mule
Alibourne, se sentant libre, ambla vers l'écurie. M. Clément, porteur de
son bandeau noué au-dessus de l'oreille gauche par un vrai nœud d'amour,
piqua des deux pour la rattraper. M. de Tourouvre le suivit, par
discrétion. Le poète désarçonné continua son chemin à pied, et Florimond
demeura seul avec Catherine.

--- Pourquoi, dit-il sans autre préparation qu'une grimace assez laide,
as-tu coupé la figure de mon valet Clément avec une épée\,? Et qui t'a
donné une épée pour cette belle besogne\,?\ldots{} Allons, parle\,!

--- Si tu crois, mon pauvre Florimond, m'intimider avec tes façons de
prévôt, tu es loin de compte.

--- Trêve de verbiage\,!\ldots{} Me répondras-tu\,?

--- Je te répondrai, Pontaillan, si bon me semble.

Florimond, à s'entendre appeler Pontaillan, se mordit les lèvres et
rougit. C'était là son nom de bâtard avant qu'il ne fût légitimé, et
Catherine seule osait l'appeler ainsi, quand les choses n'allaient pas à
son gré. Il haussa les épaules et grommela\,:

--- Cela n'est pas répondre.

--- Je te répondrai quand tu parleras sur un autre ton, et à mon heure.
Va, tu peux rouler tes gros yeux\ldots{}

Florimond, cette fois, fut piqué au vif. «\,Tes gros yeux\,!\,» ces yeux
dont le doux éclat fascinait toutes les belles assez imprudentes pour en
affronter les feux, ses yeux divins, ses yeux vainqueurs, une petite
fille élevée à la rustique osait les appeler «\,tes gros yeux\,»\,!
Florimond cria donc\,:

--- Allons, ne fais point la sotte, et parle\,!

--- Parle toi-même, si tu es capable de t'exprimer sans rugir. C'est à
toi à me demander pardon.

--- Catherine\,!

--- Non, vraiment, Florimond, tu n'es pas beau avec ces yeux de chouette
clouée à une porte. Crois-tu m'effrayer\,?

Et son rire monta, clair, menu, si léger qu'il donnait l'impression de
quelque chose d'insaisissable, quelque chose comme le bourdonnement du
moucheron qui vous entoure de ses cercles invisibles, sans qu'on puisse
savoir à quel endroit il lui plaira de piquer.

La colère de Florimond empourprait son visage, un beau visage régulier,
aux traits fins, et qui avaient cependant on ne sait quoi de mou. Malgré
l'éclat des yeux, la pureté du teint, la noblesse de l'allure, cette
figure disait un caractère lâche et cruel. Florimond, sous les paroles
hautaines et pourtant bien familières de Catherine, haletait et
soufflait comme un taureau. Les veines de son cou se gonflaient. Encore
un peu, et l'apoplexie le terrassait.

Il passa sa colère sur qui n'en pouvait mais, sur son cheval, le
tourmenta de la bride, le maintint sauvagement sous l'éperon. La bête,
folle de douleur, s'échappa de la main, rua de côté. Catherine faillit
être atteinte. Elle évita adroitement, se gara en faisant face, et, sans
se troubler, reprit\,:

--- Là, Florimond, moins de saccades et plus de courage\,!\ldots{} Tu en
viendras peut-être à bout sans brutalité. Après tout, ce Fauveau est une
bête sage, et tu l'exaspères parce qu'au fond tu n'es pas, comme ton
père, un grand écuyer. Mais ne recommence pas ce coup, mon ami, sans
quoi le marquis saura que tu as essayé de m'assassiner, comme l'an
dernier tu tentas de tuer André d'Archelet\ldots{} T'en souvient-il\,?
C'était dans la cour du Fer-à-Cheval.

De pourpre, Florimond devint bleu. Il grinça des dents, ferma les
yeux\,: les arbres tourbillonnaient, le gazon lui apparut rouge.
Cependant son visage gardait le ton bleu passé de l'olive en turquoise
qui pendait au bout de sa moustache. Jamais plus belle boucle de cheveux
blonds ne cacha l'oreille d'une femme\,; et cette boucle fournie, souple
à l'œil, plus fine que la soie, descendait jusqu'au sein gauche. Un vrai
lacet d'amour à étrangler les belles. Seule elle eût suffi à assurer la
gloire de celui qu'on appelait l'Incomparable Florimond, à Paris comme à
Bourges. Quelle moustache de cour, voire celle d'Honoré de Cadenet, duc
de Chaulnes, lui aurait-on pu opposer\,?

Mais, à cette heure, l'Incomparable Florimond ne jouissait plus d'aucun
de ses avantages, dons naturels qui lui valurent de telles séries de
bonnes fortunes, en tous lieux, que les gens les plus experts en ces
sortes de divertissements renonçaient à en évaluer le nombre. L'homme
d'amour tremblait devant la jeune fille. Rien de plus vrai, Florimond
avait peur de Catherine, et cette peur allait s'augmentant depuis la
conversation qu'il avait eue avec sa mère, Julie la Drapière, marquise
de Bannes, le matin même de cette après-midi où M. Clément Malompret fut
si fâcheusement marqué par Catherine de Lépinière.

À force de souffler et de hausser les sourcils, Florimond réussit à
retrouver un peu de calme. Cachant son trouble sous une aisance
gouailleuse, il laissa en paix son cheval et répondit à Catherine\,:

--- Le plus raisonnable cédera. Faisons la paix et permets que je
t'embrasse\ldots{} Tu ne le juges pas utile\,?\ldots{} Comme tu voudras.
Je voulais seulement savoir pourquoi tu as ainsi défiguré Clément, afin
que je le punisse moi-même s'il t'a manqué en quelque façon.

Trop fine pour abuser de sa victoire, Catherine s'empressa de raconter
la conduite du valet, et Florimond, prenant la défense de son homme de
confiance, avoua qu'il trouvait la peine grosse pour un aussi mince
méfait.

Mais Catherine releva sèchement le propos\,:

--- Qui touche la bête touche le maître. Je te l'ai entendu répéter
vingt fois. Aujourd'hui tu es bien bon seigneur. Pour une offense certes
moins grave, tu as tué, il n'y a pas longtemps, le cadet de Maubec,
après l'assemblée de Vatan.

Florimond, pris de court, répliqua à tout hasard, que «\,ce n'était pas
la même chose\,». Et il pria Catherine de lui dire, en toute franchise,
de qui elle tenait l'épée, instrument du dommage.

--- Car, enfin, tu n'en es pas encore, tant bizarre que soit ta
conduite, à sortir, comme la Marion de Mauny, à cheval, avec l'estocade
au côté.

Il avait dit cela d'un ton badin, mais son regard était noir. Catherine
leva le menton, et sa bouche mignonne parut cracher au loin un noyau de
cerise\,:

--- La gouvernante de M. de Mauny est une fort belle personne dont on
dit moins de mal que d'une lingère de Bourges, dont le nom\ldots{}

--- Catherine\,!

--- Cela te contrarie\,? N'en parlons plus. Quant à l'épée, je
l'empruntai à M. de Montenay, qui en a toujours une à mon service.

Elle avait dit ces derniers mots d'un ton froidement menaçant. Florimond
tressaillit. Mais la jeune fille continua légèrement\,:

--- M. de Montenay, qui me regardait voler à l'oiseau dans sa garenne de
Tonlieu\ldots{} Qu'as-tu à froncer le sourcil\,?

--- Ton M. de Montenay a des fourreaux d'une bien mauvaise
étoffe\ldots{} C'est bien. Mais, en attendant, Catherine, ma fille, tu
te compromets. Une fille de bonne maison ne court pas ainsi les champs
toute seule et n'emprunte pas d'épée à des gens\ldots{}

--- Florimond, mon bonhomme, ta morale vient comme la moutarde après le
rôti. M. de Montenay est le fils de mon défunt tuteur, et d'aussi bonne
maison que toi, et même\ldots{}

Il se mordit les lèvres, tracassa son cheval, dans l'attente d'une
nouvelle insolence, mais Catherine n'acheva pas. Alors Florimond crut
s'en tirer en blâmant la demoiselle de se promener ainsi avec des habits
en loques\,: «\,Encore un peu, et on la mettrait dans le même sac que
ces gueux de Primelles.\,»

--- S'ils sont gueux, du moins n'ont-ils rien volé à personne. En
pourrait-on dire autant de toi, Pontaillan\,?\ldots{} Et moi, quand je
vais avec de mauvaises hardes, cela tient à ce que ta noble mère, qui
pourtant s'y connaît en draps\ldots{}

--- Prends garde, Catherine\,!\ldots{} N'insulte pas ma mère\,!

Mais, sans l'écouter, elle continua d'un ton franchement méprisant\,:

--- \ldots{} ne m'en achète pas une aune pour mes robes, mais garde mon
argent pour t'en aider dans tes fricassées de Paris et de
Bourges\ldots{} Méfie-toi, Pontaillan, tout cela finira mal, le marquis
ne sera pas toujours exilé. Adieu\,! Retire-toi de ma présence.

Et, jetant la bride au valet d'écurie qui venait lui tenir l'étrier,
Catherine de Lépinière sauta à terre, donna son autour à un fauconnier
qui attendait, prit les pans de sa robe sous un bras, son devantier sous
l'autre, et se dirigea vers l'escalier par quoi l'on accédait à sa
chambre. L'horloge du château sonnait le coup de quatre heures.

Louis-Antoine, toujours livré dans l'herbe à sa rêverie solitaire,
entendit, lui aussi, l'horloge sonner. Ce bruit le tira de son
engourdissement.

«\,J'ai bien juste le temps, se dit-il, d'arriver chez nous pour la
leçon d'escrime. L'oncle Bouteiller est déjà, j'en jurerais, dans la
salle basse, à vérifier si le carré et son cercle inscrit sont tracés
nettement à la craie, dans les proportions requises, et si toutes les
lettres sont dans l'ordre\ldots{} Quelle sottise que ce cercle
mystérieux\,!\ldots{} Je ne suis pas grand clerc dans la science des
armes, mais cette méthode de Thibaust d'Anvers me semble quelque
chimérique\ldots{} Quelle misère que de se promener ainsi pendant deux
heures sur des lignes blanches, jusqu'à ce que la fatigue vous arrache
l'épée de la main\ldots{} Et, enfin, quel besoin ai-je de savoir manier
l'épée avec l'adresse d'un prévôt de salle\,?\,»

Et Louis-Antoine, tout en marchant, singeait les allures de Robert de
Rustigny, l'écuyer meneur de sa mère, quand elle avait des chevaux.
C'était à cet écuyer, un grand bel homme, sec, élégant, fier et brave
dans ses vêtements râpés, qu'incombait le soin de diriger Louis-Antoine
dans ses exercices\,: «\,Allons, monsieur Louis, attention, s'il vous
plaît\,!\ldots{} Eh quoi\,?\ldots{} Le pied gauche n'est pas dans la
ligne, et votre garde est mauvaise\ldots{} Le bras tendu, que
diable\,!\ldots{} Et le pied droit\,?\ldots{} Ne doit-il pas former
tangente au cercle\,?\,»

Alors intervenait l'oncle, martelant chaque parole d'un coup de canne
sur les lignes croisant leurs réseaux sur le sol, avec leurs lettres
capitales et même le contour des pieds marqués en blanc\,:

«\,Comprends donc, mon fils\,! En somme, rien n'est plus simple. La lame
de ton épée est égale aux rayons du cercle\ldots{} Mais vois donc,
Rustigny, le voici déjà hors de mesure\,!\ldots{} Remets-le en
place\,!\ldots{} Qu'est-ce ceci maintenant, malheureux\,? Tu t'avances
d'un pas de trop\,!\ldots{} Et, surtout, pas le pied sur le Z\,!\,»

Toujours marchant, Louis-Antoine se lamentait\,: «\,Et cela pendant des
heures\,!\ldots{} Moi, si j'étais obligé de me battre, je m'en tirerais
tout aussi bien qu'un autre. Je m'en irais comme cela, en me couvrant de
ma gauche armée d'un bon gant de cuir d'élan, et, quand on me
détacherait un fendant, je me raserais à terre et piquerais dans le
ventre.\,»

Et, tout en indiquant dans l'air des passes avec son épée engainée,
Louis-Antoine se hâtait vers la demeure délabrée de ses pères. Tout à
coup il s'arrêta, se frappa le front et se le meurtrit de la garde de
son épée, tapa du pied, et murmura\,:

«\,Bon Dieu\,!\ldots{} Et les collets\ldots{} J'ai oublié les
collets\,!\ldots{} Mais il est trop tard pour les aller chercher, trop
tard aussi pour prévenir la mère Jeannette\,!\ldots{} Bast\,! Je
donnerai commission à Marin de les relever. Mais trouverai-je
Marin\,?\,»

Un bon moment, Louis-Antoine hésita. Il fut même sur le point de revenir
en arrière. Mais alors il pensa aux reproches muets de sa mère quand il
avait désobéi en quelque manière. Il vit ce regard doux et triste, ces
yeux qui ressemblaient aux siens, mais qui semblaient éteints par les
larmes, cette femme pâle sous ses éternels vêtements de deuil, si
soigneusement entretenus qu'on les eût dits inusables. Il vit cette mère
vivant, telle une recluse, dans sa chambre, entre son prie-Dieu et ses
livres de raison où elle écrivait sans trêve, à moins qu'elle ne priât,
et, à côté du prie-Dieu, tout contre le pied du lit, ce coffre de chêne
où dormait couchée, sur les armes de fer noirci, dans sa bourse de
chamois, l'épée de son père, tué, alors que lui, Louis-Antoine, n'avait
pas tout à fait dix ans.

«\,Pauvre mère\,! Toujours si bonne pour nous tous\,!\ldots{} Mais
pourquoi, lorsqu'elle me regarde, ses yeux ont-ils cette même expression
désolée que je vois chez ma sœur Marguerite quand on lui enlève pour la
boucherie l'agneau enrubanné dont elle fait sa compagnie pour jouer à la
bergère\,?\,»

Et, tout en songeant à ces choses que son intelligence à la fois affinée
et obscure s'entendait mal à débrouiller pour se les représenter
clairement, Louis-Antoine, son épée toujours à la main droite et son
\emph{Saint Augustin} sous le bras gauche, s'en fut prendre sa leçon
d'escrime. Il aurait dû aussi, à la vérité, se préoccuper de ce
\emph{Saint Augustin} qu'il avait complètement négligé d'ouvrir, mais
c'était déjà bien assez de l'escrime, à cette heure, sans se créer
encore des soucis à propos de la leçon du curé.

\hypertarget{chapitre-ii}{%
\chapter{CHAPITRE II}\label{chapitre-ii}}

Charles-Armand-Alexandre-Nonpar de Neuville, marquis de Bannes, seigneur
de Corcey, Chaudey, Sarent et Saint-Valentin, fils de Gaspard-Aymon,
marquis de Bannes, et de Parménie de Donceray, comptait parmi les plus
brillants officiers aux gardes, quand sa liaison avec une femme de petit
état ruina à tout jamais sa fortune d'officier et de courtisan. De très
bonne heure orphelin, il avait dû à la sagesse de son tuteur, un Caumont
de la Force, la conservation de cette fortune que son tempérament,
impétueux jusqu'à la folie, et son caractère assez faible lui eussent
laissé totalement dilapider.

À sa naissance, c'est-à-dire au 15 mars 1590, Charles-Armand se trouvait
héritier d'un des plus grands domaines entre tous ceux du Berry. Par les
soins de son tuteur, ces terres allèrent gagnant en nombre et en
importance, s'étendant de Plaimpied à Chavannes, du nord au sud, et
rejoignant vers l'ouest Saint-Ambroix, Condé, jusqu'à Neuvy-Pailloux.
C'étaient là les biens de son père que l'on avait arrondis. Par sa mère,
il possédait d'autre part plus de trois lieues carrées de pays entre
Vatan, Paudy et Saint-Georges de Breulhamenon. Sans compter les rentes
sur l'Hôtel de Ville, sur Lyon, les pensions et autres avantages, le
jeune homme se trouva, alors qu'il avait tout juste vingt ans, maître de
soixante mille livres de revenu.

Et pourtant il avait dilapidé plus de cent mille livres pour jeter sa
gourme aux pages, et ensuite pour faire honneur à son état de capitaine
aux gardes, sans compter l'argent employé à payer sa compagnie, encore
que, profitant de la gêne d'un sien cousin qui la lui céda en 1609, il
ne l'eût point payée cher.

En tous autres temps le marquis de Bannes se fût élevé aux honneurs que
méritaient ses vertus guerrières. Mais il fut victime, comme tant
d'autres, de la trop longue paix qui succéda aux dernières convulsions
de la Ligue. Né au milieu des guerres civiles, le marquis passa sa
jeunesse dans la triste oisiveté de la paix. Il y eut aussi de sa
faute\,: sans goût pour l'étude, dépourvu d'initiative, il n'eut pas le
courage d'aller chercher à l'étranger cet enseignement militaire que les
gens de guerre sans emploi suivaient sur les marches du Danube avec les
généraux de l'Autriche, ou dans les Flandres sous les ingénieurs de
l'école du prince Maurice.

Sa nature très indolente au fond, malgré les éclats tumultueux d'une
activité qui se dépensait toujours à faux, trouva des satisfactions
suffisantes dans le pays riche et gras où il séjournait la majeure
partie de l'année, sous couleur de disputer avec ses intendants sur
l'administration de ses domaines. Mais, s'il goûta les plaisirs de la
chasse et les joies plus graves des discussions d'intérêts généraux dans
les assemblées de la noblesse et à Issoudun et à Bourges, il goûta
davantage les charmes de Julie Péréal, merveilleuse blonde de dix-neuf
ans, qui abandonna de bon cœur son mari Royer Hippeau, marchand drapier
à Bourges, à l'enseigne de la Toison-d'Or, pour vivre avec le marquis
Charles-Armand.

Ce scandale, qui commença en l'année 1611, était de ceux qui blessent à
la fois les principes moraux et les intérêts matériels de toute société
fortement constituée. On ne s'inquiéta point du consentement tacite que
le drapier Hippeau, homme riche et bien posé par ses bonne vie et mœurs,
parut donner à la fuite de la plus belle des marchandes. On parla de
séduction, voire de rapt, et le corps des bourgeois protesta en masse et
pour la bonne règle contre un privilège exorbitant de la noblesse contre
la bourgeoisie des bonnes villes. Les coupables narguèrent l'opinion,
comme d'usage, et laissèrent les bourgeois conférer. Plus fière d'un
amour qui l'élevait que Charles-Armand ne l'était d'avoir soumis à sa
loi une beauté aussi rare, Julie Péréal, tout en se cachant avec soin,
eut aussi le soin d'écrire à tous les gens de son entourage qu'elle
avait bien agi et qu'elle ne regrettait rien. Cela s'est appelé de tout
temps brûler ses vaisseaux.

Pour faire regagner à cette brebis égarée le bercail, il était de
première nécessité que l'époux outragé --- ainsi le définit-on par
politesse --- portât plainte contre le ravisseur. Cette plainte ne vint
jamais devant les magistrats de Bourges. Le curé de Saint-Bonnet signala
Julie Péréal à son prône. Le lendemain, Royer Hippeau lui annonçait
qu'il ne compterait plus parmi les fabriciens de sa paroisse, qu'il
n'irait plus à l'offrande et qu'il révoquerait tous les dons promis par
lui s'il était encore question de sa femme au prône.

Quant au marquis de Bannes, Royer Hippeau ne voyait aucun inconvénient à
ce qu'on en parlât du haut de la chaire, pourvu, naturellement et
conséquemment, que ce ne fût pas à propos de Julie Péréal.

Le curé de Saint-Bonnet se soumit, car il craignait de voir l'obstiné
drapier, laissant sa place vide entre les formettes des marguilliers,
s'en retourner vers cette religion prétendue réformée qui fut celle de
son défunt père, Isaac Hippeau, échevin de Bourges. Et comme l'on ne
savait pas trop, depuis l'assassinat du feu roi par un maître d'école,
de quel côté tournerait la girouette, le prêtre ferma les yeux sur cette
histoire, ses oreilles aux récriminations des concurrents du drapier, et
sa bouche au blâme.

Quant à Royer Hippeau, content de retenir la dot et les propres de sa
femme, naguère volage et aujourd'hui envolée, il étendit son commerce et
chercha des consolations auprès de Nicole Deleuze, sœur de lait de Julie
et veuve du mercier Augustin Pillonnet. Ce qui prouve que la morale n'a
que peu à voir dans les petites transactions de la vie courante, et que
celles-ci se règlent d'après les plus médiocres intérêts, l'amour de
soi, de ses avantages et de ses plaisirs, et la recherche de la
tranquillité à tout prix.

Un an ne s'était pas écoulé que l'on ne parlait plus à Bourges de la
belle drapière, cependant que cachée dans une maison du marquis, aux
portes mêmes de Vatan, Julie Péréal faisait ses couches, aidée par Alice
Robinet, sage-femme dudit lieu, et la discrète Nicole Deleuze, qui
valait en beauté brune cette incomparable blonde qu'était Julie.
L'enfant qui naquit dans la petite maison des champs en prit le nom de
Pontaillan, qu'il devait garder pendant bien des années. Il fut inscrit
et baptisé dans une pauvre paroisse de village, la
Chapelle-Saint-Laurian, sise sur les terres du marquis de Lépinière, qui
servit de parrain au nouveau-né. La marraine fut Nicole Deleuze.
L'enfant reçut les noms de Florimond-Charles-Claude , et fut vivement
reporté chez sa mère, qui tint à nourrir de son lait ce fruit d'un amour
où son orgueil avait eu plus de part que ses sens. Non moins vivement,
le marquis de Lépinière, ayant levé copie de l'acte de naissance, partit
en poste pour Paris.

Aussitôt arrivé, ce seigneur, qui ne négligeait rien pour s'avancer dans
le monde, annonça à cor et à cri l'aventure de son voisin
Charles-Armand, s'indigna avec qui voulut du déshonneur que la naissance
de Florimond Pontaillan infligeait nommément à la noblesse du Berry,
sans préjudice de toute celle du royaume, et trouva moyen de faire
coucher sur la feuille des bénéfices un sien bâtard que sa sœur,
M\textsuperscript{me} de Naffe, élevait à Blois.

Bref, M. de Lépinière sut si bien arranger ses affaires que le marquis
de Bannes fut mis en demeure de céder sa compagnie aux gardes, et ce fut
Lépinière qui l'obtint.

Un pareil procédé appelait une éclatante vengeance. Charles-Armand
provoqua M. de Lépinière, qui, loin de refuser l'ajournement, accourut
avec sa meilleure épée, sa plus fine dague et ses deux seconds, dont
l'un, M. de Primelles, fendit d'un irrésistible revers la tête du jeune
Langlon cousin du marquis de Bannes, qu'il assistait en cette affaire.
M. de Langlon en mourut quinze jours après. Ce fut la seule victime
d'une rencontre où les autres combattants s'étant désarmés réussirent à
s'accorder, puisqu'ils avaient satisfait aux lois de l'honneur. Et le
marquis de Bannes, rendu à la paix des champs, se consacra à Julie, à
son fils Florimond et à la surveillance de ses biens.

Pour ménager les apparences, il ne fit point maison commune avec sa
maîtresse. Installée au Coudray, Julie ne se trouvait pas à plus de deux
lieues de Lunery. A toucher ce lieu s'élevait le château de Bannes,
vieille maison fortifiée que Charles-Armand mit par terre pour bâtir sur
son emplacement nivelé une habitation vaste et plaisante, dans ce goût
italien que la reine mère avait, depuis quelque quinze ans, mis à la
mode.

Une fois logé à son aise, le marquis prit au sérieux son rôle de
propriétaire foncier. Il vécut parmi ses fermiers, courut les foires au
grand dam de ses intendants, dont il tarissait ainsi la principale
source de revenus, en vendant et achetant lui-même ses moutons et ses
bœufs. Il améliora ses bergeries et devint fameux par la qualité de ses
laines. Les plaisanteries faciles qui coururent alors sur les lainages
du marquis de Bannes et de sa bonne amie la Drapière ne portèrent pas
bonheur à leurs auteurs. Le naturel violent de Charles-Armand le porta,
quand il s'entendit ainsi brocarder, vers les solutions rapides et
extrêmes. Deux frères, MM. de Villedieu et d'Artemaille, dont l'un était
guidon de gendarmes, passèrent successivement par ses mains. Le premier
demeura sur la place avec un excès de cinq boutonnières à son
pourpoint\,; le second, c'est-à-dire le guidon, la gorge largement
ouverte, languit quelques jours avant de mourir. Leur beau-frère, M. de
Saint-Sylvain, qui les voulut venger, n'eut pas un sort meilleur. Mais,
avant de tomber le nez dans l'herbe pour ne plus se relever, il allongea
en désespéré une telle estocade au marquis que celui-ci, atteint au
ventre, demeura cloué trois beaux mois d'été dans son lit.

Julie Péréal soigna son magnifique amant avec un dévouement de sœur
grise, et Charles-Armand, aussi sensible dans ses amours que violent
dans ses colères, en chérit davantage, si possible, cette belle
bourgeoise qui s'était perdue pour lui avec un parfait désintéressement.
Une fois guéri de sa blessure, le marquis s'aperçut, non sans chagrin,
du trouble que son absence forcée avait amené dans ses affaires. Dès
qu'il put monter à cheval, il parcourut ses domaines comme une trombe,
suivi de loin par ses intendants, qui, s'il valait un prévôt, n'étaient
en rien les inférieurs des sergents.

Les métayers qui le tenaient pour moribond, tant on confond trop souvent
ses espoirs avec la réalité, faillirent tomber en chaud mal de terreur
et d'étonnement quand ils virent arriver ces cavaliers avec les
registres en porte-manteau et le fouet dans la botte. Tous les projets
de morcellement que caressaient les censiers se dissipèrent en fumée. Il
fallut représenter les fruits ou payer. On en parla longtemps dans le
Berry. Et tant que vécut Charles-Armand aucun de ses tenanciers n'osa
plus lever les cornes.

On l'aimait, toutefois, dans le pays, à cause de son esprit de justice.
Sa violence, voire sa brutalité qui n'acceptait pas de contrainte,
n'excluait pas une générosité qui se traduisait par des actions d'une
bienfaisance singulière. On le nomma toujours «\,le Bon Seigneur\,»,
parce que jamais homme, femme ni enfant ne souffrirent de la faim sur
ses terres. Par une habileté qui, chez elle, remplaçait la bonté
foncière du marquis, Julie Péréal exerçait les charités, suivant les
volontés de Charles-Armand, y ajoutait même du sien, mais sans laisser
ignorer à quiconque que de ces libéralités elle était la vraie
conseillère. Mais les gens se défiaient d'elle, et, quand ils la
croyaient loin, plus d'un se permettait des gorges chaudes sur «\,la
mignonne à M. le Marquis\,».

Facile aux petits, attentif à soulager leurs misères, toujours prêt à
récompenser la bonne volonté, haïssant le mensonge, qui n'est que trop
souvent le bouclier des faibles, lui réservait ses colères pour les
interposés de tous rangs qui opprimaient le pauvre monde en son nom.
Aussi, lorsqu'en 1620, ayant repris du service, il fut laissé pour mort
dans l'entrepont du \emph{Saint-Louis} de Nantes, étourdi et meurtri par
ce canon qui creva en tuant onze matelots et trois maîtres, le 9
septembre, quand l'escadre royale tirait à parer la pointe de la
Tranche, un deuil public attrista toute une partie du Berry. Et, pour un
peu, Julie Péréal eût été portée en triomphe, quand elle annonça, à la
fin du même mois, que le marquis était, par la grâce de Dieu, sauf dans
son corps et que le roi l'avait fait chevalier de Saint-Michel.
Certaines gens, se laissant emporter par un enthousiasme assez naïf pour
ne pas tenir compte de la voix sans pitié de l'opinion, crièrent même\,:
«\,Vive madame la marquise\,!\,» Mais on les fit taire\,; car en toutes
choses il faut considérer la fin, de même qu'il convient de ménager les
apparences, parce que le scandale est ce qu'on doit par-dessus tout
éviter. Aussi vrai que, dans la plupart des actes de la vie, le silence
est préférable aux paroles.

Le marquis en fit promptement l'expérience. Si ces cris\,: «\,Vive
madame la marquise\,!\,» avaient éveillé chez Julie des désirs qui
depuis longtemps sommeillaient, s'ils avaient flatté de secrets espoirs
qui étaient son désespoir même, parce qu'aucun argument de raison ne lui
permettait d'entrevoir une apparence même de certitude, ces cris avaient
frappé d'autres oreilles, et ces oreilles appartenaient à des envieux.
C'est pourquoi une rumeur, d'abord sourde et vague, se répandit,
s'enfla, et la voix de la calomnie souffla aux quatre coins de la
province que le marquis de Bannes poursuivait en cour de Rome
l'annulation du mariage de la Drapière afin de la prendre pour femme.

Le ridicule du propos, l'intervention du pape en cette affaire, les
impossibilités même dont abondait l'histoire n'arrêtèrent point les
détracteurs du marquis. Il n'y a, au vrai, que l'invraisemblable qui
frappe, parce qu'il se suffit à lui-même pour exciter l'intérêt. Ceux
qui craignaient de le voir reprendre sa place à la cour et à l'armée
n'eurent pas de cesse que le malencontreux amant de Julie ne retombât en
disgrâce. Et, de par le roi, Charles-Armand fut prié de résider dans ses
biens du Berry. Mais alors intervinrent ses parents du côté de La Force
pour le marier avec une jeune veuve dont la fortune était grande et les
terres ainsi placées qu'elles augmenteraient d'un tiers la totalité de
celles du marquis. Les terres de la marquise de Lépinière rejoignaient
en effet du nord au couchant le patrimoine de Bannes et descendaient
même jusqu'à Châteauroux.

Ce fut une affaire adroitement menée et qui réussit malgré les
difficultés qui semblaient inextricables à l'abord. Anne de Cuzance,
veuve du marquis de Lépinière, tué en 1621 aux Sables-d'Olonne, se
laissa persuader par son entourage que c'était œuvre pie de ramener au
bien ce Charles-Armand dont la vie scandaleuse attristait les bonnes
âmes. Depuis des années, elle connaissait celui qui avait vu, l'épée à
la main, son mari, après l'histoire de la compagnie aux gardes. La jeune
femme se résolut à entreprendre la conversion du mécréant, et
Charles-Armand prit la main de la marquise avec les meilleures
intentions dont un homme d'honneur puisse paver les chemins de l'enfer.

La position de Julie et de son fils fut réglée par un accord tacite. Ils
disparurent tous deux, et seuls quelques intimes amis surent que
Florimond Pontaillan étudiait au collège de Clermont, à Paris, où
demeurait sa mère, cachée aux yeux du monde dans une modeste maison de
la rue Saint-Jacques. Jamais d'ailleurs la marquise Anne ne demanda
compte à son second mari des écarts de sa jeunesse, non plus que de sa
liaison avec la femme du drapier.

Ses préoccupations étaient plus hautes\,: mère d'une fille âgée de trois
ans quand le coup de canon des Sables l'obligea de cacher ses boucles
blondes sous le sévère bandeau des veuves, Anne de Cuzance pensait
surtout à assurer un protecteur à l'enfant. Ce qu'elle savait du marquis
de Bannes, de son courage, de son entente des affaires et de sa probité,
l'avait décidée, plus que toute autre considération, à convoler en
secondes noces.

L'amour que témoigna le marquis à cette fille qui n'était point de son
sang prouva à Anne de Cuzance qu'elle n'avait pas compté en vain sur
cette nature généreuse et tendre dont les sympathies n'avaient pas
jusque-là rencontré leur objet. Si cette petite Catherine de Lépinière
eût été la propre fille de Charles-Armand, il ne l'aurait pas entourée
de plus d'affection ni de soins. Et cette affection se changea en un
sentiment encore plus grave, sentiment en quelque sorte de religieux
respect, après la mort d'Anne de Cuzance, qui s'éteignit d'une maladie
de langueur, après quatre ans de mariage, en l'an 1626, alors qu'elle
n'avait pas vingt-neuf ans.

Cette femme, dont la valeur morale fut peut-être égalée, sans avoir été
certainement dépassée, avait à ce point subjugué le marquis
Charles-Armand qu'il ne la trahit jamais au sens vrai du mot, car il ne
manqua point aux engagements qu'elle lui fit prendre. Indulgente aux
vices et aux défauts des hommes, par cette charité que donne seule la
religion bien comprise, trop fine aussi pour ne pas comprendre de
combien pèsent peu sur les résolutions instinctives les prières, les
objurgations et les reproches, Anne de Cuzance ferma les yeux sur les
écarts du marquis. De Julie Péréal elle s'obstina à tout ignorer. Pour
elle, l'ancienne concubine de Charles-Armand demeurait à Paris, perdue
dans la foule\,; et, si l'on venait lui apprendre avec des airs
joyeusement contristés que «\,la coquine\,» promenait son museau rose à
Vatan, Anne de Cuzance parlait d'autre chose et prouvait par ses façons
distraites que ces propos ne l'intéressaient pas.

À son lit de mort, cette femme dont on a dit qu'elle fut une sainte sur
la terre, et que François de Sales eût rangée parmi ses filles
d'élection, prit la petite main de Catherine, la mit dans la main du
marquis, réunit ces deux mains entre les siennes et dit de cette voix
claire que savent prendre quelques mourants sûrs d'eux-mêmes pour
énoncer leurs volontés dernières\,:

--- Charles-Armand, je vous confie cette enfant. Jurez-moi que tant que
vous serez de ce monde il ne lui arrivera ni peine ni dommage que vous
lui puissiez éviter. Vous la laisserez maîtresse de ses volontés, car
moi, sa mère, je la connais pure de cœur, d'entendement droit pour son
petit âge, et elle ne saurait mal faire.

Et le marquis de Bannes avait répondu simplement\,:

--- Je vous le jure, madame. Puissé-je me voir déchu de noblesse et
perdu d'honneur si je manque à ce serment\,! Et que Dieu me pardonne, en
faveur de vos mérites, les chagrins que j'ai pu vous causer.

Que la marquise lui eût toujours pardonné, cela ne faisait pas doute.
Elle répéta pourtant à Charles-Armand que le seul reproche qu'elle pût
lui adresser était d'avoir été trop indulgent aux imperfections dont
elle avait abondé sur cette terre. Et Anne de Cuzance, marquise de
Bannes, s'éteignit dans la paix du Seigneur le 30 septembre 1626. On
l'enterra en grande pompe à Bourges, dans la chapelle des Mendiantes de
Sainte-Claire, qui, ainsi que personne ne l'ignore, sont nonnes de
l'ordre des Minimes\,: et Charles-Armand demeura seul dans son château
de Bannes avec sa belle-fille Catherine, qui avait un peu moins de huit
ans.

La solitude, pour certains, est une mauvaise conseillère. Charles-Armand
se fatigua bientôt de vivre sans une femme à ses côtés qui le querellât
sans excès ou l'apaisât, au contraire, par sa soumission calculée. Le
marquis éprouvait le besoin d'occuper l'amour, l'amitié ou simplement
l'attention. C'était un homme sociable et par conséquent incapable de se
suffire à lui-même.

Quand Anne de Cuzance ne fut plus là pour meubler sa vie, il songea
fatalement à revoir de façon régulière Julie Péréal, qui passait son
temps à voyager entre Paris et Vatan, au gré de ses ordres. Julie,
d'ailleurs, depuis la mort de la marquise, avait beaucoup réfléchi de
son côté, et elle s'était résolue à revoir le marquis, mais à le revoir
de manière à ne plus le perdre de vue.

Nulle position n'était en effet plus précaire que celle de la belle
Drapière. Absente de son mari, qui lui retenait son bien, craignant sans
cesse que ce trop accommodant Royer Hippeau n'usât de son droit pour la
séparer de son enfant, elle dépendait en tout du marquis, dont la
libéralité et la protection étaient ses seuls garants en ce monde.
Depuis sa fuite, toutes les portes lui étaient fermées. Ni parents ni
amis\,; seulement des galants, dont l'audace et la brutalité étayaient
les entreprises amoureuses. L'épouse décriée de Royer Hippeau n'osait
point sortir de chez elle, car les plaisanteries licencieuses et les
compliments orduriers l'accompagnaient partout où elle montrait son joli
visage. À grand'peine, l'écuyer Ottavio Piccolomini, le quinola André
d'Archelet et le majordome Bérenger de la Butière, trois hommes de main
à la solde du marquis, réussissaient-ils à la protéger contre
l'insolence d'un chacun.

Insultée en pleine rue de Vatan par un soldat qui osa porter la main sur
elle pour lui arracher son masque, Julie tira une vengeance terrible du
drôle. Les donneurs d'étrivières dont elle vivait entourée surprirent,
le lendemain, ce soldat, qui s'appelait Franc-Cœur, dans la rue du Rouet
d'Or, au moment où il sortait, un peu aviné, de la taverne du
Petit-Lion. Avant qu'il pût se servir de son épée, Franc-Cœur fut happé,
bourré, entraîné dans un cul-de-sac. Là ses agresseurs lui coupèrent
proprement le nez avec un couteau bien affilé, et au ras de la face.
Puis ils le laissèrent évanoui dans son sang, après l'avoir dépouillé de
ses meilleurs vêtements, de son argent et de ses armes.

Comme les gens qui avaient fait ce coup étaient masqués, comme le vol
apparaissait être le vrai motif du crime, comme on ne put mettre la main
sur les coupables, Franc-Cœur ne put obtenir justice. Il eut beau crier,
intéresser à sa cause son capitaine et quelques notables, ses
accusations vagues se perdirent faute de preuves. De part et d'autre lui
arrivèrent de mystérieux avis, billets sans signature, non point écrits
de main d'homme, mais composés avec des mots ou des lettres découpés
dans les pages de divers livres. On lui promettait une mort lente dans
les tourments s'il s'obstinait à poursuivre les Enfants de la Matte.
France-Cœur ayant eu la simplicité de montrer ces lettres monitoires,
les gens de bon sens se persuadèrent que les dangereux filous qui
avaient maltraité et volé le soldat appartenaient à la redoutable
association dont les chefs se réunissaient de nuit, à Paris, sur
certains bateaux de la Seine. Chacun trembla autant pour sa bourse que
pour sa peau, et Franc-Cœur, après avoir épouvanté les bourgeoises de
Vatan par son abominable face camarde, dut s'enfuir pour ne pas se voir
jeter en prison avec les malfaiteurs et sans doute diriger quelque jour
sur le Havre-de-Grâce ou sur quelque autre port pour s'en aller
coloniser l'Amérique.

Dès lors, Julie, trop bien vengée, dut ne plus se montrer à Vatan. Il
lui fallut séjourner les trois quarts du temps à Paris, réduite à la
société des gentilshommes de service, soudoyés par le marquis, de sa
sœur de lait Nicole Deleuze, chassée par Royer Hippeau, qui avait trouvé
plusieurs fois mauvais goût à son vin, et de sa fille de chambre Maroie
Lenatier, péronnelle tournée à l'image d'une petite déesse, dont la vie
inoccupée de Julie se consumait à surveiller jalousement la conduite.

Mais l'ennui ne triompha pas de la nature froide et réfléchie de Julie
Péréal. Elle se garda de toute faute, et sa conduite fut si nette
qu'aucun calomniateur ne réussit à en détacher le marquis. Et pourtant
l'écuyer meneur de Julie, ou, si l'on préfère, son quinola André
d'Archelet, qui ne l'aimait guère, était chargé de surveiller la belle
dans toutes ses démarches. Mais cet espion domestique, trop attaché à
son maître pour discuter, voire juger ses volontés et ses actes, était
un honnête homme, silencieux et morne, peu gracieux, incapable toutefois
de s'abaisser jusqu'au mensonge. Dans ses rapports, il s'en tenait à la
vérité. Et le marquis savait qu'André d'Archelet était ainsi fait, et
Julie aussi, de telle sorte que la mission du quinola était un de ces
secrets cousus de fil blanc et qui ne trompent personne.

Julie, plus habile, avait aussi sa police. De celle-là personne ne se
doutait. Le procureur de Bourges, Marcelin Duvau, étendait une
surveillance de toute heure sur les ennemis de la Drapière, sur les amis
du marquis, et spécialement sur Royer Hippeau, depuis que la toute
charmante Nicole Deleuze n'était plus là pour le pêcher dans ses filets.
Mais Julie Péréal avait beau brûler de beaux cierges à Saint-Séverin, à
Saint-Julien et autres églises voisines, fatiguer le ciel de ses vœux,
la mauvaise fièvre n'arrivait pas qu'elle souhaitait quotidiennement et
charitablement au drapier.

Bien au contraire, la santé de Royer Hippeau s'affirmait de plus en plus
florissante. Les affaires de son négoce ne l'étaient pas moins. Chose
admirable\,! Ce drapier philosophe et entendu comptait parmi les
meilleurs clients du marquis. S'il lui achetait ses laines, réputées
entre toutes celles du Berry, c'était, au vrai, par entremise
d'intendants. Mais, enfin, il les achetait et les payait avec une
régularité louable, aux quatre termes de l'année. Cette régularité gagna
encore, si possible, après que le drapier eut fini de mettre la main sur
l'avoir de son épouse absente.

Jeune encore, puisqu'il n'avait pas quarante ans, et de forte
constitution, amateur de grands vins, de longs repas et de tous plaisirs
honnêtes, Royer Hippeau mourut cependant, et de façon subite, une année
après les obsèques de la marquise Anne. Le 15 septembre 1627, on le
trouva raide et pâle, tel un saint de pierre, en travers de sa porte, à
Molingeton. C'est là qu'à un quart de lieue de Bourges il avait une
petite maison des champs, sise entre le pont dudit lieu, la
Chapelle-Saint-Roch et le cimetière des pestiférés. Le voisinage de cet
enclos passa de tout temps pour avoir porté malheur.

Donc Royer Hippeau fut relevé, au petit jour, par des paysans qui
remarquèrent tout d'abord des trous à la pommette gauche et au front. Le
sang qui ne coulait plus avait masqué le visage d'une croûte noirâtre et
épaisse, mais qui n'empêchait pas de voir que ces trous étaient les
traces de balles, et de balles de calibre. Et les coups avaient été
tirés de près, car du crâne, éclaté au-dessus des sourcils, avait jailli
une partie de la cervelle. Des pistes de cavaliers, pour ainsi dire
fixées dans la boue du chemin par la gelée blanche, se croisaient en
face de la maison, comme si les gens à cheval, une fois la chose faite,
s'en fussent retournés à Bourges d'où ils étaient partis. Dans la
maison, pas un être vivant. Du reste, on ne retrouva jamais la servante,
une certaine Perrine Harricand, native de Charroux, que le défunt avait
à son service depuis trois mois.

La justice de Bourges informa, naturellement sans succès. On ne put
établir le vol, car personne ne savait, en dehors du défunt et de
Perrine, ce que contenait la maison avant le meurtre. On apprit
pourtant, mais par pur hasard, qu'un des meurtriers avait un faux nez,
car un superbe faux nez, en cuivre artistement peint au naturel, encore
muni des liens qui permettaient de le fixer à la tête du propriétaire,
se trouva le jour même de l'enquête à quelques pas de la maison de
Molingeton.

Aussitôt l'on se rappela à Vatan l'histoire de Franc-Cœur, le soldat au
nez coupé, à qui, par charité, une dame avait donné un faux nez\,:
«\,C'est Franc-Cœur qui a fait le coup\,!\,» Cette rumeur courut à
Bourges, s'y grossit. Dès lors ce devint une certitude. Et à la
promenade, en plein hiver, les amateurs de nouvelles s'attendaient pour
crier d'une même voix\,: «\,Nous l'avions bien dit, toujours dit, nous
l'avons dit, le disons et le dirons\,: c'est Frane-Cœur le meurtrier du
drapier\,!\,» À quoi les derniers arrivés répondaient\,: «\,Ce fut,
c'est et ce sera bien dit\,: Franc-Cœur est le meurtrier du drapier\,!
D'ailleurs ce soldat était un audacieux scélérat. Si on lui coupa le
nez, ce ne fut pas sans motifs. Entre soi les voleurs exercent une
police plus exacte que celle dont se croient protégés les honnêtes gens.
Franc-Cœur avait trompé ses associés de quelque manière\,; ils se
vengèrent en l'amputant de son nez\,!\ldots{} Gare à nous, messieurs, la
bande à Franc-Cœur est lâchée sur le Berry\,!\,»

Et chacun de se musser aussitôt le jour tombé. Pour un peu, on eût tendu
les chaînes. Et l'on ne voulut rien savoir de la déposition d'Aymon
Coquillaud, aubergiste. Seuls, trois conseillers en prétendirent tenir
compte. Le témoin fut écarté. Pourtant Aymon Coquillaud affirmait que,
le 15 septembre, à l'heure du dîner, étaient descendus chez lui, au
Serpent Saint-Georges, deux cavaliers et une femme que l'un d'eux
portait en croupe.

--- Fort bien, fit la voix publique, ce cavalier ne pouvait être autre
que Franc-Cœur, l'homme sans nez, ancien soldat. En tout fidèle aux
habitudes de rapine en honneur chez ses pareils, il a enlevé la servante
après avoir tué le maître\,!

--- Permettez, répondit l'hôtelier, j'ai connu ce Franc-Cœur, je
l'aurais bien reconnu, ce semble, à l'absence de son nez\,!

--- Il l'avait encore en ce moment, reprit la voix publique. Il l'a
perdu dans le désordre de l'action\,!

Et les propos allaient leur train sans que M. Aymon Coquillaud, hôtelier
du Serpent Saint-Georges, logeant à pied et à cheval, rue d'Aurron, en
la paroisse de Saint-Pierre-le-Guillard, daignât y opposer des
contradictions pour déplaire. Car il ne se souciait pas de mécontenter
la clientèle de choix qui se donnait rendez-vous chez lui pour traiter
de l'assassinat du drapier dans ses moindres particularités. Une légende
s'établit contre quoi, ainsi que d'usage, ne prévalut pas la raison.

Bien qu'au rapport de M. Déodat Paulmier de Poulemart, conseiller à la
Cour, l'on sût que le drapier Royer Hippeau était tombé sous la carabine
d'un ancien soldat aux gardes, assisté d'un inconnu, que ces deux hommes
avaient, en compagnie d'une femme ayant la tournure et les habits de
Perrine Harricand, fille au service dudit drapier décédé, passé la
soirée même du crime à l'auberge du Serpent Saint-Georges, la voix
publique continua d'accuser Franc-Cœur, le mousquetaire au faux nez.

De l'inconnu et du soldat aux gardes, ce dernier reconnu pour tel au
harnachement de son cheval, à la carabine pendant de sa selle à la
royale, et à divers autres signes, on ignora toujours et les noms et le
destin. Comme Perrine de Charroux, ils s'étaient dissous dans la nuit
glacée de septembre où Royer Hippeau rendit à Dieu son âme de drapier.

Julie Péréal, à qui personne ne songea en cette affaire, ne perdit point
son temps à demander par requête qu'il fût fait justice des meurtriers
de son mari, meurtriers sur qui, elle, Nicole Deleuze et M. Bérenger de
la Butière, auraient pu, peut-être, fournir plus d'un renseignement
utile. Le président Vincent La Fare de Monistrolles, qui émit ce désir
qu'on l'appelât à Bourges pour supplément d'enquête, ne fut même pas
écouté. Et même le président Claudon Harant dénonça la grave
inconvenance que c'était de troubler ainsi dans les premières douleurs
de son veuvage cette personne sur qui il eût été cruel de rappeler
l'attention.

Par les soins diligents du procureur Marcelin Duvau, Julie obtint des
jugements exécutoires, déjoua les appels et fut, en moins d'une année,
envoyée en possession non seulement de ses propres, mais encore du
principal de son conjoint, car Hippeau était mort intestat, sans
postérité, et certaines dispositions de son contrat de mariage, pour peu
qu'on les sût interpréter, établissaient que, par ses apports dotaux et
ses acquêts. Julie se trouvait associée à son commerce. Et le fonds de
la Toison d'Or fut vendu au mieux des intérêts de Julie, qui abandonna
la draperie.

Ainsi Julie Péréal, veuve Hippeau, reprit-elle, d'un même coup, sa
liberté, sa fortune et son état dans le monde. L'année n'était pas
écoulée qu'elle épousait, secrètement il est vrai, le marquis de Bannes.
Les premières nouvelles que celui-ci reçut de la Cour étaient de nature
à ne rien lui laisser ignorer des suites de sa mésalliance. On lui
mandait de rendre le collier de l'Ordre et de garder résidence, jusqu'à
nouvel ordre, dans ses terres du Berry. Le marquis répondit à ces
dispositions mal gracieuses par une déclaration de guerre. Il s'en fut
se marier, au grand jour et à la barbe de ses détracteurs, dans une
petite paroisse de Dun-le-Roy, et prit soin de faire passer son bâtard
Florimond sous le poêle, prouvant par là qu'il le reconnaissait pour
légitime.

Ainsi furent mariés, à Saint-Hostrille, le 20 décembre 1627, par M.
Étienne Rousselin, ordinaire de ladite paroisse, Charles-Armand, marquis
de Bannes et Julie Péréal. Une même bénédiction courba les têtes du
père, de la mère et de l'enfant. Puis le marquis retourna avec la
nouvelle marquise au château de Bannes, et Florimond fut reconduit à
Paris pour continuer d'y suivre, sous la direction de son précepteur
Aimeri d Olivier, les cours du collège de Clermont, et, sous la haute
main de son créat Ottavio Piccolomini, les exercices de l'Académie. M.
Bérenger de la Butière lui demeura attaché pour surveiller ses leçons
d'escrime. Enfin Nicole Deleuze fut chargée de tenir sa maison. Si,
ainsi entouré, Florimond ne devenait pas la fleur de la noblesse du
royaume, c'est qu'il faudrait nier tous les avantages d'une bonne
éducation.

Chose singulière, de cette incartade publique, de ce défi lancé à la
face de la noblesse de France, de cette déclaration contre les
institutions fondamentales et coutumières du mariage qui établissaient
le concubinage comme empêchement préalable à toute union régulière, les
suites ne furent point celles qu'on en attendait. Le marquis de Bannes
et sa marquise de boutique ne furent même pas inquiétés.

Il est certain que le cardinal Richelieu, poursuivant de concert avec
Louis XIII son entreprise contre la noblesse, dont ce prêtre ne fut pas
moins cruel ennemi que ce second Bourbon, trouvait là une occasion
merveilleuse de diminuer devant l'opinion cette caste à laquelle il
appartenait tout aussi bien que le roi et le marquis Charles-Armand. Il
se donna, jusqu'au bout, et avec son impassible sérénité, la
satisfaction quasi providentielle de vexer, sans risques, le marquis. En
ne le faisant pas poursuivre, en n'attaquant pas son mariage insensé, il
paralysait ses instincts d'indiscipline et son ardeur pour le combat.
Mettant ainsi le plus puissant propriétaire du Berry hors de la
communion de ses pairs, il le détruisait dans le présent et l'avenir et
le couvrait en même temps de cette protection insultante que l'on
accorde aux lépreux à condition qu'ils vivent à l'écart. Bien mieux,
ayant fait pressentir le marquis, le cardinal ministre fit donner des
lettres de légitimation, sous sceau royal, à Florimond Pontaillan. Les
lettres royales furent enregistrées sans remontrances à Bourges, et
Florimond devint, du jour au lendemain, héritier régulier du marquis son
père, qui le gratifia de la baronnie de Chézal-Benoît, détachée de ses
terres, en remettant à plus tard le soin de le titrer plus hautement.

Le marquis de Bannes se consacra dès lors à la gestion de ses biens.
L'âpre amour de la terre, qu'il avait hérité certainement de son père
défunt, se développait en lui de telle sorte que, s'il en vivait tout le
jour, il en rêvait aussi toute la nuit. Et, dans ses rêves, il voyait
sans répit la vilaine tache noire par laquelle Pierre-Coquille,
arpenteur juré de Vierzon, marqua, sur le plan des terres de Bannes,
celles du baron de Primelles.

Encore que peu considérable, le domaine de Primelles gênait affreusement
le marquis\,: car ce domaine allongeait, au beau milieu du sien, son
corps difforme, étranglé par places, ventru en d'autres, et ses
prolongements ramifiés rappelant les membres tors et hideux de quelque
monstre vomi des profondeurs de la mer. Du château et de la paroisse de
Primelles, qui en occupaient le centre, cet ensemble de vignes, de blés,
de pâturages, de boqueteaux et de friches, mal arrosé, mal conditionné,
de petite valeur, s'allongeait ridiculement vers l'ouest pour mourir à
Saint-Ambroix en une pointe qui n'avait pas cinquante toises de large.
Un couloir ou, pour mieux dire, un cul-de-sac encore plus mince
descendait au midi, longeant Mareuil pour s'arrêter au coteau de
Bellevue. Entre ces deux cornes du monstre, le contour se creusait,
autour des bois de Ballay, laissant la moitié de Villiers au marquis du
côté du couchant, tandis que le soleil se levait sur l'autre moitié, qui
était au baron de Primelles.

De même au sud, de même à l'est, de même aux bords du Cher, en tous
points, en tous lieux, la terre de Primelles poussait ses éperons dans
celle de Bannes. Et si, au nord, le marquis entrait vigoureusement
jusqu'au cœur même de l'ennemi par le parc de son château, que ne
séparaient pas trois cents pas de celui du baron, la pareille lui était
rendue, bien au-dessus de Lunery, par le baron, à qui une bande de terre
servait de pont pour relier à son domaine celui de Chanteloup des
Tremblaies, méchant lopin accidenté et fourni de tous gibiers de plume
et de poil.

A force de rêver de cette terre étrangère, étoilée ainsi qu'un échaudé
et posée ainsi qu'un cancre hideux sur son bien, le marquis gagna une
mélancolie maligne. Car l'on ne doit pas oublier que, si Charles-Armand
était amoureux de la terre autant qu'un avare l'est de l'or, le baron de
Primelles ne chérissait pas la terre d'un moindre amour.

Très gueux, il s'obstinait à repousser toutes les offres du marquis, qui
pouvait les grossir sans avoir cette crainte de les voir jamais
accepter. Pendant des années, ce fut une lutte sourde, acharnée,
sournoise, où le riche ne gagna rien sur le pauvre, qu'il s'essayait à
ruiner en procès. Malgré les hardies démarches de la marquise Julie,
qui, par une conduite pleine d'habileté, d'humilité feinte et de
coquetterie savante, mais aussi de réserve et de fierté, et, par-dessus
tout, de souplesse, réussit à se créer un parti tant à Issoudun qu'à
Bourges, les magistrats firent la sourde oreille. Tous ces robins
avaient une trop grande expérience des grimaces pour ne pas comprendre
que, si la marquise s'entendait à donner naissance à tous les espoirs,
sa pratique du monde lui fournissait mille moyens de les promener
jusqu'à ce qu'ils disparussent par fatigue, par vanité ou par ennui. Sa
fortune ne lui fut pas une meilleure arme. Comme aux temps du vieil
Évandre, l'équité prévalut. Le marquis de Bannes perdait tous ses
procès, à Issoudun, à Vatan comme à Bourges, et ses appels, trop
nombreux pour ne pas attirer l'attention, se virent rejetés en quelque
sorte par provision, tant les juges se fatiguaient de voir toujours les
procureurs de Bannes, avec leur Marcelin Duvau en tête, assigner le
baron de Primelles, qui ne demandait jamais rien que son dû.

Ainsi le marquis de Bannes ne put rien obtenir. Il avait essayé
d'acquérir à prix d'or les mauvaises terres de Primelles et le château
lui-même, qui n'était, au vrai, qu'une triste bicoque branlante, humide,
lézardée. A demi ruinée par les bandes de Briquemaut au siècle dernier,
on ne l'avait jamais relevée, par ce défaut d'argent qui était
héréditaire dans la famille. Quand on lui proposa d'aliéner le nid à
hiboux et à rats de ses pères, le baron répondit par un refus à
décourager toute récidive. Le marquis, en personne, visita son voisin.
Il fut bien reçu, mais pareillement éconduit. Les propositions d'échange
n'eurent pas un meilleur sort que les offres d'argent, nommément pour
cette terre des Tremblaies qui coupait au marquis ses communications
entre son château et ses fourneaux à fer de Piédechef. Les procès
n'avaient rien donné. Impuissant, le marquis se rongeait les ongles
jusqu'au ras des doigts, et le baron de Primelles, toujours gueux comme
un rat d'église, le narguait de sa méchante gentilhommière où il vivait
de pain dur, de fromage mou et de noix moisies pendant les trois quarts
de l'année, avec sa petite famille, pour se rattraper à la belle saison
sur les légumes et les fruits. À quoi servaient donc et l'argent et la
grande position dans le monde\,?

En vérité, ils ne servaient de rien au marquis, dont la position morale,
d'ailleurs, était nulle. Toutes les sympathies allaient aux Primelles,
gens d'honneur, supportant, avec cette dignité, en quoi consiste
peut-être la vraie noblesse, la pauvreté qui effraie et diminue ceux-là
seuls qui ont l'âme basse. Avec tout son argent, la marquise de Bannes
ne réussissait pas toujours à se faire saluer dans les rues de
Bourges\,: on ne lui donnait point le pas à l'église. En somme, elle
restait Julie la Drapière pour tous ceux qui se découvraient et
s'inclinaient jusqu'à terre devant M\textsuperscript{me} de Primelles
lorsque, dans ses modestes habits, assise sur une planchette, en croupe
d'un écuyer rustique, elle s'en venait à la ville pour visiter ses amis.
Et elle en avait à revendre.

Contre ces mépris, l'épée du marquis ne valut elle-même rien. A force de
briller au soleil, on eût dit qu'elle avait perdu son poli. Les
provocations de Charles-Armand, si elles gagnaient en fréquence,
perdaient en autorité. En l'année 1627, il se battit cinq fois en duel,
avec des fortunes diverses, car, à mesure qu'il se discréditait, se
dressaient devant lui des adversaires de moindre maison, mais de sorte
certainement plus redoutable. Il tua deux hommes, en blessa un, mais fut
lui-même atteint trois fois. Après chaque rencontre, des épîtres
officieuses parties de l'entourage du cardinal ministre lui rappelaient
que Sa Majesté entendait que l'on respectât ses édits, et que ces édits
étaient pour les grands comme pour les petits. Les lettres d'abolition
qu'on lui avait toujours accordées jusqu'ici contre espèces sonnantes
lui furent désormais refusées.

Son dernier duel contre M. de Cresancy, personnage décrié, et où il
n'eut pas l'avantage, lui valut plus d'ennuis qu'il n'en avait jusque-là
éprouvé. Les avertissements vinrent de toutes parts\,: «\,S'il allait à
Paris, il tomberait sous la prise de corps. Tout ce que l'on pouvait
faire en sa faveur, c'était d'ignorer son existence. Mais qu'il n'y
revint plus\ldots\,»

Le caractère du marquis, devant ce qu'il s'obstinait à appeler une
persécution, se fit de plus en plus renfermé et sauvage. Dans son
intérieur la vie lui devint à charge, car la marquise Julie, maintenant
assurée dans son état, ne gardait plus grands ménagements. Par
l'insolence, elle confinait à la grande dame, mais on retrouvait vite la
bourgeoise parce qu'elle tournait à l'aigre. Ne voulant pas ouvrir ses
yeux sur sa position très fausse, oubliant de parti pris que le marquis
s'était perdu aussi bien en s'abaissant jusqu'à elle qu'en l'élevant
jusqu'à lui, elle le rendait responsable de son injurieux isolement. Et
elle était à ce point vaine qu'il lui échappait de dire que «\,c'était à
n'y rien comprendre\,».

Tout l'amour de la marquise allait d'ailleurs à son fils Florimond, dont
les désordres précoces montraient qu'il avait de qui tenir. C'était le
portrait de sa mère. On parlait déjà dans les ruelles de sa charmante
figure, mais M. Aimeri d'Olivier prétendait attendre qu'il eût ses seize
ans révolus pour l'y conduire. Les nouvelles que recevait de Paris cette
mère pleine d'une affection jalouse, impétueuse et sauvage, suffisaient
à occuper son temps. Chaque jour arrivait une lettre, quelquefois deux
ou trois. Nicole Deleuze, Piccolomini, M. Aimeri et M. de La Butière
étaient de grands écrivains. La poste ne marchant ni assez vite ni assez
souvent à leur gré, ils trouvaient d'autres courriers\,; la marquise en
expédiait de son côté. Qu'on l'en eût crue, et tous les bidets du Berry
eussent amblé sur les routes pour le service de Florimond de Neuville,
baron de Chézal-Benoît. Quant à lui, il bâillait pendant les cours du
collège, où M. Aimeri prenait des notes à sa place, puis il allait à
l'Académie, montait à cheval avec plaisir, mais goûtait plus de plaisir
encore au noble jeu des armes, s'attachant curieusement aux coups peu
usités et qui vous mettent un homme par terre avant même qu'il ait pris
sa garde. Il aimait aussi la chasse à l'oiseau et se réjouissait à voir
les proies se débattre entre les mains des faucons ou les oiselets
ouvrir leurs ailes palpitantes sous le bec d'acier qui leur fouillait le
crâne. Quand il pleuvait, Nicole Deleuze lui tenait compagnie dans sa
chambre, et M. Aimeri d'Olivier leur lisait quelques pages de
\emph{l'Astrée}, puis il leur en expliquait les beautés cachées.
Florimond s'endormait, et M\textsuperscript{lle} Nicole souriait
agréablement au poète, qui caressait sa barbiche en roulant de gros yeux
avec une expression lourde, alanguie et frivole.

Le marquis de Bannes, séparé de ce fils qu'il ne connaissait pour ainsi
dire point, reporta toute son affection sur sa belle-fille Catherine. Il
en dirigeait l'éducation avec une attention jalouse, ne laissant à
personne le soin de la gouverner. La marquise ne se souciait en rien de
cet enfant du dehors, dont un mariage raisonnable la débarrasserait dans
peu d'années. Les femmes de service, obéissant au doigt et à l'œil,
n'agissaient donc que d'après les ordres du marquis. D'un couvent de
Bourges l'on obtint qu'une dame veuve y retirée vint trois fois la
semaine donner des leçons à Catherine, en attendant qu'on la prit comme
élève aux Augustines, où elle apprendrait, avec des filles de son âge et
de sa condition, la couture, la broderie et tous les arts utiles à la
conduite d'une maison.

Si Catherine, dont les aimables qualités et la vigueur d'esprit
dépassaient de beaucoup le niveau commun des enfants de son âge, eût
possédé quelques années de plus, son influence aurait été suffisante
pour arrêter le marquis de Bannes sur la pente escarpée où le
précipitaient sa violence et son obstination. Mais Charles-Armand se
dirigeait vers la quarantaine, et Catherine de Lépinière avait bien
juste neuf ans.

La mollesse et l'indécision du marquis, sous l'influence de l'âge et
aussi des contrariétés, laissaient la place à l'obstination et à la
violence, ou plutôt s'en aidaient, pour se dissimuler sous des
apparences honnêtes. Il y avait autre chose\,: son âme, comme appelée
par la terre, à quoi il pensait sans cesse, se modelait sur celle des
paysans et devenait âpre et vulgaire. À vivre courbé sur la glèbe on
n'élargit point son horizon\,; et ce qui est la qualité foncière des
petites gens se mue en vice dégradant chez qui fut créé pour des
destinées plus hautes.

Les belles qualités du marquis, à qui répugnaient les œuvres du pur
esprit, et qui, autant par la faute d'une aveugle rancune que par
l'absence d'heureuses occasions, ne pouvaient se développer dans les
durs travaux de la guerre, allèrent se perdant. Son esprit se fit de
plus en plus court. A voir son front obstiné, son œil inquiet et voilé,
on eût dit d'un taureau en quête d'un coup de corne à placer. Enfin, le
marquis de Bannes en vint à ne plus voir que ses intérêts et son honneur
sur la terre. Toute affaire où on ne lui cédait pas ce qu'il souhaitait
lui apparut comme spécialement dirigée contre lui. Une méchante histoire
de garenne à lapins devint ainsi la cause de sa ruine.

Un de ses fidèles amis, M. Gaspard de Montenay, de quelque quinze ans
son aîné, et propriétaire aux environs de Saint-Florent, ayant appris
que le baron de Primelles se trouvait cruellement gêné, crut bien agir
en lui proposant d'acheter, à conditions libérales, deux petits morceaux
de sa terre. Grand chasseur devant l'Éternel, M. de Montenay avait jeté
son dévolu sur la fameuse pièce des Tremblaies, dont la nature
rocailleuse et accidentée faisait une remise de gibier sans pareille, et
aussi sur une autre pièce, pas plus grande qu'un mouchoir, la garenne de
Tonlieu, méchant coteau abritant un peuple de lapins dont la chair
sentait le thym, le romarin et la sauge, sans compter les perdrix qui y
vivaient par centaines.

Désireux avant toutes choses d'obliger son ami le baron de Primelles, M.
de Montenay prit ses dispositions avec tant de délicatesse que cette
aliénation demeurait pour ainsi dire fictive. Alors que de beaux et bons
écus tombaient dans l'escarcelle du baron, celui-ci ne perdait aucun de
ses droits dans la pratique\,:

«\,Tant que vous et votre fils vivrez, avait dit M. de Montenay, ces
terres continueront d'être vôtres, et vous y aurez droit de chasse, tout
comme moi.\,»

Quand il connut cet arrangement entre honnêtes gens désireux de se
plaire, Charles-Armand fut pris d'une colère dont les éclats
épouvantèrent tout le château pendant deux jours. Le calme ne revint
qu'après qu'on eut couché, saigné et purgé l'irascible marquis. Mais ce
calme était celui qui précède la tempête, car, un mois après la cession
des Tremblaies et de Tonlieu à M. de Montenay, le baron de Primelles
était tué par le marquis de Bannes, à la Fête-Dieu d'Issoudun, en pleine
procession, devant le second reposoir à la porte du lormier Claudon
Tortorel. Le drap blanc de cet autel extérieur en demeura taché de sang.

Les rares partisans du marquis prétendirent que M. de Primelles avait
voulu prendre le pas sur lui en sortant de la grande paroisse Saint-Cyr
et l'avait même bousculé, répondant par des propos malsonnants à ses
observations raisonnables. On ne les crut pas. Au vu et au su de tous,
les choses n'étaient pas allées ainsi.

Apercevant le baron qui s'avançait paisiblement derrière le \emph{Corpus
Domini}, qu'on portait, ainsi que d'usage, sous son dais de drap
d'argent croisé d'or, le marquis, qui rejoignait le cortège, avait
marché sur lui\,: «\,Vous avez, monsieur, tué naguère mon aimé cousin M.
de Langlon. Vous m'en rendrez raison, et sur l'heure\,!\,» Et, ne
laissant au baron de Primelles ni la place ni le temps de se mettre en
défense, il lui avait passé au travers du corps la courte épée qu'il
tenait à demi dégainée et cela avant même que sa victime eût fait voir
le jour à la sienne.

On répéta, en cette occasion, le mot de circonstance\,: «\,C'est là tuer
un peu en prince.\,» Mais on ne vivait plus sous cette bonne régence où
l'on pouvait assassiner ses ennemis en toute tranquillité, comme le
chevalier de Lorraine fit du vieux baron de Lux, en plein Paris, au
grand jour. L'émotion causée par cet attentat abominable n'était pas
apaisée que le marquis de Bannes, menacé de prise de corps gagnait les
Flandres sous un déguisement de courrier. Sur ses traces trottaient,
sans hâte, les gens du roi, beaucoup plus pressés de chasser ce proscrit
au delà des frontières que de le saisir. Le cardinal ministre préférait
toujours l'exil à l'échafaud, quand il ne s'agissait pas de complot
contre la sûreté du royaume. Et, dans la circonstance, il ne se souciait
pas de voir recommencer un procès semblable à celui de
Montmorency-Bouteville et de Rosmadec des Chapelles, procès engagé
contre sa volonté et dont il ne pouvait se rappeler les débats et
l'exécution sans se signer par trois fois. Et c'était au moment même où
Bouteville et des Chapelles portaient leur tête en grève que le marquis
de Bannes s'enfuyait dans les Flandres\,!\ldots{} «\,Qu'il aille se
faire justicier ailleurs\,! s'écria le cardinal, et qu'on ne l'arrête
pas, sur toutes choses\,!\ldots{} Qu'on lui donne à savoir que je désire
n'entendre plus parler de lui\,!\,»

C'en était fini du marquis de Bannes. Jamais il ne pourrait rentrer en
France. Un instant, Julie se demanda s'il ne lui faudrait pas rejoindre
à Bruxelles son maître et mari. Une lettre du marquis vint la tirer
d'embarras. Ses volontés étaient telles\,: «\,La marquise séjournerait à
Bannes, s'occuperait des intérêts avec les conseils du procureur
Marcelin Duvau. De Catherine, M. de Montenay était constitué tuteur. À
lui désormais incombait le soin de surveiller l'éducation de l'enfant.
Quant à Florimond, le marquis, son père, laissait à Julie, sa mère, la
charge de diriger ses premiers pas dans la vie, De ce fils,
Charles-Armand ne s'occupait que pour régler l'état de sa dépense et de
sa maison.

Julie demeurait donc en aussi bonne position qu'avant la catastrophe.
Dame et maîtresse à Bannes, se déchargeant sur Marcelin Duvau et ses
intendants des ennuis inséparables d'une grosse fortune en terres, mais
gardant de sa première condition les qualités ménagères, elle ne pensa
plus qu'à son Florimond. Pour lui ménager des ressources, car à mesure
qu'il croissait le jeune homme augmentait ses exigences, elle vécut
d'épargne, rançonna les fermiers, se montra impitoyable pour tout ce qui
touchait à l'argent. Son chagrin était de ne pas pouvoir mettre le nez
dans les finances de Catherine. Elle avait bien essayé de le faire, mais
M. Gaspard de Montenay l'éconduisit, dès le premier jour, avec une tant
courtoise ironie que la Drapière se retira confuse. Et Marcelin Duvau ne
réussit pas mieux dans ses entreprises.

Catherine de Lépinière entra, sur son désir, chez les Augustines de
Bourges, et M. de Montenay, deux fois le mois, la visitait pour
s'informer de son éducation. Chaque semaine, il écrivait au marquis\,:
«\,Notre enfant, lui répétait-il sur tous les tons, est vraiment une
merveille sur la terre\,; et c'est grand'pitié que vous ne soyez pas ici
pour jouir avec nous de sa gentillesse et de son exemplaire sagesse. Les
Dames du couvent me disent qu'elles n'ont au grand jamais rien rencontré
de pareil\,: «\,C'est une fée pour les travaux de la main, une sainte
Cécile pour la musique. Elle apprend tout, sait tout, entend tout, danse
à ravir, monte déjà à cheval, fait tout avec une grâce sans pareille,
et, avec cela, se montre si modeste qu'aucune de ses jeunes compagnes
n'en ressent de l'envie.\,»

\hypertarget{chapitre-iii}{%
\chapter{CHAPITRE III}\label{chapitre-iii}}

--- Vous êtes bien bon, dit M. de Tourouvre en allongeant sa jambe
bottée sur un coffre dont son éperon égratigna la couverture de
brocatelle, vous êtes vraiment bien bon de supporter les insolences de
cette péronnelle. À votre place, je dirais à madame votre mère de la
faire fouetter par la gouvernante. Elle en a le droit et, j'ajouterai,
le devoir.

--- On voit bien, maître Acresin, que tu ne connais pas M. le marquis
mon père. Si tu le connaissais, Acresin de Tourouvre, tu saurais que
quiconque lèverait le doigt sur Catherine serait, peu après, écorché vif
pour le moins. Crois-moi, ne t'y frotte pas, mon garçon.

M. de Tourouvre sourit, haussa doucement les épaules et répondit d'une
voix mielleuse\,:

--- Le marquis n'est pas ici. M. de Montenay, le père, est mort. Cette
pécore n'a pas de tuteur. Demain, si vous le voulez, elle en aura un de
votre main. Qui s'occupe d'elle à cette heure\,?\ldots{} Et si\ldots{}

Florimond l'interrompit avec impatience\,:

--- Qui s'occupe d'elle\,?\ldots{} Tout le monde et personne. Je ne sais
ce qui se passe dans ce misérable pays, mais il me semble que derrière
chaque buisson, derrière chaque arbre, derrière chaque motte de terre,
se tient embusqué un esprit familier, un espion qui nous guette. Ma
noble mère s'est aperçue de cela. Sa première parole, à mon arrivée, fut
pour m'avouer sa peur. Cette peur redouble ses migraines, et, par
surcroît, elle a ses lunes\ldots{} Le diable emporte les femmes, et
particulièrement Catherine\,!\ldots{} Quelle misère de s'exiler ici\,!

--- Prenez patience\,! On finira par vous trouver de l'argent.

--- Il me faudra attendre des semaines\,!\ldots{} Ne remue pas ainsi,
Tourouvre\,! Tu mènes un bruit\,!\ldots{} Ignores-tu qu'elle dort, là,
sous nos pieds, entourée par ses femmes, qui aspergent son visage avec
du benjoin, de l'eau d'ange, je ne sais quoi encore\,?\ldots{}

--- Elle a raison\,; car, mon cher Florimond, votre mère est encore très
belle\ldots{} Une taille\ldots{}

--- Attends qu'elle soit veuve, tu l'épouseras.

--- Cette plaisanterie, ne vous en formalisez pas, n'est peut-être point
d'un bon fils\ldots{} Mais vous aimez tant vos amis\,!\ldots{}

--- Tourouvre, tu divagues\ldots{} Appelle donc, pour qu'on nous apporte
du vin.

M. de Tourouvre s'en fut sur la pointe des pieds jusqu'à l'antichambre.
Bientôt parut le vin, présenté, par M. Clément Malompret en personne,
dans une cimare d'argent qui disparaissait jusqu'au col dans un seau à
glace. Le valet avait remplacé son mouchoir aux trois couleurs par une
ligne de mouches en taffetas. La plus grande, en façon de volet,
couvrait l'œil atteint. À cette vue, M. de Tourouvre parla du chapon
truffé. Et il exacerba la colère de Clément par des plaisanteries
savamment perfides, où il laissa percer quelques sentiments d'amicale
pitié. Et il murmura jusqu'à trois fois\,:

«\,Si j'étais le maître, pareilles choses ne se passeraient point sous
mon toit.\,»

Ce qui amena cette réponse de Florimond\,:

--- Tu es une bête, Tourouvre\,! On voit bien, je te le répète, que tu
ne connais pas le marquis\ldots{} Laisse-nous, Clément, et t'en va te
bassiner la face. Avec tes coutures et tes mouches, tu me gâteras le
goût du vin.

Clément se retira, comme il était entré, discrètement, non sans avoir
échangé avec M. de Tourouvre un regard auquel Florimond ne fit aucune
attention. S'il avait su lire dans ces regards, Florimond aurait compris
que ces messieurs de service n'étaient pas contents de lui. Mais il ne
vit rien et continua de morigéner M. de Tourouvre\,:

--- Oui, Tourouvre, quoiqu'on parle avantageusement en tous lieux de ton
esprit, je t'assure que tu n'es qu'une bête\,!

--- Vous me le daignez dire si souvent, monsieur, que je commence à
croire que vous vous trompez\ldots{} ou plutôt qu'on vous trompe.
N'était la crainte de vous déplaire, j'oserais insinuer que M. de la
Butière, jaloux de l'amitié dont vous voulez bien m'honorer, vous
entretient de moi sans douceur.

--- Ton imagination, Tourouvre, combat avec trop de succès ton peu de
raison. Pourquoi diable, veux-tu que La Butière soit jaloux de toi\,?
Vous êtes mes amis, continuez de l'être et ne me rompez pas\ldots{}

M. de Tourouvre sourit et murmura, mais si bas que Florimond ne
l'entendit pas\,: «\,Je ne m'étais pas trompé, cet ivrogne me dessert
auprès du jeune homme.\,»

Florimond se mordit la langue et reprit en bredouillant\,:

--- Ne me romps pas la tête\,! Non, ne me romps pas la tête avec tes
propos hors de saison. En mai, on ne doit penser qu'à l'amour. Ah\,!
Tourouvre, cette petite lingère de Bourges\ldots{}

On grattait à la porte. Avec la permission de Florimond, M. de la
Butière entra dans la chambre haute, tapissée de verdures de Flandres,
où l'héritier de Bannes avait élu domicile. C'était une vaste chambre
carrée, prenant son jour par trois fenêtres qui donnaient sur la cour
d'honneur du château. Sise au second étage, cette chambre était placée
juste au-dessus de celle où couchait la marquise Julie. Voilà pourquoi
Florimond recommanda tout d'abord à M. de la Butière, gentilhomme
entretenu ou, si l'on préfère, père nourricier du jeune baron de
Chézal-Benoît, de modérer le bruit qu'il menait en marchant sans
précautions, avec ses bottes fortes, car Florimond ne ménageait jamais
tant sa mère que quand il avait besoin d'argent.

--- Prends garde, Bérenger, tu vas réveiller ma mère\,! Elle a ses lunes
et tracasse ses filles d'atour. Si elle sort de son lit, ce sera pour
nous rebattre les oreilles avec ses lamentations. Assieds-toi et bois\,!
Ce vin n'est pas sans mérite.

M. de la Butière traversa la pièce comme si le parquet ciré, où se
reflétait sa longue personne, eût été formé d'œufs frais pondus. À ce
métier il trébucha plus d'une fois, s'embarrassa dans ses éperons, prit
enfin place sur un banc, s'adossa au mur, ramena soigneusement son épée
entre ses jambes maigres bottées de veau noir. Jamais homme plus noir de
poil ne s'assit sur un banc de chêne. Du bout carré de ses bottes à la
forme de son chapeau, de la garde de son épée à sa barbiche que ne
striait aucun fil d'argent, M. Bérenger de la Butière était noir. Noirs
ses gants, dont les gardes de velours atteignaient les coudes\,; noirs
ses cheveux, dont la moustache descendait enrubannée de noir sur son
pourpoint de taffetas noir, rattaché par des aiguillettes de soie noire
à son haut-de-chausses en camelot noir. Il avait le regard noir. Et pour
qui le connaissait son âme était d'un aussi beau noir que le reste. Au
vin seul M. de la Butière empruntait, à l'occasion, c'est-à-dire assez
régulièrement chaque jour que Dieu fait, une acrimonieuse gaîté. Quand
il n'avait pas bu, M. de la Butière rappelait le corbeau\,; quand il
avait bu, c'était le coq, dressé sur ses ergots, prompt à l'attaque et à
l'amour.

Enfin, M. de la Butière était tout noir. Seuls son visage pâle, éclairé
par un nez souvent rouge, et son col de chemise à deux pointes, rattaché
par des cordons à glands, échappaient à cette loi d'uniforme noirceur.
Il ne comptait guère plus de quarante ans\,: et on lui en eût donné
plutôt moins que plus, tant il se montrait bien conservé, alerte, svelte
et aisé dans ses mouvements. Il avait servi comme anspessade au régiment
de Vaillac, puis comme exempt dans la police. Les raisons pour
lesquelles il rentra dans la vie privée demeuraient obscures. On
prétendait qu'il avait tiré l'épée, et cela à plusieurs reprises, dans
des affaires qui étaient moins d'honneur que d'argent. Le marquis de
Bannes l'avait attaché à sa personne sur la recommandation de Julie,
alors que celle-ci, simple drapière sans lisières, devait s'entourer de
protecteurs qui fussent gens de tête et de main. La renommée, qui est
presque toujours une menteuse parce qu'elle s'en va partout semant des
propos dont elle serait bien embarrassée pour expliquer l'origine, la
renommée chargeait M. de la Butière de crimes sans nom et d'aventures
sans date. Toutes fois qu'aux entours de Julie et du marquis il se
passait quelque chose de violent, la rumeur publique dénonçait Bérenger
de la Butière. Mais, si la rumeur fait trop souvent loi, elle ne fait
pas foi, de telle sorte que la justice n'informait pas contre ce noir et
redouté personnage. Ou bien, si elle informait, elle perdait son temps.
Les babillages ne constituent point de charges au criminel.

On essaya d'impliquer M. de la Butière dans l'affaire de Franc-Cœur,
quand ce soldat eût le nez coupé à Vatan par des inconnus. Ce fut en
pure perte. Second du marquis en divers duels, il profita de ses lettres
d'abolition et ne fut pas inquiété. Aussi, quand on murmura qu'après
tout ce Bérenger avait bien pu tremper dans l'assassinat du drapier
Royer Hippeau, la justice se hâta de le mettre hors de cause. On l'avait
accusé jadis en vain, on ne voyait pas de raisons pour que les
accusations présentes fussent plus solides. Il avait mal versé dans la
police\,? Où et quand\,? --- C'était sous un faux nom\,! Lequel\,? C'est
pourquoi M. de la Butière, cautionné d'ailleurs par le marquis de
Bannes, produisit pour la forme des témoins qui jurèrent sur tout ce
qu'on voulut que ce noble homme, ancien anspessade, blessé au service du
roi, demeurait à Paris, rue Saint-Jacques, où il surveillait, par ordre
de son père, le jeune Pontaillan. Or, comme on ne peut à la fois tuer
aux portes de Bourges et éduquer un enfant à Paris, il fut décidé que M.
Bérenger de la Butière n'avait rien à voir dans ce fâcheux assassinat.

M. de la Butière, ayant bu un coup de vin, se plaignit en termes
cérémonieux et mesurés du mauvais esprit qui régnait sur le peuple des
campagnes, notamment aux environs de Lunery.

--- Que t'est-il arrivé\,? Parle\,!

À cette question de Florimond, M. de la Butière ne répondit d'abord
qu'en portant sa santé. Et M. de Tourouvre remarqua que de l'épée du
noir personnage un quillon était faussé, que son pourpoint, que ses
chausses portaient des traces terreuses, ainsi que ses gants.

«\,L'ivrogne, pensa-t-il, se sera battu contre des arbres dans un chemin
boueux\,: il en garde les marques.\,»

M. de la Butière toussa pour éclaircir sa voix et commença de parler
avec noblesse et lenteur\,:

--- Je longeais, ce tantôt, une pièce de terre qui confine à la garenne
de Montenay quand un spectacle aussi curieux que plaisant s'offrit à mes
regards distraits. Quoi de plus rare, en effet, de plus merveilleux
même, que de rencontrer dans les champs une jeune beauté vêtue en
bergère de comédie, portant une houlette à flocard de soie et tenant en
laisse, au moyen d'un galon, un agneau plus blanc que le lait de sa
mère, ayant pour collier une faveur de satin vert pré\,?

--- Tu te moques, La Butière\,! Et ton histoire sent trop son travail
d'auteur. Nous en avons entendu d'autres, à Paris, quand nous nous
amusions à jargonner le langage précieux des ruelles avec ces folles qui
s'amourachaient de toi.

--- Moins que de vous, mon cher Florimond\,!\ldots{} Moins que de
vous\,!

La phrase fut criée en même temps par M. de Tourouvre et par M. de la
Butière. Mécontents de s'être ainsi prévenus dans leur flatterie, les
deux favoris se turent un instant. Et M. de la Butière reprit\,:

--- Je vous jure que ce que je raconte est aussi vrai que voici, à cette
heure, M. de Tourouvre qui boit son troisième verre de vin.

M. de Tourouvre, vexé, tenta d'expliquer que porter trois fois son verre
à sa bouche n'est pas vider trois fois son verre. Et il ajouta avec une
conciliante prudence\,:

--- Je ne prétends pas, monsieur, mettre en doute ce que vous racontez
si plaisamment. J'ai assez vécu pour savoir que tout est possible.

--- Tout cela ne nous dit pas pourquoi tu te plains du peuple de la
campagne.

--- Ayez donc un peu de patience, et vous saurez, cher Florimond, de
quoi je me plains\ldots{} Donc, cette jeune beauté apparut au détour
d'un chemin, galamment accoutrée d'un simple habit gris de ramier, avec
cette simplicité sans apprêt qui sied avant tout aux bergères. Sa tête
charmante, aux mille boucles d'or qui se pressaient de chaque côté de
son visage clair et rosé, était coiffée d'un chapeau en écorce de
tilleul, avec un petit bouquet de fleurs des champs pour plumet. Bref,
c'était une vraie bergère de \emph{l'Astrée} car elle portait avec cela
des gants à franges et des souliers dont les bouffettes dépassaient un
jeune chou en ampleur. Ai-je dit\,: bergère de \emph{l'Astrée}\,?

--- Vous l'avez dit, en effet, répondit M. de Tourouvre, et tout dans
votre fidèle description est pour nous prouver que c'en était une.

--- Votre louange me flatte, monsieur, et vous ne savez pas si bien
dire. Derrière cette bergerette de comédie marchait une jeune paysanne,
ou plutôt une servante à demi rustique, qui portait un livre sous son
bras.

--- Et quel était ce livre\,? demanda Florimond en bâillant.

--- Je le sus bientôt\,: le livre, messieurs, était \emph{l'Astrée}\,!

--- Tu te moques de nous, Bérenger. Va conter à d'autres tes sornettes.

--- Ce sont si peu des sornettes que je puis vous nommer la bergère et
la porteuse de \emph{l'Astrée}.

--- Voire\,! La belle avance quand j'aurai appris que quelque fille de
théâtre, descendue du chariot avec une Isabelle de la troupe, se promène
par les champs pendant qu'on répare un brancard. Tu nous ennuies avec
tes comédiennes\ldots{} Enfin, si elle était aussi jolie que tu nous le
veux faire croire, tu aurais bien pu nous l'amener. On aurait laissé la
servante en bas pour divertir les laquais.

--- Mon cher Florimond, vous allez vite en besogne. Sachez d'abord que
ces deux beautés n'étaient pas sur vos terres.

--- Tu railles\,?

--- Mais sur celles de M\textsuperscript{me} de Primelles.

Florimond rit avec une méchanceté méprisante\,:

--- Tu appelles cela des terres\,?\ldots{} Tu pourrais dire\,: potager
ou clapier.

--- Clapier si vous voulez, mais le clapier avait à cet endroit une
lieue de largeur. Quant à sa longueur, je vous laisse le soin de
l'apprécier, ou plutôt les jambes de mon cheval la prouvent, car, pour
en sortir, il s'est rendu boiteux.

--- Ah çà, fit Florimond d'un ton aigre, tu ne vas pas t'amuser à crever
mes chevaux\,?

M. de la Butière fronça ses noirs sourcils et répondit d'une voix que la
colère rendait dure et insolente\,:

--- D'abord, ce cheval appartient à la marquise votre mère.

--- S'il en est ainsi, Bérenger, tu peux le crever à ton aise et en
prendre un autre après. Mais quel besoin avais-tu de faire des lieues de
pays\,? Tu n'avais qu'à couper au plus court à travers les mauvaises
herbes de ces gueux.

--- Je l'ai voulu faire, mais j'ai reçu des coups de bâton.

M. de Tourouvre, à entendre cela, pensa laisser choir son verre. Et
Florimond, laissant le fauteuil où il était plus qu'à demi couché, se
dressa sur ses pieds\,:

--- Tu railles, Bérenger\,!

--- Je raille si peu, messieurs, que mon chapeau en porte encore les
marques.

En effet, la haute forme dure du chapeau montrait deux cassures assez
nettes sur son sommet et le bord était déjeté, rompu, irréparablement
sans doute. La médaille qui tenait à la passe et servait d'enseigne
avait disparu ainsi que le piquet d'aigrette blanche.

Alors Florimond, ne pensant plus aux lunes de sa mère, trépigna, hurla,
tapa du poing sur la table\,:

--- On t'a bâtonné, toi, La Butière\,?\ldots{} Où sont les oreilles du
drôle\,?\ldots{} Çà\,! parleras-tu\,?

--- Chaque chose en son temps, s'il vous plaît. Laissez-moi conter mon
histoire, puisqu'elle a le bonheur de vous intéresser maintenant.
Monsieur de Tourouvre, je porte votre santé\,!\ldots{} Et donc, quand je
vis s'avancer ces deux nymphes villageoises, je les saluai très
gracieusement en portant mon cheval de côté de manière à barrer le
chemin, car je ne voulais pas qu'elles pussent s'échapper. Alors la
bergère, sans lâcher houlette ni mouton, dit à la suivante\,:
«\,Colbert, pose mon \emph{Astrée} au pied de cet arbre --- ainsi
appris-je que le livre était \emph{l'Astrée} --- et tu demanderas à ce
cavalier, qui se trompe sans doute de route, ce qu'il vient chercher
chez nous.\,» Je répondis aussitôt\,: «\,Ce que je cherche, mes déesses,
c'est la vue de vos aimables attraits\ldots{} Si vous êtes bergères,
j'entends être berger avec vous\ldots{} au moins jusqu'au coucher du
soleil. Laissez-moi vous courtiser librement.\,» --- Alors, vous me
croirez si vous voulez, la bergère au mouton dit à la bergère au
livre\,: «\,Le drôle est ivre sans doute\ldots\,»

--- Et c'était une vraie calomnie\,!

À cette exclamation de M. de Tourouvre, M. de la Butière n'opposa qu'un
rire sardonique. Son nez, qui s'enflammait, parut s'allonger, et
l'amateur de plaisirs champêtres continua\,:

--- La première bergère dit donc\,: «\,Le drôle est ivre sans doute, il
vaut mieux revenir sur nos pas. --- Oui, mademoiselle, fit la seconde
bergère, mais s'il s'obstine à nous importuner\,?\ldots\,» Je n'entendis
pas le reste, car les deux mignonnes chuchotaient derrière l'arbre où
était déposée \emph{l'Astrée }\ldots{} Qu'auriez-vous fait à ma place\,?
Je mis pied à terre et m'avançai vers ces deux belles en les suppliant
de ne point se montrer cruelles pour le berger Céladon\ldots{}

--- Écoute, Bérenger, tu fus peu généreux d'abuser ainsi de tes
avantages contre ces deux effrayées\,; continue\,!

--- Mon cher Florimond, avant de me blâmer, écoutez la fin de
l'histoire\,! Je les voulus embrasser. Mes bras se refermèrent sur du
vent. Abandonnant \emph{l'Astrée} au pied de l'arbre, mes deux coquines
sautent sur le revers du chemin, et, tirant d'une poche je ne sais
quelle flûte ridicule, la suivante envoie des sifflements pareils à ceux
de ces serpents qui dévorèrent, sous les yeux d'un peuple consterné,
Laocoon et ses fils.

--- Au fait, Bérenger\,! Au fait\,! Laisse là nos souvenirs de
collège\,: ils ne profitent qu'à toi\,!\ldots{} Au fait\,!

--- Moi, tout à mon amoureux caprice, je m'avançais les bras ouverts
pour enlacer d'un même temps les deux oiselles qui sifflaient sur leur
perchoir. Porté par l'amour, je gravissais le talus\ldots{} quand mon
chapeau, brusquement enfoncé sur mes yeux, me priva de la clarté du
jour, et une grêle de coups de bâton s'abattit sur mes épaules et mon
dos\,!\ldots{} Porter la main à mon épée\,?\ldots{} Vaine entreprise\,!
Plongé dans des ténèbres profondes, accablé par assez de coups pour me
croire assommé sans remède, je me laissai aller à terre, où je m'assis,
sans oser relever mon chapeau, de peur d'avoir les mains brisées par
l'invisible bâton\ldots{} Enfin l'on cessa de me battre et j'entendis
une voix dure et menaçante qui, de toute évidence, s'adressait à moi\,:
«\,Tu peux t'en retourner, bélître. Mais ne vagabonde pas plus longtemps
sur les terres de M\textsuperscript{me} de Primelles. Et si jamais tu te
permets de manquer de respect à sa fille ou à ses femmes, ta tête n'aura
plus besoin de chapeau. Va, tu es corrigé pour quelque temps, je
l'espère. Rentre chez toi, par la gauche\,!\,» Et une autre voix plus
grossière continua\,: «\,Monsieur, si vous m'en croyez, on lui baillera
encore une douzaine de bons coups pour épousseter son manteau.\,» La
première voix reprit\,: «\,Non, Martial, il a son compte. S'il y
revient, lui ou quelque autre valet de Bannes, il sera mis en tel état
que l'exemple profitera aux autres.\,» Les voix se turent. Je me
débarrassai à grand'peine de mon chapeau, regardai partout. Personne\,!
Les bergères, les gens à bâtons, le mouton, tout avait disparu comme par
enchantement, même \emph{l'Astrée}, qui n'était plus au pied de son
arbre. Altéré de vengeance\ldots{}

--- On l'eût été à moins, dit M. de Tourouvre en remplissant avec
politesse le verre que M. de la Butière gardait vide à la main, on l'eût
été à moins\,! Si encore vous aviez trouvé, près de cet arbre qui abrita
\emph{l'Astrée} de son ombre, la source du Lignon\ldots{}

Florimond, impatienté, éleva la voix\,:

--- Te tairas-tu, Tourouvre\,? Ces plaisanteries sont lourdes et froides
comme un carré de veau\,! Finis-en, Bérenger, mais apprends, si cela te
peut satisfaire, que je ne quitterai pas le Berry que Martial n'ait été
pendu.

--- Donc, messieurs, tout à ma vengeance, je remontai à cheval et
galopai à travers champs, l'épée à la main\ldots{}

M. de Tourouvre osa encore interrompre\,:

--- Mais l'on a dû vous prendre pour un faucheur et vous faire remarquer
que vous devanciez la saison\ldots{}

--- Sans me soucier des blés et des seigles, entendez-vous, monsieur de
Tourouvre, qui ne lasserez pas ma patience, dans l'espoir de mettre la
main sur les drôles qui m'avaient ainsi arrangé, je galopai. Mal m'en
prit\,: une sotte barrière dissimulée par les herbes se dressa devant
moi\,; mon cheval buta\ldots{} Et devinez qui me secourut dans ma
seconde chute\,!

--- Parbleu\,! s'écria M. de Tourouvre, les deux bergères, le mouton,
\emph{l'Astrée} et quelques porteurs de bâtons. Ainsi l'on vous a cette
fois encore secoué la poussière de votre manteau\,? Voilà qui est
vraiment fâcheux, mon bon monsieur, et souffrez que je vous plaigne\,!

--- Non, monsieur, répondit M. de la Butière avec une dignité empreinte
d'amertume, non, monsieur de Tourouvre\,! Ce furent deux sergents
blaviers, assistés d'un messier et d'une douzaine de croquants\,; et ils
me firent un procès pour avoir gâté les seigles, les blés, je ne sais
quoi encore\ldots{}

--- Eh bien, fit M. de Tourouvre, quoi de surprenant en cela\,? La loi
est telle. On ne laboure ni ne sème, je pense, pour que les cavaliers
s'amusent à fouler sans rime ni raison les récoltes.

--- Paix, Tourouvre\,! Le procès, ma mère le payera. Mais je jure Dieu
que le petit Primelles tombera, avant peu, sous mon épée\,!

Et Florimond, blanc de rage, brisa son verre sur le plancher.

--- Prenez garde, dit froidement M. de Tourouvre, la gloire est petite
et le péril très grand de tuer un enfant de seize ans.

Florimond, sans l'écouter, continuait ses menaces\,:

--- Quant à Bouteiller, ce vieil imbécile, dont je jurerais qu'il fut le
principal artisan de ce bel exploit, périsse mon nom si je ne l'assigne
pas demain, avec l'épée et la dague, comme on se battait de son
temps\ldots{}

--- Prenez garde, dit encore plus tranquillement M. de Tourouvre, le
baron de Mordicourt est un vieillard de soixante-dix ans.

--- Tourouvre, tu m'ennuies\,!

Entendant cela, M. de la Butière déclara qu'il était rompu, moulu,
rendu, et qu'il allait se mettre au lit sans souper\,:

--- Bonsoir\,! À demain les affaires sérieuses.

Et il sortit, aussi noir, mais avec le nez plus rouge qu'il n'était
entré, tant le vin du château de Bannes abondait en généreuses vertus\,!

Mais la brouille dont il escomptait déjà les profits n'éclata point
entre son rival Tourouvre et Florimond, baron de Chézal-Benoît. Dès que
Bérenger fut parti, M. de Tourouvre reprit l'avantage en montrant au
jeune homme l'abîme qui s'ouvrait sous ses pas.

--- Tout cela, mon cher Florimond, n'est-il pas pour vous prouver
combien vous raisonniez justement, suivant votre habitude, lorsque vous
me disiez que les espions nous entouraient\,? Outre que l'on est jaloux
de votre mérite, l'on vous fait aussi la guerre parce que de sottes
langues, et certes moins sottes que perfides, répandent ce bruit que
vous voulez continuer l'entreprise de votre père sur les terres du
voisin.

--- Verbiage\,! murmura Florimond. Verbiage\,! Misérables commérages\,!
Si l'on me connaissait mieux, l'on saurait que je n'ai aucun projet dans
ce triste Berry, où je ne viens, Dieu m'en est témoin, que contraint et
forcé par le manque d'argent\,!

--- Sans doute\,! Mais ces histoires nous prouvent que nous devons
redoubler de prudence. Loin de moi l'idée de porter contre M. de la
Butière une accusation téméraire. Cependant il est de toute évidence que
notre ami avait un peu bu, si j'ose dire. Ne convient-il pas de
rechercher dans son intempérance, je ne dirai pas coutumière, non,
certes\,! mais enfin, dans son intempérance, la cause de cette
aventure\,?

--- Après tout, fit Florimond en grognant, ce n'est pas une raison qu'un
homme soit saoul pour le battre à plates coutures\ldots{}

--- D'accord. Au reste, de cette aventure, le ridicule, je veux
l'espérer, n'éclaboussera pas votre nom.

--- Il ferait beau voir, Tourouvre\,!\ldots{} Et, quand le diable y
serait, je me moque des sots.

--- Nous nous en moquons tous avec vous. L'ennui, c'est qu'ils sont très
nombreux\ldots{} N'empêche que si La Butière n'avait pas eu les sens
troublés par Bacchus, l'idée ne lui serait pas venue d'insulter
M\textsuperscript{lle} Marguerite de Primelles tandis qu'elle jouait,
innocemment, chez elle, à son jeu de bergerie.

--- Ne voilà-t'il pas un grand malheur\,?\ldots{} Alors nous n'aurons
plus le loisir de nous amuser aux champs\,?

Suivant toujours son raisonnement, M. de Tourouvre continua sans
répondre à ce propos, dont la niaiserie ne le fit point sourire.

--- Est-il admissible que La Butière, qui connaît le pays depuis tantôt
dix ans que votre mère le prit à son service, n'ait pas deviné
M\textsuperscript{lle} de Primelles sous son déguisement\,?\ldots{}
Est-il quelqu'un, entre Lunery et Primelles, à ignorer les manies de
cette jeune fille\,?

Florimond, gêné par la justesse du discours, haussa les épaules et se
mordit les lèvres. Il s'en tira par un mensonge\,:

--- Moi qui te parle, je n'ai jamais vu cette sotte. Les gens de mon
entourage peuvent aussi bien ne l'avoir jamais rencontrée.

--- Il se peut. Vous observerez cependant que tout le monde vous rendra
responsable de l'insolence. On répétera qu'elle fut commise par vos
ordres\ldots{} Cela vous nuira.

--- Tu deviens joliment prudent, Tourouvre, et je t'admire avec ta mine
de prêcheur. Allons, cesse dérailler\,!\ldots{} C'est un coup monté, un
coup monté, te dis-je. On nous espionne, on nous tend des pièges. Je ne
sais qui me retient de partir à la tête de mes paysans pour mettre tous
ces gueux à sac\,!\ldots{} Et tu me chanteras qu'il est naturel que La
Butière ait été ainsi surpris, houspillé, dans ce lieu où personne ne se
laissait voir avant la bastonnade sous quoi c'est miracle qu'il n'ait
pas succombé\,!\ldots{} Je te dis, moi, que le coup était monté. On
l'avait suivi, le pauvre homme.

M. de Tourouvre sourit et appuya\,:

--- Oui, le pauvre homme\,!

--- On l'avait suivi, Tourouvre\,! Et les croquants de Primelles avaient
été avertis\,!\ldots{} Mais par qui\,?

--- En cherchant bien, mon cher Florimond, en cherchant\ldots{}

--- Ah\,! Tu as des soupçons\,?\ldots{} Allons, parle\,!

--- Vous avez sous votre toit, cela soit dit en confidence, des ennemis
attachés à vous nuire. Il n'y a pas une heure, vous dûtes subir leurs
malveillantes insinuations\ldots{}

--- Oui, j'entends\,! Catherine, à ce que tu crois, est la complice de
ces Primelles. As-tu des preuves\,?

--- Il faut prévoir, mon cher Florimond, avant même que de savoir. La
balafre de Clément ne vous semble-t-elle pas cousine des coups de bâton
si libéralement distribués à notre ami Bérenger\,?

--- Tu reconnais donc qu'il s'agit d'un coup monté\,! Et monté par
qui\,? Par Catherine\,?\ldots{} Fort bien\,!\ldots{} Et maintenant\,?

--- Je n'ose vous conseiller. Il me semble, toutefois, qu'un bon tuteur,
choisi par votre habile procureur Duvau, réussirait à mater cette
amazone\ldots{} On a vu des filles réfléchir dans la solitude bien
habitée d'un couvent. Entourées de bons conseils, elles acquéraient la
pleine sagesse et devenaient des épouses dociles\ldots{} comme je vous
en souhaite une.

Florimond bâilla, étira ses jambes, ses bras, et s'écria avec un accent
lamentable\,:

--- Toujours la même chanson\,! As-tu juré, Tourouvre, de me rompre la
tête avec ces projets insensés\,? Devrai-je te répéter cent fois que le
marquis nous surveille et que ma mère elle-même n'oserait rien
entreprendre contre Catherine, tant elle redoute le retour, toujours
possible, de mon père\,?\ldots{} Un caprice du roi, la retraite ou la
mort du cardinal, et le voici de retour\,! Alors nous n'aurons plus qu'à
gagner au pied\ldots{} Agréable perspective\,!\ldots{} Tu parles encore
d'un mariage avec cette petite furie\,?\ldots{} N'es-tu pas fou,
Tourouvre\,? ou bien mon vin te monte à la tête, car tu oublies que j'ai
été repoussé avec perte. Laissons cela\,! Catherine ne m'épousera
jamais.

--- Toujours et jamais sont, mon cher, des mots qui n'ont aucun sens,
surtout dans les choses de l'amour\ldots{} Et ai-je assez vécu pour
entendre ainsi parler l'Incomparable Florimond, l'impitoyable vainqueur
de tant de belles, le héros des fêtes galantes, terreur des pères au
moins autant que des maris\,?\ldots{} Foin de vous, baron de
Chézal-Benoît, je renie Dieu si l'on ne vous a pas changé en
voyage\,!\ldots{} \emph{Quantum mutatus ab illo\ldots{} Hectore}\,!

Et M. de Tourouvre fit semblant de pleurer dans son verre.

--- Ah çà\,! toi aussi, tu te mêles de réciter du latin\,!\ldots{} C'est
une maladie de collège\ldots{} Tu oublies, Tourouvre, que nous sommes
ici loin de Paris et que Catherine n'est pas une précieuse.

--- D'accord, mais bien une Diane chasseresse\ldots{} Voyez-vous,
Florimond, à votre place je sortirais souvent à cheval avec elle.

Pour cynique et enfoncé dans le mal que fût Florimond, le regard dont M.
de Tourouvre accompagna son conseil le gêna au point qu'il rougit et
baissa les yeux. Au même instant, il pensa à l'écuyer de sa mère, André
d'Archelet, qu'il avait essayé de tuer naguère, en faisant ruer son
cheval, parce qu'il le soupçonnait d'espionner sa mère, et lui aussi
Florimond. Catherine savait l'histoire\,; est-ce que Tourouvre la savait
aussi\,?

--- Va, laisse-moi, Tourouvre\,! C'est l'heure où ma mère me visite, et
elle voudra, comme à son ordinaire, causer seule à seul avec moi. Veille
à ce que personne ne nous dérange.

M. de Tourouvre sortit aussitôt, et Florimond, hochant le menton,
murmura\,:

--- Singulier et douteux personnage\,! Toujours prêt à semer le vent,
mais qui s'enfuit dès que lève la tempête. Si La Butière possédait le
quart de son esprit, il serait roi sur la terre. Mais ils ne s'entendent
pas\ldots{} Ou bien, quand ils s'entendent, c'est qu'ils conspirent
contre ma bourse. Jamais elle ne fut aussi plate\,!\ldots{} Et ma mère
crie misère maintenant\,: c'est son nouveau genre. Loin de m'assister,
elle se désole comme s'il était si difficile d'emprunter\,!\ldots{}
Qu'elle vende ses bijoux\,! Elle n'en a nul besoin, puisque personne à
Paris, dans le beau monde, ne lui veut ouvrir sa porte\ldots{} Tout cela
m'ennuie\,!\ldots{}

Il promena son regard distrait le long des tapisseries, sans même les
voir, se renfonça dans son fauteuil, prit son pied dans sa main et
soupira\,: «\,Quel ennui\,!\ldots{} Si j'avais le génie de la finance,
je trouverais une combinaison\ldots{} Mais je ne l'ai pas. Tout le monde
tombe d'accord pour en gratifier ma mère, de ce génie. Qu'elle s'en
aide, par les cornes du diable, et ne me laisse pas ainsi dans
l'embarras\,!\ldots{} Je me peux pourtant pas priver Madeleine du
nécessaire\,!\,»

Cette Madeleine était une lingère de Bourges allant sur ses dix-neuf ans
et qui s'appelait Brossin. Florimond la gardait comme fille entretenue
pour se distraire pendant les mois de l'année qu'il passait auprès de sa
mère, c'est-à-dire quand il manquait d'argent. Plus riche en beauté
qu'en vertu, cette noguette lui avait été procurée par Clément, le valet
de chambre qui s'en servait comme instrument propre à entretenir son
influence sur le jeune baron. Jamais âme plus vile n'anima carcasse
revêtue d'une mandille de laquais. Clément Malompret ajoutait à la liste
des péchés capitaux des vices singuliers et rares. Ses moindres défauts
étaient la lâcheté et l'hypocrisie. Et, comme il y joignait l'avarice,
ses amis étaient peu nombreux. Par contre, il avait des associés, pris
non point dans la valetaille, qu'il méprisait pour n'en rien ignorer des
petitesses, mais dans la domesticité supérieure. MM. de Tourouvre et de
la Butière étaient les associés de Clément.

Sur la recommandation de ce dernier gentilhomme, ce valet était entré au
service de Florimond, à charge de l'espionner avec art et de faire
naître des occasions profitables. Clément, bientôt, découvrit M. de
Tourouvre et le lia par un semblable contrat.

Celui-là était son ancien lieutenant aux chevau-légers, que des affaires
mystérieuses, se traduisant par défaut d'argent, avaient obligé de
renoncer à son grade. Il battait le pavé de Paris, assez embarrassé de
sa personne, quand il s'aboucha avec Clément Malompret, qui l'engagea au
service de son maître, parce qu'il avait besoin de quelqu'un pour
surveiller La Butière. Ayant ainsi embrassé la profession de gentilhomme
de service, M. Acresin de Tourouvre ne pensa plus qu'à faire la cour au
jeune Florimond et aussi à miner le gouverneur La Butière pour lui
prendre sa place. Malgré les ressources de son esprit et son habileté
incontestable à flatter sans trop s'amoindrir, M. de Tourouvre ne put
aboutir à ses fins.

Au demeurant, cet aventurier n'était ni meilleur ni pire que la plupart
de ses pareils. Mais sa gueuserie était telle, car son pauvre argent
s'en allait par le jeu, qu'il aurait vendu Dieu, le ciel, le royaume et
le roi à qui serait venu lui proposer le marché. Cependant M. de
Tourouvre jouissait auprès de la marquise d'une considération notable.
Parce que, sans doute, jeune encore, joli homme, bien disant et de
belles manières, il plaisait à la dame\,? Nullement. La raison
principale de cette faveur était autre\,: au vrai, M. de Tourouvre
détestait cordialement Catherine de Lépinière, autant par
incompatibilité d'humeur, car la belle-fille du marquis rendait au
gentilhomme de service pareille mesure d'aversion, que parce qu'il
redoutait sa perspicacité. Mais il ne dénigrait jamais Catherine devant
la marquise qu'en la présentant comme complice avouée des Primelles.
Puis il renchérissait sur la perfection de ses attraits et la rare
qualité de ses vertus.

Quand il eut remarqué la jalousie féroce dont souffrait M. de Tourouvre
à l'endroit de M. de la Butière, il exploita ce sentiment, il l'excita
de manière à ce que chacun d'eux en pâtit, et gouverna à son gré ces
deux sangsues attachées à la bourse de Florimond. Clément était le chef
de ce trio d'hommes de choix\,; d'ailleurs n'avait-il pas, à toute heure
de la nuit et du jour et en toute occurrence, l'oreille de son maître\,?
Et surtout, dans les moments difficiles, il lui prêtait de l'argent,
tandis que les deux autres compères, toujours à court, ne pouvaient que
tendre la main.

Ministre des plaisirs de Florimond, Clément se montrait aussi habile à
les exciter qu'à les satisfaire et à les entretenir, jusqu'à ce que la
satiété commandât d'en inventer de nouveaux. Quand le jeune homme
commença de courir les ruelles, les brelans et le guilledou à Paris, M.
Clément apparut à ses côtés ou à sa suite, gardant discrètement les
manteaux, ainsi que le dieu Mercure, son patron, faisait pour Jupiter
quand celui-ci descendait sur la terre pour répandre ses faveurs parmi
les filles et les femmes des hommes. M. Clément gardait les manteaux,
gardé lui-même par MM. de Tourouvre et de la Butière, capitaines d'une
douzaine de laquais grisons. Cette compagnie de donneurs d'étrivières
valait bien les fameux Simons de M. d'Epernon.

On s'engraisse vite à ce métier, quand le complaisant est ingénieux, et
surtout quand le galant est novice. M. Clément touchait de toutes mains
et ne ménageait pas ses services. Cet accordeur était sourd, muet et
aveugle. Son mutisme ne cessait que lorsqu'il s'agissait de chanter la
gloire de son maître. Si le fils du marquis de Bannes gagna à Paris ce
surnom d'incomparable qui aurait suffi à la gloire des plus grands, il
le dut à M. Clément, qui, pour l'avoir entendu prononcer par une dame
dont il abaissa un soir le marchepied du carrosse, s'en alla le crier
dès le lendemain sur les toits.

L'Incomparable Florimond s'abattit sur la belle société de Paris tel un
ouragan dévastateur. Avec ses dix-huit ans, sa jolie figure, sa
chevelure de Phœbus Apollon, ses manières mignardes de petit chat qui ne
montre pas trop les griffes, il tourna les têtes, ravit les cœurs,
bouleversa les ménages. M. Aimeri d'Olivier, son précepteur, homme d'un
esprit délié, plus souple qu'une liane, moins haut qu'une sole, ou, si
l'on préfère, plus plat qu'un turbot, possédant en un mot ce caractère
qui aide à se pousser auprès des grands, écrivait les lettres d'amour de
Florimond. Promenées partout sous le manteau, ces épîtres remplissaient
d'aise tout un chacun, particulièrement pour ce qu'elles fournissaient à
la chronique scandaleuse d'une société polie où l'on n'aurait su que
bâiller si le premier souci n'avait été de se divertir aux dépens
d'autrui. M. Aimeri tournait aussi, polissait de jolis vers que
Florimond recopiait et débitait, comme de son cru, d'un air à la fois
provocant et modeste qui lui valait, beaucoup plus que la poésie, poésie
un peu d'antichambre, des applaudissements attendris. Ses extravagantes
amours avec la comtesse de Rouilley, dont il tua le frère derrière les
Minimes, affirmèrent sa réputation.

De pareils succès ne vont pas sans exciter des indignations sauvages.
Alors l'Incomparable Florimond joua des griffes. Ses duels, presque
toujours heureux, le rendirent aussi fameux que ses galanteries.
Désormais solidement assise sur l'insolence et la fatuité, sa gloire
défia les envieux. Elle avait des bases trop solides pour qu'on les pût
sérieusement saper. Et même on lui pardonna sa mère, pour le courage
qu'il apporta à la défendre. Couverte par l'épée de son fils tout aussi
sévèrement qu'elle le fut par celle du marquis son mari, Julie fut à
l'abri des outrages. Le marquis, dans son exil, dut se consoler avec les
compliments qu'il recevait, de toutes parts, au sujet de son héritier.

Et Julie sentait redoubler pour l'enfant de ses entrailles une affection
où son orgueil intraitable de mère s'étalait avec une exagération
touchante.

À la vérité, cette femme, dont la vanité mondaine aurait suffi à remplir
la vie et qui ne put satisfaire cette passion, reportait sur son fils
toute sa capacité d'aimer. Elle n'aimait que son Florimond sur la terre.
Pour lui, elle fût allée par la ville et par les champs, en camisole de
nuit et en petit jupon, eût dîné d'un maigre potage, si de ces
sacrifices Florimond eût bénéficié pour augmenter ses plaisirs, pour
souper en plus riche compagnie chez les Guillemins et à l'Épée Royale,
pour entretenir des comédiennes et payer les violons aux dames du
Marais. Ce que coûtèrent à la marquise drapière les Isabelles, les
Agnès, les Dorines, les dames du Marais, les violons, les collations,
les rigodons et les dondons eût suffi à tenir sur pied, pendant deux
campagnes, une armée du roi. Julie paya sans mot dire, ou plutôt sans
dire autre chose que\,: «\,Si mon fils ne prend pas du bon temps pendant
sa belle jeunesse, quand s'amusera-t-il, s'il vous plaît\,?\,»

Mais la prudence du marquis, qui m'était plus dans sa belle jeunesse,
avait prévu ces inconvénients. Sa fortune, que des influences
toutes-puissantes avaient sauvée de la confiscation, fut administrée, de
loin, par lui-même. Des Flandres, il adressait à ses intendants et à son
notaire de Bourges, M. Audouin Trémolat, des ordres que l'on exécutait
malgré les entreprises sournoises de la marquise, et Florimond, dont les
dépenses allaient toujours s'augmentant, vit bientôt se tarir la source
des libéralités de sa mère.

Tant que celle-ci avait puisé dans les réserves que son ingénieuse
économie avait accumulées en vue de suffire aux prodigalités de son
fils, ce fils avait pu vivre sans compter. Il fallut bientôt en
rabattre. C'est alors que le bien de Catherine de Lépinière tenta la
cupidité des usuriers qui avançaient de l'argent à la marquise. Parmi
ceux-ci le procureur Marcelin Duvau était le plus actif et le plus
qualifié. Conseil de Julie aux temps difficiles où elle se débattait,
simple drapière en rupture de ménage, contre des ennemis ameutés, ce
fesse-Mathieu sans honneur et sans scrupules n'ignorait rien des
affaires du marquis, rien non plus des propres de Julie, rien non plus
de la fortune de Catherine. La mort de M. de Montenay le père permit,
pendant ces derniers mois, à Julie et à son procureur de malverser aux
dépens de M\textsuperscript{lle} de Lépinière. Constitué tuteur de
celle-ci six ans auparavant, M. Gaspard de Montenay avait administré la
fortune de sa pupille avec une telle sagesse que cette fortune s'était
augmentée d'un bon quart en ce court laps de temps.

Par des influences de notaires peu consciencieux et des corruptions de
greffiers, Marcelin Duvau réussit à mettre la main sur une partie de
l'argent, et la danse des écus recommença, aux dépens, cette fois, de
Catherine, et toujours au bénéfice de Florimond, et aussi du complaisant
procureur, prêteur à la courte semaine. Étonnée de se voir refuser les
sommes que son défunt tuteur lui comptait à dates fixes pour ses
dépenses personnelles, sa fauconnerie, ses pauvres, ses chevaux et ses
chiens, la belle-fille du marquis lui en écrivit. L'on sait comment la
marquise Julie étouffa ces plaintes et comment le marquis, une fois
averti, répondit.

À craindre la gêne pour son enfant bien-aimé, la marquise devint une
lionne en fureur. Ce qui la désespérait dans le cas était autant son
impuissance que la présence au château de Bannes de cette injurieuse
fille étrangère, qui y vivait entourée de soins et d'égards sous
l'invisible égide du marquis. Et dire qu'il n'aurait tenu qu'à cette
Catherine de devenir la femme de Florimond et de tout arranger en
apportant sa fortune à l'incomparable coureur de ruelles\,!

Car des autres courses de son fils la marquise ne voulait rien savoir.
Ses œuvres de coupe-jarrets, les querelles de tripots, les violences sur
les gêneurs et les empêcheurs de danser en rond, les guets-apens et
autres distractions du jeune baron de Chézal-Benoît, cela regardait
Bérenger de la Butière, M. de Tourouvre et la compagnie choisie des
grisons donneurs d'étrivières. Mais de ces hommes de main l'on ne
pouvait pas tout attendre. Les mauvaises besognes qui leur souriaient à
Paris leur répugnaient en province. Quand la marquise proposa à La
Butière d'enlever adroitement et courageusement les comptes de tutelle
de Catherine déposés chez le notaire du marquis, M. Audouin-Trémolat, ce
gentilhomme se refusa très carrément à tenter le coup\,: «\,Couper le
nez d'un soldat ne tirait pas à conséquence. A la rigueur même, la mort
subite d'un mari pouvait passer pour un accident\ldots{} Quant à voler
des pièces de justice, cela menait au gibet et pour le moins aux
galères\ldots\,» La marquise ne put rien obtenir, et elle s'en trouva
bien, car elle apprit bientôt que non seulement M. de Montenay le fils
avait levé copie enregistrée de tous les papiers, mais qu'encore le
méfiant notaire faisait coucher chaque nuit un homme armé sur le coffre
ferré, cadenassé, scellé, où reposaient les pièces originales de la
tutelle en question.

La marquise ne fut pas plus heureuse avec M. de Tourouvre.

Et pourtant on a prétendu que cette mère désespérée serait allée jusqu'à
l'offre de sa personne, personne entre toutes belle et plaisante. Pris
entre la terreur salutaire des justes lois et la terreur non moins
salutaire du marquis absent, mais dont l'ombre semblait se promener dans
toutes les chambres pour retourner à Bruxelles et lui raconter ce qui se
disait et se passait chez lui, M. de Tourouvre parla d'autre chose\,:
«\,Si vous m'ordonniez, madame, de tordre le cou à votre amie Catherine
et de l'aller jeter dans les douves de Primelles, je n'oserais vous
refuser\ldots{} Pour le reste, veuillez m'excuser, s'il vous plaît.\,»

La marquise ne revint plus sur ce sujet. Loin d'arranger les choses, la
mort de Catherine de Lépinière n'aurait fait que les embrouiller, et la
haine aveugle de Tourouvre demandait plutôt à être calmée. Cette haine,
la marquise s'essaya à la diriger contre les Primelles. M. de Tourouvre,
en bon courtisan, affecta de la partager, tout en prêchant la patience,
l'oubli des injures, car cet homme était entre tous expert à répandre de
l'huile sur le feu sous prétexte de l'éteindre.

Julie exécrait les Primelles parce qu'elle ne voyait en eux que les
auteurs de l'exil du marquis, exil qu'elle considérait, bien à tort,
comme la cause unique de ses disgrâces. Ce sentiment occupait dans son
cœur le peu de place qu'y laissait son amour déréglé pour Florimond.
Elle s'était juré de détruire cette famille misérable, d'acheter à vil
prix toutes ses terres et d'obtenir du marquis, en récompense de ce
service, des facilités d'argent dont son fils aurait profité. Mais,
aussi patiente et réfléchie dans sa haine qu'elle se montrait impatiente
et violente dans son affection, Julie la Drapière, marquise de Bannes,
attendait, pour le faire tuer par Florimond, que le petit Louis-Antoine
fût en âge de tenir une épée.

\hypertarget{chapitre-iv}{%
\chapter{CHAPITRE IV}\label{chapitre-iv}}

La marquise de Bannes entra chez son fils avec cet air majestueux et
aisé qui était un véritable air de cour. Cette tranquille assurance
s'alliait à merveille avec l'opulence de sa beauté épanouie. Loin de la
flétrir, les années n'avaient fait que la rendre plus glorieuse, ainsi
que ces beaux fruits que veloute et dore l'amoureux soleil d'automne.
Julie tendit à Florimond sa main droite à baiser, et sa main gauche, où
se cachait à moitié un mouchoir brodé en point de Gênes, pendait entre
les plis savamment pressés de sa robe à queue. Mais Julie ne put garder
longtemps ce maintien cérémonieux. Lorsqu'elle vit cette tête blonde qui
s'inclinait devant elle, ses deux mains blanches constellées de bagues
la saisirent d'un mouvement avide, et elle la couvrit de baisers pressés
en murmurant\,:

--- Enfin je te tiens, mauvais sujet\,! Mon enfant chéri, mon trésor,
mon roi\,!

Désireux d'échapper à cette étreinte qui dérangeait à la fois sa
coiffure et son col de dentelle empesé, Florimond répondit à ces
embrassements de Médée avec une lassitude maussade que sa mère ne
remarqua pas tout d'abord. Quand il eut réussi à se tirer des mains de
sa mère, qui laissèrent sur sa personne un parfum de frangipane et de
musc, quand la dame se fut assise mollement dans un grand fauteuil et
eut tapoté sa robe pour en ramener les fronces à leur ordonnance
première, il lui dit d'un accent pénétré\,:

--- Ma mère, vous êtes une très belle femme, et c'est plaisir que de
vous admirer.

La marquise rougit de plaisir\,; et, menaçant de son doigt effilé le
gracieux Florimond, elle répondit d'un ton détaché\,:

--- Venant de vous, mon fils, ce compliment a sa valeur. Il me
transporte d'aise, car l'on sait que vous êtes un grand connaisseur.
Votre encens est de grand prix. Je ne suis, hélas\,! plus telle qu'au
jour trois fois heureux où je vous mis au monde. En ce temps, pour moi
trop éloigné, et quoique vous soyez dans la pleine gloire de votre
triomphante jeunesse, le marquis, votre père, --- puisse la grâce de
Dieu toucher le roi et le décider à rappeler ce père tant regretté\,!
--- le marquis, votre père, voulait bien reconnaître que j'étais une
femme digne en tout d'honorer son état.

La belle Julie débitait cela avec son air de cour, un air de cour
d'excellent aloi qui était vraiment d'une grande dame. Le malheur avait
formé la Drapière, et l'isolement où elle vivait lui avait certainement
profité. L'on comprenait, à la voir et à l'entendre, les jalousies
féroces qu'elle excitait encore aujourd'hui, alors qu'elle avait dépassé
quarante ans. Sa taille souple et superbe et sa gorge plus fière que
celle des Flores et des Pomones dont la nudité de marbre animait les
allées ombreuses de son parc ne perdaient rien à être emprisonnées dans
le corsage de drap de soie noir qui allait s'évasant au-dessus de ses
vastes jupes élargies encore par le vertugadin de proportions modestes
qui les bombait suivant le galbe amplifié des hanches. Depuis la fuite
du marquis, Julie ne portait plus que du noir. Mais le velours de
Piémont, la serge et le camelot ondé, le satin et l'écarlate la plus
fine compensaient par leur richesse la sombre sévérité de cette mise. La
marquise était de ces blondes dont la chair, pareille à la nacre des
coquilles, éclaire le noir. De même pour ses bijoux, qui étaient des
perles, des pierres taillées et des émaux noirs. Cette joaillerie à feux
infernaux chargeait les oreilles, le cou, les épaules, s'enroulait en
colliers, s'allongeait en pendants, se balançait en pentacols, se
lançait en sautoirs, retombait en chaînes de poitrine et de cou, semait
de ses traînées luisantes les lisérés des taillades, arrêtait par ses
fermoirs les divisions symétriques des chiquetades. Et Florimond prisait
toutes ces gemmes, songeant que le prix de la moitié eût plus que
largement suffi à lui assurer pendant une année tout entière ces
plaisirs de Paris dont il se désolait d'être sevré.

Le compliment qu'il avait adressé à sa mère était un compliment de bonne
foi. Et elle n'avait pas menti en lui disant qu'il était un grand
connaisseur. Toutefois, dans ce compliment l'affection n'occupait aucune
place. Florimond n'aimait pas sa mère, car, en réalité, il n'aimait que
lui, ou bien, s'il l'aimait, c'était à la façon de ces dévots qui
honorent, au delà des monts, leur saint avec beaucoup de génuflexions et
à grand renfort de cierges, à la condition que ce saint exauce leurs
vœux. Qu'il cesse de se montrer propice, et aussitôt on le met en
pénitence, dans un coin, le nez au mur, jusqu'à ce qu'il ait fourni de
nouvelles et solides preuves de sa bienveillance et de son influence en
haut lieu.

Florimond s'assit, regarda attentivement sa mère, ainsi que les bas
officiers examinent les soldats avant une montre du commissaire des
guerres, et, d'un ton docte, redit son compliment\,:

--- Oui, ma mère, vous êtes très belle.

La marquise sourit agréablement, et son visage un peu mou se fit rose
sous le blanc et le rouge dont il était légèrement rehaussé. Elle eut
très bien pu se passer de ces artifices, étant de celles dont l'éclat
n'a rien d'emprunté, comme aussi des mouches de taffetas qui se
répondaient de la tempe au coin de la bouche\,; mais la bonne façon
exigeait qu'il en fût ainsi. Sa peau, tout à la fois transparente et
mate, avait la pâleur diaphane des fleurs d'un arbre fruitier, de son
front à son cou plein cerclé d'un fin pli, collier que Vénus donna aux
femmes pour leur prouver qu'elles vivent courbées sous sa loi, et à la
naissance de sa gorge, que découvrait décemment le col rabattu avec ses
larges dentelures obtuses de réseau des Flandres où les rosaces
alternaient avec des étoiles à huit rais.

Ses yeux bleus, à fleur de tête, légèrement battus et bridés, avaient
sous leurs paupières lourdes, toujours un peu clignotantes, une
expression de langueur dédaigneuse et sournoise que ne contredisaient
pas ses sourcils fauves, franchement arqués, relevés vers les tempes, et
qui n'étaient jamais d'accord, car leur mobilité était excessive et,
pour ainsi parler, contradictoire. Quand un coin de la bouche --- et
cette petite bouche, une belle œuvre de Dieu, rappelait le bec de
quelque petit oiseau qui a volé une cerise --- se retroussait pour
sourire en exagérant sa fossette qui se continuait en un tout petit pli
qui chatouillait la narine battante, le sourcil du côté opposé se
haussait, et cela produisait le meilleur effet du monde, en dépit de ces
esprits chagrins qui voudraient nous obliger à croire que le manque de
symétrie dans le jeu des traits est la preuve d'une nature oblique et
d'un caractère incertain.

Le nez droit, gentiment arrondi à son fin bout, le menton bien dessiné
quoique fuyant, et pas trop empâté, s'harmonisaient avec l'ovale court
du visage. Sur le front poli et bombé descendaient les boucles pressées
des garcettes. Par une raie transversale les rangs de ces crochets ténus
se séparaient de la chevelure rejetée en arrière, mais maintenue par le
travail du fer en deux touffes ondées et largement bouffantes sur les
côtés, de telle sorte que ces boucles extrêmes couraient en hameçons
rasant les joues, dans la même ordonnance que celle du front, et
encadraient cette jolie face de leurs menues arcades, qu'on eût dit de
verre filé et couleur de miel. Et encore de cette chevelure ainsi massée
pour rendre la tête plus ronde, le chignon perché à mi-hauteur se
sommait d'une courte et large aigrette blanche, en manière de brosse,
maintenue à sa racine par un tortil de perles. Et enfin on avait, pour
obéir à la mode, éteint avec un œil de poudre blonde ces cheveux d'un
blond d'or pâle, pour en adoucir, si possible, les contours crêpelés et
mousseux.

Et voilà ce que Florimond admirait le plus chez sa mère. Pour lui, la
première vertu d'une femme était d'être brave et galante dans ses
ajustements, sa coiffure et ses airs, car c'est sur la mine que l'on
juge les gens et d'après elle qu'on leur assigne la place qu'ils
méritent d'occuper dans le monde. Il approuva la touffe d'aigrette,
l'eût souhaitée un tantinet plus serrée, loua le petit piquet de roses
que soutenait une broche d'émaux au-dessous du sein droit. Par contre,
il critiqua l'échancrure du corsage, dont la courbe était passée de
mode, la façon des manches, qui sentait un peu sa province, demanda si
les sœurs Aubert de Bourges taillaient toujours d'après des patrons
venus de Paris, regretta de n'en avoir pas apporté et parut curieux de
savoir si Jacqueline Amelin était toujours réputée parmi les bonnes
lingères.

La marquise sourit, tira du large ruban qui lui servait de ceinture un
minuscule miroir que rattachait à son cou une chaîne à fins anneaux
imbriqués, y mira le bout de son nez, s'éventa avec son mouchoir parfumé
d'ambre, et dit d'une voix pointue\,:

--- Puisque vous parlez de lingères, je vous en parlerai aussi, et,
entre nous, on en parle trop à Bourges\ldots{} Aussi vrai que tu habites
dans la peau du plus charmant garçon du monde, monsieur mon fils, tu
n'es qu'une bête\,!

Sans se rappeler qu'il avait gratifié, une heure auparavant, M. de
Tourouvre d'une épithète identique, Florimond regarda sa mère de son air
le moins gracieux. Et, si celle-ci n'eût pas été habituée à voir dans
son enfant chéri le détenteur de l'humaine perfection, elle aurait crié
d'épouvante devant ces yeux que semblaient obscurcir d'épaisses vapeurs.
Seuls, les yeux des taureaux en fureur ont cet aspect nuageux. Rarement
yeux plus sincèrement méchants n'arrêtèrent leurs regards sur une femme.
La marquise possédait, au vu et au su de tous, des yeux bleus assez
sournois. Mais ils étaient d'un joli bleu, sans profondeur, avec la
pupille encerclée d'orange, de manière que, lorsqu'ils s'animaient,
c'était pour répéter l'image du ciel, dont l'azur est constellé de
points d'or\,; tandis que les yeux de Florimond, tout aussi larges,
étaient d'un vert glauque qui s'assombrissait, aux heures mauvaises,
pour devenir presque noirs. Seul le marquis de Bannes, quand l'égaraient
ses fureurs, ouvrait de tels yeux, sombres, glauques, voilés, sans fond
comme l'abîme et vitreux.

La marquise ne voulut pas remarquer cette colère qui commençait de
monter\,; elle continua de sa voix de tête\,:

--- Tu n'es qu'une grosse bête, et l'on se moque de toi à Bourges. Ta
mignonne, Madeleine Brossin, la lingère, travaille justement chez
Jacqueline Amelin. Cela est fâcheux, Florimond, parce que j'ai dû cesser
de me fournir chez Jacqueline. Ouvrières, apprenties et maîtresses
répètent à qui veut bien écouter que tu as promis à Madeleine de
l'épouser.

Florimond répondit en grondant\,:

--- Si l'on retenait tous les commérages, la vie deviendrait impossible.
Puis-je empêcher les sottes femmes de bavarder\,? Jacqueline Amelin est
une coquine. On en a fouetté pour moins au cul d'une charrette\ldots{}
Si je la tenais\,!\ldots{}

Il passait sa colère sur les crépines de son fauteuil. Sous ses doigts
impatients la passementerie volait en pièces.

La marquise, très attentive à mirer le coin de son œil dans la glace
montée en or, ne releva pas l'interruption\,:

--- Florimond, je te le répète, tu n'es qu'une bête. Est-ce là, entre
nous, une distraction digne de toi que de faire l'amour aux lingères\,?
Un gentilhomme qui se respecte et à qui tant de grandes dames veulent du
bien ne s'attache pas à une femme de petite condition\ldots{}

Et, comme son fils la considérait avec une insolente insistance, elle
marqua et reprit avec une héroïque simplicité\,:

--- Le marquis, ton père, m'a épousée, c'est vrai\,! Mais, mon enfant,
les ennuis dont il n'a cessé de souffrir et dont je prends encore ma
juste part prouvent la vérité de ce que j'avance.

Si peu délié que fût l'esprit de Florimond, il comprit ce que cet aveu
dépouillé de tout artifice avait de grand. Il se leva, s'agenouilla
devant sa mère, qui, une fois ses mains prises dans les boucles blondes
de l'Incomparable, se tut, muette et ravie. Florimond, toujours à
l'affût d'une position à exploiter, ne quitta pas sa place. Il se
plaignit, d'un ton pleurard, de la méchanceté des hommes. «\,Jamais il
n'aurait pensé à cela\,! Sa mère était sa mère\,! Il continuerait de la
défendre contre tous\,! Et puis\ldots\,»

Et puis il ne sut plus que dire, parce qu'il n'osait pas profiter de cet
émoi pour demander de l'argent. Il s'attendrit sur sa solitude, essaya,
sans succès, de pleurer, supplia sa mère de ne plus rire de lui. «\,Que
voulait-elle qu'il devint dans ce trou de campagne\,? S'il y restait,
c'était pour l'amour d'elle\ldots\,» Il s'embrouilla dans ses
explications, ses récriminations, et termina par une de ces scènes de
feinte colère dont seuls les enfants gâtés et les amoureux ont le
secret.

La marquise l'écoutait, ravie, en souriant d'une façon tout à la fois
narquoise et joyeuse qui n'avait rien de maternel en apparence. Au fond,
Julie prenait à regarder ainsi se débattre son fils ce plaisir
instinctif des bêtes féroces qui surveillent du coin de l'œil les
allures de leurs petits quand ils se roulent à l'orée des tanières.
Quand Florimond eut bien tempêté, quand il se fut rassis, boudeur, dans
le fauteuil où il recommença de plumer les houppes, elle dit du ton le
plus indifférent\,:

--- Mais, nigaud, regarde donc autour de toi.

Niaisement, Florimond inspecta la chambre du regard.

--- Non, ce n'est pas ici, fit Julie en se levant, mais là-bas. Tiens,
en face\,!

Elle le mena à la fenêtre en s'appuyant nonchalamment sur son bras, lui
montra le pauvre manoir des Primelles que buvaient les ombres du soir.
Sur le chemin passait une jeune fille tenant un mouton en laisse. Une
autre fille la suivait avec un livre sous son bras.

Alors la marquise dit à l'oreille de son fils\,:

--- Tiens, la voilà, Marguerite de Primelles, avec son \emph{Astrée} et
son agneau\,! Ne crois-tu pas, Incomparable Florimond, que ce serait là
pour toi une conquête plus noble que cette petite lingère de Bourges qui
se vante de se marier quelque jour prochain avec toi, baron\,!

--- Y songez-vous, ma mère\,? Moi épouser cette pauvresse dont le père a
été tué par le mien\,!\ldots{}

--- Eh\,! grande bête, qui te parle d'épouser\,?\ldots{} Allons, je me
sauve. N'oublie pas que ce soir nous avons ce bon Duvau et quelques amis
de Bourges à souper. Et surtout ne les traite pas de Turc à More. Ces
robins connaissent plus d'une manière de se procurer de l'argent.

Florimond dressa l'oreille. La marquise, déjà sur le seuil de la porte,
menaça son fils du doigt et lui jeta ces paroles, de sa voix maintenant
basse et sifflante\,:

--- Pense à la bergère\,!\ldots{} On en dit beaucoup de bien. Et lis
\emph{l'Astrée}, seul, sans trop bâiller sur ces inepties
champêtres\ldots{} En attendant que vous la lisiez à deux, tu m'entends.

Demeuré seul, il retourna à la fenêtre. M\textsuperscript{lle} de
Primelles avait disparu. Il haussa les épaules, bâilla, suivant son
habitude, se rassit dans son fauteuil et, tout en tirant sur les
houppes, s'abîma dans ses réflexions.

«\,Ma mère me croit plus sot que je ne suis, ma parole\,!\ldots{} Dans
son désir de me voir débarrassé de Brossin, elle s'imagine que je vais
me lancer dans une aventure galante avec quelque beauté rustique qui
répondra mal poliment à mes avances en me lançant son sabot à la tête et
en appelant au secours tous les croquants du pays\ldots{} Il m'a semblé
cependant que cette bergère de fantaisie avait une jolie tournure. Bast,
sottises que tout cela\,!\ldots{} Des aventures sans élégance ni
éclat\,!\ldots{} Ma mère a profité de ma simplicité pour éluder la
question financière\ldots{} Et je n'ai pas eu le temps seulement de lui
parler de Catherine\,!\ldots{} Ah\,! qui me débarrassera de
celle-là\,!\ldots\,»

Florimond en fut débarrassé, au moins pendant le souper, car
M\textsuperscript{lle} de Lépinière ne sortit point de sa chambre, où
ses femmes la servirent. La satisfaction qu'il éprouva de ne pas voir
cette figure ennemie fut empoisonnée par les mauvaises nouvelles
qu'apportait Marcelin Duvau\,: le marquis avait écrit, et ces messieurs
de Caumont aussi. De tous les résolutions étaient unanimes et
absolues\,: il fallait obéir.

M. de Montenay le fils succédait à son défunt père comme tuteur de
M\textsuperscript{lle} de Lépinière. Le notaire Audouin Trémolat
s'occupait d'apurer les comptes. Et le procureur Duvau se désolait à
l'idée de rapporter. Bien plus, M\textsuperscript{lle} Catherine, avec
qui il avait conféré avant le souper, s'était donné le plaisir de
traiter M. Duvau de haut en bas. «\,Et cette méchante fille, non
contente de fournir ces preuves de sa perversité et de son audace en
laissant éclater une joie aussi insolente que déplacée, avait déclaré
qu'elle continuerait de séjourner au château de Bannes\,!\,»

--- Oui, madame, criait maître Duvau, elle l'a déclaré, parlant à ma
personne\,! Et cette demoiselle --- dont Dieu me préserve de penser du
mal\,! --- m'a ri au nez, sauf votre respect. Désormais, elle vivra à
part\,; dans cette aile gauche dont le marquis absent --- puisse-t-il
bientôt revenir\,!\ldots{} Je bois, madame, à son prochain
retour\ldots{}

On but sans grand enthousiasme. Dans le fond, personne ne se souciait de
voir revenir le marquis à Bannes\,:

--- Quelle lessive, sainte mère de Dieu\,! Quelle lessive s'il revenait
jamais, mes enfants\,!

C'était l'opinion de M. de Tourouvre, et tout le monde la partageait,
sauf peut-être André d'Archelet et les femmes de service. Mais on ne les
consultait point.

M. Marcelin Duvau vida son verre, soupira profondément et reprit\,:

--- Le marquis en a ainsi décidé. M\textsuperscript{lle} de Lépinière
aura sa maison montée, comme une princesse, madame\,! Votre écuyer
meneur, André d'Archelet, passe à son service\ldots{}

--- Grand bien lui fasse, dit la marquise d'un ton glacial.

--- Bon débarras pour nous, s'écria Florimond, je n'ai jamais pu le
souffrir\,!\ldots{} Ah çà\,! Duvau, mon père a-t-il pris quelques
dispositions pour me tirer d'embarras\,?

Julie fronça le sourcil droit\,; aussitôt son sourcil gauche s'ouvrit
gracieusement en arc tendu, ce qui donna à sa jolie figure un air
d'indécision comique et charmant.

--- Laissez cela, mon fils, nous en parlerons demain à loisir.

Mais, emporté par son ressentiment, le procureur ne put s'arrêter de
dire qu'aucune disposition n'avait été prise.

--- Hélas\,! non, monsieur\,! Votre père ordonne que vous restiez ici
jusqu'à ce que l'on ait épargné sur votre pension de quoi payer vos
dettes. Il a parlé aussi de vous acheter une compagnie aux gardes. Les
oncles Caumont s'occuperont de la chose.

--- Et encore\,?

--- Ah\,! laissez cela, mon fils\,! Vous savez bien que je suis là et
qu'avec Duvau nous arrangerons toujours ces choses que vous êtes trop
jeune garçon pour entendre. Ne vous mêlez pas d'affaires, je vous en
prie. Vous êtes né pour filer le parfait amour avec les bergères de
l'Astrée.

Et la marquise échangea un regard d'intelligence avec Duvau.

«\,Elle y tient décidément, se dit Florimond. Mais quel est son plan en
me lançant sur cette piste\,? Ma mère n'est pas d'un tempérament
bucolique. Elle doit manigancer quelque ténébreuse intrigue\ldots{}
Demain j'en parlerai avec Olivier. Ce soir, soyons tout à la joie. La
mine épanouie de ma mère, malgré les avis graves et recueillis sous quoi
elle s'amuse à la voiler, est pour me faire croire qu'elle a trouvé de
l'argent.\,»

L'intrigue que manigançait Julie la Drapière était simple. Ayant tout
appris de l'aventure du vertueux La Butière, d'abord par Nicole Deleuze,
qui avait écouté à la porte de Florimond la conversation de ces
messieurs, ensuite par La Butière en personne, dont elle tira
délicatement les vers du nez, elle résolut aussitôt de tendre ses
filets. La victime devait y tomber fatalement. Julie ne doutait pas un
seul instant que d'une fille ainsi entêtée de poésie, de bergeries et
d'autres fadaises, l'Incomparable Florimond n'avait qu'à paraître pour
triompher.

Il ne s'agissait que de provoquer les occasions avec esprit et prudence.
Julie Péréal, marquise de Bannes, ne manquait ni de l'une ni de l'autre.

Outre que cette entreprise galante plaisait à sa nature perverse, ---
car cette femme froide par les sens avait un caractère assez immoral
pour ne redouter aucune comparaison, --- elle y trouverait, en cas de
réussite, une satisfaction double, voire triple. La première serait
d'assurer les plaisirs de son fils\,; la seconde de le garder plus
longtemps auprès d'elle\,; la troisième, enfin, de se réjouir dans la
honte et le désespoir de ces Primelles exécrés, qui disparaîtraient
même, puisque nécessairement un duel fatal au petit Louis-Antoine
s'ensuivrait. Après ce qu'elle faisait depuis plus de cinq ans pour
Florimond, une peccadille de plus ou de moins n'entrait pas à ses yeux
en ligne de compte. Quant au marquis, il ne manquerait pas de la
féliciter de ce nouveau et maître coup porté à ces ennemis mortels dont
il avait déjà tué le chef. Et encore, et enfin, et toujours, il fallait
que Florimond s'amusât.

Rarement, peut-être, despote ruthène ou principicule hongrois se permit
ce que la complicité aveugle et attentive de sa mère rendit facile à ce
fils gâté dans les moelles dès le berceau. Florimond, enfant, eût
demandé la lune qu'on serait parti en poste pour la lui chercher.

Dès qu'il eut seize ans, le château de Bannes devint un mauvais lieu où
les belles filles de service, soigneusement choisies hors du pays,
voisinaient avec les coupe-jarrets entretenus pour sa sûreté et ses
plaisirs. Pour un peu, la marquise eût installé chez elle, pour son
enfant bien-aimé, un sérail à la turque avec la garde obligée
d'eunuques, d'icoglans et de janissaires.

Dans ce château enchanté, où il pouvait croire commander à des génies et
à des lutins familiers, Florimond ne voyait personne résister à ses
caprices. Tout lui était accordé. La terreur ou l'argent tenait les
bouches muettes. Et, comme preuve des brutalités sauvages de cet
adolescent ainsi éduqué, on pouvait voir une fille de chambre de la
marquise, Gilette Léchanson, fleur de beauté pour toujours défigurée. Le
poing de Florimond l'avait marquée au visage, en punition de son
indocilité. Toutes les bagues armant les doigts de cette main
seigneuriale avaient porté, déchirant les lèvres, brisant deux dents. Et
cela parce qu'une nuit de fête, après une chasse enragée dans les
escaliers et les couloirs, la bande joyeuse avait rabattu la petite
chambrière dans la chambre de Florimond, où l'on sonna la curée au son
des cors. Meurtrie et sanglante, Gilette se débattait sous le poing du
maître, palpitant comme un pauvre oiseau blessé, quand la porte s'ouvrit
toute grande. Tourouvre, La Butière, Clément et les autres reculèrent
stupides de mauvaise honte et de peur, et Florimond lâcha Gilette, car,
le fouet de chasse à la main, Catherine de Lépinière marchait sur eux.
Sans un mot, sans regarder la troupe lâche et féroce qui se glissait
vers la porte, elle prit par la main la malheureuse enfant qui criait
d'épouvante dans le désordre et l'angoisse de sa chair dévêtue, et elle
rentra avec Gilette dans son appartement.

De ce qui se passa entre M\textsuperscript{lle} de Lépinière et la
marquise le lendemain nul ne parla par la suite, On sut seulement que
Julie garda le lit huit jours et que l'Incomparable Florimond partit
pour Paris. La maladie de Julie reprit de plus belle quand elle reçut
une lettre du marquis, écrite de sa meilleure encre\,: «\,N'oubliez pas,
lui mandait-il entre autres choses, qu'il y a telles folies de jeune
homme qui n'en sont point, parce qu'elles sont accomplies par le conseil
de gens qui ont la tête froide et que le feu de la jeunesse n'égare
plus. La violence sur les filles et femmes est, hors du cas de guerre,
un crime qui mène un gentilhomme en Grève et un manant au gibet. J'ai
grand'peur que mon fils soit justiciable plutôt de la corde. C'est à
vous d'y pourvoir, madame, à vous dont la faute est d'avoir créé à ce
garçon de petit esprit des facilités pour mal faire. Si une pareille
histoire me revenait aux oreilles, mes ordres sont donnés pour qu'on
embarque Florimond à destination des îles d'Amérique. Mes oncles de
Caumont y tiendront la main. Ma volonté, à laquelle je vous supplie
d'obéir, est que cette Gilette demeure dans votre immédiate domesticité
et qu'elle trouve auprès de vous tous les soins dont son malheur la rend
digne. C'est moi qui la marierai et la doterai quand il en sera temps.
Dieu qui nous juge, madame, m'inspirera en ces circonstances\,; je le
prie de vous avoir en sa sainte garde dont, tous tant que nous sommes,
avons grand besoin\ldots\,»

Julie et Nicole Deleuze durent en passer par où le voulait le marquis.
Gilette, quand elle fut remise de son effroi et de ses meurtrissures,
reprit son service, et personne ne souffla mot de l'aventure. Mais tout
le monde avait senti passer, comme le souffle de la tempête, la colère
du marquis. Il est à remarquer que, depuis cette histoire qui arriva
lorsque Florimond avait vingt-cinq ans et Catherine quatorze, il ne fut
plus question au château de violences ni de filles de chambre envoyées
en pénitence dans ces maisons de force d'où l'on ne sort guère que les
pieds en avant et la croix entre les mains. La haine de M. de Tourouvre
contre l'héritière de Lépinière datait de cette fête nocturne. Seul
parmi ces braves il avait peut-être alors son sang-froid. M. de la
Butière, le gouverneur, saoul comme une grive, avait perdu de tout la
mémoire. Quant à M. Aimeri d'Olivier, en sa qualité d'homme attaché aux
seules œuvres de l'esprit, il dormait dans sa chambre, pendant que
l'orgie et les cors grondaient, son bonnet de nuit enfoncé sur les
oreilles et jusqu'aux yeux.

S'il est permis de supposer que l'indignation et la colère rendirent
seules la marquise malade après son entretien secret avec Catherine,
l'on sait très bien que l'effet produit par la lettre du marquis fut
celui d'un seau d'eau glacée sur le dos d'un malheureux tombé de fièvre
en chaud mal. Cette fois, la marquise s'abattit assommée. La
stupéfaction paralysa d'abord toutes ses énergies. La première violence
du choc passée, elle réfléchit et s'avoua qu'elle n'y comprenait
absolument rien.

«\,Alors, c'était ça, la noblesse\,! Au prix du déshonneur, des avanies
publiques, de la réprobation générale, elle s'était glissée dans cette
caste, sans autre but, du reste, que d'y faire pénétrer son fils, et ce
fils se trouvait traité comme le commun des humains\,! Donc, elle était
marquise, son fils baron, en attendant mieux, --- et, entre nous,
Nicole, Bannes aurait bien pu lui donner un titre de comte\ldots{} ---
et, pour une petite histoire de servante sainte nitouche, voilà qu'on
parlait d'envoyer Florimond aux îles, avec les boucaniers, les
flibustiers, les tireurs de laine et les filles des cagnards de
l'Hôtel-Dieu\,!\ldots{} C'était à n'y pas croire\ldots{} Que les
bourgeois n'eussent pas le droit de houspiller, cela pouvait se
comprendre, mais les nobles\,! Mais Florimond surtout\,!\ldots{} Car,
après tout, de la noblesse elle se moquait comme de ses bigoudis. Pour
Florimond, qui se fût moqué de lui aurait eu les yeux arrachés. Son
fils, son fils aux îles\,!\ldots{} Son fils à elle, la chair de sa
chair, la fine fleur de son sang, son portrait vivant, son Florimond,
enfin\,!\ldots{} menacé, par un père barbare et qui ne le connaissait
seulement pas, du traitement le plus infamant\,!\ldots{} Non, certes\,!
Non, mille fois non\,!\ldots{} Elle défendrait Florimond avec ses
griffes, le couvrirait de son corps\,! Il ferait beau voir qu'on essayât
de le lui arracher\,!\ldots{} Si l'on entraînait Florimond vers les îles
d'Amérique, eh bien\,! elle partirait avec lui\ldots\,»

Et Nicole Deleuze, la marraine de Florimond, s'était écriée qu'elle
partirait aussi\,: «\, À tout prendre, il était dans son droit, cet
enfant. Oui, il avait droit à tout, aux femmes, à l'argent et au reste,
avec sa charmante figure, sa gracieuse ardeur, son merveilleux esprit\,!
Qu'il s'amusât, quoi de plus naturel\,? Avec ça que le marquis ne
s'était pas distrait en son temps\,!\ldots\,»

Et Julie Péréal, soupirant autant qu'Ariane put le faire jamais quand
cette princesse fut abandonnée dans son île par Bacchus, renchérit sur
les reproches de sa sœur de lait\,: «\,Il n'était pas si fier,
Charles-Armand, quand il mendiait mes bonnes grâces en cachette. Une
certaine nuit, nous dûmes le dissimuler dans une armoire\ldots\,»

--- Et certain jour dans un coffre, avait repris Nicole. T'en
souvient-il\,? Je demeurai assise sur ce coffre, attentive à ravauder un
bas, tandis que cet imbécile Royer t'examinait avec des mines de singe
mourant\,!

--- Ne me parle pas des hommes, Nicole, ce sont tous des égoïstes et
plus dénués de cœur que le propre valet du bourreau\,!

--- Excepté Florimond, naturellement, Julie.

Et les deux femmes unies par cette affection et cette haine communes
avaient déchiré le marquis à belles dents\,: «\,Puisse-t-il ne jamais
revenir\,!\ldots{} attraper un mauvais coup à la guerre\,!\ldots{} Alors
Florimond serait marquis.\,»

Pour Julie Péréal, il n'y avait pas trente-six manières de voir les
choses. Il n'y en avait que deux, la bonne et la mauvaise. Et, de même,
le monde se divisait en deux catégories de gens\,: ceux qui étaient bons
à quelque chose et ceux qui n'étaient bons à rien, c'est-à-dire ceux qui
pouvaient concourir aux plaisirs de Florimond, flatter ses goûts,
satisfaire ses passions, le pousser dans le monde, l'aider de leur
personne, de leur bourse ou de toute autre façon, et ceux qui
traversaient ses désirs ou contrariaient ses projets. Le marquis fut
donc placé dans la seconde catégorie, la catégorie des ennemis mortels.
Car Julie tenait le monde tout entier pour créé afin que Florimond y
trouvât ses aises, dût-il pour cela en exterminer les principales
populations. De ces populations et des autres elle se moquait comme les
gardes de la batterie des Suisses qu'on appelle Colin Tapon.

Le procureur Duvau, à qui elle alla conter ses peines, ne réussit pas à
lui faire entendre qu'il est des puissances avec quoi l'on est obligé de
compter. L'autorité du père, chef de la famille, est une de ces
puissances et non des moindres. L'homme de loi tenta vainement
d'expliquer à cette mère exaspérée que «\,l'autorité paternelle est un
frein nécessaire\,»\ldots{} La marquise s'était écriée\,: «\,Vous ne
m'obligerez pas à croire qu'il n'y ait rien au-dessus\,!\ldots\,» Et le
procureur avait répondu très froidement\,: «\,En effet, madame, il y a
l'autorité du roi agissant comme père de famille de sa noblesse en
présence de cas infamants. --- Taisez-vous, Duvau, ne me forcez pas à
vous traiter de sot\ldots{} Pour un peu je vous battrais\ldots{} ---
L'honneur serait grand pour moi, madame, que de passer par vos
mains\ldots\,» Déjà Julie était loin. Son mépris s'en accrut pour ce
chicaneau dont les sentiments lui apparaissaient absolument dénués de
noblesse\,: «\,Ses idées sont aussi crasseuses et plates que ses
assignations et autres grimoires.\,»

Comme c'était une femme persévérante, elle continua de s'informer, sans
toutefois raconter les exploits de Florimond. La réponse fut partout la
même. Alors elle se loua d'avoir deviné que le monde était mal fait.
Mais, obéissant à l'humaine prudence, elle laissa Florimond libre de
mener la danse des écus à Paris, puisque là, au moins, il pouvait se
divertir sans qu'on entravât son aimable gaîté. Pour suffire aux
libéraux ébats de son fils, la marquise emprunta en tous lieux, sans
pouvoir se ruiner, puisque de son bien elle n'avait pas l'absolue
disposition et que le notaire Audouin Trémolat, homme formaliste et
allié aux ennemis de l'Incomparable Florimond, refusait toute
combinaison, et toujours sous ce prétexte qu'il n'avait pas
l'assentiment, c'est-à-dire la signature, dûment certifiée valable, du
marquis absent. Et cette signature, M. Audouin Trémolat la connaissait
entre toutes, puisqu'il ne se passait pas de semaine que M. de Bannes ne
lui écrivit longuement.

Au milieu de ses ennuis, la marquise crut voir une grâce du ciel dans
cette idée qui la visita de lancer Florimond sur Marguerite de
Primelles. «\,Je crois, Dieu m'assiste, que le gaillard en tient
déjà\,!\,» Et elle commença de comploter avec Nicole Deleuze et
d'escompter la chute prochaine de la fille du défunt baron. Ces deux
femmes, qui avaient sucé le même lait, étaient créées pour s'entendre.
Marraine de Florimond, Nicole l'aimait d'un amour bestial, qui serait
allé jusqu'aux pires compromissions. Heureusement que l'âge la mettait à
l'abri et des entreprises et du soupçon. Son amour, brouillé et obscur,
était avant tout maternel. Elle aussi fût partie chercher la lune en
poste si Florimond la lui eût demandée. De ces deux femmes, vaines,
oisives, sans religion solide, et au vrai sans mœurs, tant elles
avaient, au fond, conscience que leur vie était manquée, l'Incomparable
Florimond était le dieu unique. Elles ne parlaient que de lui, ne
pensaient qu'à lui, ne se levaient et ne marchaient que pour lui. Il
habitait leurs rêves. Les soins qu'elles donnaient à leur personne,
leurs élégances d'ajustements, tout cela était pour flatter ses regards.
Un compliment de lui, et leur cœur en dansait la sarabande pendant des
jours et des nuits. Et elles eussent été bien scandalisées si on leur
eût dit que ce garçon, ainsi élevé dans leurs jupes, avait perdu à ce
contact tout ce que le cœur d'un gentilhomme possède d'honneur, de
courage moral, d'endurance et de probité. Quand M. de Montenay accusait
Florimond de cacher sous ses boucles blondes la cervelle d'un courtaud
de boutique, on ne pouvait plus justement parler. Tant il est vrai que
la caque sent toujours le hareng. Hors de la chaleur généreuse, mais
violente à l'excès, du sang qu'il hérita de son père, le jeune baron de
Chézal-Benoît tenait tout de Julie la Drapière.

--- Je crois, dit Nicole en roulant les plus beaux yeux bruns du monde
sous ses lourdes paupières, qu'il est allé causer avec M.
d'Olivier\ldots{}

«\,Il\,», ce ne pouvait être que Florimond. La marquise le comprit bien,
car elle répondit\,:

--- Pourvu qu'Aimeri le conseille avec sa finesse coutumière\,!\ldots{}
En tous cas, Nicole, pour le commerce épistolaire, chose indispensable
dans tout commerce amoureux, notre écrivain n'a point son pareil\ldots{}
Il vous tourne une lettre d'après les lois du suprême bon ton.

--- Je ne sais qui me retient d'aller écouter à la porte, répondit
Nicole. --- Elle passa la fine pointe de sa langue sur ses lèvres roses
et gourmandes et reprit\,: --- Ce qu'ils disent doit être bien
amusant\ldots{} Si vous m'en croyez, j'irai\ldots{}

--- Y penses-tu, Nicole\,?\ldots{} Non, cela ne serait ni
convenable\ldots{} ni prudent. La porte d'Aimeri donne sur l'escalier du
service, et tu risquerais d'être surprise par une fille de chambre ou
quelque valet. Sois patiente. Aimeri nous racontera tout, par le menu.
Siffle donc pour qu'on prépare mon chapeau de paille, mon parasol, et
nous nous promènerons dans le parc, autour du bassin. Vois comme le
temps est beau\ldots{}

La marquise cligna des yeux, sourit d'un air entendu et murmura sur le
ton d'une colombe qui roucoule\,:

--- Piccolomini nous accompagnera. Je veux qu'il remplace, à partir de
ce jour, André d'Archelet, dont le marquis --- que Dieu le bénisse pour
cette heureuse détermination\,! --- me débarrasse enfin. À compter
d'aujourd'hui, ton Piccolomini sera mon meneur\,; ses gages se
ressentiront de son nouvel état. Fie-t'en à moi.

Nicole rougit de plaisir, car le bel Ottavio Ranucio Piccolomini, qui se
disait bâtard d'un cousin du grand Octave d'Aragon, général au service
de l'empereur, était son galant avoué.

Ainsi la marquise Julie et sa sœur de lait Nicole Deleuze, qui tenait
auprès d'elle l'état de gouvernante, de trésorière, de confidente et de
femme à tout faire, trompaient-elles les ennuis de l'attente, cependant
qu'enfermé avec son ancien précepteur, Florimond parlait de choses
graves. Il nourrissait en M. Aimeri d'Olivier une trop absolue confiance
pour ne pas s'ouvrir à lui des confidences singulières dont l'avait
honoré la marquise sa mère touchant M\textsuperscript{lle} de Primelles.
Et Florimond ne pouvait mieux s'adresser, puisque le poète entretenu
l'assistait, de fondation, dans ses amours. Et, dans l'espèce, M.
Aimeri, secrétaire à la fois de la mère et du fils, n'ignorait rien des
projets de la marquise.

Se renversant dans le fauteuil garni de cuir doré, s'appuyant fortement
au dossier dont un des angles supportait sa perruque montée sur une
calotte noire, M. Aimeri épongea son front chauve avec un mouchoir qui
empestait le tabac et parla. D'entrée, il abonda en aperçus ingénieux
sur l'amour, les cent manières de le faire naître, et celles de s'en
débarrasser, qui sont beaucoup plus de mille. Puis il aborda les
solutions pratiques\,:

--- Moi, à votre place, je mettrais mon meilleur habit, et je
chevaucherais paisiblement\ldots{}

--- Bon pour toi, Olivier\,! «\,Paisiblement\,» est le terme qui
convient à tes allures discrètes. Pour moi, je préfère piquer mon genêt
et paraître à mon avantage dans quelqu'une de ces cabrioles qui font
valoir le cavalier\ldots{} Mais que je ne t'interrompe pas plus
longtemps\ldots{}

M. Aimeri, qui se bourrait le nez de tabac, acquiesça à cette concession
et reprit en glissant sa tabatière dans une poche\,:

--- Les airs vifs ne sont pas favorables à votre entreprise. La timide
bergère qu'il vous faut séduire ne sera point sensible, croyez-moi, à
vos exercices d'écuyer. Vous avancez donc tranquillement sans pousser
votre cheval, et, à ce moment où vous apercevez la demoiselle, vous
empruntez l'air le plus modeste que vous puissiez trouver\ldots{}

--- Le tien, Olivier, le tien\,!

--- Vous simulez une douce surprise, vous rougissez si vous en êtes
capable\ldots{}

--- Certes oui, Olivier, il me suffira pour cela de penser à tes
amours\ldots{}

--- Et vous saluez de la meilleure grâce du monde, ce qui vous est
particulièrement aisé.

--- Assurément, Aimeri, cela m'est fort aisé. Et, ensuite, je tourne un
compliment\ldots{}

--- Gardez-vous-en bien, ce serait tout perdre\,!\ldots{} Vous saluez et
vous disparaissez. Si, le lendemain, la belle se trouve au même endroit,
c'est évidemment qu'elle y sera ramenée par le désir de vous revoir.

--- Aimeri, tu as du génie à rendre jaloux tous les poètes de France, et
je m'étonne que ton auguste chef ne soit pas couronné de
lauriers\ldots{} Est-ce bien tout\,?

--- Oui, pour aujourd'hui\ldots{} Demain, je vous lirai deux chapitres
de \emph{l'Astrée}\ldots{}

Florimond, qui, rien qu'à entendre parler de ce roman à jamais fameux,
se sentait envahi par une invincible somnolence, interrompit son
précepteur\,:

--- Ne pourrais-tu, Olivier, lire ce maître livre sans moi\,?

--- Hélas\,! non, mon cher enfant\,! Si, dans cette entreprise amoureuse
d'où tout m'indique que vous sortirez vainqueur, j'étais personnellement
en cause, point ne me serait besoin de relire le livre de M. d'Urfé. De
ce roman sans pareil, ainsi qu'on l'appelle, je sais par cœur les
passages les plus remarquables. Si vous vous intéressez aux disgrâces de
Céladon, de Tircis et de la nonpareille Cléon, je puis vous réciter
leurs discours\ldots{} Ainsi, par exemple, ce portrait du volage
Hylas\ldots{}

--- Non, mon cher Olivier, ne te mets pas en frais\,: la peine serait
perdue. Dis-moi, pourtant, ce qui te paraît digne d'être retenu dans ce
livre où tu m'assures que se trouvent tous les artifices propres à faire
réussir mes projets. Pour moi, je me désole de voir revenir à chaque
page l'éloge d'une insipide vie champêtre et de ce bonheur parfait qui
consiste à tresser des corbeilles, à presser des fromages, tout en
devisant sur des subtilités misérables que peuvent seuls éclaircir les
druides, prêtres de Teutatès\ldots{} Quelles fadaises\,!

--- Que voulez-vous\,? c'est la mode\ldots{} Le monde est ainsi
construit que, lorsqu'un sot de mérite réussit à attirer quelques autres
sots de son acabit sur ses pas, le reste suit, et aussi bientôt les
sages, tant les hommes ont un tempérament moutonnier\,! Est-ce ma faute
si la cervelle des dames s'est laissé tourner vers les choses champêtres
au point que tout ce qui n'a pas une physionomie pastorale les ennuie
maintenant\,? Or, mon cher Florimond, ce sont les dames qui dirigent,
soutiennent et portent les auteurs à la gloire. Il faut leur plaire, ou
tout au moins y tâcher, sans quoi l'on n'arrive à rien. Quand elles ont
associé le mot «\,ennuyeux\,» à votre nom, auriez-vous tout le génie
d'Homère et de M. de Malherbe, vous êtes condamné à une obscurité sans
remède.

--- Aimeri, tu t'élèves à des hauteurs telles que mon faible esprit te
suit avec peine. Continue cependant, car, dans les choses de l'amour,
ton esprit, alerte et fécond en ressources, excelle à diriger mes pas
chancelants\ldots{}

--- Riez tant que vous voudrez, c'est de votre âge. Il n'en est pas
moins vrai, noble Florimond, que l'ennui est avant tout question
d'usage. Tant qu'une femme aime un homme, elle ne le trouve jamais
ennuyeux. De même pour les livres\,: c'est la mode qui les sacre
amusants ou ennuyeux, sublimes ou ridicules. Que \emph{l'Astrée} soit un
livre peu récréatif, j'en demeure d'accord avec vous. Mais, pour rien au
monde, je ne tirerais de l'écritoire ce jugement. Il restera inédit.
Car, si je l'écrivais, ce jugement, si je médisais en quoi que ce soit
de cette \emph{Astrée}, bréviaire adopté par la belle société qui file
le parfait amour, bréviaire de la sagesse de ce temps qui revêt ses
molles passions de l'attirail bucolique, j'aurais tôt fait de perdre la
faveur dont les gens du bel air me veulent bien honorer, et
vous-même\ldots{}

--- Non, mon cher Olivier, mon amitié te sera toujours fidèle. Mais je
crois avoir compris. Tu me conseilles de jouer de \emph{l'Astrée} avec
cette sotte mais charmante fille que le roman du sieur d'Urfé a rendue
folle ou peu s'en faut.

M. Aimeri d'Olivier opina de son bonnet, c'est-à-dire de sa calotte
porte-perruque dont il s'était recoiffé. À ce moment, on gratta à la
porte.

--- Eh là, qu'y a-t-il\ldots{} Et me dérangera-t-on sans cesse\,? cria
Florimond. Entrez\,! Entreras-tu, bélître, ou laisseras-tu ta mine
couturée et réparée de taffetas entre les battants de la porte, comme
enseigne de la maison d'un barbier\,?

Ainsi invité à produire sa personne tout entière, M. Clément Malompret
s'avança discrètement dans la pièce, salua à peine le poète, tant la
domesticité se méprise soi-même, et exposa son affaire\,:

--- C'est, monsieur, votre dévoué Cottebleue qui désirerait vous
entretenir, avec votre gracieuse permission, d'une histoire de
braconnage.

Et Cottebleue, qui suivait Clément, exhiba des collets de crin. Il les
avait trouvés en contrebas du vieux mur du parc, et, avec, cinq lapins
étranglés, «\,dont un lièvre\,».

--- Tenez, les voilà\,!

Cottebleue avait tiré de son balandran râpé les preuves du délit et les
présentait par les oreilles.

--- Marin seul a pu faire le coup\ldots{} Mais patience\,! Le soleil ne
se couchera pas avant que je vous aie amené le drôle. Puissé-je perdre
mon nom s'il ne couche pas ce soir dans une cave du château\ldots{}
C'est Marin qui a posé ces collets, j'en suis sûr, et Clément ne me
contredira pas.

Loin de contredire Cottebleue, le valet de chambre déclara que, tant que
l'on n'aurait pas purgé le pays de ce Marin, les honnêtes gens ne
dormiraient pas tranquilles, les filles non plus, car il était au su de
tous que le fils du vieux berger des Primelles dépassait en noirceur les
coquins les plus réputés. M. Clément parla ainsi en toute bonne foi, car
il ne dormait plus tranquille depuis que Marin lui avait promis de
rompre ses précieux os si ledit Clément Malompret s'avisait encore de
tourner autour de Francine, la jeune sœur dudit Marin. Toutefois M.
Clément ne se crut pas obligé de relater ces particularités dans son
réquisitoire.

Le vieux Cottebleue, grand, sec, borgne et boiteux, approuvait en
hochant le menton, ce qui faisait onduler sa longue barbiche grise.
Cette touffe de poils capricante lui donnait l'air d'un bouc, et son œil
d'un roux jaunâtre augmentait la ressemblance. Avec son épée antique,
son bâton blanc de porteur d'exploits, son brassard, ses guêtres
crottées, son balandran couleur de terre et son chapeau plat, crasseux,
d'un gris pisseux, que rehaussait une plume rouge, déteinte et cassée,
Andoche Cottebleue imitait ces épouvantails qui se dressent dans les
champs en défiant tour à tour les ardeurs brûlantes du soleil et les
caresses orageuses de la pluie.

C'était un ancien valet d'armée que deux blessures, reçues dans une
bagarre où il s'occupait vertueusement à piller les bagages des maîtres
pendant que ceux-ci avaient l'ennemi sur les bras, obligèrent de
renoncer au service. À ce métier de valet et de voleur il avait gagné
assez de bien pour acheter à Lunery une charge de porteur d'exploits.
Son zèle à servir les haines du marquis de Bannes lui avait valu comme
récompense le don, à bail, de quelques carrés de terre. Aussitôt il les
avait ensemencés avec des plantes fourragères propres à attirer le
gibier, dont il trouvait son avantage à pratiquer le plus actif et le
moins avoué des commerces. La concurrence déloyale qu'il avait à
supporter l'exaspérait donc contre Marin, puisque, à son idée, le seul
braconnage légitime était le sien, à lui Cottebleue, porteur d'exploits,
muni du bâton blanc et du brassard, homme considéré de tous. Il
détestait encore Marin, comme d'ailleurs tous les gens de Primelles,
maîtres, tenanciers et valets, pour d'autres raisons. S'étant un jour
permis, sous couleur de remettre une assignation en mains propres, de
pénétrer dans le manoir de Primelles, Cottebleue s'en était vu
reconduire à beaux coups de canne par le vieux baron de Mordicourt,
oncle de la baronne, et cela d'une façon tout à la fois si militaire, si
galante et si secrète que le porteur d'exploits de Lunery en avait dû
garder le lit pendant trois semaines.

Comme si sa malchance l'eût ainsi voulu, aucun témoin n'avait vu cette
bastonnade, entendu les cris de Cottebleue, qui pourtant piaillait à
rendre jaloux les geais des environs en invoquant en vain l'autorité
tutélaire des justes lois du Berry, M. Blaise Le Bouteiller, baron de
Mordicourt, ayant eu l'indélicatesse de donner cette dégelée à
Jean-Andoche Cottebleue dans l'avant-cour du château, à une heure du
jour où personne ne se trouvait en ce lieu. Et s'il est une chose triste
à dire, et qui prouve le mauvais esprit du populaire, personne ne
plaignit Cottebleue. Et même quelques méchantes gens trouvèrent là
prétexte à se réjouir. Cottebleue endossa donc ses coups de bâton en
remettant au ciel et à son habileté dans l'espionnage, qui était grande,
l'exemplaire et inexorable vengeance de cet attentat du vieux baron.

Pour Florimond, le nouveau crime de Marin Labrande couronnait une série
de méfaits qui demandait une punition exemplaire. M. Aimeri d'Olivier se
vit donc rabroué de la bonne façon quand il tenta de pallier le délit,
et surtout de demander un supplément d'informations avant que l'on se
décidât à sévir. Sèchement, le jeune seigneur interrompit son poète
conseiller\,:

--- Eh\,! qu'avons-nous besoin de preuves\,? En vérité, Aimeri, je
t'admire. Personne n'a vu Marin pour ces collets, dis-tu\,? La belle
raison\,! Mais il en a posé vingt fois, cent fois, sous le nez de mes
gardes. Ils n'ont rien osé faire, parce que ce sont des poltrons et
qu'ils redoutent plus les coups de ce Marin que ma colère\ldots{} Tandis
que mon brave Cottebleue\,!\ldots{} Au moins, j'ai un homme sous la
main\,! Va, Cottebleue, mon brave, cueille-moi ce drôle vivement et me
l'amène\,!\ldots{} Quant aux lapins et au lièvre preuve du délit, je les
saisis entre tes mains, comme juge\ldots{}

La figure de Cottebleue s'allongea quand il entendit ces mots, et sa
barbiche en descendit de plusieurs pouces sur son balandran graisseux.
Mais bientôt ses traits reprirent leur sérénité, car Florimond
continua\,:

--- Et je te les donne comme seigneur. Quand on évoquera la cause, tu
présenteras les peaux ainsi que les collets. J'ai dit.

Cottebleue se retira avec ses cinq lapins «\,dont un lièvre\,», et les
collets de Louis-Antoine, qu'il alla tendre aussitôt dans les coulées de
Bannes, à toucher sa terre de Lunery, et Florimond demeura seul avec M.
Aimeri d'Olivier.

--- Je ne sais, fit celui-ci d'un air docte tout en bourrant son nez de
tabac, je ne sais si mes conseils valent qu'on les écoute, mais, si vous
m'en demandiez un, je sais bien quel est celui que je vous donnerais.

--- Ah oui\,! Toujours ton Astrée avec le berger Céladon\ldots{} Va, mon
cher Olivier, je suis tout oreilles.

--- Cela reviendra en son temps\ldots{} Moi, à votre place, je ne
molesterais pas les gens de M\textsuperscript{me} de Primelles à cette
heure où vous entreprenez la conquête de mademoiselle sa fille. Un
point, c'est tout.

Et, regardant avec intérêt deux mouches qui se faisaient des politesses
sur un livre ouvert, M. Aimeri s'offrit une prise.

Florimond comprit le raisonnement. Ami des solutions faciles, il saisit
ce que l'avis de son précepteur avait de pratique. Il dénonça donc son
ferme propos d'arrêter toutes les poursuites contre Marin.

--- Gardez-vous en bien, dit alors M. Aimeri sans cesser d'observer ses
mouches.

--- Mais, alors, comment diable veux-tu que je m'en tire\,?

Et Florimond, qui maintenant n'y comprenait plus rien, tirailla son
pinceau de barbe d'un air anxieux. Jouissant de son avantage, M. Aimeri
se barbouilla encore le nez de tabac, cessa de s'intéresser aux mouches
et daigna rassurer son élève\,:

--- Vous vous en tirerez de la manière la plus simple. Laissez
Cottebleue appréhender Marin, mais recommandez-lui, sur toutes choses,
de ne pas le maltraiter. Et quand vous tiendrez Marin vous aurez un
otage\ldots{} Un otage de première qualité, monsieur\,! Car vous pourrez
en jouer auprès de M\textsuperscript{lle} de Primelles, vous donner tous
les bénéfices de la générosité, car les dames aiment tout ce qui revêt
les apparences héroïques\ldots{} Sans compter qu'en relâchant ce drôle
pour les beaux yeux de la demoiselle, cet arrangement, naturellement
secret, crée entre vous une sorte de complicité qui engage votre
bergère\ldots{} Et encore vous vous rendez sympathique aux gens de
Primelles, qui fermeront les yeux sur vos allées et venues\ldots{}
Est-ce clair\,?

--- Aimeri, ton génie dépasse de beaucoup la portée de mon jugement.
Quoiqu'il paraisse prouvé que nous autres possédons par droit de
naissance les dons les plus précieux de l'esprit sans compter les
autres, ta subtilité m'offre chaque jour de nouveaux sujets
d'étonnement. Me daigneras-tu expliquer\,?

--- Pas aujourd'hui, mon enfant. Partez sans tarder à la découverte.
C'est l'heure où M\textsuperscript{lle} de Primelles, jalouse d'égaler
Célidée et autres bergères fameuses dans l'histoire, se dirige vers le
carrefour de Mercure avec son mouton, son livre, sa confidente Colbert
et sa rêveuse mélancolie pastorale. Cette mélancolie ne peut réussir à
faire naître un autre Thamire. Apparaissez donc ainsi qu'une incarnation
de cet admirable berger, et saluez-la comme il convient\ldots{} Allez la
saluer, vous dis-je, et revenez, après votre promenade, me raconter vos
succès.

Florimond partit sans en demander davantage. L'aventure lui plaisait par
son caractère chimérique et flattait ses mauvaises passions.

\hypertarget{chapitre-v}{%
\chapter{CHAPITRE V}\label{chapitre-v}}

Florimond avançait dans la campagne au pas dansant de son barbe et se
haussait de temps à autre sur ses étriers pour voir si personne ne
paraissait sur le petit chemin où M\textsuperscript{lle} Marguerite de
Primelles avait coutume de se promener sous ses habits de bergère. Ce
chemin, bien garni d'arbres, partait de la ferme de Lunerette pour
aboutir au bois des Usages, où il se bifurquait. Sa branche droite se
dirigeait en ligne droite vers la paroisse, tandis que la gauche
n'arrêtait pas de serpenter dans les taillis jusqu'au delà de Toux, de
telle sorte que ce chemin appartenait dans toute sa longueur au domaine
de Primelles. Par un accord très ancien, les gens de Bannes
l'empruntaient quand ils devaient se rendre de Primelles même aux
Écobeilles, sans quoi il leur eût fallu contourner la pointe du domaine
de la baronne, en prenant par Mareuil, puis tourner à gauche et passer
par Maleray et l'Ecoron.

Ce ne fut pas M\textsuperscript{lle} Marguerite et son mouton familier
qu'aperçut Florimond, mais deux cavaliers qui lui semblèrent des
maîtres, car cinq ou six hommes à cheval les suivaient, ayant tout l'air
de pages et de laquais. Par désir de garder l'entière liberté de ses
actions, et surtout par crainte d'attirer l'attention de ses gens,
Florimond chevauchait seul. À peine arrivé à Lunerette, il avait renvoyé
M. de la Butière et Clément, celui-ci sous couleur de porter un message
à sa mère, celui-là avec de l'argent pour acheter une de ces fameuses
bouteilles de liqueur ménagère que seule savait fabriquer la mère
Bouhant de Maleray. M. de la Butière, enchanté de cette occasion qui
s'offrait de déguster, sans bourse délier, quelqu'un des produits
distillés par la célèbre veuve, avait piqué des deux sur l'Échalusse,
laissant son jeune seigneur et ami continuer sa promenade solitaire.

Les deux cavaliers, qui venaient par un sentier de traverse du côté des
Avant-Bois, furent bientôt à une assez courte distance pour que
Florimond les reconnût\,:

--- Ah çà, pensa-t-il, est-ce une gageure, et les amoureux de ma
charmante bergère se sont-ils donné rendez-vous dans ce chemin\,? Voici
Mauny d'Anrieux, que la renommée nous dit très épris de la demoiselle et
qui ne se déclare point par la peur unique qui le tient de sa
gouvernante Marion. Et, à sa droite, cette longue figure de
carême-prenant est le rigide moraliste Ludovic de Montenay, hier nommé
tuteur de l'amazone Catherine. Celui-là, si l'on en croit la rumeur
publique, est le type des amoureux platoniques chantés dans les
pastorales. Sa réserve est telle qu'il ne se déclare qu'aux arbres des
forêts et grave le nom de sa belle ainsi que le sien sur leur écorce, à
l'exemple des chevaliers errants. Si la touchante Marguerite aime autant
\emph{l'Astrée} que cet animal d'Aimeri me le voudrait laisser croire,
son amour ne doit pas s'égarer ailleurs que sur ce Montenay\ldots{} Mais
nous sommes à deux, sinon à trois de jeu, ma divine, et, quoique je vous
aie à peine entrevue jusqu'ici, le plaisir que je ressentirais en damant
le pion à ces deux hobereaux me décide. J'entends gagner la partie\,:
«\,Serviteur, messieurs\,!\,»

Les trois hommes se saluèrent poliment, et, Florimond ayant cru habile
de demander à MM. de Montenay et de Mauny la permission de se joindre à
eux, ils la lui donnèrent et le remercièrent d'un honneur dont ils
sentaient tout le prix.

--- Je m'amusais, dit Florimond, qui se complaisait à accumuler les
mensonges, à travailler ce cheval, quand mon laquais se laissa emmener
par le sien du côté de Lunery. Alors j'ai continué de marcher seul, sans
m'attendre à cette heureuse fortune de vous avoir pour compagnons.

M. de Montenay ne répondit rien. Mais M. de Mauny d'Anrieux, qui était
d'un caractère ouvert et facile, complimenta le jeune baron sur la
beauté de sa monture. Il avait trouvé à peu près la pareille, naguère, à
une foire d'Issoudun. Les trois cents écus qu'on en demandait étaient
malheureusement un trop gros prix pour sa bourse\,:

--- Tandis que pour vous\ldots{} à la bonne heure\,! On voit bien que
votre mère n'y regarde pas quand il s'agit de vous retenir auprès
d'elle. Car je gage que ce cheval fut acheté pour elle à Bourges, chez
Nicolas Lallemand, aussi vrai qu'il vient de Flandre et a été dressé par
les Espagnols. Un peu piaffeur, peut-être, et faible des reins\ldots{}
Croyez-moi, il le faut seller plus en avant, car ces bêtes sont toujours
portées sur leurs épaules\ldots{} Belle bête, en tout cas. Votre mère y
dut mettre le prix\,!

Et M. Lucien-Timoléon-Hannibal de Mauny d'Anrieux, sieur de Torchefelon,
Pradeaux, Naboullet, baron de Château-Chevrier, prit M. de Montenay à
témoin\,: «\,Ce barbe valait quatre cents écus, au bas mot, rien n'étant
si rare qu'une pareille pureté de robe chez un cheval ferrant.\,»

M. de Montenay n'y contredit point. Et il demanda à Florimond s'il
emmènerait ce cheval à Paris quand il aurait sa compagnie aux gardes.

--- Oh\,! la chose n'est pas encore faite\,! répondit Florimond. La
volonté de mon père est que j'aie une compagnie. Encore faut-il la
payer, et je ne sais quand\ldots{}

--- Vous le saurez bientôt, fit en souriant M. de Montenay, car le
marquis votre père a chargé\ldots{} des amis à moi de suivre l'affaire.
On espère que M. le cardinal consentira\ldots{}

M. de Mauny l'interrompit avec un sans-gêne qui lui était habituel\,:

--- Il y consentira, monsieur, j'en jure par mon chapeau\,! Si vous
n'avez pas encore vécu au service, celui des gardes vous sera un
apprentissage sans douleur. Le seul ennui, c'est que l'on doit résider à
Paris quand on se trouve de quartier\ldots{} Ah\,! Montenay, vous n'êtes
pas souvent à votre régiment, vous\,!\ldots{} Votre vie coule ses jours
tranquilles et exempts de soucis, soit à Bourges, soit à Lavergne,
cependant que le régiment de Leuville tout entier regrette et vos
lumières et vos austères principes dans ce qui regarde le commandement.

--- Et vous, Mauny, répondit M. de Montenay, votre vie se passe sous la
discipline de Marion, qui vous mène à la croate.

--- Il est vrai que ma gouvernante s'entend à gouverner, Montenay. Mais
c'est surtout la vie des champs qui plaît à mon caractère. Depuis que
j'ai fait mes adieux à la guerre et à ce métier de soldat que j'exerçai
sous le grand roi de Suède\ldots{}

Et M. de Mauny salua. M. de Montenay leva son chapeau encore plus haut.
Florimond, sans comprendre que l'usage parmi les gens de guerre était
d'honorer ainsi la mémoire de celui en qui ils voyaient leur maître à
tous, donna au hasard un coup de chapeau, et M. de Mauny d'Anrieux
continua\,:

--- Rien ne m'est plus doux que ces loisirs champêtres dont je goûte
tout le prix dans ma petite maison de Magny. Quand y viendrez-vous,
Montenay, chasser la perdrix\,?\ldots{} Pour vous, monsieur, qui comptez
parmi les riches de la terre, vous trouveriez sans doute mon hospitalité
petite\ldots{} Cependant, si le cœur vous en dit\ldots{}

Florimond n'eut pas le temps de remercier. Sur le talus du chemin qui
s'abaissait doucement sous son épais tapis de gazon, se dressa
M\textsuperscript{lle} Marguerite en personne, avec son mouton, sa
suivante et son \emph{Astrée}. Dérangée par l'arrivée des cavaliers,
elle s'était levée brusquement de la souche moussue qui lui servait de
siège, et le livre, ayant glissé de ses genoux sur la pente, s'étalait
maintenant tout ouvert sous les sabots des chevaux. Vivement Florimond
mit pied à terre, ramassa \emph{l'Astrée}, et, le chapeau à la main,
grimpa jusqu'au sommet du talus. Là, il présenta de la meilleure grâce
du monde le livre à la jeune fille, s'inclina devant elle, redescendit
et fut en selle avant que ses deux compagnons, surpris de cette
gracieuse vivacité, eussent remarqué les dangers courus par
\emph{l'Astrée}.

MM. de Montenay et de Mauny, gênés peut-être par la présence de
Florimond, ne s'arrêtèrent pas pour causer avec M\textsuperscript{lle}
de Primelles. À peine mêlèrent-ils à leur salut quelques courtoisies
banales. Ils rejoignirent Florimond, qui avait pris l'avance sans se
retourner, et bientôt les trois cavaliers disparurent à un tournant du
chemin.

--- Quel est donc, Colbert, ce jeune homme qui m'a si galamment rapporté
ma pauvre \emph{Astrée}, miraculeusement échappée aux fers de ces
coursiers farouches\,?\ldots{} demanda la jeune fille à sa suivante que
la modestie de sa condition retenait assise sur une pierre à quelques
pas de sa maîtresse.

--- Hélas\,! mademoiselle, répondit Françoise Colbert, c'est M.
Florimond, le baron de Chézal-Benoît\ldots{} le fils de\ldots{} le fils
du\ldots{}

--- En finiras-tu\,?

--- Mon Dieu, mademoiselle, c'est que\ldots{} Enfin, puisque vous tenez
à le savoir, c'est le fils du marquis de Bannes\,!

--- Divine bonté, est-ce possible\,?\ldots{} dit M\textsuperscript{lle}
Marguerite avec beaucoup de sang-froid. Je croyais que ce jeune homme
avait la mine des bêtes sauvages\,! Vois, Colbert, comme on nous en
donne à garder. Il m'a semblé, au contraire, que ses manières étaient
exquises et qu'il saluait avec autant de grâce qu'il maniait bien son
cheval.

--- Certainement, mademoiselle. Vous avez trop bon goût pour vous
tromper.

--- Ne trouves-tu pas, Colbert, que c'est sous de pareils traits qu'on
aime à se figurer l'aimable berger Céladon\,?

--- Lui-même, mademoiselle, lui-même\,! C'est tout à fait ainsi que je
pensais.

Ainsi approuvée, M\textsuperscript{lle} Marguerite de Primelles rentra
chez elle avec sa houlette, son mouton et sa suivante Colbert, qui
portait \emph{l'Astrée} sous son bras. La nuit, elle rêva de Florimond
et se vit devisant avec lui sur les choses de l'amour supérieur, aux
bords ombragés du Lignon. Renonçant à son appareil cavalier, cet unique
Florimond endossait le costume des bergers, dont il avait le cœur
simple, l'âme généreuse, la subtilité, l'éloquence, la délicatesse et la
capacité d'aimer.

Tout d'abord elle l'avait distingué entre ses deux compagnons par la
noblesse et l'élégance aisée de ses manières. Son empressement à la
servir était la conséquence même de toutes les vertus dont elle lui
faisait crédit. Comme il se montrait supérieur, et par la douce fierté
de son regard et par l'élégance de sa personne, à ces deux hommes, dont
la vue ne réussissait jamais qu'à augmenter son ennui\,! M. de Montenay,
avec sa longue figure froidement attentive, ses yeux interrogateurs,
tout à la fois vifs et voilés, lui déplaisait au delà du possible. Elle
se trouvait incapable de lui pardonner sa gravité un peu morne, la
mesure de son geste et son air toujours, quoi qu'il en fit, protecteur.
Car M\textsuperscript{lle} Marguerite ne désirait pas qu'on la
protégeât, mais bien qu'on se soumît à elle, qu'on lui confiât
l'éducation de son âme, qu'on la prit pour guide et pour souveraine
maîtresse de ses pensées, qu'on méritât son amour par la belle façon
dont on se serait exprimé pour lui avouer qu'on se mourait de tendresse.

M\textsuperscript{lle} de Primelles avait le courage et l'énergie en
petite estime quand ces qualités ne s'abaissaient pas devant elle pour
se muer en un renoncement complet de la personne et se convertir en une
éloquente et diserte adoration. Ce qui importait en amour, c'était plus
ses causes que l'amour lui-même. Capitaine de gens de pied, riche,
indifférent quoique haut à la main, et de caractère résolument
philosophique, habitué au commandement comme à l'obéissance passive qui
est la marque du bon soldat, M. de Montenay n'avait rien qui pût lui
plaire.

Quant à M. de Mauny d'Anrieux, son physique énergique et agréable, ses
yeux noirs, sa barbe rousse, n'empêchaient pas Marguerite de l'avoir en
exécration. Sans doute il était généreux et honnête, --- mais qui ne
l'est pas\,? --- chevaleresque, mais chevaleresque à l'exemple de M. de
Montenay et de tous ces héros, taillés sur le patron d'Hercule, qui
croient ravir leur belle en pourfendant les ennemis.
M\textsuperscript{lle} Marguerite méprisait les combats, pour ce que la
guerre comporte de vulgarité brutale. M. de Mauny, à faire la guerre, en
avait gardé quelque chose de hardi et de décidé. Seule son indolence
aurait pu trouver grâce aux yeux prévenus de la jeune fille si cette
indolence ne se fût accompagnée d'une certaine joyeuseté foncière qui
lui semblait encore plus vulgaire que la brutalité du soldat et lui
apparaissait d'autant plus odieuse que cette joyeuseté allait de pair
avec la vivacité de l'esprit, un amour indéniable du beau et une
connaissance également manifeste des hommes, des livres et des choses.

Que M. de Mauny d'Anrieux fût bien disant, cela était hors de doute.
Alors pourquoi n'usait-il de cet avantage que pour se moquer des
sentiments nobles et élevés, les seuls qui exaltent l'âme au-dessus des
médiocres tracas de la vie\,?\ldots{} Tandis que Florimond\,! Il n'avait
point parlé, mais son regard mouillé disait un cœur de choix, capable
d'une soumission à nulle autre pareille, un cœur sensible, un vrai cœur
de berger\,!\ldots{}

Aussi les jours qui suivirent cette rencontre se passèrent-ils pour
Marguerite de Primelles à s'entretenir avec sa suivante Françoise
Colbert des vertus sans secondes du jeune baron de Chézal-Benoît. Car,
pour ajouter à ces vertus, Florimond était jeune, tandis que MM. de
Montenay et de Mauny étaient vieux, au regard de Marguerite, dont l'âge
atteignait juste dix-sept ans\,! Le premier en avait trente et le second
trente-trois, c'est-à-dire près du double de son âge. Et, s'amusant
ainsi à supputer les années, M\textsuperscript{lle} de Primelles
calculait que M. de Mauny eût pu à la rigueur être son père. Et elle ne
s'arrêtait pas un seul instant sur cette idée que Florimond était le
fils de l'homme qui avait tué son père, à elle, le baron de Primelles,
--- l'amour pastoral présentant cette particularité que ses actions
n'ont rien à voir avec les faits vulgaires de la vie.

Donc ces messieurs étaient vieux, et Florimond, Colbert le savait très
bien, était dans sa vingt et unième année, et c'est là le véritable
printemps des bergers. La fille de chambre était trop fine mouche pour
ne pas saisir l'importance des profits que pourrait lui valoir une
passion aussi violente. Le principal pour elle était que Florimond la
partageât. Elle se promit d'y aider de tout son pouvoir et de pratiquer
avec adresse et prudence M. Clément Malompret, qui, la veille encore,
lui avait offert sa protection.

Dès lors la fille de chambre de Marguerite, à tout passage que lui
lisait sa maîtresse, --- car c'est avec \emph{l'Astrée} que celle-ci
tuait le temps, --- Colbert s'écriait avec une admiration trop vive pour
ne pas être sincère\,: «\,Ah\,! mademoiselle, quels galants propos\,! Je
ne suis pas sûre que M. Florimond ne parlerait pas mieux\,! Mais, par
exemple, je gage qu'il parlerait aussi bien\,!\,» Ou bien elle insinuait
que des cheveux de M. Florimond le lustre et le soyeux valaient presque
ceux de mademoiselle\,! Et souvent elle répétait à mi-voix, d'un ton
pénétré\,: «\,Pensez donc, mademoiselle, il paraît que dans les réunions
des grandes dames, à Paris, on ne l'appelle que l'Incomparable
Florimond\,!\,»

--- «\,Hélas\,! ma mie. --- répondait Marguerite en levant au ciel les
plus beaux yeux du monde, --- plaise à Dieu que tant de succès mérités
n'amènent point ce charmant jeune homme à se gâter comme le vilain
Hylas, dont la légèreté et la fatuité nous sont données comme le
terrible exemple de ces fruits empoisonnés que produit
l'égoïsme\,!\ldots{} Donne-moi \emph{l'Astrée}, Colbert, et écoute\,! Je
veux te lire le portrait de cet être volage et odieux.\,»

Et M\textsuperscript{lle} de Primelles se mettait à lire, après avoir
posé sa houlette à terre. Son mouton enrubanné tondait l'herbe fraîche à
ses pieds, et la chambrière bâillait discrètement en ourlant une
collerette ou en tressant des couronnes de fleurs afin de ne pas
s'endormir à la longue sous les flots de l'éloquence qui la
submergeaient.

En vérité, la lecture des pastorales et en particulier celle de
\emph{l'Astrée} avait tourné la tête de cette jeune fille, victime à la
fois de la solitude des champs et de l'indifférence des siens. Son
éducation n'avait pas été cependant négligée. Élevée d'abord par sa
mère, puis mise, peu avant la mort de son père, dans un couvent de
Bourges, elle y avait appris tout ce qui s'enseigne aux filles de sa
condition, excepté ce que peut seule leur enseigner une mère. Le meurtre
de son mari avait à tout jamais ébranlé l'esprit de la baronne de
Primelles. À dater de ce jour funeste où l'on rapporta le baron Louis
tombé sous l'épée du marquis de Bannes, sa veuve se désintéressa en
quelque sorte de sa fille. Les Ursulines de Bourges ne purent réussir à
donner à l'enfant qu'on leur avait confiée ces bonnes notions pratiques
grâce auxquelles leurs élèves entraient dans le monde maîtresses de
maison accomplies. La lingerie, la couture, l'économie domestique ne
l'intéressèrent point.

Par contre, elle s'abandonna délicieusement aux idées qui couraient
parmi les filles de la noblesse et les trouva toutes fixées dans les
pastorales et les romans d'amour qu'on se passait en cachette, quand on
ne les étudiait pas ostensiblement comme le vrai code du bon ton.
\emph{L'Astrée}, particulièrement, enchanta cette jeune fille abandonnée
à qui la mort violente de son père fut une cause de haïr la force et les
armes. Cette conception ingénieusement et artificiellement idéalisée de
la vie suffit à fausser son esprit, qui, effrayé par la solitude que
crée la dévotion, se lança dans la solitude encore plus aride des
divagations chimériques. M\textsuperscript{lle} de Primelles prit pour
bon argent les déclamations des bergères et des bergers contre tout ce
qu'ils appelaient passion sauvage ou simplement action. Quant à la
religion, elle se persuada que ce n'était qu'une manière de se
représenter agréablement les attributs, la grâce et l'amour divins. Elle
admira dans François de Sales l'élégance de l'expression et feuilleta, à
ses heures, le \emph{Traité de l'Amour de Dieu} entre deux chapitres de
\emph{l'Astrée}, tout comme les graveurs en pierres fines --- ainsi que
le dit l'évêque de Genève lui-même --- ont au-dessus de leur tour une
émeraude pour y reposer leurs yeux fatigués par le travail minutieux des
pierres qu'ils entaillent.

Quand elle revint chez sa mère, Marguerite de Primelles avait tout juste
quinze ans. Depuis quatre années qu'elle vivait séparée des siens, elle
s'était habituée à se suffire moralement à soi-même, de telle sorte que
sa mère et son frère lui apparaissaient comme des étrangers. Son oncle,
M. Le Bouteiller, baron de Mordicourt, dont la nature rude et pourtant
bienveillante ignorait toutes ces délicatesses qui seules étaient
capables d'attacher cette fille réservée et sensible, ne réussit pas à
gagner sa confiance. Indifférente à tout en apparence, Marguerite
s'étudia à dissimuler la chaleur de cœur qui, dans ce milieu aussi
simple que sévère et qu'elle trouvait abominablement vulgaire, ne
trouvait aucun objet propre à l'entretenir. Elle se fit muette, sourde
et aveugle, attendant d'événements improbables, dont son imagination
exaspérée se flattait de provoquer la naissance, la réalisation de ses
vœux non moins ardents qu'incertains. Les livres qu'elle avait rapportés
du couvent de Bourges devinrent son unique société.

Pour ne point trop attirer l'attention sur ce que sa vie présentait de
singulier, elle pratiqua l'humaine prudence. Elle se composa habilement
une attitude suffisamment puérile et folâtre pour qu'on pût imputer à la
bizarrerie de sa jeunesse les extravagances qui, autrement comprises,
auraient pu devenir pour elle une source de journaliers ennuis. Ses
manies, sa défroque pastorale, son agneau enrubanné, furent pris pour
signes de ses goûts champêtres. Et, alors qu'elle se moquait
supérieurement du gouvernement des basses-cours, des bergeries, des
laiteries et des aumailles, chacun demeurait convaincu que la jeune
fille avait l'œil sur tout le ménage campagnard et s'instruisait dans la
vie rustique, où sa pauvreté la condamnait à se confiner, en ne s'en
rapportant qu'à ses yeux.

Passant ainsi pour une fermière modèle, M\textsuperscript{lle} de
Primelles put parcourir librement, du matin au soir et dans tous les
sens, le domaine et s'arrêter où bon lui semblait avec sa fille de
chambre Colbert. Celle-ci, dans sa tête de rusée commère, roulait des
projets non moins vagues et chimériques que ceux de sa maîtresse.
Toutefois, sa nature plus grossière s'orientait vers le meilleur parti à
tirer de la vie. Avec une personne aussi capricieuse et mélancolique que
M\textsuperscript{lle} Marguerite, on pouvait s'attendre à tout en
dehors du raisonnable. Et le hasard est si grand qu'il amène des
résultats inespérés pourvu que l'on soit capable de l'administrer en
faisant naître des aventures. Car où les aventures manquent l'on ne
saurait naturellement en profiter.

C'est pourquoi Françoise Colbert, qui n'était pas moins astucieuse et
intrigante que jolie et parfaitement tournée, augura bien de la
rencontre. Le magnifique et décrié Florimond était riche, cela ne
faisait pas question. M\textsuperscript{lle} Marguerite en paraissait
fortement éprise, cela était également prouvé. Donc Françoise devait
tirer le plus possible de l'affaire. Elle avait remarqué que, depuis ce
jour où Florimond avait salué Marguerite dans le chemin de Lunerette, la
jeune fille dirigeait invariablement ses promenades de ce côté. Il
fallait donc y ramener Florimond.

Mais, par une malchance déplorable, M. Clément Malompret ne se montrait
plus dans le pays, son maître non plus, et une semaine venait de
s'écouler. Françoise Colbert, tout en se méfiant de Marin et de tous les
Labrande, grands et petits, dont elle sentait la surveillance occulte et
hostile attachée à ses pas, alla aux renseignements chez M. de Montenay,
en se substituant complaisamment à Margot Larçonnière chargée par le
baron de Mordicourt d'une lettre à porter.

Margot Larçonnière, native de Saint-Christophe, remplissait auprès de la
baronne de Primelles les fonctions de fille de chambre. Onques créature
plus effarée et plaintive ne brossa les habits, ne plia les robes et
n'eut charge du linge dans une honnête maison. Sa petite taille plate
s'alliait bien avec sa mine pointue, son profil en lame de couteau, son
menton fuyant et son front étroit et bombé. Le ton de ses cheveux blonds
était si pâle qu'ils imitaient la filasse. Ses yeux éplorés rappelaient
des fleurs de mauves cuites dans du lait. C'était un abrégé de femme, ou
plutôt de fille, puisqu'elle ne comptait que dix-huit ans et se pouvait
comparer, pour la pureté, à sainte Agnès en personne. Margot Larçonnière
avait été donnée à la baronne de Primelles, depuis tantôt sept ans, par
la nourrice de Marguerite, Ursule de Segry, femme du portier-garde
Roquelin Saboureau, dont le caractère geignard sympathisait avec celui
de Margot.

La chambrière Larçonnière ne possédait que de bonnes qualités. Dix écus
par an, une robe aux étrennes et quelques vieux jupons à Pâques, cela la
contentait, mais ne pouvait l'empêcher d'aller courbée sous une crainte
perpétuelle et des autres et de soi. La peur de commettre une bévue,
celle des araignées, des revenants et de mille autres choses de pareille
importance, gâtait sa vie et la condamnait à accumuler les sottises.
Bien juste, par sa douceur angélique, la baronne de Primelles
réussissait-elle à la tenir en confiance. Un jour, Margot s'abattit tout
en larmes auprès de sa maîtresse. À grand'peine put-on consoler cette
grande coupable qui se désespérait d'avoir cassé un moutardier de cinq
sous et demandait à le payer sur ses pauvres gages.

M. Le Bouteiller, baron de Mordicourt, était pour Margot le seigneur des
épouvantes. Quand il lui adressait la parole, elle s'arrêtait
tremblante, joignait les mains comme qui prie\,; des pleurs
obscurcissaient sa vue, voilaient sa voix, et ses pauvres jambes, dont
la maigreur s'exagérait par la largeur de ses pieds, flageolaient, comme
prêtes à jouer des cliquettes. Et pourtant M. Le Bouteiller n'était pas
méchant. Mais il appartenait en tout au siècle précédent, et ses
manières étaient brusques. Enfin il avait une façon à lui de regarder le
monde par-dessus la tête à quoi Margot Larçonnière, fille de pauvres
gens morts de la peste, ne put jamais s'habituer.

Ce fut à cette créature falotte et découragée que M. Le Bouteiller
confia, de l'aveu même de M\textsuperscript{lle} de Primelles, une
lettre pour la porter à M. de Montenay\,; il s'agissait d'une battue aux
loups\,:

--- Il est à sa maison de Lavergne, tu la lui remettras tout à l'heure.
Lorquin te prendra en croupe, puisqu'il se rend à Lunery. Il te
reprendra avec la réponse\ldots{} Entends-tu, pécore\,?\ldots{} Non, je
veux dire, Margot, ma bonne fille\,!\ldots{} Surtout, ne perds pas cette
lettre. Elle contient des graines dans une enveloppe de papier ficelée
sous la cire\ldots{} Je te la confie, coquine\,!\ldots{} Allons, allons,
ma pauvre enfant, c'est une façon de parler\ldots{} À pleurer ainsi, tu
vas gâter ta collerette\ldots{} Si je te confie ce paquet, c'est que tu
es soigneuse, tandis que Lorquin est un lourdaud qui ne manquerait pas
de l'écraser dans la poche de sa botte\ldots{} Allons, va\,!

Sans oser répondre, l'infortunée se coula le long du mur avec sa lettre
à la main. On eût dit d'une musaraigne surprise au soleil levé sur le
seuil d'une grange. Or, si Margot Larçonnière redoutait deux choses sur
terre, c'était de voyager à cheval et le tête-à-tête avec Jacques
Lorquin dit le Brave. Margot redoutait le cheval parce qu'elle avait
peur de tomber. Elle redoutait Jacques Lorquin dit le Brave parce que ce
garçon de charrue, homme de confiance de l'oncle Le Bouteiller,
jouissait d'une réputation de jovialité qui épouvantait la chaste et
timorée chambrière. Les plaisanteries de ce grand et robuste garçon, qui
suffisaient à dérider les mines les plus revêches, produisaient sur
Margot Larçonnière un effet tout différent. En un mot, c'était une fille
de Lévi, et Lorquin un Amalécite, un Holopherne ou quelque chose
d'approchant. Margot tenait Lorquin pour un réprouvé. Quant à Lorquin,
il n'avait aucune idée précise sur Margot. Il mangeait comme quatre,
buvait comme huit, --- à l'occasion, car rarement plus pauvre homme
n'aiguillonna bœufs à la charrue, --- possédait une inaltérable bonne
humeur et n'aurait pas levé sa forte main sur un enfant. Il ne l'aurait
mise sur la taille d'une fille non plus, à moins que ce ne fût avec son
consentement. Mais de Jacques Lorquin Margot ne voulait rien savoir,
parce qu'il s'était une fois moqué, et avec irrévérence manifeste, du
curé de Primelles.

L'idée d'entreprendre un voyage en croupe de ce dangereux impie paralysa
les faibles moyens de Margot. Dès que M. Le Bouteiller eut tourné les
talons, elle s'assit sur un banc de la cour et pleura amèrement. C'est
alors que Françoise Colbert, qui s'entendait à rôder partout, l'aperçut.
Margot ayant conté son chagrin, Françoise prit sur elle de la remplacer
comme messagère. Justement M\textsuperscript{lle} Marguerite gardait la
chambre, accablée par la migraine et aussi par le chagrin de ne pas
avoir revu Florimond. Donc, quand le puissant Lorquin parut, botté aussi
haut qu'un pêcheur de la Hollande et sa souquenille de courrier sur le
dos, ce fut Françoise Colbert qu'il trouva prête à s'asseoir sur la
croupe du grand cheval à tous crins qu'on achevait de brider. Il accepta
le changement avec plaisir, parce que cette compagne de route bien
frisée lui agréait mieux que «\,la fille Jérémie\,», ainsi qu'il
appelait d'ordinaire Margot. Déjà la dolente chambrière s'enfuyait en
trottant menu vers la cuisine et sa planche à repasser les chemises.

Une fois qu'il eut garni ses fontes de deux longs pistolets à chenapan,
accroché une lanterne de corne à l'arçon, --- précaution utile puisqu'on
ne reviendrait qu'à la nuit tombée, --- bouclé sa ceinture, où pendait
un bon couteau de Turquie dont la lame plus large qu'une faucille se
recourbait avec fierté, pris les commissions d'un chacun et suspendu au
flanc gauche de sa bête une sacoche qui ferait contrepoids aux jambes de
Françoise, Lorquin se mit en selle. Il donna l'étrier à l'agile
servante, qui s'assit sur le panneau et, empoignant la ceinture de son
compagnon, déclara qu'elle ne s'était jamais sentie tant à l'aise.

La bête partit à l'amble. Sur le pas de la porte du pont, le vieux
concierge Roquelin Saboureau, qui tenait le guichet ouvert, souhaita bon
voyage à Françoise, lui conseilla de cacher ses mollets, avec un
compliment dont la nature s'excusait par cela que ledit Roquelin, ancien
arquebusier au régiment de Goas, avait gardé de son métier de soldat une
incoercible liberté de langage. Sa figure héronnière disparut bientôt
aux yeux amusés de Françoise, et Lorquin abonda en discours hardis et
plaisants.

Pour accéder à la terre de M. de Montenay, point n'était besoin de
passer sur celles de Bannes. En longeant celles-ci entre les Colombiers
et Germigny, on s'y rendait en ligne presque droite, et c'était une
affaire de deux petites lieues de pays. Devait-on toutefois, avant que
de s'engager dans cette coulée, passer par Lunerette, puis sous les murs
du parc de Bannes, murs sans cesse détruits par la malveillance des
paysans, sous prétexte de droit de fouage, de secondes herbes et autres
servitudes depuis longtemps rachetées. Mais le marquis, qui entendait
demeurer clos puisqu'il avait payé pour cela, les avait toujours
relevés. Depuis son départ pour l'exil, les brèches s'étaient rouvertes
d'elles-mêmes, et la marquise aimait mieux fermer les yeux, car elle
désirait se rendre populaire.

Par une de ces brèches, Françoise Colbert aperçut la marquise Julie qui
se promenait majestueusement dans une allée et que précédaient quelques
paons avides à se disputer les morceaux de pain qu'elle leur distribuait
de sa main gantée de velours sombre. Elle avait à sa droite Nicole
Deleuze, avec une corbeille à pain en écharpe, et à sa gauche l'abbé
Rousselin, son chapelain, qui marchait, pareil à un gros rat soyeux,
bien en point dans sa courte soutane de satin noir. Derrière la dame de
Bannes, un petit Maure coiffé d'un turban vermeil, vêtu d'un habit long
de brocart, portait un large parasol de damas bleu turquin, dont il
pouvait, grâce à la longueur du manche, abriter la tête soigneusement
crêpelée de sa maîtresse. Un page en mandille armoriée tenait, à côté de
cet avorton barbaresque, la traîne de la robe à pleins bras. Venaient
ensuite les filles de chambre avec leurs collerettes à mille tuyaux et
leurs simples jupes de serge avec le vertugadin plissé en éventail, et
leurs corps à ailerons. L'une tenait le chapeau de paille de la
marquise, chapeau léger d'Italie, plus large qu'une roue de charrette et
tellement chargé de plumes qu'on pensait, en le voyant, à une culière de
carrousel. Une autre fille, Maroie Lenatier, célèbre autant par sa
beauté que par la faveur dont l'honorait la marquise, portait la boîte à
broderie\,: et une troisième avait un pliant passé à chacun de ses bras.
Et, enfin, les feutres empanachés de MM. de Tourouvre et de la Butière
balançaient leur aigrette à l'arrière-garde de cette caravane de choix.
Mais en avant, marchant avec grâce et prudence à reculons, et au grand
mécontentement des paons, M. Aimeri d'Olivier, semblable à un radis
noir, ressemblance qu'augmentait une couronne de feuillage ceignant son
front, sa calotte et ses cheveux rapportés, paraissait occupé à réciter
des vers ou à prononcer une allocution dont la force poétique se
devinait à l'ampleur de son geste et à la façon despotique dont il
levait le menton.

Pour mieux jouir de ce merveilleux spectacle, Françoise pria son écuyer
de se mettre au pas, et elle accompagna cette demande d'un si vigoureux
pinçon dans le dos que Lorquin, flatté dans sa chair, malgré l'épaisseur
de sa souquenille en drap de ménage, par cette caresse cavalière, arrêta
du coup son roussin\,; et cela si heureusement que, si le pan de mur lui
cachait la vue du cortège, Françoise, en tournant la tête, pouvait par
le talus de la brèche avoir cette vue tout entière. Elle put même
entendre M. Aimeri d'Olivier comparer la marquise à Phœbé, son fils à
Phæbus\,; elle put voir surtout M. Clément Malompret en l'honneur de qui
elle était montée à cheval, car elle le cherchait ou l'attendait en vain
depuis huit longs jours. M. Clément, qui venait tout à fait derrière, en
tête de six grands laquais, les laissa marcher et s'arrêta sur un signe
de Françoise, dont la mine se montrait au-dessus du monceau de pierres.
Ce signe ne fut pas saisi par le complaisant Lorquin, qui s'occupait de
rajuster la mèche de son fouet. Il poussa tout aussitôt son cheval en
avant, quand il entendit Françoise lui crier sur le mode aigu\,:

--- Allons, qu'attends-tu\,?\ldots{} En route\,! Il y a loin d'ici à la
maison de M. de Montenay\,!

Elle avait prononcé ce nom d'une telle voix de trompette que MM. de
Tourouvre et de la Butière s'en retournèrent au bruit. Ils ne purent
rien voir, car ils étaient déjà à quelque vingt pas de la brèche, et la
tête de Colbert avait disparu. Mais M. Clément avait entendu et vu, de
telle sorte qu'à la hauteur des Bornes il apparut brusquement au beau
milieu du chemin, salua avec une politesse excessive Lorquin, échangea
un nouveau signe avec Françoise, puis s'éloigna à la rapide allure de
son courtaud dans la direction de Maleray. Lorquin déposa son précieux
fardeau à la porte de la maison de Lavergne. Il apporta à ce faire un
tel soin que la sage Colbert crut devoir l'en blâmer\,:

--- Lorquin, mon ami, point n'est besoin pour me mettre à terre de
m'embrasser dans le cou. Outre que cette façon n'est pas honnête, elle a
l'inconvénient de chiffonner les collerettes. Fuyez-vous-en,
malotru\,!\ldots{} Vous avez failli fracasser le paquet de graines à moi
confié par M. Le Bouteiller\ldots{} T'en iras-tu\,? Je n'ai pas besoin
de toi pour entrer.

Ainsi semoncé, Jacques Lorquin, dit le Brave, fit claquer son fouet avec
désinvolture et allégresse et, sans s'excuser autrement, partit à fond
de train pour gagner Lunery. Et Françoise, tournant le dos à la porte de
M. de Montenay, regagna vivement le chemin des Bornes, où M. Clément
l'attendait, la bride de son cheval à la main.

Leur conversation fut brève, car ils redoutaient d'être surpris dans ce
lieu découvert où les abritait à peine un buisson. Colbert apprit que
Florimond était à Bourges, où il se cachait, n'ayant pour suite qu'un
méchant petit valet.

--- Il ne sait comment se défaire de cette Brossin, dont la jalousie
devient furieuse. Croirais-tu que, non contente de le tracasser par ses
lettres, la diablesse voulait le relancer jusqu'au château\,?\ldots{}
Encore un peu, et elle y pénétrait sous couleur de proposer des
mouchoirs\,!\ldots{} J'ai eu toutes les peines du monde à la
renvoyer\ldots{} Si tu trouves que cela ne mérite pas le fouet, je me
demande à qui on devra le donner\,!\ldots{} Son enfant\,?\ldots{} Eh
bien, qu'elle l'élève, et nous laisse en paix\,!\ldots{}

M. Clément, qui ne s'abaissait pas à ces vulgaires détails, négligeait
de dire que c'était du manque d'argent que souffraient la mère et le
nouveau-né. Florimond se débattait dans une gêne cruelle, et M. Clément,
qui désapprouvait ces amours roturières avec la lingère Madeleine
Brossin, fermait sa bourse.

Tout en prêchant la morale, le valet de chambre prit le menton de
Colbert. Puis il reconnut qu'il était urgent de rappeler Florimond,
puisque M\textsuperscript{lle} Marguerite séchait sur pied en attendant
son retour\,:

--- Je partirai demain, de grand matin, et vous le ramènerai, ma belle.
Comptez toutes deux sur nous pour après-demain, à la place ordinaire, et
sur le coup de trois heures après midi\ldots{} Et retiens ta
langue\ldots{} Fais-moi le plaisir de cacher cela sous ton corset.

Françoise, qui reculait avec méfiance, risqua un pas en avant et prit
les deux écus que lui tendait le généreux Clément. Cette manière
d'accompagner les pourparlers lui parut de bon augure pour les
entreprises à venir, d'autant qu'elle ne voyait quasiment jamais
d'argent à Primelles, où cette marchandise était fort rare. Ses gages
minimes, elle devait les attendre souvent tout un an.

Quant aux velléités amoureuses de M. Clément, elle les relégua dans le
domaine sans limites des espoirs qui n'ont pas de terme précis. De même
qu'avec Lorquin, elle usa de procédés dilatoires, et le valet de chambre
de Florimond, qui avait d'ailleurs des préoccupations plus hautes,
s'éloigna en répétant\,: «\,À mercredi, la belle enfant\,! À
mercredi.\,»

Et lorsque Colbert fut de retour à Primelles elle versa sur le cœur
endolori de M\textsuperscript{lle} Marguerite, qui se morfondait dans
son lit, le baume des consolations en lui annonçant que, dans deux
jours, M. Florimond, au retour d'un voyage, passerait par le chemin de
Lunerette. De ce chemin Marguerite rêva toute la nuit.

\hypertarget{chapitre-vi}{%
\chapter{CHAPITRE VI}\label{chapitre-vi}}

Lorsqu'il revint de Bourges, Florimond était de l'humeur la plus
détestable. D'abord il avait dû passer une nuit entière enfermé dans une
armoire chez la femme du conseiller Godefroy Harant, ensuite il avait
subi les fureurs jalouses de Madeleine Brossin, dite Madelon, ensuite il
s'était vu accabler de compliments par la dame Macette, au sujet de son
fils Joachim, dont il se souciait comme de son dernier bonnet de nuit.
L'incartade de la jolie lingère, mère de cet enfant de l'amour, racontée
par Clément à Françoise Colbert, n'était en effet que trop véridique\,;
et Florimond, s'il avait pu la faire renvoyer quand elle s'était
présentée hardiment au château de Bannes, n'en avait pas moins dû partir
pour Bourges afin d'éviter de nouvelles complications.

Tout ainsi que son père le marquis, Florimond avait versé dans les
amours basses. Lors de son séjour à Bourges, en 1631, il s'était épris
d'une petite lingère que sut lui procurer une personne recommandable,
répondant au nom de Macette Péronet, toujours prête à obliger son
prochain contre récompense honnête. L'aventure, qu'il se promettait
brève, dura grâce à l'adresse de la demoiselle et de son chaperon
Péronet. Au bout d'un an, Florimond se trouva père et obligé de pourvoir
aux dépenses de trois personnes, sans compter la nourrice de l'enfant,
qu'il fallut cacher à la campagne. Mais, chaque fois que Florimond
passait par Bourges, on faisait venir la nourrice et le nourrisson, et
il se voyait condamné à une méchante vie de famille qui exaspérait tous
ses instincts de vie libre et joyeuse. Sans compter que les trois
commères, Madelon, Macette et la nourrice Scolastique ne parlaient que
d'argent, comme si elles ne comprenaient pas ou ne voulaient pas
comprendre que Florimond ne séjournait dans la bonne ville du Berry qu'y
contraint par le manque de cet argent.

Avec quel plaisir il aurait envoyé promener tout ce monde si Madelon,
excitée par la sangsue Macette, qui lui racontait, en l'ornant pour les
besoins de la cause, le mariage de Julie Péréal avec le marquis de
Bannes, n'eût pris cette précaution de répandre dans toute la ville le
bruit de sa féconde union avec l'héritier de la noble maison alliée aux
La Force\,! Faire enfermer la lingère dans un couvent de filles
repenties, où personne ne l'eût réclamée puisqu'elle était orpheline,
envoyer la Macette en prison, abandonner à l'hospice le fils que rien ne
prouvait, n'était point chose difficile. Mais une grande honte en eût
rejailli sur Florimond et sur sa mère la marquise, qui n'avait pas
besoin de cette addition à son impopularité. Il aurait pu, à la rigueur,
expédier ce ménage de hasard à Paris, où tout se perd dans le flot de
peuple qui y grouille, puisque Bourges tout entier tenait les yeux sur
lui et clabaudait avec le sans-gêne de ces petites villes oisives et
endormies où l'on ne dort que d'un œil, l'autre toujours appliqué à la
fente de quelque volet. Cela n'eût pas éteint les difficultés, et les
dépenses auraient triplé peut-être. Or, Florimond était toujours à court
d'argent. Comme dernière ressource, il y avait Bérenger de la Butière ou
M. de Tourouvre. Marier Madelon à l'un d'eux était un expédient qui en
valait bien un autre, à condition toutefois qu'une dot fût versée.
Comment réunir les quelques milliers d'écus nécessaires\,?

Le procureur Marcelin Duvau n'en trouvait pas le moyen, d'autant que
personne n'osait aviser le marquis exilé de cette particularité de son
héritier. Le marquis, du reste, était renseigné là-dessus, mais il n'y
attachait pas d'importance. Touché par la détresse de Florimond, adouci
peut-être aussi par l'heureux succès d'une instance où il venait de
triompher et de gagner le droit de s'appeler non plus Duvau mais
Marcelin de Vaulx, chose à ses yeux considérable, l'homme de loi lui
consentit cependant un prêt, au denier douze, il est vrai, mais enfin un
prêt de six mille livres. Grâce à cet argent, dont il garda toutefois la
majeure partie, Florimond put donner la pâture à son ménage affamé et en
clore pour quelques mois les becs. Aux scènes de jalousie il était trop
habitué par son métier d'homme à bonnes fortunes pour leur accorder une
attention même minime. Donc, léger de cœur et d'esprit, il s'occupa de
M\textsuperscript{me} Jeanne de la Pelice, femme du conseiller Godefroy
Harant, homme de soixante ans, alors que celle-ci n'en comptait pas
trente. M\textsuperscript{me} Jeanne possédait cette beauté diminuée qui
emprunte son mérite à l'élégance des habits et aux artifices de la
toilette. Mais sa taille était ravissante, et la blancheur mate de son
teint se rehaussait par le noir soyeux d'une chevelure
extraordinairement longue et fournie qu'elle s'amusait, par un caprice
singulier, à couvrir de poudre blonde, de telle sorte qu'elle
ressemblait à un chou de crème saupoudré de cannelle.

Cette femme de magistrat, fille d'échevin ayant acheté la noblesse,
séchait d'envie, tant sa vanité la torturait en lui prouvant combien sa
condition était infime à côté des nobles d'épée. Cette fausse bourgeoise
pointue, acide, sentimentale et dolente, suivant les gens, vivait
persuadée qu'elle avait déchu en épousant un membre du Parlement. Elle
crut se relever en prenant pour amant un jeune homme qui allait entrer
aux gardes. Riche, avare et défiante, elle crut Florimond plus riche
qu'il n'était, et se flatta de trouver en lui l'idéal de magnificence et
de galante perfection après quoi elle n'avait cessé de soupirer.
Introduit chez elle par des officiers du gouverneur à qui elle donnait à
jouer, Florimond put dire qu'en un même jour il était venu, avait vu et
avait vaincu. La discrétion du jeune homme était petite, celle de la
dame encore moindre, l'indifférence du conseiller Harant, par contre,
dépassait les limites du croyable. N'eût été la jalousie de Madelon,
avivée par la cautèle de la mère Macette, Florimond aurait trouvé dans
l'hôtel de ce magistrat cette paix du cœur, indispensable accompagnement
de toute liaison intelligente, c'est-à-dire où chacun des intéressés
récolte ce qu'il a semé, c'est-à-dire encore autant comme rien.

Malheureusement Madelon ne voulut pas se pénétrer des simples nécessités
de la vie. Elle poursuivit de billets sans signature le conseiller
Harant, qui n'en pouvait mais et les semait avec insouciance dans les
couloirs, les cours, ou même dans la rue, la marquise Julie, qui s'en
gaussait avec ses femmes, et quelques gentilshommes en droit de se
considérer comme étant les uniques favorisés de ces distinctions
amoureuses que conférait M\textsuperscript{me} de la Pelice à ses
heures. La lingère en fut pour ses frais de papier. Et si Florimond,
surpris par le conseiller, qui rentra à une heure où l'on ne l'attendait
pas, dut se blottir certain soir dans une armoire à linge et y passer la
nuit entière accroupi parmi les serviettes et les taies d'oreillers,
cette disgrâce ne fut imputable qu'à un malheureux hasard.

Son caractère impérieux et violent d'enfant gâté dès le berceau en garda
l'impression d'une rare et cruelle offense. Et il jura une haine
sauvage, déclara une guerre perpétuelle et sans merci au robin
malencontreux dont les ronflements avaient offensé ses oreilles pendant
cette nuit de mai où lui-même avait ronflé et dormi, loin de la désolée
Madelon, et à poings fermés, sur le linge fin de la conseillère. Et il
n'eut pas une pensée bienveillante à l'égard de cette dame, tout à la
fois perfide et sensible, qui, dans sa cruelle insomnie, avait compté
les heures jusqu'à ce que le lever du soleil appelât son mari au Palais
de Justice, où, de mémoire d'homme, il n'était jamais arrivé autrement
que premier. Ce matin-là, le conseiller Harant entra modestement, comme
à son ordinaire, mais les épaules couvertes d'un beau manteau de revêche
gris de lin, doublé de peluche rose sèche, manteau qui n'était pas le
sien, puisque sur le coup de huit heures on vint le réclamer de la part
de M\textsuperscript{me} de la Pelice.

--- Donnez, donnez, fit répondre M. Harant, ma femme sait bien quels
sont mes habits.

M. Clément avait trouvé son maître dormant du sommeil du juste à
l'hôtellerie des Jeux Marins, où il se cachait pendant ses escapades de
Bourges. Une fois réveillé, Florimond oublia ses grands serments de
vengeance et reconnut que le retour à Bannes s'imposait. Pendant les dix
lieues de route, M. Clément ne cessa de le blâmer à cause du temps qu'il
perdait dans de pauvres divertissements sans gloire, alors que la fleur
de Primelles se penchait amoureusement vers lui, impatiente d'être
cueillie. Ainsi les propos légers de La Butière et de Tourouvre, les
hautaines et perfides insinuations de sa mère, les conseils ingénieux
d'Aimeri et les avis savamment flatteurs de Malompret, d'autant plus
flatteurs qu'ils se rehaussaient de reproches sentant la franchise du
serviteur dévoué, poussaient tous Florimond sur la même voie\,: «\,C'est
une conspiration amoureuse contre moi, --- songeait-il, balancé à
l'amble régulier de son cheval de poste, --- et la charmante Marguerite
en est la principale conjurée, puisqu'elle me rappelle, sans aucun
doute\,!\ldots{} Aimeri fut bien inspiré en me poussant à m'éloigner
pour quelques jours\,!\ldots{} Avant que de la revoir, je veux qu'il
m'explique les plus beaux passages de \emph{l'Astrée}, qu'il m'en bourre
la tête\ldots{} Alors pourrai-je aborder la belle et débuter
ainsi\ldots\,»

Ses réflexions furent brusquement interrompues par Cottebleue, qui le
salua, à la tête d'un gros de paysans et de laquais, au milieu de
l'avenue du château\,:

--- Bonne nouvelle, monsieur\,! J'espère que vous serez content\,! Nous
avons pincé Marin\,!\ldots{} Le drôle nous assomma trois hommes, ou à
peu près, mais nous le tenons\,!

Florimond, à la grande stupéfaction de Cottebleue, n'entra nullement en
joie. Fronçant le sourcil, il tourna la tête vers Clément\,:

--- N'aurais-tu pu m'avertir\,?

Le geste du valet prouva au maître que la nouvelle était fraîche et
qu'ils étaient deux à l'apprendre. Florimond toisa alors le porteur
d'exploits, qui demeurait bouche bée, le chapeau à la main droite, le
bâton à la gauche, devant le cheval qui fumait. Et il lui demanda, d'un
ton rogue si on n'avait pas maltraité Marin.

--- Pas plus que de raison, monsieur\,!\ldots{} Pas plus que de raison.

--- Juste Dieu\,!\ldots{} Vous ne l'avez pas blessé, au moins\,?

--- Ah monsieur, il sera toujours assez frais pour\ldots{}

--- Tais-toi maroufle\,!\ldots{} Allons, vous autres, parlez\,! Où
est-il\,?

Tout le monde se tut. Cottebleue prit sur lui de répondre en reculant
toutefois prudemment hors de la portée du fouet que Florimond venait de
tirer de sa botte\,:

--- En bon lieu, monsieur\,!\ldots, Dans le caveau de la vieille tour du
nord, et bien ficelé, je vous en réponds. Cette fois il ne nous glissera
plus entre les doigts\,!

Florimond jeta un écu à Cottebleue\,: «\,Pour ta peine\,!\,» et, sans
l'écouter davantage, rentra chez lui. Son premier soin fut d'ordonner à
Clément d'amener Marin\,: «\,Qu'on le détache, et qu'on ne le brutalise
pas, tu m'entends\,! Et tu le feras monter ici, dans ma chambre\ldots{}
Qu'on ne me dérange point. Je n'y suis pour personne. Inutile d'aviser
ma mère de mon arrivée\ldots{} Envoie-moi Aimeri et reviens avec
Marin\,!\,»

Clément s'inclina en souriant d'un air confit\,: «\,Monsieur, prenez
garde. C'est un scélérat capable de tout, et il a juré ma mort\ldots{}
Mais j'obéis, monsieur, j'obéis.\,» Il descendit les escaliers quatre à
quatre, bouscula MM. de Tourouvre et de la Butière, qui tentaient de se
glisser chez Florimond, en criant\,: «\,Non, non\,! La consigne est pour
tout le monde\,! On n'entre pas\,!\ldots{} Nous avons de grosses
affaires, oui, messieurs, c'est ainsi\,!\,» M. Clément semblait avoir
chaussé les talonnières ailées de Mercure\,; il se hâtait, porté en
quelque sorte par ses pressentiments\,: «\,Tout va à merveille\,! Ce
misérable Marin devient la cheville ouvrière de ma
combinaison\,!\ldots{} Au diable Francine\,! Je suis prêt à jurer que je
ne connais même pas la couleur de ses cheveux.\,»

M. Aimeri d'Olivier ne mit pas longtemps à se produire. Il s'assit
gravement, négligea de dire, tant sa modestie était grande, que tout
allait pour le mieux depuis qu'il avait pris la queue de la poêle,
saisit sa tabatière, où il paraissait puiser ses maîtresses pensées, et
s'informa de la santé de son cher Florimond avec une affectueuse
indifférence. Celui-ci, impatient, mit aussitôt les fers au feu\,:

--- Aimeri, tout dépend de ta prudence. J'ai confiance en toi. A toi le
soin de diriger l'interrogatoire de ce Marin que le diable confonde,
mais que nous devons soigner mieux qu'un enfant Jésus. Ne m'abandonne
pas\,! Quand je te semblerai errer, tu taperas avec ta boîte à tabac sur
le bras de ton fauteuil\,; alors je me tairai, et tu parleras.

--- Mon cher enfant, votre sagesse dépasse trop la mienne pour que je
puisse faire mieux que vous. Profitant toutefois de votre permission,
j'essayerai de placer un mot, à l'occasion, sans plus.

Dirigé par M. Clément, qui clignait de l'œil d'une façon tout à la fois
paterne et condescendante, Marin entra. Ses vêtements en lambeaux
disaient la violence de la lutte inégale qu'il avait soutenue contre
plusieurs, et la trace des cordes sur ses poignets saignants prouvait la
force des liens autant que l'acharnement de ceux qui les avaient serrés.
Sa tête brune, embroussaillée, était balafrée et meurtrie\,: de sa barbe
rousse on avait arraché une touffe. Il avait perdu un soulier, et ses
chausses déchirées montraient ses genoux à vif.

Florimond considéra le braconnier sans colère\,:

--- Il n'y a pas de bon sens à secouer ainsi les gens. Clément, tu
manderas à Cottebleue de modérer son zèle à l'avenir. Ce garçon, pour
coupable qu'il soit, ne devait pas être mis en pièces. Emmène-le sans
tarder, donne-lui du linge et des habits convenables, et explique-lui
qu'il n'y a pas de quoi se désespérer. Regardez-le, monsieur
d'Olivier\,: ne dirait-on point d'un sanglier coiffé par un vautrait
tout entier\,? Mes ordres ont été dépassés, comme toujours, c'est clair.
Ah\,! qu'on est mal servi par le temps qui court, et que les mœurs des
champs sont sauvages\,!\ldots{} Depuis combien de temps est-il au
cachot\,?

--- Depuis tantôt deux jours, monsieur\,! --- Et Clément ajouta avec un
accent de compassion dont la sincérité apparaissait évidente\,: --- Et
je crains bien qu'on ne l'ait pas nourri à sa faim\ldots{} Si j'avais pu
savoir\,!\ldots{}

Encouragé par les hochements de menton d'Aimeri, qui, les mains croisées
sur sa panse, les jambes allongées, examinait Marin avec intérêt,
Florimond haussa les épaules et, affectant une pitié très profonde,
grommela\,:

--- Tant pis\,! Tant pis\,! Tout cela est contraire à la justice. On ne
procède pas autrement chez les Turcs\,!\ldots{} Ce garçon, après tout,
n'est encore qu'accusé\,; je ne puis prendre sur moi de le traiter comme
un coupable, hem\,!\ldots{} hem\,!\ldots{} comme un coupable.

--- Sans doute, sans doute, monsieur, un accusé, simplement\,!

Et M. Aimeri d'Olivier, ayant ainsi fourni des preuves de sa modération,
tira sa tabatière de son haut-de-chausses et la garda dans sa main.

Florimond, n'entendant pas le bruit du fauteuil heurté, comprit qu'il
pouvait continuer à défendre les droits du faible. Ramenant sur son
genou gauche sa jambe droite encore bottée, il en examina les plis
poussiéreux avec intérêt et reprit\,:

--- C'est fort ennuyeux, fort ennuyeux, Clément, et je suis fort
mécontent\ldots{} Hem\,! Hem\,!\ldots{} Qu'on donne donc à ce garçon, et
sur l'heure, à manger suivant sa faim\,!\ldots{} Et je veux aussi qu'il
boive à sa soif, qu'il fume même, pourquoi pas, s'il le
désire\,?\ldots{} Je l'interrogerai après.

Marin, qui jusque-là avait gardé obstinément ses yeux baissés, regarda
Florimond de côté, furtivement. Alors Florimond, qui l'observait en
dessous, dit\,:

--- Çà, Clément, qu'attends-tu pour lui délier les mains\,?\ldots{} Sans
doute que la corde ait froissé les os\,?\ldots{} Voilà de la belle
besogne.

Clément se crut autorisé à discuter cet ordre\,: «\,L'homme était
dangereux, le laisser ainsi\ldots\,»

--- Allons, tranche-moi cela, et vivement, et sans le couper. Prends ce
couteau, et que ce soit fait soigneusement\,!

La figure énergique de Marin, crispée par la colère, s'adoucit. Il leva
sur Florimond ses yeux qui, sous leurs épais sourcils, brillaient de
vivacité et d'intelligence. Puis il parla\,:

--- Bien sûr que j'ai faim\,!\ldots{} Et pourquoi se sont-ils jetés à
huit sur moi pendant que je rentrais chez nous avec des bourrées\,? Je
ne faisais pas de mal\ldots{} Et l'on n'a pas le droit de saisir les
gens sur la terre de leur seigneur\ldots{}

--- Dirait-il vrai\,? demanda vivement Florimond, dont M. Aimeri,
maniant avec nonchalance sa tabatière inutile, admirait avec
stupéfaction la force de dissimuler. Mes gardes t'ont mis la main au
collet chez M\textsuperscript{me} de Primelles\,? Dirais-tu vrai\,?

--- Pourquoi mentir\,?\ldots{} Et puis, ce ne sont pas vos gardes qui
m'ont pris. Ils n'oseraient pas venir sur nos biens,
peut-être\,!\ldots{} C'est Cottebleue, avec sept valets de ferme. Je les
connais bien, les sept valets de Landry Vaillard, votre fermier des
Aubrois\ldots{}

Florimond serra les poings, se mordit les lèvres et cria d'une voix
blanche\,:

--- Ce n'est pas possible\,! Prends garde\,!\ldots{} Je\ldots{}

M. Aimeri donna un coup de sa tabatière sur le bois du fauteuil, comme
pour s'aider à l'ouvrir. Florimond se tut, et le poète prit la parole\,:

--- Ce garçon, monsieur, n'a pas, à mon humble avis, ce mauvais aspect
des malfaiteurs de profession, et je suis convaincu qu'il parle franc.
Si vous vouliez me permettre de vous donner mon opinion\ldots{}

Florimond, qui regrettait son accès de colère et craignait surtout de
l'avoir laissé percer, acquiesça gracieusement, tout en enrageant contre
cet Aimeri, dont il avait attendu si longtemps le secours. M. Clément
approuva d'un murmure et d'une révérence. Il avait gardé le couteau
après avoir coupé les liens de Marin et se tenait sur la défensive, tant
il redoutait ce hardi braconnier\,: «\,Pourvu qu'il ne nous bouscule pas
d'un bond, ne saute après par la fenêtre\,! Il nage mieux qu'un poisson.
En deux temps il aura passé l'eau, gagné le parc, et adieu\,! Puis,
quelque soir, il me donnera un bon coup de bâton\,!\ldots{} Attention,
je crois qu'il va s'élancer\,!\,»

Mais Marin ne nourrissait pas des intentions aussi aventureuses.
Meurtri, faible de faim, il frottait machinalement l'une contre l'autre
ses mains délivrées. Elles étaient gonflées, douloureuses\,; le sang
extravasé avait empouacré les ongles.

--- Mon Dieu, monsieur, à votre place je donnerais d'abord à ce pauvre
diable la nourriture et les habits que votre bonté lui
promettait\ldots{} Et après on causera.

Marin refusa avec fierté. Au vrai, il était plein de défiance. Ce que
venait de dire M. Aimeri n'était sans doute qu'une plaisanterie
concertée avec le seigneur de Bannes. Florimond fut tout simplement
admirable. Sa mansuétude dompta l'ombrageux braconnier, qui accepta ses
offres, avec celle restriction toutefois\,:

--- Votre valet Clément ne me plaît guère\ldots{} Et puis nous avons un
vieux compte à régler\ldots{} Sauf votre respect, monsieur, je
m'accommoderais mieux d'un autre gardien. Foi de Marin, je ne chercherai
pas à fuir\,!

Sans sourciller, Florimond renvoya Clément en lui ordonnant d'appeler
Barrois. Clément accepta cette combinaison avec joie, car il avait
autant peur de Marin que celui-ci avait de haine et de mépris pour lui.
Le valet de chambre en second, Barrois, vieux serviteur du marquis, se
présenta avec la gravité d'un bedeau, reçut les instructions et emmena
Marin, qui salua en cherchant machinalement son chapeau absent.

Alors M. Aimeri félicita Florimond de sa mansuétude\,:

--- Mon cher enfant, vous méritez mieux que des compliments pour cette
puissance de commander à vous-même. Si madame votre mère vous avait vu
ainsi décidant en seigneur et juge, --- et avec quelle ferme modestie\,!
--- je gage que les larmes lui en auraient coulé des yeux\ldots{} et
c'eût été grand dommage, car je n'en contemplai jamais d'aussi
beaux\ldots{} si ce n'est les vôtres. \emph{Macte animo, generose
puer}\ldots{} Je veux dire\,: achevez maintenant ce que vous avez si
bien commencé. Déployez la même longanimité, pratiquez votre avisée
patience dans l'interrogatoire\ldots{} J'écouterai\ldots{} Ne vous
engagez à rien\ldots{} Laissez croire à ce drôle que votre chagrin
serait d'être calomnié chez vos voisins. Aspergez son vilain museau
d'eau bénite de cour\ldots{} Aussi bien vous découvrez-vous tant habile
que je me demande quel secours je pourrai bien vous apporter dans le cas
présent.

Florimond huma cet encens vulgaire avec plus de plaisir que celui du
curé de Lunery quand celui-ci lui faisait, suivant le droit et l'usage,
honneur avec son encensoir, sous le dais seigneurial, en sa paroisse.

--- Tu te moques, Aimeri, tu te moques\,!\ldots{} Je retiens seulement,
de ton appréciation bienveillante, ce que tu dis de l'habileté. Dieu
merci, je ne crois pas manquer de cette marchandise. Grâce à mon
adresse, Aimeri, ton élève a échappé aux pires dangers\,! J'ai déjoué
les entreprises jalouses du conseiller Harant. Non content de me voler
mon manteau, que sa femme sut lui faire lâcher d'ailleurs, ce mari
ridicule, --- est-il, Aimeri, un mari qui ne soit pas ridicule\,? --- ce
pleutre, ce robin essaya de me donner la mort sournoisement, cette
nuitée, en me retenant dans une armoire où je pensai périr au milieu de
pièces de linge. J'en sortis pour longtemps parfumé, tant la conseillère
avait serré de capricornes musqués parmi ses mouchoirs. En cherchant
bien, tu trouverais, je gage, quelques débris de ces mouches dans mes
manchettes et mes cheveux\ldots{} En veux-tu\,? Rien de meilleur pour
aromatiser le tabac. Et encore j'ai réussi à soutirer quelque trois
mille livres à ce fesse-mathieu de Marcellin\ldots{} Quand je dis trois
mille\ldots{}

Florimond pouvait regretter ses paroles, mais non les rattraper. Il s'en
mordit la langue, car M. Aimeri, saisissant la balle au bond, cria
aussitôt misère\,: son habit montrait la corde\,; ses rabats effilochés,
ses manchettes élingées prouvaient sa ruine. Florimond dut lui lâcher
cent livres, et M. Aimeri jura de garder le secret sur cette libéralité
et particulièrement sur sa source. En échange, le jeune homme reçut une
grande quantité de bons conseils, et aussi force citations à relire dans
\emph{l'Astrée} pour son entretien du lendemain\,:

--- Si vous daignez ne pas oublier mes recommandations, la demoiselle
tombera amoureuse de vous comme une bête. Elle palpite déjà comme un
oiseau blessé. Ne vous a-t-elle pas vu\,?\ldots{} Heureux Florimond, qui
ne connut jamais de cruelles\,! Allez de l'avant. Tout m'annonce que
Marguerite ne vous repoussera pas. Le voudrait-elle que l'amour ne lui
en laisserait pas la force. Toutefois, ne vous montrez pas impatient.
Caton, pendant de longues années, demanda la ruine de Carthage. Ces
années seront pour vous de brèves journées, sans plus. Mais j'entends
votre braconnier. Attention\,! Il fait partie du piège où va tomber
votre belle. Ne brusquez rien\,!

Marin rentra. Lavé, vêtu et restauré, il avait assurément meilleure
mine. Il regarda avec une tranquille assurance, et sans forfanterie,
Florimond, qui l'admonesta avec une hautaine douceur\,:

--- Écoute-moi, puis tu répondras tout à ton aise. Personne ici ne te
veut de mal, quoi qu'on ait pu te dire à Primelles. Je te mettrais bien
en liberté\,; encore ne suis-je pas seul en cause. Que tu aies tendu des
collets sur le talus de mon parc, la semaine passée\ldots{}

Marin s'agita, murmura quelques mots inintelligibles, haussa les
épaules. Puis il baissa la tête et ne bougea plus.

--- Oui, tu les as tendus, tu en conviens toi-même. Et mieux, tu les as
fabriqués. Personne autre que toi n'est capable de tresser ainsi le
crin\ldots{} Ou bien, c'est ton père\,?\ldots{} Tu restes
muet\,?\ldots{} Après tout, le crime est petit, et je ne sais même pas
si je t'obligerai à répondre de cela en justice.

Marin respira plus librement. Nul ne savait mieux que lui que c'était le
petit baron Louis-Antoine qui avait posé les collets. Sa terreur était
qu'un autre pût le savoir, et aussi que le petit seigneur se mêlât de le
délivrer en se déclarant coupable. Il surveilla plus attentivement
Florimond, qui, maintenant débotté, jouait avec sa pantoufle.

--- Enfin, tu les as posés, ces collets, au pied de mon mur, tu
l'avoues\,?

Marin rougit\,; ses lèvres tremblèrent. Mais il se renferma dans son
silence obstiné.

--- On t'accuse malheureusement d'actions plus graves. Tu as battu mes
gardes, et tu en blessas même un grièvement, à la Toussaint
dernière\ldots{} Et encore n'as-tu pas cassé le bras de Mathieu Laurent,
mon sergent blavier, quelques jours avant Pâques\,?\ldots{} Tout cela,
mon garçon, est plus grave que tu ne le penses\ldots{} Pour moins on va
aux galères\ldots{} Parlerai-je enfin de cette attaque contre mon fidèle
gentilhomme\,?\ldots{}

M. Aimeri d'Olivier eut à ce moment avec sa tabatière, qui ne voulait
pas s'ouvrir, un colloque d'où il sortit pourtant victorieux. Au bruit
de la corne heurtant le bois, Florimond s'était tu. Il reprit bien
vite\,:

--- Voyons, je vous en fais juge, mon cher monsieur Aimeri.

--- Que vous dirais-je, monsieur\,? vous avez mille fois raison. La
sagesse et la charité parlent par votre bouche. Les galères, c'est une
bien lourde punition pour des peccadilles de ce genre, et ce pauvre
garçon est bien jeune. Malheureusement le roi décide le plus souvent en
toute sévérité contre les contrevenants aux édits sur la chasse. En se
conformant aux désirs de Sa Majesté, les juges ne font que leur petit
devoir. La chose est fâcheuse\ldots{} Ce pauvre diable, monsieur, est,
je le répète, bien jeune. Il s'amendera\ldots{} Comme juge de Lunery,
vous n'êtes pas tenu de l'envoyer au tribunal d'Issoudun\ldots{} Si
j'étais à votre place\ldots{}

Florimond, n'étant plus obligé de parler, avait repris pied. Son calme,
sa dignité, son sérieux avaient on ne sait quoi d'auguste. Balançant sa
pantoufle au fin bout de son pied, il interpella Marin, qui se tenait
coi, sans rien comprendre\,:

--- Tu entends, Marin Labrande, ce que dit M. Aimeri d'Olivier. C'est un
homme droit et doux entre tous\ldots{} Eh bien, mon cher monsieur
Aimeri. Comment agiriez-vous à ma place\,?

--- Mon Dieu, monsieur, je garderais ce bon garçon ici, sous clef. Je ne
le laisserais manquer de rien. Et, quand j'aurais sérieusement examiné
son affaire, je me déciderais à prendre un parti\ldots{} pas avant,
monsieur, de m'être informé de toutes les circonstances.

En conséquence, Florimond décréta que Marin Labrande serait tenu sous
les verrous jusqu'à plus ample informé. Le braconnier, ainsi interrogé
sans qu'il eût pu répondre, se retira abasourdi de cette extraordinaire
mansuétude\,: «\,Ils manigancent quelque chose, bien sûr\,! Mais
qu'est-ce qu'ils veulent\,? Soupçonneraient-ils M. Louis-Antoine\,? En
tous cas, ce n'est pas moi qui le vendrai.\,»

M. Clément Malompret, qui, l'oreille collée à la porte, avait suivi la
conversation juridique, ne se fit point faute d'avertir Françoise
Colbert\,:

--- Ma fille, ta demoiselle est, à cette heure, maîtresse souveraine du
sort de Marin. Je t'avouerai, entre nous, que ce drôle s'est laissé
aller contre moi à des menaces que je méprise, du reste, parce qu'il
m'accuse de courtiser sa sœur Francine. Or, peut-on songer à cette
Francine quand on a l'avantage sans pareil d'être distingué par une
beauté de ton mérite\,?

Colbert retint ce propos, et aussi un bel écu blanc que lui glissa le
valet de chambre avec une familiarité galante. Elle eut bientôt instruit
M\textsuperscript{lle} Marguerite des malheurs du fils Labrande\,:

--- Ah\,! mademoiselle, quand je pense qu'un seul mot de vous suffirait
pour sauver ce pauvre garçon des galères\,!

Flattée de ce discours qui lui prouvait sa toute-puissance sur
l'Incomparable Florimond, Marguerite de Primelles attendit avec une
nerveuse impatience ce lendemain où elle verrait à ses pieds ce modèle
accompli des bergers qui ne sortait plus de sa pensée.

Il se rencontrèrent, à point nommé, dans le chemin de Lunerette.
Aussitôt que Florimond parut, Colbert s'écarta discrètement et se percha
sur un bloc de pierre d'où elle pouvait tout surveiller du côté des deux
châteaux. Clément Malompret, posté à cheval beaucoup plus loin sur la
gauche, se cachait derrière un buisson qui ne l'empêchait pas de voir au
loin, et il avait le chemin sous lui. Florimond apparut donc seul aux
regards charmés de Marguerite, qui, par contenance, garda son mouton
auprès d'elle et son \emph{Astrée} sur les genoux. Assise sur la roche
moussue qui lui servait de fauteuil, elle reçut le salut respectueux et
caressant de Florimond avec une réserve pleine de gravité et de grâce,
qu'elle sut garder malgré les battements précipités de son cœur.
L'émotion la faisait rose, et son visage long, au petit nez aquilin, aux
mâchoires peu saillantes, en prenait une grâce et un éclat qu'ils
n'avaient jusque-là possédés. À qui l'eût vue en ce moment, Marguerite
de Primelles eût apparu transfigurée. Et sa mine, discrètement radieuse,
s'efforçait de garder la fierté d'une fille de bonne maison et la
supériorité d'une bergère prête à recevoir les hommages d'un berger
digne de ce nom.

Ses cheveux blonds, ondulés plus que de raison par Colbert, étaient bien
ceux qu'avait loués sans art M. de la Butière. Sans se croire autorisé à
mettre pied à terre, Florimond flétrit les procédés de ce butor. Il en
exprima tous ses regrets, se confondit en plates excuses, sans rien
perdre de son prestige, car déjà Marguerite admirait tout en lui. La
Butière en ce moment était également béni par Florimond et par la jeune
fille, puisqu'il surgissait comme un prétexte honnête à cet entretien
tant désiré.

C'est pourquoi Florimond avait, au premier temps, abordé ce sujet, après
les rapides et premières excuses pour cette audace d'adresser la parole
à une femme qu'il ne connaissait que par ce qu'un chacun proclamait de
ses mérites\,: «\,Souffrez que je me présente, puisque personne n'est là
pour se charger de ce soin. Mon indiscrétion me rend coupable, sans
doute, mais vous me devez pardonner, car, en vous voyant occupée à lire
cette merveilleuse Astrée dans la campagne sauvage, j'ai compris la
hauteur de vos sentiments et gagné le désir de m'incliner devant votre
sagesse. Vous êtes l'oracle qui m'indiquera le chemin, la lumière qui me
guidera pendant la route, l'étoile de la nuit\ldots{} Ah, mademoiselle,
laissez votre humble serviteur baiser la trace de vos pas\,!\,»

Marguerite ne sut pas mauvais gré à ce soupirant cavalier du peu de poli
de son langage\,: «\,Il ne possède pas encore, se dit-elle, cette
facilité dans l'expression que donnent seuls l'exercice et la lecture
assidue du grand Honoré d'Urfé. Le code de l'honnête amitié ne lui est
pas familier\ldots{} Je veux lui en enseigner tous les articles, et il
sera plus savant que moi.\,» Ainsi pensant, M\textsuperscript{lle} de
Primelles avait répondu d'une façon tout à la fois protectrice et
évasive. Hésitante et malaisée d'abord, la conversation se fit familière
et confiante. Ce fut Florimond le plus timide. Suivant avec un
sang-froid assez rare à son âge les recommandations minutieuses
d'Olivier, il affecta un trouble qui n'agitait point son cœur. Jamais
oiseau de carnage n'emprunta plus sournoisement les espèces de la
colombe pour pénétrer dans le pigeonnier. Le loup revêtit la peau de
l'agneau. Florimond réussit même à éteindre ce regard brillant dont la
flamme avait au premier contact communiqué son ardeur dévorante à des
coquettes expertes dans l'art d'aguicher les hommes. Il plaignit sa vie
triste et solitaire, son amour des champs, où errait son désir à la
recherche d'une âme pareille à la sienne, mais plus riche en qualités de
délicatesse, plus noble, plus ornée, et qu'il pût prendre pour modèle
sans se piquer de la jamais égaler. Mais, hélas\,! où, et comment la
trouver\,?\ldots{} Et voilà que le hasard le plus heureux, --- si l'on
peut être assez dénué d'esprit pour voir là l'effet du hasard, --- que
la fortune, plutôt, le conduisait aux pieds de celle dont le moindre
désir deviendrait pour lui un ordre, trop heureux si en l'exécutant il
pouvait garder cet espoir qu'elle lui permit d'obéir toujours à ses
lois\ldots{}

--- Je veux donc, dit Marguerite radieuse, vous mettre tout à l'heure à
l'épreuve. Si je vous demandais une grâce\,?

--- Les grâces viennent des reines, et moi je suis votre humble sujet.

--- Cela n'est pas répondre. Si je vous demandais\ldots{} comment
dire\,?\ldots{} un service\ldots{}

--- Mademoiselle, il est accordé d'avance.

--- Jurez ça\,!

--- Je le jure\,! Si je ne l'accomplis point, c'est que je n'en aurai
pas le pouvoir.

--- Je ne veux pas de restrictions\,!

--- J'écoute avant que d'obéir.

--- Il y a\ldots{} un\ldots{} certain\ldots{} --- M\textsuperscript{lle}
de Primelles hésitait\,; au fond, elle était abominablement troublée,
--- un certain Marin Labrande que vous retenez en prison\ldots{}

La figure de Florimond était de marbre. Les yeux baissés sur la crinière
de son cheval, il s'obligeait à ne pas sourire. «\,Vraiment Aimeri était
sorcier. Ou bien alors une mystification\ldots{} Mais pourquoi\,?\,»

Très vite, comme si elle craignait de ne pouvoir aller jusqu'au bout de
sa phrase, elle dit\,:

--- Je vous prie de le mettre en liberté\ldots{} S'il vous plaît\,!

Ces derniers mots furent prononcés avec une tremblante humilité qui
donna à Florimond un avant-goût de plus solides victoires. Il ne
répondit pas tout d'abord. Fronçant les sourcils, il parut n'accorder
son attention qu'aux oreilles de son cheval. Évidemment il hésitait, et
M\textsuperscript{lle} de Primelles, haletante, attendait. Enfin, elle
parla. Sa voix, un peu brisée, décelait son angoisse\,:

--- Eh bien, monsieur, vous ne le voulez donc pas\,?

--- Mademoiselle, --- fit Florimond avec une émotion contenue que lui
eût enviée le meilleur comédien, --- pourquoi me le demander à
nouveau\,? N'ai-je point juré\,? Cet homme sera en liberté dès que je
serai moi-même rentré.

Elle lui tendit la main, qu'il baisa avec une ardeur discrète. Le regard
triomphant de la jeune fille prouva à Florimond combien elle
s'enorgueillissait de sa victoire, et il se jura de lui faire payer un
tel prix que toute sa vie ne pourrait réussir à l'acquitter, vie de
honte, de désespoir et d'opprobre, et lui se réjouirait de ses larmes.
Ainsi le fils de la drapière Julie administrait l'avenir de la fille du
baron de Primelles, tué par son père à lui, le marquis de Bannes.

A la même heure, M\textsuperscript{lle} Catherine de Lépinière,
rencontrant Cottebleue qui s'en allait d'un pied léger à Houet, lui
tenait ce langage\,:

--- N'essaye pas de te sauver, Cottebleue, ou je jure Dieu que mon
écuyer te baillera autant de coups de bâton que t'en distribua naguère
M. Le Bouteiller. Allons, approche et m'écoute\,!

Cottebleue, voyant qu'il ne pouvait se sauver dans le sentier qui
courait entre les vignes, avança jusqu'à toucher le cheval de la jeune
fille et attendit, le chapeau à la main. Depuis deux jours, il ne savait
plus sur quel pied danser. Rabroué la veille par Florimond alors qu'il
en attendait des compliments et une récompense honnête, voilà
qu'aujourd'hui il se trouvait exposé à la colère de la belle-fille du
marquis. M\textsuperscript{lle} Catherine, en effet, ne semblait, pas de
bonne humeur. Sous son regard, où se lisaient le mépris et le courroux,
Andoche baissa la tête. Au vrai, une peur bleue travaillait le drôle
chaque fois que sa mauvaise chance le mettait en face de
M\textsuperscript{lle} de Lépinière. Et Cottebleue enviait à cette heure
la condition des lapins, ses protégés bien qu'il n'en eût pas la garde.
Au moins ces bêtes à longues oreilles pouvaient-elles, en cas de danger,
rentrer dans les profondeurs de leurs trous et y braver la fortune
contraire, tandis que lui était exposé, sans secours, aux violences de
parole et d'action de cette furie en jupons qui montait à cheval et s'y
tenait à califourchon comme un homme. Fort entre tous de la protection
de la marquise ou de Florimond, il ne répugnait à aucune besogne
vexatoire, opprimait, rapinait, se parjurait, payait d'audace et
terrorisait tout le pays. C'était bien. Mais en face de
M\textsuperscript{lle} Catherine il perdait tous ses avantages et
tremblait de peur. «\,Un mot d'elle, songeait-il, et le marquis, dont le
bras est plus long qu'une nuit d'hiver, m'obligera à vendre ma charge.
Alors il me faudra quitter le pays, où je ne pourrai plus vivre. Chacun
se vengera de moi\,; je serai peut-être assassiné\,!\ldots{} Qu'ont-ils
donc tous contre moi, et Marin est-il devenu subitement un saint pour
que sa juste prise m'amène de tels ennuis\,?\,»

Malgré sa venette, Cottebleue tenta de payer d'audace. Puisqu'il ne
pouvait s'enfuir dans les vignes à cause de M\textsuperscript{lle}
Catherine, d'André d'Archelet et du laquais, qui accompagnaient
celle-ci, pouvait-il au moins exciper de son mandat. Porteur d'exploits,
il détenait une part de l'autorité royale. Ce fut donc au nom de cette
autorité qu'il fit un effort pour élever la voix.

--- Toute révérence gardée, mademoiselle, je dois me rendre de ce pas à
Houet, puis à Lapan et aux Loges de Laiguillon, pour les devoirs de ma
charge. Et le temps me reste à peine d'y arriver avant le coucher du
soleil. Veuillez donc me laisser passer, et que Dieu vous garde\,!

--- Cottebleue, --- fit Catherine en portant son cheval de côté, --- tu
me réponds de Marin sur ta tête\,! Je t'ordonne de me dire comment tu
trouvas ce courage de l'arrêter sur les terres de M\textsuperscript{me}
de Primelles, ou plutôt qui te le donna.

Andoche pensa un instant à tourner les talons. Si le chemin lui était
ainsi barré du côté du Cher, rien ne semblait l'empêcher de se couler
dans les friches embroussaillées de la Rille. Il se reprocha de ne pas
l'avoir fait plus tôt\,; Catherine comprit son dessein et appela son
écuyer\,:

--- D'Archelet, fais-moi le plaisir de garder la route derrière le sieur
Cottebleue. Il faut que je lui parle.

Alors le porteur d'exploits s'excusa d'un ton impudent où perçait
néanmoins son inquiétude\,:

--- Est-ce ma faute, après tout\,? Je dois faire mon devoir. Monsieur le
baron est seigneur de Bannes, c'est lui qui commande en l'absence du
marquis\ldots{} Et puis ces histoires-là ne regardent pas les
demoiselles\ldots{} Serviteur, je suis pressé\,!\ldots{}

--- S'il tente de gagner au pied, roue-le de coups, d'Archelet, et que
Jeannet t'assiste avec son fouet.

Andoche Cottebleue, très rouge, cria qu'il était sous la protection des
justes lois du royaume. Il exhiba son brassard et son bâton, jura qu'il
porterait plainte devant le bailli de Lunery qu'il serait entendu,
dût-il en appeler jusqu'à la table de marbre, qui tranche, en dernier
ressort, des cas forestiers.

--- Tais-toi, mauvaise bête\,!\ldots{} Maintenant, rappelle-toi ma
question. À qui obéissais-tu en empoignant Marin derrière sa
maison\,?\ldots{} Si tu ne réponds pas, je te jure, moi, que le marquis
te chassera. Entends-tu, lâche et triste valet qui te charges des basses
œuvres du faux maître\,?\ldots{} Écoute-moi\,! Si Florimond ose traîner
Marin en justice, je viendrai témoigner en personne\,! Oseras-tu
soutenir que c'est Marin qui posa les collets de crin, quand tu sais
bien que\ldots{}

Andoche, en tant que Normand, saisit le côté avantageux de cet
interrogatoire où, entraînée par la générosité de son sang, la jeune
fille multipliait les demandes sans attendre les réponses. Retrouvant
son sang-froid, il dit avec dignité\,:

--- Mademoiselle, sur tout ce qu'il y a de plus sacré, je porte serment
que je ne le sais pas.

Catherine ne comprit pas que Cottebleue s'en tirait à peu de frais et
que ce n'était pas là répondre. Toute à son idée, elle cria\,:

--- Pourquoi donc as-tu déclaré que Marin était le coupable\,?

--- Ce n'est pas moi, mademoiselle. Le pauvre diable l'a avoué,
puisqu'il n'est pas allé contre\ldots{}

--- Il a eu tort\,! Retiens donc, imbécile, que c'est moi qui les avais
tendus, ces collets\ldots{} pour m'amuser. Je les avais payés à la mère
Labrande. Symphorien en fabrique souvent pour moi\ldots{}

Cottebleue eut l'impression que la jeune fille mentait\,: «\,Mais dans
quel intérêt\,? Que vient-elle chercher là dedans\,? Il y a autre
chose.\,» Il répliqua avec une indifférence admirablement jouée\,:

--- Du moment que c'est comme ça, il ne me reste plus qu'à en aviser M.
Florimond\ldots{}

Catherine rougit, et l'autre continua\,:

--- Pour sûr, mademoiselle, vous avez le droit de vous amuser avec des
collets. Toutefois c'est bien la première fois que je vois une
demoiselle chasser à la façon des braconniers.

Catherine rougit plus fort. Les larmes lui vinrent aux yeux, tant elle
haïssait le mensonge. Pour cacher son trouble, elle exagéra sa colère\,:

--- Maroufle\,! Je te vais couper la figure\,!\ldots{} En attendant, tu
as frappé, empoigné, ligoté Marin, et cela sur la terre de Primelles\,!
Tu as de la chance que le vieux ne t'ait pas entendu\ldots{}

--- Oh\,! le vieux Bouteiller ne passe jamais par là\,!

A cette interruption cynique, Catherine leva son fouet sur le drôle, qui
para machinalement du coude un cinglon qui ne vint pas.

--- Cottebleue, si tu parles en ces termes du baron de Mordicourt, je
commanderai à mes gens de te bâtonner\ldots{} Je te dis, méchant valet,
que si le vieux Labrande t'avait entendu tu restais mort sur la
place\ldots{}

--- Ça se serait bien pu, fit Cottebleue en ricanant.

Mais l'écuyer André d'Archelet prit alors la parole\,:

--- Dis plutôt, l'ami, que vous étiez assez nombreux pour ne rien
craindre. Et tu t'en es vanté hier encore dans la grande écurie.

--- Bien ça\,! s'écria Catherine heureuse de reprendre l'avantage.
Ainsi, tu avoues, face de traître, que tu avais emmené du monde avec toi
pour enlever Marin\,!

Impatienté, humilié, très sérieusement inquiet sur le dénouement de
cette aventure, il nia, s'embarrassa dans ses finasseries. Cependant
Catherine et ses gens raillaient et le menaçaient. Cottebleue perdit la
tête. Trépignant de fureur, il jeta par terre son chapeau, le foula aux
pieds, brandit son bâton\,:

--- En voilà assez, n'est-ce pas\,?\ldots{} Sommes-nous ici au
tribunal\,? Et qui êtes-vous pour me commander\,?\ldots{} Vos
témoignages, voilà ce que j'en fais\,!

Son geste ignoble s'acheva sous une grêle de coups de fouet. Il
tourbillonna, se garant la tête et la face, se heurta aux chevaux,
glissa au pied du talus, hurlant de peur et de rage\,:

--- Ah\,! vous n'allez pas me tuer\,!\ldots{} Je suis sous la garde du
roi\,!

Et les coups de pleuvoir\,: «\,A toi, pleutre\,! Pied plat\,! Faux
témoin\,! Butor\,!\,»

Alors il demanda merci, promit de parler, avoua toute sa pauvre
conspiration, dénonça ses complices, les garçons de la ferme des
Aubrois. Il déplora son succès, s'aplatit, incrimina Florimond, La
Butière, la marquise et Tourouvre\,: «\,S'il avait su, il n'aurait pas
risqué de danser une pareille gaillarde pour un petit écu\,!\,» Il alla
jusqu'à s'engager à délivrer Marin, sachant bien qu'il n'en avait plus
le pouvoir. Catherine l'écoutait avec une telle expression de mépris que
Cottebleue put à peine se retenir de pleurer. Et il se promit une
inexorable vengeance.

Enfin, M\textsuperscript{lle} de Lépinière dit à l'écuyer et au
laquais\,:

--- Placez-le entre vous deux et liez-lui les mains\,!\ldots{} Là\,!
Serrez son bâton dans la poche de ma selle, il viendra le réclamer chez
moi. Tais-toi, Cottebleue, ou, j'en jure par ma défunte mère, que ta
scélératesse eût certainement lassée, je te fais mourir sous le fouet.
Et en route\,!

Sous son chapeau, Cottebleue sentit la sueur mouiller son front et ses
cheveux se dresser de terreur. «\,Où M\textsuperscript{lle} Catherine
prétendait-elle l'emmener\,?\,» On était sur les terres de Primelles,
maintenant, et l'on se dirigeait vers la bergerie de Tonlieu. C'était là
que, deux jours auparavant, Marin avait été saisi et enlevé par
Cottebleue et les garçons de Landry Vaillard. La chose avait été si
vivement exécutée que, bien qu'on l'eût attaqué à cinquante pas de sa
maison, le fils aîné de Symphorien le sorcier ne fut pas entendu des
siens. Le père était encore aux champs, les femmes préparaient le
souper, la nuit tombait, tout demeurait désert. Serré, étranglé par
seize mains, Marin avait mordu, rué, frappé en vain. Cottebleue et les
sept valets de ferme étaient restés vainqueurs, au prix de deux
mâchoires cassées, d'un crâne fêlé, de trois poignets foulés et d'un
genou froissé. Si Marin eût trouvé le temps de tirer son couteau, le
porteur d'exploits et ses acolytes n'auraient pas gardé l'avantage. Mais
il fut assailli à l'improviste, pliant sous le poids de trois gros
fagots qu'il avait chargés au bois des Usages.

La lande gardait encore les traces de la lutte. Un petit pommier galeux
se penchait à moitié rompu. Des lambeaux de drap demeuraient accrochés
aux ajoncs\,; et, entre les maigres touffes d'herbes foulées, des
empreintes de pieds s'imprimaient nettement dans la terre durcie, car il
n'avait pas plu depuis cette bataille rustique.

--- Tu vois, Cottebleue, dit Catherine, le terrain de ta victoire.
Regarde-le bien, mon homme, tu ne seras libre qu'après avoir demandé,
sur cette même place, à Marin de te pardonner, et à deux genoux encore.

Cottebleue recommença de multiplier ses serments\,: «\,Qu'on lui donnât
seulement une heure, et Marin serait libre comme l'air, foi
d'Andoche\,!\,»

--- Et, foi de Lépinière, je t'assure, moi, Andoche Cottebleue, que tu
ne sortiras pas de chez le vieux Labrande que son fils ne lui soit
rendu. Allons, marcheras-tu\,?

Alors, Cottebleue fut mordu par la peur aux entrailles. Il se désespéra.
«\,Tout, plutôt que d'être livré à Symphorien, le berger qui lisait dans
les astres, et à sa femme Jeannette, la jeteuse de sorts, à Honorin, le
tondeur, et aux autres.\,» Il invoqua encore les justes lois, les
privilèges de son état, le bailli de Lunery et le seigneur de Bannes. Il
rendit M\textsuperscript{lle} de Lépinière, le sieur d'Archelet, le
laquais Jeannet, responsables de son sort devant Dieu, le roi et les
juridictions inférieures, et s'effondra sur le banc où la vieille femme
de Symphorien avait coutume de filer sa quenouille. La Baude accourut\,;
ses yeux d'or luisants s'arrêtèrent sur le porteur d'exploits. Et la
chienne gronda, menaçante, fonça sur le misérable. Avant qu'on la pût
écarter, elle lui avait levé un morceau du drap de ses chausses et un
échantillon de peau du mollet.

--- Mère Jeannette, cria Catherine, mère Jeannette, viens-t'en sans
tarder\,! Je ne te ramène pas ton fils, mais en voici un qui le vaut
presque. Regarde\,!

La vieille paysanne parut sur le pas de sa porte. Derrière, par-dessus
son épaule abaissée par l'âge, brillaient les larges yeux noirs et les
cheveux ébouriffés de sa fille Francine, brune et ambrée, telle un joli
pain bien doré. Puis sortirent ses deux frères, Honorin, le second
berger de Primelles, et Médard, garçon de charrue qui promettait d'être
un rude homme. Ainsi livré, les poings liés, à ses particuliers ennemis,
Cottebleue pensa défaillir, et ce que dit Catherine ne lui remonta pas
les esprits.

--- Regardez-le tous, mes braves gens\,! Le voilà, ce valeureux
Cottebleue, soutien de la justice\,! Il a tant d'honneur qu'il fit entre
mes mains un grand serment de ne vous plus quitter avant que Marin
revienne. Ficelez-le bien, pourtant, parce que son assiduité à
s'acquitter de sa charge est telle qu'à la première occasion il
retournerait à Lunery pour chercher du papier timbré et vous écrire une
promesse dans les formes.

Pareil à un \emph{ecce homo} de paroisse campagnarde, Cottebleue, le nez
baissé, demeurait exposé aux regards malveillants de la famille
Labrande. Honorin, qui ne se recommandait point par la mansuétude, opina
pour qu'on envoyât le porteur d'exploits au fond de la mare, afin d'y
assigner les grenouilles. Son cadet Médard était d'avis qu'on le donnât
en pâture aux porcs que l'oncle Baudel engraissait dans l'enclos voisin.
Mais, d'autorité, la vieille Jeannette imposa silence à ses fils.

--- Méchant valet, dit-elle de sa voix creuse, tu me réponds de mon
garçon. S'il lui arrive malheur, tu mourras. Qu'on lui attache les pieds
et qu'on le porte au grenier\,! Le père décidera à son retour.

Cottebleue, grinçant des dents, menaça tous les Labrande des fourches de
justice\,: «\,On verrait bien qui aurait le dernier. Il représentait le
seigneur de Lunery, qui saurait bien le défendre\,!\,»

Il s'arrêta de parler et poussa un cri de douleur. Sous prétexte d'aider
ses frères à le ficeler, Francine, la fille ébouriffée en corset de
toile, lui avait sournoisement piqué la jambe avec une aiguille à laine
et Honorin serrait les menues cordes d'une telle force que les bas
drapés de Cottebleue entraient dans sa peau.

Ainsi ces petits se vengeaient-ils à leur manière du misérable suppôt de
Florimond. Et ils allaient l'emporter, insensibles à ses protestations
juridiques dont ils se moquaient sans délicatesse, quand la mère
Jeannette dit\,:

--- Attention, enfants, voici le père\,!\ldots{} Taisez-vous et
l'écoutez\,!

Mais dans le crépuscule où l'on distinguait à peine les gens et les
choses quelqu'un s'avança rapidement. Et des exclamations de joyeuse
surprise s'élevèrent.

--- Non, ce n'est pas Dieu possible\,!

--- Ce n'est pas le père\,!\ldots{}

--- C'est Marin\,!

--- Mon Marin\,? Vous riez\,!\ldots{}

--- Non, non, c'est bien lui\,!

C'était lui, en effet, ayant encore sur le dos les habits qu'il devait à
la libéralité de Florimond. Alors on sauta de joie\,; Francine dansait,
et les ailerons de sa coiffe voltigeaient ainsi que des papillons
blancs. On s'embrassait. Catherine, sur sa petite jument, battait des
mains. Raide et muette, la vieille Jeannette se laissa accoler par son
fils aîné. Mais des larmes roulaient sur son visage plus âpre qu'une
antique muraille. Combattant son émotion où elle ne voyait que
faiblesse, cette vieille femme, enfant de la glèbe, se dressait dans
l'ombre grandissante comme un emblème de la patiente et sauvage fierté
de ceux qui firent de la seule terre et leur raison d'être et leur
force\,; --- leur raison d'être qui est de nourrir l'humanité, leur
force qui est dans l'effort ininterrompu du quotidien travail
courageusement supporté sans espoir d'un avenir meilleur. Et le vieux
Symphorien, qui venait de rejoindre le groupe, n'était ni moins sérieux
ni moins superbe.

--- Voilà, dit-il, appuyé sur son haut bâton d'épine, voilà celle qu'il
faut remercier\,! La demoiselle de Lépinière est un ange parmi les
hommes. Souffrez, belle et bonne demoiselle Catherine, vous que nous
voudrions tous saluer ici pour notre dame et maîtresse, souffrez que je
vous salue en vous bénissant\,!

Et, pliant un peu son genou raide, le vieux berger inclina sa longue
figure blanche et grise sur les mains de la jeune fille, qu'il baisa. Et
les pleurs du vieillard et de l'enfant se mêlèrent sur les gants de peau
de chien. Doucement, elle tendit son front pour que le bon vieux
l'embrassât entre ses boucles brunes, elle le tendit avec tant de grâce,
que chacun joignit les mains comme qui prie et cria\,:

--- Noël pour M\textsuperscript{lle} Catherine\,! Que Dieu l'assiste et
la garde de tout mal\,!\ldots{} C'est la chère demoiselle qui a tout
fait. Par elle, voici Marin de retour parmi nous. Çà, Francine, n'as-tu
pas un bouquet de roses\,?\ldots{} Ma fille, cours au potager\,!

Et Catherine, étouffant de joie et de tendresse, avait beau crier\,:

--- Mais, bonnes gens, je n'y suis pour rien, hélas\,! Ce bonheur est
venu sans moi\,!

Tous reprenaient en chœur\,:

--- Noël à la gentille demoiselle\,! Tous ici nous sommes prêts à mourir
pour vous\,! Donnez-nous-en l'occasion\,!\ldots{}

Alors, Catherine les supplia de la laisser parler.

--- Mon père Symphorien, et vous, ma mère Jeannette, vous avez retrouvé
votre fils, et malheureusement sans que je vous aie pu aider.
Maintenant, si vous m'aimez, il faut rendre la liberté à
Cottebleue\ldots{}

Un murmure désapprobateur s'éleva parmi les enfants\,: «\,Eh quoi\,! on
allait lâcher cette bête puante\,!\ldots{} Pas avant de l'avoir
écorché\ldots{} bâtonné\ldots{} lardé\,!\,»

La voix du vieux Symphorien domina ce maussade concert de regrets\,:

--- Qu'entends-je\,?\ldots{} Et ce Cottebleue n'est pas encore libre\,?
Osez-vous bien attendre avant que d'obéir\,?

Tout se tut. Puis dans la nuit qui descendait, opaque et humide,
montèrent les menaces de Cottebleue, qui s'éloignait à grands pas. Alors
la vieille Jeannette parla à son tour\,:

--- Holà\,! les garçons\,! Chacun une torche, et qu'on s'arme\,! Vous ne
quitterez la demoiselle que lorsqu'elle sera rentrée sous son
toit\ldots{}

\hypertarget{chapitre-vii}{%
\chapter{CHAPITRE VII}\label{chapitre-vii}}

Quinze jours entiers Andoche Cottebleue garda le lit, où le cloua une
mauvaise fièvre, car, la pluie n'ayant pas cessé de tomber pendant qu'il
courait à travers champs au risque de s'égarer dans les ténèbres
épaisses, il était arrivé à Lunery trempé jusqu'aux os. Malade encore
plus de peur que de froid, il trouva bien juste la force de se coucher.
Une heure après, le porteur d'exploits délirait entre sa femme
Philomèle, qui levait les bras au plafond sans aider à rien, deux
matrones expertes dans la fabrication des purgatifs et des onguents, et
le barbier de Lunery, qui le saignait en toute abondance.

Cottebleue se rétablit. Aussitôt il s'occupa d'exercer sa vengeance.
Puisque c'était comme ça, il combattrait Marin par les moyens directs.
Compter sur le seigneur de Bannes, autant bâtir sur le sable mouvant. Ce
Florimond de malheur avait prouvé, en relâchant Marin contre toute
justice, ce qu'on en pouvait attendre. Avec un pareil écervelé,
capricieux et bizarre, il valait mieux ne plus lier partie. Cottebleue
agirait sans le prévenir. Porter plainte contre M\textsuperscript{lle}
de Lépinière, contre André d'Archelet ou le laquais Jeannet était
dangereux et complètement inutile. Procéder contre Marin tout bonnement,
telle était la simple sagesse.

Fort de son mandat, Cottebleue leva, à ses frais, tant la haine impose
silence à la plus sordide avarice, copie de citations au greffe du
bailliage. Avec un seul de ces papiers il avait le droit de requérir
aide et protection de tout venant. Armé de ses grimoires, il requit
quelques sergents de justice, leur promit une récompense honnête, et
marcha à leur tête sur Tonlieu, où il espérait bien mettre la main sur
l'audacieux braconnier qui bravait son autorité et sa légitime colère.
Dans la bousculade de la grande battue aux loups que M. de Montenay
dirigeait sur les trois domaines, rien de plus facile que de cueillir
Marin au milieu des rabatteurs.

«\,Une fois que je le tiendrai, ce gredin, je ne serai pas si sot que de
le livrer à M. Florimond. Je le traînerai à Lunery, et le bailli en fera
son affaire. Il y a là, dans ma sacoche, douze plaintes et autant
d'ordres d'informer contre ce drôle. C'est largement suffisant. Le
bailli n'osera pas lui donner la clef des champs, le tribunal d'Issoudun
connaîtra des délits par ma diligence.\,»

Déjà le vindicatif Normand voyait son ennemi Marin dans les cachots
d'Issoudun, puis dans ceux de Bourges, voire de Paris, aux geôles de la
Conciergerie, et enfin sur les côtes de Provence, où archers à cheval et
argousins poussaient la chaîne des forçats, dans la poudre aveuglante
des routes sans fin, jusqu'à la mer. Là, Marin devrait logement et
nourriture à la générosité du roi\,: «\,Tu le faucheras, le grand pré,
canaille, avec la manille au pied et une bonne rame aux
poings\,!\ldots{} Et, allez, le nerf de bœuf\,!\,»

Pour mieux poursuivre son équitable vengeance, Cottebleue songeait à
laisser le Berry pour Marseille ou Toulon et à prendre rang parmi les
comites. Quel plaisir d'émoucher les épaules nues de Marin avec un bon
nerf de bœuf, tandis que le drôle, désespéré, haletant, devrait suivre
le mouvement et ne point perdre la mesure, sous peine d'avoir la tête
fracassée par la poignée de la rame d'arrière. Enthousiasmé par ce
spectacle, Cottebleue allongeait le pas, encourageait son monde\,:

--- Et allez donc, mes enfants\,! Le nerf de bœuf, le nerf de bœuf des
galères, voilà qui est bon pour ces vauriens\,!

Et les sergents, émoustillés par le vin gris que Cottebleue leur avait
libéralement versé et payé au cabaret de Lunery, répétaient en
s'appliquant à marcher droit et à ne pas trop fouler les récoltes\,:
«\,Le nerf de bœuf, voilà qui est bon, en vérité\,!\,»

La bonne fortune souriait à Marin. Elle voulut qu'il tombât à la
descente du coteau de la Rille sur l'homme promis par lui aux galères de
Sa Majesté. Agenouillé, Marin, le couteau à la main, était en train de
dépecer un gibier\,: «\,Attention, vous autres\,!\,» Et Cottebleue, se
glissant entre les ajoncs, toucha de son bâton blanc, que
M\textsuperscript{lle} Catherine de Lépinière lui avait rendu quand il
partit de la maison des Labrande, l'épaule du braconnier\,: «\,Je t'y
prends, vermine, à voler les chevreuils de M. de Montenay\,!\ldots{} À
moi, les enfants, empoignez-le sans crier gare\,!\ldots{} Et
s'il\ldots\,»

Andoche n'acheva point sa phrase. Entre Marin et lui deux cavaliers
surgirent, et Marin, penché sur sa pièce, ne se dérangea pas. Suant
d'angoisse, l'homme au papier timbré avait reconnu Florimond et le baron
de Mordicourt en tenue de veneurs, le jeune homme vêtu de drap rouge, le
vieillard de drap gris, et tous deux l'épée de chasse au côté. Derrière
se pressaient des valets de chiens, des porteurs d'arquebuse, des
piqueurs avec leurs épieux, dont les larges fers en feuilles de laurier,
croisés à leur nœud par la billette, étincelaient au soleil de juin.

Florimond et le baron de Mordicourt, ou si l'on préfère M. Le Bouteiller
en personne\,! Le vieux Blaise, l'oncle de M\textsuperscript{lle} de
Primelles, raide et cérémonieux sur son cheval vair aussi vieux que son
maître\,! M. Le Bouteiller, qui l'avait bâtonné impunément, lui, Andoche
Cottebleue, le porteur d'exploits, bras droit des huissiers et soutien
de toutes justices\,! Et M. Le Bouteiller causait amicalement avec
Florimond\,! C'était à n'y rien comprendre\,!\ldots{} Ou bien Cottebleue
ne comprenait que trop\,: une conspiration générale, ourdie contre lui
spécialement, réunissait tout le pays en ce lieu afin de consommer sa
ruine.

Et, pour donner à Cottebleue le dernier coup de massue, voici que Marin,
dégageant un grand loup gris qui gisait sous lui, en offrait la patte à
M. Florimond, baron de Chézal-Benoît, représentant le marquis de Bannes,
seigneur de Lunery en l'occurrence\,! Florimond donna deux pièces d'or à
Marin\,: «\,C'est bien, mon brave\,! Jamais je ne vis mieux travailler.
N'oublie pas que la tête est pour M. de Montenay, qui mène la chasse.
Décolle-la proprement et la lui porte\,!\ldots{} En vérité, monsieur de
Mordicourt, votre Marin est un veneur accompli\ldots\,»

Cottebleue aurait voulu rentrer sous la terre qui buvait le sang du
loup. Moins heureux que les sergents qui s'étaient prudemment perdus
dans les broussailles, il demeurait bien à découvert. M. Le Bouteiller
l'interrogea aussitôt sans douceur\,:

--- Que cherches-tu ici, vermine\,?\ldots{} Est-ce vous, monsieur, qui
l'avez appelé\,?

--- Non, certes, monsieur, répliqua Florimond, qui rougit de contrariété
et foudroya le malencontreux Andoche du regard. Par ma foi, je n'en sais
pas plus long que vous\,!\ldots{} Peut-être at-il été engagé parmi les
rabatteurs\,? En tout cas, il n'est pas à sa place. Vous plaît-il qu'on
le renvoie\,?

--- Cela me plaît, monsieur, et je veux y veiller moi-même. Holà\,!
maroufle, qui te permit de te mêler à la battue avec ta vilaine figure
et les ridicules insignes de ta triste profession\,? Çà, parleras-tu\,?

Cottebleue balbutia, se gratta la tête, cacha son bâton sous son
balandran, usa de moyens dilatoires\,: «\,Si tel était le bon plaisir de
ces messieurs, il s'en irait\ldots{} Venu en simple curieux\ldots\,»

Un valet de chiens eut l'indélicatesse de l'interrompre\,:

--- Espèce de menteur\,!\ldots{} En curieux\,!\ldots{} Avec son bâton et
cinq sergents\,!\ldots{} Si on me permet de découpler les chiens, je
vous ramène les sergents tout à l'heure\ldots{}

Un rire malveillant couvrit la voix de Cottebleue. Il ne renonça
pourtant pas à se défendre\,: «\,Chacun avait le droit\ldots\,» Mais on
le huait, et l'on parlait de lâcher les chiens sur les sergents. Le
porteur d'exploits, gagné par la peur, commit la maladresse de parler de
ses fonctions\,: «\,Si tel était le bon plaisir de ces messieurs, il
remettrait la chose à un autre jour. Il avait cru bien faire en\ldots\,»

M. Le Bouteiller lui coupa la parole avec rudesse\,:

--- Quelle chose, face de Mathieu\,?\ldots{} Et que prétendais-tu
faire\,?\ldots{} À qui en as-tu ici, avec ta sacoche, et ton bâton, et
ton papier timbré, dont je vois un échantillon dépasser de ta manche\,?

Andoche perdit tout à fait la tête. Il avoua stupidement que son papier
timbré était pour Marin. Et, la lâcheté détruisant toute son habituelle
prudence, il s'écria, les larmes aux yeux\,:

--- Dame, vous comprenez, mes bons messieurs, la justice doit suivre son
cours. Les ordres de M. le bailli sont formels. Les sergents qu'il m'a
adjoints\ldots{}

Élevant la voix au-dessus des rumeurs de la cohue qui s'esclaffait comme
un cent de mouches, le vieux baron garda son sérieux et demanda\,:

--- Où sont-ils\,?

Où ils étaient\,? Cottebleue s'en doutait bien, et il eût souhaité les
rejoindre. Partis, envolés les sergents\,! Ils s'étaient dissous en
fumée. Et l'abandonné songeait\,: «\,Quelle volée de bois vert je vais
recevoir, Dieu seul le sait, à moins qu'il ne m'assiste\,!\,» Il ôta
poliment son chapeau, voulut prendre congé\,:

--- Serviteur, messieurs\,!\ldots{} Chacun son ouvrage\,!\ldots{} Je
vous salue bien humblement\,!

Mais le baron de Mordicourt n'en avait pas ainsi décidé\,:

--- Attends ici, maraud\,!\ldots{} Attends ici mes ordres\,! Qui t'a
permis de partir\,?\ldots{} Marin, mon brave, prends ta baguette et
frotte-moi les épaules de M. Cottebleue jusqu'à ce que j'aie dit\,:
«\,Assez\,!\,»

Le rêve d'Andoche où il se réjouissait à épousseter Marin avec un nerf
de bœuf tournait décidément court. La réalité était tout autre.

Marin présentait alors à Florimond la tête du loup, sanglante, hérissée,
avec ses yeux glauques, voilés et encore hagards. Un pied de langue
jaillissait entre les crocs formidables. Derrière les oreilles pointues,
une triple collerette de crins raides encerclait le cou.

--- C'est un des plus beaux loups que j'aie vus\,! s'écria le jeune
homme avec une joyeuse admiration\,; admiration aussi feinte que le
compliment. Car, n'aimant pas la chasse, Florimond n'était pas grand
connaisseur en loups. Il examina encore la lourde tête d'où le sang
tombait goutte à goutte, et reprit\,:

--- Cette tête accompagnera dignement les dix ou douze qui ornent la
porte d'entrée de Montenay. Tiens, Marin, prends encore ceci.

Et il donna au fils de Symphorien une pièce d'or. Distrait par la tête
du loup que Florimond offrait à son admiration, M. Le Bouteiller oublia
le porteur d'exploits, qui essaya de gagner au pied. Mal lui en prit,
car le grand-oncle de M\textsuperscript{lle} Marguerite ne regardait le
loup que d'un œil\,:

--- Empoigne-le, Marin, et le rosse.

Cottebleue se plaça alors sous la protection de Florimond et saisit sa
botte avec une ardeur telle qu'il faillit le désarçonner.

«\,Stupide animal\,!\,» cria le jeune baron. Et il allait détacher à
Cottebleue un coup d'étrier dans les côtes, quand il se retint. «\,Non,
non, pensa-t-il, c'est le moment de ne mécontenter personne. Je suis
venu ici pour me gagner des amis\ldots{} Montrons à ces gens combien je
suis bon prince. Procédons avec prudence et habileté.\,»

A parler franc, cette habileté lui avait été inculquée par M. Aimeri
d'Olivier, qui ne cessait de lui répéter depuis huit jours\,:
«\,Endormez-moi tout ce monde par une bienveillance sévèrement
observée\,! Défiez-vous du premier mouvement\,: il ne pousse jamais
qu'aux sottises. Montrez-vous homme conciliant sur tous les terrains\,!
On ne prend pas les mouches avec du vinaigre. Quand tout le pays frémira
d'émotion en citant vos actions charitables, vous pourrez faire le
diable à quatre. Personne ne croira vos accusateurs. Endormez-les, vous
dis-je, et après les fricassez suivant votre volonté\,!\,»

C'est pourquoi Florimond intercéda pour Cottebleue auprès du baron de
Mordicourt\,:

--- Je vous demande, monsieur, la grâce de ce malheureux. Voyez, il est
aux trois quarts mort de peur, et il s'agrippe à ma botte comme le noyé
serre la branche qu'il a pu empoigner. Je réponds de lui. Sur ma foi, il
ne reparaîtra plus sur vos terres. Accordez-moi cela, et vous me
trouverez en tout empressé à vous plaire. Je me tiens d'ailleurs pour
votre dévoué serviteur. Ce que vous ordonnerez sera bien.

--- Mon Dieu, monsieur, vous me retirez l'envie qui me tenait de
bâtonner ce drôle\ldots{} Qu'il s'en aille donc, puisque cela vous
agrée\,!

Cottebleue respira plus librement. Mais il n'osa pas quitter l'abri
tutélaire qu'était la botte de Florimond, car la baguette de Marin,
vraie baguette de veneur, en beau coudrier non écorcé, puisqu'on allait
entrer dans l'été, et longue de sept pieds du gros bout à la pointe, lui
inspirait plus que du respect.

Florimond remercia très poliment le baron de Mordicourt, secoua sa
jambe, que le porteur d'exploits dut enfin lâcher, et parla avec une
bonhomie condescendante\,:

--- Écoute, Cottebleue, et retiens bien ce que je vais dire. Tu vois
bien ce bon garçon, --- et du manche de son fouet il désignait Marin,
dont la main robuste tenait à bonne hauteur sa baguette, --- j'entends
que partout où tu le rencontreras tu sois poli avec lui et passes ton
chemin sans le molester\,!\ldots{} Et toi, Marin, à compter de ce jour
je te donne permission d'accompagner mes gardes et de chasser avec eux
sur mon bien, sans exception de lieux\ldots{} Mais ménage mes chevreuils
et mes cerfs, car je n'en ai plus guère\ldots{}

M. Le Bouteiller se récria. Il trouvait la chose excessive\,:

--- Vous le gâtez, monsieur, vous le gâtez\,! Il ne fera plus que
musarder\,!

Florimond reprit avec une déférente gracieuseté\,:

--- Laissez-moi, monsieur, ce plaisir de rendre heureux un brave homme
que vous aimez\ldots{} Et puis, c'est un peu mon intérêt, après tout. Je
suis mangé par les loups\,!\ldots{} À partir d'aujourd'hui mes moutons
dormiront tranquilles.

--- S'il en est ainsi, monsieur, je n'ai plus qu'à approuver et à me
réjouir pour ce mauvais sujet de Marin\ldots{} Allons, remercie M. le
baron de Chézal-Benoît\ldots{} seigneur de Lunery.

M. Le Bouteiller ne prononça ces derniers mots qu'à regret. Il fronça
même le sourcil. Feignant pourtant d'oublier les terribles discordes qui
divisaient Bannes et Primelles, il reprit d'un ton enjoué, en
s'adressant à Florimond de rechef\,:

--- Ma parole, vous me plaisez, monsieur\,; et je regrette d'avoir tant
attendu pour vous connaître\ldots{} Eh\,! là, Marin, ne nous prendras-tu
pas un autre loup\,?

Cottebleue s'était esquivé sans demander son reste. Quant à Marin, tout
fier de son succès, il déclara que la journée n'était pas finie. Qu'on
longeât les coteaux en tirant sur Lapau, et certainement on trouverait
encore quelque belle occasion\,! Pour lui, le plus gros de la besogne
était fait. Ce loup gris, qu'il avait rabattu jusque sous l'arquebuse de
Florimond, était\,une bête comme on n'en tuait pas tous les vingt ans\,:
«\,Voilà, monsieur, plus de cinq ans que je le traque, et toujours il
m'a échappé. Heureusement qu'il n'a pu gagner l'eau, sans quoi nous ne
l'aurions pas eu, je vous le garantis. Ce compagnon-là nous a saigné
plus de cent moutons, à lui seul, en deux saisons\ldots{} sans compter
ceux qu'il a enlevés chez vous\,!\,»

Les sons aigres des huchets, que l'on entendait du côté de l'Échalusse,
apprirent aux chasseurs que M. de Montenay continuait la battue de ce
côté avec M\textsuperscript{lle} Catherine et M. de Mauny d'Anrieux.
Florimond demanda à M. de Mordicourt s'il ne vaudrait pas mieux
rejoindre le gros de la troupe\,:

--- À vous de commander\,: j'obéis.

Flatté par ce témoignage de respect, le vieux gentilhomme s'inclina\,:

--- Puisque vous voulez bien prendre mon avis, je crois qu'il vaut mieux
suivre Marin, avec nos quelques valets et nos chiens de tête. Nous avons
chance de réussir en battant les coulées qui sont familières à ce
garçon. Il m'a parlé d'une louve et de ses louveteaux.

--- À vos ordres, monsieur. Que Marin nous guide\,! Passez en avant, je
vous prie.

M. Le Bouteiller, ayant salué de son petit chapeau à l'antique, piqua
des deux, et à sa suite tout le monde s'enfonça dans les broussailles.
Alors les sergents qui avaient si courageusement soutenu Cottebleue se
levèrent du taillis où ils reposaient à plat ventre et reprirent le
chemin de Lunery, en riant avec une impudence qui n'avait rien de commun
avec l'importance de leurs fonctions.

Au-dessus des maigres buissons, des genêts et des ajoncs qui couvraient
les terres incultes au pied des petits coteaux où se rangeaient les
échalas inégaux des vignes, M. Le Bouteiller dressait sa longue et
maigre personne vêtue à la mode du temps passé. Habillé de chamois et de
drap gris de fer, avec le pourpoint tailladé, le haut-de-chausses étroit
ceint du lodier à coupons de carpe, les bottes étroites, les manches
plates à épaules remontées par des bourrelets, il portait sa mine
basanée, à barbe en coin et à cheveux ras sur une collerette à plis
pressés et sous un bonnet en façon de mortier, orné d'une enseigne
d'émail, le tout avec une morgue véritablement espagnole.

Cela s'expliquait par le long séjour qu'avait fait M. Blaise Le
Bouteiller, baron de Mordicourt, dans les camps de Sa Majesté
catholique. Partout où la Ligue avait tenu tête au défunt roi, on
l'avait vu payant de sa personne. De ce vieux débris des guerres civiles
le corps était blessé en dix endroits. Seule, la tête était restée
bonne, si l'on ne voulait s'arrêter à deux balafres qui se croisaient à
hauteur du front, et à un trou laissé par une balle de mousquet à
l'angle de la mâchoire, du côté gauche.

Né en 1564, M. Le Bouteiller n'avait que huit ans lors de la
Saint-Barthélemy. On eût pu croire cependant. à la haine dont il
favorisait ceux de la religion, qu'il avait été mêlé à cette affaire,
dont il ignora le détail, car sa mère le garda avec elle, dans sa
chambre, rue du Battoir, pendant que son père courait par les rues dans
l'espoir vain de rétablir l'ordre. M. Le Bouteiller père tenait état de
gentilhomme auprès du chevalier d'Angoulême. Il est indéniable que son
plus amer regret fut de ne pas avoir joué, faute d'années, sa partie
dans ce mémorable tumulte. Mais les guerres qui suivirent et durèrent
plus de vingt ans lui fournirent des compensations majeures. A la
journée de Coutras, où il chargea avec les gendarmes de la compagnie
Carrouges, il eut la tête fêlée d'un coup de mousquet, à ce moment même
où un boulet le décoiffait de son armet et où un autre tuait son cheval
entre ses jambes. Il demeura deux jours sous les morts. À peine guéri,
il reprit du service et guerroya sans repos ni trêve contre la maison de
Bourbon jusqu'au jour où la paix générale l'obligea de planter ses
choux.

Il possédait dans le Vexin une mauvaise gentilhommière plus qu'aux trois
quarts ruinée, où il se terra pour y vivre de pain noir et de fèves. Les
mauvais sentiments qu'il nourrissait contre Henri IV --- et le baron de
Mordicourt ne l'appelait jamais autrement que le parpaillot --- ne
devinrent pas meilleurs pour le fils qui succéda à ce grand roi tué par
l'eustache de Ravaillac. Il ne reporta pas cependant sur Louis XIII sa
haine tout entière, puisqu'il choisit le cardinal ministre pour en
supporter une bonne moitié.

Malheureusement les seuls bons troubles dans lesquels il eût trouvé à
satisfaire son humeur aventureuse furent ceux où les protestants
s'allièrent aux Anglais contre le roi. Ainsi pris entre les huguenots
qu'il abhorrait et le roi qu'il ne pouvait pas sentir, le baron de
Mordicourt se réfugia dans l'inaction et continua de manger du pain
noir, des fèves et de ce maigre gibier qu'il tuait d'occasion sur sa
terre dont un enfant de cinq ans pouvait faire, sans se hâter, le tour
complet en une petite heure.

Cependant il s'était laissé embaucher, avec quelques gentilshommes sans
ouvrage, quand la reine mère leva contre son fils l'étendard de la
révolte. Sans avoir jamais compris les raisons de cette pauvre émeute
qui prit sa fin ridicule à l'échauffourée des Ponts-de-Cé, il paya de sa
personne en cette journée mémorable où l'on ne fit pas de mal à un chat,
trouva moyen de recevoir un coup de pique à la jambe, et rentra chez lui
profondément découragé par le manque d'entrain des générations
présentes. Celui qui avait combattu en bataille ouverte contre le petit
roi de Navarre à Coutras, contre le même petit roi, devenu roi de
France, à Arques, à Ivry, sous les murs de Paris, où il le moqua
publiquement à l'allaire de Corbeil, et en vingt autres places, avec les
lansquenets, les gendarmes français, les Italiens, les Wallons et les
Espagnols, faillit mourir de chagrin après la débandade des Ponts-de-Cé.
Son chagrin se changea en maladie noire lorsque le cardinal ministre,
arrivé au pouvoir, après avoir abandonné la bonne dame de reine mère, se
mit à persécuter sournoisement la noblesse. Le ligueur irréductible
déclara la guerre au prêtre qui déshonorait son habit en s'associant aux
rancunes du Bourbon. Avant que Richelieu n'exerçât, en son nom, le
pouvoir, Louis XIII, fils du parpaillot, demandait à sa noblesse la
fidélité sans plus. Maintenant, il en exigeait la soumission.

La gentilhommière du Vexin était, heureusement, à trop grande distance
des oreilles du cardinal, qui pourtant écoutait et entendait tout, pour
que les propos libres, et partant malsonnants, du baron de Mordicourt
pussent parvenir jusqu'à elles et les offenser. Sans quoi ledit baron
aurait pu en pâtir malgré sa chétive position. Quelques amis de
Mordicourt, confidents de ses secrètes pensées, et, à son exemple,
ennemis de la maison régnante, comme de toute autre maison qui aurait
régné, essayèrent de l'aboucher avec le duc d'Orléans, dont les
conspirations de comédie suffisaient à flatter leurs espoirs désordonnés
et imprécis. Il refusa de prendre parti pour un Bourbon contre un autre
Bourbon, sous ce prétexte qu'à l'arbre on connaissait les fruits et que
de cet arbre toutes branches étaient également mauvaises. M. Le
Bouteiller, regrettant des illusions dont il lui eût été malaisé de
définir la nature, vécut seul, avec l'admiration de ces temps abolis où
les sentiments et les actions héroïques couraient les rues, tandis que
le temps présent se recommandait par l'absence de tout esprit
chevaleresque.

Sans se donner la peine, du reste inutile, de réfléchir, sans comprendre
que c'était surtout de sa jeunesse qu'il éprouvait les regrets beaucoup
plus que des luttes politiques où elle s'était consumée, le baron de
Mordicourt vit ses cheveux blanchir et ne renonça pas à ses rêveries
chimériques. Ce vieil homme ne voulut rien comprendre ni rien savoir de
la nouvelle société qui se formait et qui tendait à substituer aux
libertés anarchiques que durent tolérer les derniers Valois, faute de
les pouvoir détruire, un principe d'ordre général et de régularité sous
l'observation des lois.

Triste et solitaire, ce féodal sans famille, ni seigneur, ni Vassaux,
avait atteint ses soixante-deux ans quand sa nièce
Marie-Célestine-Françoise de Saudres, baronne de Primelles, que l'épée
du marquis de Bannes faisait veuve, s'adressa à lui dans sa détresse et
le pria d'accepter, avec le gouvernement de sa maison décapitée, la
tutelle de ses enfants sans père.

M. Blaise y consentit volontiers, dans l'espoir de former un gentilhomme
en la personne de son petit-neveu, Louis-Antoine, et poussé aussi par ce
secret désir de protéger et d'aimer auquel il n'avait jamais pu
sacrifier dans sa vie aventureuse et vagabonde. Sa pauvreté et son
complet isolement l'avaient ensuite condamné à refouler ce désir au plus
profond de son cœur.

Le caractère du baron de Mordicourt était de ceux qui défient l'examen
des gens superficiels dont le monde est pour les trois quarts composé.
Son extraordinaire désintéressement laissait froids les indifférents et
blessait ceux qui ne vivent que pour se pousser vers les situations
profitables. À qui jugeait suivant les apparences, M. Blaise était une
manière de fou pas trop dangereux, mais dont il était préférable
d'éviter le contact, d'autant que l'extraordinaire liberté de ses
jugements déplaisait à cause de l'espèce de responsabilité que cette
façon de s'énoncer inflige en quelque sorte aux auditeurs. À moins
qu'ils ne soient directement menacés dans leurs intérêts, leur vanité ou
leurs plaisirs, les gens n'aiment point prendre parti. S'ils se
décident, contraints et forcés, c'est toujours contre le plus faible,
c'est-à-dire contre celui dont on sait qu'il ne possède pas la capacité
de nuire.

Trop bon et aussi trop facile dans le tréfonds, M. de Mordicourt
apportait tous ses soins à cacher ce qu'il considérait comme des
faiblesses. Et la dureté tout apparente de son écorce empêchait de
reconnaître l'excessive tendresse de son cœur. C'est qu'il était
ombrageux et bizarre, ne se fiait pas à quiconque, et molestait ainsi
ces personnes d'élite dont la seule occupation est de provoquer des
confidences pour s'en forger des armes, car l'on ne sait jamais sous
quel vent tourneront les choses, et il est bon de prendre ses
précautions contre ses meilleurs amis.

Ainsi le baron Blaise, tenu pour passionné, violent et brutal, avait
éternellement été la dupe des ambitieux réfléchis qui l'avaient employé
comme un instrument qu'on rejette dès qu'il ne peut plus servir.
Toujours à la peine, jamais on ne le vit aux honneurs. Ceux-là mêmes qui
chantaient ses louanges en prenaient acte pour le dénigrer plus
commodément. On l'aimait à ce point qu'on s'en autorisait pour dire de
lui pis que pendre, et surtout répéter qu'il était bien heureux de
n'avoir pas de besoins. De telle sorte qu'on se trouvait dispensé par
cela même de la pénible nécessité de lui rendre service. Sa brutalité
était passée en proverbe alors qu'on eût été assez embarrassé pour en
fournir des preuves sérieuses. Au vrai, on n'aurait pu trouver homme,
catholique ou protestant, auquel il eût vraiment fait du mal. En dehors
du champ de bataille et des duels où on l'avait provoqué, jamais M. de
Mordicourt ne versa le sang. Néanmoins sa réputation était celle d'un
homme sanguinaire. Sa franchise, son courage et son indépendance
gênaient.

M. de Mordicourt, une fois installé à Primelles, où les trompettes de la
renommée le précédèrent pour annoncer ses méfaits, reconnut vite
l'ingratitude de la tâche à quoi il s'était attelé. Pris entre sa nièce,
qu'il vénérait comme une sainte et dont il admirait et la piété fervente
et l'énergie que ne rebutait aucune privation, et deux enfants dont
l'éducation était trop singulière pour qu'il ne s'en avouât pas choqué,
ce chef sans autorité d'une famille ingouvernable ne s'appliqua plus
qu'à la défense des intérêts matériels du domaine de Primelles. Son
activité, que ne pouvait refroidir sa verte vieillesse, fut dès lors
tendue tout entière vers les améliorations pratiques. Il s'adonna à
l'agriculture, retrouva son goût pour la chasse dans les friches
giboyeuses, vécut à cheval, courant en un même jour de Saint-Ambroix à
Corquoy et de Mareuil à Tonlieu. Il surveilla les fermiers, les bergers,
s'occupa des foins, des seigles et des vignes. Heureux de pouvoir enfin
boire à sa soif et manger à sa faim dans cette maison où son industrie
ramenait peu à peu l'abondance, il se laissa aller à une vie grasse où
il se serait épanoui si sa famille lui eût montré un semblant d'amitié.

Mais, s'il réussit à se faire craindre et même à se faire aimer par les
tenanciers et les paysans du domaine, il n'y réussit pas au château.
Louis-Antoine le redoutait assez peu en dehors des exercices de
l'escrime, et Marguerite le méprisait comme grossier et barbare malgré
sa culture d'humaniste qui transportait d'admiration le curé de
Primelles. Quant à la baronne sa nièce, la seule chose qui l'intéressât,
c'était les progrès de son fils dans la science difficile de l'épée.

Le vieux Blaise avait trop longtemps compté sa vie comme rien pour tenir
à bien haut prix celle des autres. Il estimait cependant que la carrière
de spadassin n'était point celle qu'un gentilhomme devait choisir. «\,Ma
nièce Françoise, pensait-il, s'imagine à tort que le fils du marquis de
Bannes vit uniquement dans l'espoir de donner un mauvais coup d'épée à
Louis-Antoine pour se venger de l'exil de son père. Elle a tort
assurément. Les hommes de ce temps ne sont plus d'un tempérament aussi
valeureux. Ce Florimond, dont MM. de Montenay et de Mauny me racontent
de temps à autre les escapades, est un galant qui ne se remue ici-bas
qu'à se divertir. Et on lui prête de bien noirs projets. Il s'en ira à
la cour ou à l'armée, en quête d'occasions de pousser sa fortune, et
augmentera le nombre des chiens couchants qui entourent le Bourbon et
son ministre en robe rouge. Louis-Antoine ne fera jamais, ou je me
trompe fort, un duelliste bien redoutable. Il tire l'épée comme un
autre, mais pas mieux. Il a surtout du goût pour la chasse et pour la
flânerie. Mais qu'y faire\,? Sa mère ne veut pas que je m'en
mêle\,!\ldots{} Et ainsi du reste. Si Louis-Antoine passait seulement
sur ses auteurs le dixième du temps que sa sœur perd à lire des romans,
cela n'en irait que mieux. Cette fille est folle sans remède. Vaine,
enflée de vent poétique, méprisant tout, elle rougit de sa pauvreté sans
être capable d'y apporter le remède salutaire du travail. Est-ce bien,
après tout, par simplicité qu'elle s'habille à la campagnarde et joue à
la bergère\,?\ldots{} Ne serait-ce point plutôt par mauvaise honte de
porter de simples habits\,?\ldots{} Et quand je pense que depuis que
j'entrai ici sa sainte femme de mère se contente de la même
robe\,!\ldots{} Tout cela, c'est l'époque\,!\ldots{} Et aussi la faute
du Bourbon\,!\ldots{} De notre temps, les filles des petits
gentilshommes ne se croyaient pas tenues à vivre parmi les livres, ainsi
que des princesses. Elles occupaient leurs dix doigts, devenaient de
bonnes femmes de ménage\ldots{} Aujourd'hui\,!\ldots{} Allez donc parler
raison à ces pécores\,!\ldots{} Autant en emporte le vent\,!\,»

Parfois M. de Mordicourt se demandait encore si sa nièce Françoise ne
nourrissait pas, en poussant ainsi son fils Louis-Antoine vers les
exercices d'escrime, des projets soigneusement cachés. Que cette veuve
ne pût apaiser dans les abîmes mystiques de la dévotion le trouble de
son cœur, qu'elle ne vécût que pour la vengeance, cela lui paraissait
insensé. «\,Ainsi cette douce Françoise rêverait de lâcher
Louis-Antoine, quand le temps serait venu, sur Florimond\,? Je ne le
croirai jamais\,!\,»

A la vérité, quand il se sentait ainsi envahi par le brouillard des
rêveries, M. Le Bouteiller appelait le bon vin à son aide, et la
généreuse liqueur des coteaux du Cher réussissait, presque en tous
temps, à ramener ses pensées vers un horizon moins pluvieux.

Parfois, cependant, la mélancolie s'affirmait, triomphante. Alors le
vieux baron s'accusait, sans réussir à se trouver coupable. S'il eût été
moins vieux, il aurait appelé Florimond pour le tuer et frapper ainsi le
marquis de Bannes dans ses plus chères affections. Mais, s'il était trop
vieux pour se battre utilement, il l'était aussi assez pour aller au
fond des choses. Le marquis --- il le savait de bonne source ---
n'aimait que peu ce fils, légitimé par faiblesse, dont il n'avait
jusqu'ici tiré aucune satisfaction pour son orgueil de père. Et voilà
pourquoi le vieux gentilhomme se demandait jusqu'à quel point il était
opportun de risquer la vie de son petit-neveu Louis-Antoine, qu'il
aimait sans daigner le dire, contre l'héritier assez contestable du
meurtrier de son défunt neveu, le baron de Primelles, qu'il n'avait
jamais vu ni connu.

C'est qu'avec l'âge la sagesse avait forcé l'entrée de son cœur. Le
vieux coureur d'aventures qui avait semé son sang, sans en compter les
gouttes, sur les champs de bataille de la Ligue, se découvrait avare du
sang de ceux qu'il aimait. Au fond, il ne croyait pas un accord
impossible, car pour l'homme d'action tout est préférable à la paix
armée qui use les forces et amoindrit lentement les courages. Donc le
baron de Mordicourt avait saisi, et sans répugnance cette occasion qui
s'était offerte à lui de lier connaissance avec Florimond pendant la
battue aux loups. En agissant ainsi, rien ne l'empêchait de voir venir.
Sa première impression lui avait semblé bonne. Celle de Florimond fut
encore meilleure, mais, pour parler ainsi que les géomètres, dans un
sens diamétralement opposé.

--- Tu n'as pas idée, disait-il le soir même à M. Aimeri d'Olivier en
lui rendant compte de ses faits et gestes, tu n'as pas idée de la
simplicité de cet antique imbécile. Les pères, tuteurs et autres barbons
de comédie ne sont rien au prix de ce hobereau ridicule. Je l'ai
entortillé à mon gré, et la chose me semble à présent sans intérêt, car
elle fut vraiment trop facile.

--- Vous avez eu cette prudence de ne pas lui souffler mot de sa
nièce\,?

--- Ah çà, Olivier, me prends-tu pour une bête\,?

--- Loin de moi une pareille pensée\,!\ldots{} Mais je redoute toujours
quelque éclat de cette magnifique gaieté. Enfin tout est pour le mieux.
Vous avez conquis l'oncle, éteint la défiance. Ah\,!\ldots{}
dites-moi\ldots{} M\textsuperscript{lle} Catherine n'assista point à cet
entretien\,?

--- Non pas, certes\,! Elle est restée avec Montenay et Mauny d'Anrieux,
et aussi avec Tourouvre et La Butière.

--- Savez-vous si elle est rentrée au château\,?

--- Oui, je le sais. Elle est tellement rentrée que la cour des écuries
retentissait de ses cris quand je revins moi-même. Elle défendait à ce
pauvre Tourouvre de reparaître jamais devant ses yeux\ldots{} Quelque
nouvelle sottise\,!

Florimond ne vit pas la figure retournée et livide de son précepteur,
parce qu'il s'occupait de se faire débotter. Barrois emporta les bottes,
et M. Aimeri, qui avait repris son sang-froid, demanda à son élève s'il
ne croyait pas le temps venu de recommencer à investir Marguerite de
Primelles\,:

--- Vous la négligez trop. Depuis quinze jours que vous l'avez
abandonnée, elle doit se dessécher, tel un arbre sans eau. Et ne
craignez-vous pas que l'isolement où vous la condamnez refroidisse son
amoureuse ardeur\,?

--- J'attends, mon cher Aimeri, que son oncle Bouteiller lui chante mes
louanges\ldots{}

Florimond accompagna ces paroles d'un claquement de doigts à ce point
désinvolte, et conquérant, et gracieux, que sa mère, qui entrait pour
prendre sa part de cet entretien honnête, se pâma d'aise et cria, prête
à tomber en faiblesse, tant son orgueil maternel était flatté\,:

--- Mais regarde-le, Aimeri, regarde-le\,! Il est divin, plus que
divin\,!\ldots{} Que n'ai-je des mots pour exprimer ce que je
pense\,!\ldots{} Florimond, vous êtes le phénix, le\ldots{} Ah\,! que
sais-je\,?

Et la marquise s'assit, sans prendre garde aux regards consternés et aux
signes que lui envoyait le poète.

--- Madame, fit alors Florimond en se rengorgeant, il m'est venu une
idée\ldots{}

--- Pas possible\,!

Cette observation que laissa échapper imprudemment M. Aimeri ne fut pas
relevée, parce que la mère et le fils étaient, à cette heure et comme
toujours, trop pleins d'eux-mêmes pour égarer leur attention sur cet
homme étriqué, que l'envie ne réussissait pas à augmenter de volume. Il
en crevait cependant avec sa mine jaune comme un coing, coiffée d'une
calotte noire. Florimond, bombant le torse, continuait\,:

--- Une idée, madame, qui est si belle que vous voudriez l'avoir
trouvée, je gage\,! Et j'ajoute que sans votre précieux concours cette
idée ne saurait aboutir.

--- Parlez, Florimond, parlez\,! Ne nous faites languir\,!\ldots{}
Aimeri, ne remue pas ainsi et écoute\,!

Florimond toussa, tira sa barbe en pinceau, caressa complaisamment sa
moustache et sa turquoise, et daigna s'expliquer.

Il s'agissait de donner un bal à Bourges. Le conseiller Harant prêterait
son hôtel, sa femme payerait les violons, quelque autre la collation\,;
lui, Florimond\ldots{}

--- Vous, mon amour, vous viendrez\,!\ldots{} Ne trouvez-vous pas cela
suffisant\,?

--- La difficulté, madame, à ne vous rien cacher, et je la tiens pour
grande\ldots{}

--- Ah\,! mon fils, que vous vous exprimez élégamment\,!\ldots{}
N'est-ce pas, Olivier\ldots{} que\ldots{}

Elle avait enfin compris les signes désespérés d'Olivier. Mais, trop
maîtresse d'elle-même pour déceler son trouble, elle ne s'interrompit
pas de parler\,:

--- Que c'est plaisir de l'entendre\,!

--- Assurément, madame, assurément. Monsieur votre fils, en cela pareil
à sa mère, et pour user de votre propre langage, est un phénix, le seul,
l'unique phénix\,!

Florimond laissa s'évaporer cet encens et reprit en souriant\,:

--- La difficulté, madame, sera d'attirer à ce bal la petite
Primelles\ldots{} Qu'elle me voie danser, et la pauvrette en demeurera
férue au cœur jusqu'à la fin de ses jours.

--- Cette idée, s'écria la marquise qui avait retrouvé son sang-froid et
s'éventait avec son mouchoir trempé d'ambre, cette idée est tout
simplement sublime\ldots{} Aurais-tu trouvé cela, toi\,?

M. Aimeri, ainsi interrogé, se garda bien de répondre la vérité. L'idée
était de lui. Toutefois, l'avait-il soufflée avec une telle discrétion
que Florimond se l'était appropriée de bonne foi sans même s'en rappeler
l'origine.

--- Je pense tout comme vous, madame. Encore faut-il que
M\textsuperscript{lle} Marguerite assiste à ce bal\,: \emph{Hic jacet
lepus}\ldots{} Hem\,!\ldots{} Je veux dire ainsi que c'est là où le bât
nous\ldots{} le bât me blesse\,! Jamais ces Primelles ne laisseront la
douce enfant danser à Bourges.

--- Eh\,! qu'en sais-tu, Olivier\,? répliqua aigrement la marquise.
J'ai, grâce à Dieu, conservé d'assez belles relations à la ville pour ne
pas désespérer d'enlever cette affaire. J'y songerai cette nuit\ldots{}
Bonsoir.

Tout en l'accompagnant avec force salutations, Aimeri trouva moyen de
murmurer à l'oreille de Julie\,:

--- Tourouvre a manqué son coup.

Elle pâlit, se mordit les lèvres\,:

--- L'imbécile\,!\ldots{} Que Florimond n'en sache rien, et que
Tourouvre se taise\,!

Aimeri s'inclina plus bas encore, baisa la main qu'on lui tendait\,:

--- Il se taira.

C'est peut-être à cette histoire que la marquise pensa, car elle ne
dormit pas de la nuit. Sans doute aussi pensa-t-elle au bal de Bourges,
car le lendemain, comme son fils la saluait dans une allée du parc, elle
l'accueillit par ces mots\,:

--- Vous pouvez, mon cher Florimond, commander les violons. Je les veux
payer de mon argent. Et votre belle viendra, ou j'y perdrai mon nom.

«\,Cela pourrait bien lui arriver quelque jour si elle continue à
manigancer ainsi.\,» Mais, de cette réflexion tout intérieure, M. Aimeri
ne daigna donner communication à personne. Il eût été d'ailleurs désolé
de voir s'arrêter en si beau chemin une intrigue où il prêtait la main
encore plus pour la gloire que pour l'argent. M. Aimeri d'Olivier
participait de la nature des singes, qui font le mal pour le seul
plaisir de mal faire et se moquent de se couper les doigts pourvu qu'ils
brisent en mille pièces quelque précieux vase de cristal.

M\textsuperscript{lle} Catherine de Lépinière, à qui cette comparaison
juste en soi fut empruntée, redoutait M. Aimeri plus que Florimond et
tous ses estafiers réunis. Mais Louis-Antoine, à qui elle communiquait
ses secrets dans les guérets sauvages de Tonlieu, ne s'intéressait pas
au précepteur de Florimond. Ce qui l'intéressait, c'était la battue aux
loups de la veille, où son amie lui avait défendu de paraître.

--- Est-il vrai, Catherine, que tu avais un habit tout rouge avec des
galons d'or et de l'or sur toutes les coutures, un habit, enfin, pareil
à celui de Florimond\,?

--- L'as-tu donc vu\,? demanda-t-elle avec une vivacité dont elle ne fut
pas maîtresse.

--- Oui, pendant qu'il parlait à l'oncle Bouteiller, en contrebas du
coteau de la Rille. J'étais là, moi, caché dans l'herbe, et si près
d'eux que j'aurais pu tirer les chiens par la queue.

--- Si l'on t'avait vu, Louis-Antoine\,!

--- Eh bien, je me serais laissé glisser dans le grand trou, derrière le
rocher. Malin qui m'y eût déniché\ldots{} Avec tout ça, Catherine, tu ne
me dis pas pourquoi tu n'as point voulu que je me mêle à la chasse.
L'oncle voulait pourtant m'emmener\ldots{}

--- Et toi, tu ne m'as pas dit ce que racontaient ton oncle et
Florimond.

Sans remarquer que la jeune fille éludait sa question en y répondant par
une autre, Louis-Antoine raconta ce qu'il avait entendu et vu\,:
l'intervention de Florimond en faveur de Cottebleue, la faveur où le
fils de la marquise tenait Marin, l'amicale attitude de l'oncle
Bouteiller.

--- Si tu avais pu les écouter, ta surprise n'eût pas été mince. Ils
causaient comme une paire d'amis. Un moment, Florimond a invité l'oncle
à venir chasser chez lui.

--- Tu en es bien sûr\,?

--- Me crois-tu donc sourd\,?\ldots{} Et l'oncle l'a quitté, à la fin de
la battue que j'ai suivie en me faufilant dans les broussailles, avec de
belles paroles. Encore un peu, et ils allaient s'embrasser\,!

--- Et Marin\,?

--- Oh\,! Marin, je l'ai rencontré ce matin. Il danse de joie parce que
Florimond lui a donné licence de tirer son gibier dans tout le domaine.
Il ne jure plus que par le jeune maître de Bannes et déclare qu'il
passerait par le feu pour lui être un tant soit peu agréable.

--- Et ta sœur Marguerite\,?

--- Oh\,! celle-là, qu'il pleuve, qu'il vente, qu'il grêle, elle se
promène chaque jour du côté de Lunerette avec son mouton, son livre et
Colbert.

«\,Que complote Florimond\,? se demandait Catherine. Il cherche à se
rapprocher des Primelles\,; dans quel intérêt\,? Pour l'heure, il ne
semble pas en avoir après Louis-Antoine\ldots{} Marguerite, peut-être\,?
Mais est-ce probable\,? Et comment se relie à ses machinations
ténébreuses le coup qu'il fit hier tenter contre moi\,? Je le gêne, rien
de plus certain.\,»

Cela n'était que trop vrai. M. de Tourouvre avait essayé d'assassiner
Catherine pendant la battue aux loups. Se souciant peu de chasser avec
l'équipage de Florimond, qui devait rejoindre le baron de Mordicourt
dans les bois de Toux, M\textsuperscript{lle} de Lépinière était partie
pour la Vergne avec son écuyer André d'Archelet, ses chiens et ses
piqueurs à elle. À la Vergne, elle trouva le rassemblement au complet,
et aussi M. de la Butière. Celui-ci avait demandé au maître des chiens
la permission de suivre, et M. de Montenay ne crut pas devoir lui
refuser ce qu'il avait accordé à dix gentilshommes des environs.

Catherine, flairant un danger dans la présence de l'estafier de
Florimond, se promit de ne pas quitter M. de Montenay et se rangea botte
à botte avec lui au moment du départ. Elle n'avait pas couru trois cents
pas que sa selle tourna, et, sans M. de Mauny d'Anrieux, qui la
flanquait à droite, elle fût tombée rudement et de haut, car son cheval
était grand. Le gros des chasseurs s'arrêta juste à temps pour ne pas la
fouler. André d'Archelet accourait pour lui porter secours, quand M. de
Montenay s'aperçut que la sangle avait été tranchée sous le panneau avec
un fer dont le fil devait être irréprochable, tant la coupure était
nette. Il garda ses réflexions pour lui. On répara vivement le dégât, et
Catherine remonta sur sa selle. Elle ne parut pas attacher d'importance
à l'accident, et même son enjouement fut remarqué. Tout en plaisantant
avec son tuteur Montenay et M. de Mauny d'Anrieux, elle cherchait à se
rappeler certaines particularités qui l'avaient frappée lors de
l'arrivée de La Butière\,: «\,C'est lui qui a coupé la sangle, j'en suis
sûre, maintenant\,!\,» En effet, pour entrer dans la salle basse de la
Vergne, où M. de Montenay offrait une collation matinale, Catherine
avait laissé son cheval dans la cour de la maison de chasse. Et M. de la
Butière était resté presque seul, au milieu des chevaux, très occupé à
vérifier divers détails du harnachement de sa propre monture.

Catherine ne doutait plus. Mais l'absence de preuves la condamnait au
silence, quoique le sourire et l'air soucieux de M. de Montenay fussent
pleins de sous-entendus. Fuyant leurs regards, M. de la Butière s'était
cantonné à la queue de la troupe. Puis il se laissa distancer et on ne
le revit plus de la journée.

À la croisée des mauvais chemins qui mènent des fermes de l'Échalusse à
Tureau, un accident plus singulier arriva\,; et celui-là non plus ne
pouvait être mis sur le compte du hasard. M. de Montenay, averti que
deux loups, dont une bête accompagnée, allaient débucher des
broussailles que l'on battait du côté de Chevrais avec la huée, posta
Catherine à trente pas en avant, en lui recommandant de tirer droit,
sans s'écarter\,: «\,Vous êtes trop grande chasseresse pour que je me
croie obligé à d'autres recommandations. Ici l'on met pied à terre. Deux
piqueurs sont là derrière leurs épieux, en cas de retour, car il y a un
vieux loup, paraît-il. Et voici Robineau, mon maître garde, qui vous
assistera en cas de besoin. Au ralliement, revenez sur les chevaux.\,»

Ainsi placée, à pied, dans un lieu découvert, M\textsuperscript{lle} de
Lépinière, avec son habit de chasse rouge et doré, ne pouvait, en bonne
conscience, se confondre avec une bête rousse ou grise. Il advint
cependant qu'à cet instant même où elle tirait, et assez heureusement
pour le blesser, sur l'un des loups qui se suivaient en trottant la tête
basse et en secouant d'un coup de gueule qui fauchait les chiens assez
malavisés pour rompre la distance, un coup de feu brilla et résonna en
face d'elle. Le plomb traversa son chapeau au sommet de la forme, rasa
le plumet et donna contre une pierre. Tremblant d'émotion encore plus
que de fureur, le vieux garde Robineau voulait foncer sur le maladroit.
De celui-ci Catherine entrevit la tête empanachée qui plongea dans les
seigles, bien loin, au-dessus du talus.

La sévère ordonnance de la chasse défendait à M\textsuperscript{lle} de
Lépinière et au garde de quitter leur poste. Ils demeurèrent donc en
place, et Catherine répara adroitement son chapeau de manière à cacher
les traces de la balle. Mais Robineau, qui avait vu les deux trous qui
traversaient le feutre, serra les poings et grommela\,: «\,Ce n'est pas
là du plomb à loup. Je jure Dieu que le coquin sera pincé, ou je
mourrai\,!\,» Et, s'éloignant, il quêta, derrière la jeune fille, parmi
les rocailles. Quelques minutes après, il revenait avec une balle de
calibre aplatie qu'il présenta sur la paume de sa main\,: «\,Pauvre
demoiselle, le coup était bien pour vous\,! Quelle
abomination\,!\ldots{} Gardez-la, mademoiselle, gardez-la comme
preuve\,!\ldots{} Le meurtrier ne m'échappera pas, je vous en fait
serment\,!\,» Catherine, ayant pris le plomb, dit alors au garde
stupéfait de son courage\,: «\,Non, Robineau, tu te tairas\,! Et ton
serment, mon brave, sera de ne parler à personne sur terre de ce
coup\ldots{} Plus tard, je te dirai le nom de l'homme\ldots{}
Aujourd'hui comme demain, tu te tairas, si tu m'aimes.\,»

M\textsuperscript{lle} de Lépinière ne se vantait pas. Elle eût pu
nommer l'homme. Il s'appelait M. de Tourouvre, et elle l'avait bien
reconnu. Quand Catherine rentra au château de Bannes, la première figure
qu'elle vit dans la cour de la grande écurie fut celle du gentilhomme
aux gages de Florimond. Il venait de descendre de cheval, et une
arquebuse de carabin était accrochée à sa selle. Il eut l'audace de
saluer la jeune fille et de lui tenir l'étrier quoique André d'Archelet
fût là pour ce faire.

Tout en acceptant l'aide, Catherine dit tranquillement\,:

--- D'Archelet, donne-moi la carabine de Tourouvre qui est suspendue à
son harnachement.

Et elle sauta légèrement à terre sans lâcher l'épaule de M. de
Tourouvre, qui n'osa pas s'en aller. Elle prit l'arme des mains d'André
et cria\,: «\,Tourouvre, valet trop bas pour qu'on te châtie, fuis-t'en
de ma présence et ne reparais jamais devant moi\,!\,» Interdit,
l'estafier de Florimond passa du rouge au vert et au blanc, baissa la
tête et s'éloigna sans mot dire.

Maintenant, avec l'arquebuse et la balle, elle tenait aussi l'homme.
Toutefois, prudente et circonspecte autant qu'assurée dans son courage,
elle ne voulut rien brusquer et se promit de tirer l'affaire au clair
sans scandale. La tentative criminelle de La Butière serait moralement
prouvée quand elle le voudrait. Le regard de M. de Montenay lui en avait
dit long, sans paroles\,: «\,Le jour viendra, songeait-elle, où les
misérables tomberont dans les filets qu'ils tendent autour de moi. Ils
commettront quelque faute grossière, et je les aurai à merci\,!\ldots{}
Tant qu'ils ne toucheront pas à Louis-Antoine, je les laisserai en
repos\ldots{} Mais qu'ils n'y viennent pas\,! Pour lui, pour le
défendre, je me changerais en tigresse\,!\,»

Et c'est pourquoi, en quittant Louis-Antoine, elle lui renouvela ses
recommandations habituelles\,: «\,Évite Florimond\,! Je te défends de le
rencontrer, tu m'entends\,!\ldots{} Pour l'amour de moi, je te le
commande. Il ne faut pas, Louis-Antoine, que vos regards se croisent.
C'est déjà trop de respirer le même air. Devrais-tu te tapir dans les
broussailles quand il passe, je t'ordonne de le faire. Et, plus tard, tu
comprendras\ldots{} Louis-Antoine, puisque tu sais le latin,
rappelle-toi la devise de mon tuteur Montenay, car elle est fort
belle\,: \emph{Gaudet patientia duris}\ldots{} Allons,
traduis\,!\ldots{} Quoi\,? Tu es jaloux de Montenay\,?\ldots{} Pauvre
sot, comme si je pouvais aimer un autre que toi\,!\ldots{} Montenay\,!
Ma joie serait de le marier à ta sœur\,!\ldots{} Allons, embrasse- moi,
je me sauve\,!\,»

Louis-Antoine la regarda partir, ainsi que chaque jour, avec une obscure
tristesse. Ce garçon était semblable aux fleurs qui ouvrent leur corolle
sous les baisers du soleil et les referment à son coucher. Son soleil
était Catherine. Il ne vivait que quand elle était là, dans l'herbe, à
lui tenir compagnie. Il lisait dans ses yeux les secrets des choses.
Tandis que, lorsqu'elle n'était plus là, tout perdait sa signification.

Et, regagnant son pauvre logis où l'attendaient invariablement
\emph{Saint Augustin} et la leçon d'escrime, il songeait\,:

«\,Elle ne parle jamais sans raison, et je gagne toujours à l'écouter.
C'est clair. Si je lui avais obéi il y a tantôt trois semaines je
n'aurais pas laissé mes collets en place, et Marin ne serait pas tombé
aux mains de Cottebleue et des gardes\ldots{} Non, pas des
gardes\ldots{} Ils sont les amis de Marin parce qu'ils ont peur de
lui\ldots{} D'ailleurs, je n'y comprends rien\,! Voilà maintenant que
Marin est l'ami des Bannes, de la Drapière et de son fils Pontaillan. Et
encore, pourquoi Pontaillan\,? Il est baron de Chézal-Benoît, c'est son
droit d'être appelé ainsi\ldots{} N'empêche que le vieux Symphorien l'a
rudement secoué, Marin, pour s'être laissé habiller et payer par
Florimond\,! La mère Jeannette ne décolère pas. Elle jette des sorts du
côté de Bannes. Quelque jour, Symphorien tarira leurs sources, ils
n'auront plus de bonne eau à boire\ldots{} Et la mère Jeannette me fait
des signes\ldots{} En voilà, des mystères\,!

«\,Que veulent-ils\,? Je n'en sais rien. Croient-ils que Florimond veut
m'assassiner\,?\ldots{} D'abord je me défendrais\,: je suis fort\ldots{}
Et au fond nous nous moquons l'un de l'autre. Seulement, Florimond a de
l'argent, tandis que moi je n'en ai pas, et mes chausses sont
trouées\ldots{} Et puis après\,? Qui cela gêne-t-il\,?\ldots{} Il y a
autre chose\,: son père a tué le mien, il y a fort longtemps de
cela\ldots{} Est-ce donc une raison pour qu'il me tue aussi\,? Je n'ai
pas peur de lui\,!\ldots{} Mais Catherine me défend de parler de tout
ça, et même d'y penser\,: «\,Louis-Antoine, me dit-elle, tu auras ton
heure\,!\,»\ldots{} Quelle heure\,? Celle où je me marierai avec
elle\,?\ldots{} Ça, c'est un secret\,!\,»

Ainsi livré en proie à des rêveries vagues et dont il s'avouait
incapable de tirer quoi que ce fût de rassurant, le jeune baron de
Primelles marchait dans la campagne en dessinant, suivant sa coutume,
des passes d'escrime avec son bâton, sa fidèle épée rouillée restant
accrochée à sa hanche. Et pensant tantôt à \emph{Saint Augustin} et au
curé qui lui enseignait l'amour du prochain, l'horreur du duel, tantôt à
son oncle Bouteiller dont revenait sans cesse la maxime favorite que
tirer l'épée est le premier devoir d'un gentilhomme, Louis-Antoine se
sentait extraordinairement fatigué.

\hypertarget{chapitre-viii}{%
\chapter{CHAPITRE VIII}\label{chapitre-viii}}

Malgré ses embarras d'argent, la marquise Julie avait trouvé de quoi
stipendier La Butière et Tourouvre. Au premier elle avait compté deux
cents écus, et au second trois cents, ces deux sommes à valoir sur
l'entreprise qu'ils devaient tenter contre M\textsuperscript{lle} de
Lépinière. En cas de succès, La Butière toucherait encore trois cents
écus, et Tourouvre six cents, car les risques qu'il courait étaient
considérables. En somme, couper les sangles d'une selle n'est pas un
méfait tellement noir --- et encore le faut-il prouver --- qu'il puisse
mener son auteur au gibet. Tandis que loger à une demoiselle quelques
onces de plomb dans le corps, voire par imprudence, y conduit son homme
sans rémission.

M. Aimeri d'Olivier, après en avoir conféré longuement avec la marquise,
s'était chargé de traiter l'affaire avec les deux braves. Il l'avait
menée avec son habituelle prudence. Prenant chacun d'eux en particulier,
il laissa entendre à La Butière combien il faisait peu de cas de
Tourouvre, et à M. de Tourouvre la petite estime où il tenait M. de la
Butière. De telle sorte que ces deux gentilshommes partirent munis
d'argent et convaincus qu'ils jouissaient sans partage de la confiance
et de la marquise et de M. Aimeri.

«\,Tourouvre, se disait La Butière, est à ce point désespéré par la
misère où l'enfonce sa passion de jouer qu'il a perdu tout courage. Il a
les foies blancs, c'est visible. Aussi n'est-ce pas à lui qu'on a pensé
pour le coup. Voilà qui fut agi sagement.\,» Et M. de la Butière ne s'en
reposa sur personne du soin de repasser son couteau de poche, qu'il
rendit tranchant à l'imitation d'un rasoir.

De son côté, M. Acresin de Tourouvre songeait, en caressant ses écus\,:
«\,La Butière, ce misérable ivrogne, n'a aucune part à la faveur de la
marquise. Et son indiscrétion est telle que jamais on ne se serait
ouvert à lui d'un aussi important projet. D'ailleurs, incapable de
loger, même à vingt pas, une balle dans le fond d'une assiette, sa main
tremble et son cœur est trop peu généreux pour de telles actions. Moi,
c'est une autre chanson\,! Ma vieille arquebuse de guerre me vaudra
demain six cents écus de plus\ldots{} sinon mille\ldots{} car la bonne
marquise hait encore plus que moi la méchante pécore dont la Parque va
sous peu trancher le fil des jours. Toi, ma belle, tes heures sont
comptées. Un accident, dans ces chasses à la huée, est chose trop
commune pour tirer à conséquence\ldots{} Au vrai, j'aurais accepté de
faire le coup pour la moitié de l'argent\,!\,»

Mais ni M. de Tourouvre ni M. de la Butière n'eurent de revenant bon
selon ce qu'ils s'en promettaient. Ils durent se contenter des arrhes
qu'ils avaient touchées, sans plus. Chacun de son côté essaya bien de
montrer les dents. Ce fut en vain. Et même M. de Tourouvre se vit
arrangé de la belle manière quand M. Aimeri apprit l'histoire de sa
carabine, si prestement enlevée par André d'Archelet, suivant le
commandement de M\textsuperscript{lle} Catherine. «\,Si la marquise
apprend cela, monsieur, votre ruine est certaine. Taisez-vous, vous ne
réussirez pas à m'effrayer avec vos grands airs. Tant de mal que vous
pensiez de mon ami La Butière, je vous le dis tout net\,: jamais il
n'aurait montré et une telle imprudence et une pareille pusillanimité.
Taisez-vous et tâchez de rentrer en possession de votre arme. Je vous le
conseille, sans autrement vous en indiquer les moyens.\,»

Et il avait recommandé aux deux maladroits de garder le silence devant
Florimond\,: «\,M. le baron de Chézal-Benoît n'est, Dieu merci, pour
rien dans tout cela. Qu'il l'apprenne, et il vous chassera, soyez-en
sûrs. Votre intérêt commande donc qu'il demeure de tout ignorant.\,»

M. de la Butière répondit qu'il avait déjà tout oublié. Comme il
empestait le vin, M. Aimeri ne mit pas sa sincérité en doute. Il lui
emprunta dix écus et le renvoya. M. de Tourouvre, qu'il interrogea
ensuite, parla de même, refusa de prêter même un denier, et s'en fut
mécontent, parce qu'à sentir de l'argent tinter dans ses goussets il
retrouvait quelque superbe. Et puis il ne rêvait que vengeance. Il se
croyait lésé par Catherine de Lépinière. Cette jeune sotte lui coûtait
peut-être quinze cents écus à cette heure, car les prétentions de M. de
Tourouvre croissaient en proportion de ses regrets.

«\,Hélas\,! madame, disait, le soir même de cette fâcheuse aventure,
Aimeri d'Olivier à la marquise, qui, la tête chargée de papillotes,
l'écoutait mollement couchée dans son lit, hélas\,! on ne peut plus
compter sur personne par le temps qui court. Ces deux imbéciles nous ont
claqué dans la main, l'un par sa précipitation, l'autre par sa
maladresse. Et, quand je vois l'incapacité de tous ces gens qui ont
porté les armes, je sens s'évanouir chaque jour davantage les regrets
qui me poignaient jadis de n'être pas homme d'action. Croyez-moi,
madame, il nous faut trouver quelque autre chose.\,»

Accablée non point par sa perte d'argent, mais par le mauvais succès
d'un projet dont elle avait cru jusque-là la réussite infaillible, la
marquise avait répondu en soupirant\,:

«\,Puisses-tu trouver, Olivier\,! Pour moi, je désespère de vaincre
cette espionne domestique qui me brave jusque sous mon toit, Elle me
dessert auprès du marquis et entretient avec lui une correspondance où
je n'ai aucune part. Quand j'écris à mon mari, j'en reçois des lettres
dures et chagrines où jamais il n'est question de cette Catherine dont
je n'arrête pas de me plaindre\,! Elle fera tant, Aimeri, que Florimond
sera sacrifié par son père\,!\,»

Et, déchirant son mouchoir brodé à belles dents, martelant de ses poings
gantés de peau grasse, afin d'entretenir la finesse et la blancheur de
sa peau, la courtepointe en velours amaranthe, Julie, couronnée de
papier, s'écria d'un accent tragique plus digne d'une reine de tragédie
que d'une marquise douillettement allongée dans un lit drapé\,: «\,Ah\,!
Olivier\,! Olivier\,! qui nous débarrassera de Catherine\,?\,»

Sa voix s'était élevée, si claire et si haute que la belle Maroie
Lenatier, sa fille d'atours, couchée dans le cabinet voisin, n'en perdit
pas un mot, non plus que du reste. Julie, exaspérée, montrait à nu sa
nature boutiquière. Les imprécations les plus vulgaires, les injures les
moins nobles dévalaient en torrent de ses lèvres passées au cérat qui
défend contre les gerçures. Son visage, si plaisant d'ordinaire, était
défiguré par la grimaçante colère. On eût dit d'une tête de Méduse
rendue luisante par la pommade et dont chacun des serpents, tenant
office de tresses, eût été entortillé dans un cornet de papier. Oubliant
sans doute que c'était de la seule grâce du marquis qu'elle tenait son
lit drapé de tabit et de moire cerise, alternant par bandes avec le
velours amaranthe, son lit de marquise à quenouilles sommées de panaches
d'autruche, Julie la Drapière chargeait le marquis de Bannes, son maître
et seigneur qui l'avait élevée jusqu'à lui\,:

«\,Toujours il a pris parti contre moi\,!\ldots{} Dieu sait pourtant ce
que je lui ai sacrifié\,!\,»

M. Aimeri le savait aussi. Toutefois, il garda le silence. Étant de ceux
qui dépendent des grands de la terre, il n'ignorait rien de ce qui leur
peut déplaire, et notamment de ces intempérances de langage dont on
n'est jamais le bon marchand. Encourageant les fureurs de la marquise
par son attitude pitoyable, il huma sournoisement une prise et gémit
avec componction.

Et toujours Julie accumulait les griefs. Tous, --- du marquis son mari à
M\textsuperscript{lle} de Lépinière, du notaire Trémolat aux Primelles,
tous, tout le monde aussi, --- s'unissaient dans un infâme complot
contre Florimond. L'envie, la jalousie, la haine entouraient l'enfant de
ses entrailles, son fils, enfin\,! «\,M'en pourrais-tu citer, même à la
Cour, de plus accompli\,?\ldots{} Est-ce pour rien qu'on l'a surnommé
l'incomparable\,?\,»

Puis elle en revint à La Butière et à Tourouvre. Flétrissant leur
maladresse, elle la porta au compte de leur lâcheté, parla de poison, de
noyade\,:

--- Il paraît qu'on peut, en battant les gens avec des sacs longs et
étroits remplis de sable, les tuer sans laisser traces des coups\,!

Effrayé par son exaltation, Aimeri approuvait sans discuter.

--- Cela est bien possible\,!\ldots{} De grâce, madame, parlez moins
haut\,!

Mais la rage de Julie coulait à pleins bords, tel un torrent bourbeux
qui charrie même les pierres gisant en travers de son lit. Et Maroie
Lenatier se signait en retenant son souffle.

Aimeri tremblait devant cette fureur barbare\,: «\,C'est, se disait-il
en soi, une prêtresse de Thrace\,!\ldots{} Non, plutôt une tigresse
d'Hyrcanie\,!\,» Et il suppliait la marquise de se calmer. Peut-être
valait-il mieux appeler ses femmes\,:

--- Songez, madame, que l'on pourrait entendre les éclats de votre voix,
écouter aux portes\ldots{} N'y a-t-il personne à côté\,?

--- Laisse-moi, Olivier, ce n'est rien\,! Ma fille de chambre, rien de
plus. Elle dort et ne nous entend pas. D'ailleurs, elle m'aime\ldots{}
Et, si par malheur elle se montrait indiscrète, j'ai plus d'un moyen
pour assurer à jamais son silence\ldots{} Je pense, Olivier, que le plus
sage est de nous tenir cois et d'attendre\ldots{} Tu vois, maintenant je
suis bien calme\ldots{} Quant à cette petite Primelles, dussé-je périr à
la tâche, je veux que Florimond arrive à ses fins. Demain, pas plus
tard, j'irai à Bourges, et tu m'y accompagneras. J'ai besoin de tes
conseils. Oui, j'irai à Bourges, j'y retournerai dix fois, vingt fois
s'il le faut. Le bal se donnera, et notre Marguerite, la bergère, y
assistera. Sur ma part de paradis, cela se fera, te dis-je\,!\ldots{} Tu
peux te retirer, bonsoir\,!\ldots{} Ah\,! envoie-moi Nicole, je veux la
voir avant de dormir\,; bonsoir\,!

Ainsi congédié, M. Aimeri d'Olivier avait baisé un doigt du gant gras de
Julie, et s'en était allé coucher, tout en murmurant\,: «\,Sa part de
paradis\,!\ldots{} Elle se vante\ldots{} Moi, si je n'étais qu'elle, je
m'affairerais à trouver à cette Catherine un galant. La fille amoureuse
perd toute capacité de nuire, sinon à son amoureux, comme de juste.
Ah\,! si j'étais né comte, marquis, ou simplement baron\,!\,»

Et M. Aimeri, ayant enfoncé, comme à l'ordinaire, son bonnet de nuit
jusqu'à ses yeux, s'endormit du sommeil du juste en s'imaginant qu'il
avait toujours ses vingt ans et les grâces que comporte cet âge\ldots{}
Seule la noblesse lui manquait\,; c'est pourquoi il ne pouvait songer à
M\textsuperscript{lle} de Lépinière.

Deux jours après ce colloque avec son poète domestique, la marquise de
Bannes s'en fut à Bourges. Elle en battit le pavé avec une telle
constance qu'elle trouva, en moins d'une semaine, une dame très noble
pour prêter sa maison, dix autres dames qui en amèneraient trente, et
une dame, enfin, qui se consacra à cette besogne d'attirer Marguerite de
Primelles au bal de M\textsuperscript{me} de Chazeron, qui prêtait sa
maison. Cela, par exemple, coûta fort cher, dix fois plus peut-être que
les violons, le luminaire, les collations et les autres frais\ldots{}
sans compter les humiliations. Et elles furent considérables.

Julie possédait trop de finesse naturelle pour ne pas s'arrêter sur ce
que le choix fait par Florimond de M\textsuperscript{me} Jeanne de la
Pelice, épouse du conseiller Godefroy Harant, avait de désastreux. À
supposer qu'on voulût bien oublier ce que tout Bourges savait des
rapports de cette personne avec Florimond, la noblesse ne viendrait pas
chez cette robine qui avait déchu en se mariant avec un magistrat, ni
Marguerite de Primelles non plus par conséquent. D'autre part, la
position de Julie elle-même lui défendait de nourrir l'espoir d'agir de
sa personne sur cette noblesse où on l'accusait de s'être glissée à la
façon des filous. La femme du marquis de Bannes s'en rendait très bien
compte. Mais ce qu'on ne pouvait l'empêcher de faire, c'était d'acheter
les gens, de les diriger, d'ordonner leurs mouvements ainsi qu'un
montreur de marionnettes actionne, caché dans un réduit de son théâtre,
les fils de ses poupées. Et ces poupées seraient M\textsuperscript{mes}
de Chazeron, de Saint-Agoulin, de Creulles, de Rochefort, d'Alloigny, de
Saint-Aubin et de Puyferrand.

A les mouvoir elle s'employa sans marchander, car Marcelin de Vaux, qui
espérait pouvoir acheter un titre pour mieux orner son nouveau nom si
avantageusement changé, semblait avoir repris depuis quelque quinze
jours confiance dans sa signature. Ses correspondants de Bruxelles et
d'Anvers, avec qui il négociait du papier, lui avaient écrit que le
marquis était malade et qu'on le disait très bas. Le procureur ne
communiqua pas cette nouvelle à Julie, parce que ce n'était pas son
affaire et qu'il ne voulait pas qu'on le sût si bien renseigné. Marcelin
n'apprit pas davantage à Julie qu'il avait vu le testament du marquis,
déposé chez le notaire Trémolat, et dont un clerc acheté lui avait
permis de lire les moindres détails. Dans ce testament, qui ne semblait
pas avoir été révoqué, Julie la Drapière, constituée marquise douairière
de Bannes, se voyait assurer une réserve considérable en dehors des
reprises à exécuter légitimement sur ses propres biens. Maître Marcelin
compta donc à la marquise, trop fine pour montrer sa joie de l'aubaine,
quelques milliers d'écus. Elle les emporta dans son carrosse, en une
belle cassette, qu'elle mit sous clef aussitôt rendue chez sa cousine,
M\textsuperscript{me} de Chazeron, où elle était descendue.

M\textsuperscript{me} Madeleine de Chazeron, née de Maillecornet, veuve
d'un gentilhomme de M. le Prince, vivait mal d'une méchante pension,
payée à regret par celui-ci, mais que triplait heureusement une seconde
pension assurée par le marquis de Bannes. Bien qu'elle fût sa tante, son
âge ne dépassait pas celui d'Armand-Alexandre. Très belle encore malgré
ses quarante-trois ans, Madeleine de Chazeron ne se remaria pas, parce
que, à s'en rapporter à la rumeur publique, son mari défunt la battait
fort et buvait d'autant. L'hôtel qu'elle habitait, avec sa façade sur la
rue des Arènes et ses jardins adossés à ceux de
Saint-Pierre-le-Guillard, lui appartenait en son entier. C'était un des
beaux de la ville, avec ses portes antiques, sa cour en fer à cheval et
ses trente fenêtres à meneaux, avec des estanfiques si agréablement
travaillées à rinceaux, à figures d'hommes et de bêtes, que tout
étranger à la ville ne pouvait la quitter avant d'avoir admiré l'hôtel
des Cent Marmousets. De ce vieux nom les curieux retrouvaient l'origine
dans les petits personnages qui se jouaient parmi les chicorées de
pierre. Et M\textsuperscript{me} de Chazeron se désolait de n'être pas
assez riche pour pouvoir jeter par terre toutes ces drôleries gothiques
et de mauvais goût, pour les remplacer par de beaux cordons de brique
encadrant des bossages de pierre à pointe de diamant, comme ceux que
l'architecte Perrin Gâchelard prodiguait dans la nouvelle maison du
président Prémolin.

Le défaut d'argent condamnait M\textsuperscript{me} de Chazeron à
d'autres privations encore. Celle de ne pouvoir offrir le pain bénit
pour faire pièce à M\textsuperscript{me} de Saint-Aubin lui était la
plus sensible pour l'heure. Il est vrai que M\textsuperscript{me} de
Saint-Aubin avait offert ce pain bénit avec un luxe tel, à la dernière
Pentecôte, que bien des dames en pâlissaient encore d'envie. Toute la
paroisse de Saint-Pierre-en-Guillard fut émue, tant le pain dépassait
par ses proportions tout ce qu'on avait vu jusque-là. De mémoire
d'homme, jamais pareil pain ne fut offert à l'admiration du populaire.
Quatre sacristains en court surplis pliaient l'épaule sous le poids de
la planche. Et celle-ci, sous sa housse de damas blanc et bleu à franges
d'argent, ployait aussi sous ce pain circulaire, pyramidal en son
milieu, blond, roussi par places, doré à l'œuf, et pétri dans la plus
fine fleur de gruau par Simonet Tardif, le meilleur boulanger de
Bourges, et célèbre jusqu'à Issoudun par son art à fabriquer les
palices. En grande pompe, M. le vicaire Torton l'était venu prendre, ce
pain, chez M\textsuperscript{me} de Saint-Aubin, à l'hôtel de la rue de
Suez. Et M. Torton marchait avec quatre suisses, autant de diacres et
des enfants de chœur en nombre indéterminé, mais dont les soutanelles et
les petites calottes rouges transformèrent à l'instant la rue des Arènes
et le carrefour Saint-Hippolyte en un champ de coquelicots.

M\textsuperscript{me} de Chazeron en fut plus irritée qu'un taureau par
un jupon couleur de sang. Elle avait demandé qu'il lui fût permis de
joindre une couronne de pain à cette offrande unique. Elle en fut pour
ses frais. L'altière Saint-Aubin avait répondu\,: «\,Que
M\textsuperscript{me} de Chazeron offre le pain bénit à son jour, quand
bon lui semblera. Pour moi, j'entends que ce soit aujourd'hui le mien.
Il m'en coûte cinquante écus, avec la cire des cierges, l'encens et le
vin de la messe, c'est tout dire. Qu'on me laisse et la paix et la
gloire de mon pain\,!\,»

Et, de même, lorsque M\textsuperscript{me} de Saint-Aubin avait quêté
dans l'église, aucune dame paroissienne ne l'accompagnait. Seule au
milieu de ses femmes de service qui tenaient la queue de sa robe ou la
soutenaient par les coudes pour que la foule ne l'offensât pas, la
pimbêche était allée à l'offrande, sautillant sur ses talons hauts comme
des cornets à jouer aux dés, et l'argent et le cuivre pleuvaient dans sa
sébile d'or.

Et enfin, non contente d'avoir ainsi humilié tout un chacun par son
faste, cette Saint-Aubin avait envoyé dans chaque maison non point un
morceau de pain bénit, ainsi que d'usage, mais un beau petit pain mollet
où des grains d'anis et des nonpareilles reproduisaient la figure du
Saint-Esprit.

--- Ah\,! ma chère, j'en ai encore le goût dans la bouche\,!\ldots{} La
coquine n'avait-elle pas fourré, avec six ou sept douzaines de ces
petits pains, la masse principale de son pain bénit\,!\ldots{} Quelle
insolence\,!\ldots{} Si vous l'aviez vue s'avancer, heureuse de montrer
aux hommes son museau démasqué, précédée par tous les suisses la
pertuisane au poing, et par les mortes-payes des Jacobins avec la
hallebarde sur l'épaule\,!\ldots{} Quelle pitié\,! S'exhiber ainsi en
plein jour, à visage découvert\,! On eût dit de cette reine Vasthi que
son orgueil obligea Assuérus à répudier, ou de quelque autre Candace.
Voyez-moi la belle reine de Saba\,!

--- Qui vous empêche de lui rendre la pareille\,? demanda Julie avec
tranquillité.

--- Ah\,! ma pauvre enfant\,!\ldots{}

M\textsuperscript{me} de Chazeron appelait toujours ainsi la marquise sa
nièce, qui, au vrai, n'était sa cadette que de trois ans. Ainsi la veuve
affirmait-elle sa supériorité, puisqu'elle prenait d'emblée l'avantage
avant même qu'on sût ce qu'il lui plairait de raconter.

--- Ah\,! ma pauvre enfant, comme on voit bien que tu n'es pas dans ma
bourse\,!\ldots{} Dépenser cinquante ou soixante écus pour du pain
bénit, Seigneur\,! Et où veux-tu que je les prenne\,?\ldots{} Quant à
l'artifice des petits pains anisés, je le juge détestable et en tout
contraire à l'esprit de l'Église. Ce pain bénit, d'une seule pièce, puis
proprement coupé en morceaux que l'on distribue aux paroissiens,
n'est-ce pas une fidèle image de la communion des fidèles\,?\ldots{}
Et\ldots{}

--- Je suis bien de votre avis, ma tante, mais\ldots{}

--- Mais tout ce qui passe par la tête à l'évent de cette Saint-Aubin
paraît admirable. Sa richesse lui tient lieu d'esprit, et le curé de
Saint-Pierre-le-Guillard n'a d'oreilles que pour elle\ldots{}
Enfin\,!\ldots{} Ne médisons de personne\,!\ldots{} Quand je pense
qu'elle lui a encore donné, dernièrement, une chapelle de tabis vermeil,
avec les chasubles, les chapes, sans compter les dalmatiques des
diacres, les aubes, les collets et les amicts\,!

--- Voyons, ma tante, soyez raisonnable. Il y a bien plus beau à
Saint-Pierre de Rome.

--- Sans doute, répliqua M\textsuperscript{me} de Chazeron avec une
touchante simplicité. N'est-ce pas au moins M\textsuperscript{me} de
Saint-Aubin qui l'a donné\,?

--- Elle a certainement beaucoup à se faire pardonner.

A cette remarque envoyée par Julie d'un ton détaché, la veuve joignit
les mains et s'écria avec une vive compassion\,:

--- Serait-ce Dieu possible\,? Et saurais-tu quelque chose de nouveau
sur cette évaporée\,?

Julie ne savait rien. Pourtant elle baissa le menton et leva son index
sur sa bouche pour indiquer sans doute que la tombe ne garderait pas
mieux son secret que les lèvres avivées par une discrète pointe de
rouge.

--- Je sais, je sais\,! fit M\textsuperscript{me} de Chazeron qui ne
savait rien non plus. Que veux-tu, ma pauvre enfant, nous n'avons pas de
quoi offrir à la paroisse des chapelles de brocart et de satin\ldots{}
Quand je dis «\,nous\,», je parle pour moi seule, car toi, Julie, il est
à la connaissance de tous que tu puises dans des coffres sans fond.

La marquise ne releva point le propos. En vertu du proverbe qui nous
apprend qu'il vaut mieux inspirer l'envie que la pitié, elle ne se crut
pas obligée de prendre M\textsuperscript{me} de Chazeron pour confidente
de sa gêne presque continuelle. Bien mieux, elle confirma les paroles de
sa tante par une déclaration dont la banalité lui permit de gagner du
temps pour voir venir.

--- Ma chère tante, l'argent ne fait point le bonheur\ldots{} Chacun de
nous porte sa croix sur la terre\,!

Elle présenta le tableau de sa vie abandonnée, solitaire, entre un mari
absent par la volonté du roi et un fils absent par goût des plaisirs.
M\textsuperscript{me} de Chazeron compatit à ses peines\,: «\,Tous les
hommes étaient pareils. Les meilleurs ne valaient pas mieux que les
pires.\,» Elle se répandit en récriminations contre son défunt mari, qui
buvait à se dessécher le corps, la battait et lui mangeait son argent.

--- Et, pourtant, lui était gentilhomme\,!\ldots{} Comprends alors que
ton fils\ldots{} naturellement\ldots{}

Julie eut le courage de ne pas relever cette allusion perfide. Elle
commanda à sa colère. En s'adressant aux parentes de son mari, elle
acceptait d'avance toutes les humiliations. Elle leva les épaules,
soupira et se plaignit à son tour en approuvant le sévère jugement de sa
tante\,:

--- Oui, les meilleurs ne valent pas les pires\ldots{}

Puis, légèrement, elle en revint au pain bénit.

--- Pourquoi, chère tante Madeleine, ne rendriez-vous pas à Saint-Aubin
son coup de la Pentecôte, en offrant le pain bénit ce prochain dimanche
qui suit la fête de la Trinité\,?

--- Tu en parles facilement, ma fille. Te répéterai-je que je réussis
bien juste, avec mes pauvres ressources, à joindre les deux
bouts\,?\ldots{}

--- Mais, ma tante, ne suis-je pas là, trop heureuse si vous me
permettiez de vous aider\,? Donnez-moi ce plaisir de vous mettre à même
de payer le pain bénit à votre tour\ldots{}

Si pleine de bonne grâce que se présentât l'offre, M\textsuperscript{me}
de Chazeron l'accueillit avec une méfiante froideur\,: «\,Vers quel but
caché tendait cette intrigante\,? Sans doute s'imaginait-elle que la
bonne tante Chazeron lui permettrait de quêter à ses côtés en pleine
église\,? Plutôt vivre toute sa vie de pain et de ces fruits secs qu'en
raison de la couleur de leur robe on appelle les quatre mendiants\,!\,»

Elle répondit donc très prudemment du fin bout de son bec rose et en
chassant d'une chiquenaude une mite qui prétendait s'installer sur son
corsage noir sévèrement busqué\,:

--- Tout cela n'est que vanité, mon enfant. Nous ne pouvons pas, tu es
trop fine pour ne pas comprendre, offrir le pain bénit à
Saint-Pierre-le-Guillard\ldots{} Mais, à Lunery, tant que tu voudras.

M\textsuperscript{me} de Chazeron avait appuyé sur ce «\,nous\,» d'une
façon assez insultante pour que Julie se sentit dévorée par l'envie de
lui arracher les yeux, de grands yeux bruns, veloutés, candides, doux,
lumineux, adorables. La marquise répondit avec une parfaite
mansuétude\,:

--- Pourquoi dire «\,nous\,», ma bien-aimée tante\,? Jamais je n'ai
nourri cette sotte prétention d'aller à l'offrande avec vous. Quand vous
offrirez votre pain bénit, je serai chez moi, à Bannes, et non pas ici.

«\,Cette fille est pleine d'esprit, se dit M\textsuperscript{me} de
Chazeron. Je conçois qu'elle ait entortillé Charles-Armand, qui aurait,
du reste, plus sagement agi en m'épousant. Dieu voulut qu'il prit cette
traînée à femme. Inclinons-nous devant sa volonté. Le malheur est
arrivé\,: qu'y faire\,?\ldots{} Et puis, j'oublie que Charles-Armand est
mon neveu\,!\,»

Ainsi pensant, elle ne laissa pas de sourire agréablement et dit\,:

--- Ma chère enfant, loin de moi une pareille idée\,! Tu es la femme de
Charles-Armand devant Dieu et les hommes\ldots{} Il n'y a plus de
remède\ldots{}

La langue lui avait fourché, bonté divine\,! M\textsuperscript{me} de
Chazeron toussa, essaya de rattraper ses paroles, se moucha, surtout
pour cacher la rougeur de sa face. Stoïque, la marquise ne bougea pas.
Elle avait emporté de Bannes une telle dose de patience que cette
première attaque n'était pas capable de l'épuiser. Une lourde bourse en
damas noir pendait au bras droit de son fauteuil. Julie la prit,
l'ouvrit, en tira des écus d'or, dont elle compta vingt que sa main
glissa dans celle de sa tante.

--- Voyons, très chère, pas d'enfantillages\,! Offrez donc votre pain
bénit\,! Vous en mourez d'envie. Le marquis vous aime tant qu'il en eût
fait tout de même à ma place. Je ne suis que son intendante.

M\textsuperscript{me} de Chazeron rougit. L'aumône était trop grosse
pour que l'idée lui vint de la refuser. Elle prit donc les espèces\,:
«\,Où veut-elle en venir, la coquine, pour m'acheter aussi
cher\,?\ldots\,»

Julie ne la laissa pas longtemps sans lui apprendre ce qu'elle
attendait. «\,Il s'agissait d'une assemblée, d'un bal\ldots\,» Bref,
elle défila son chapelet. L'étonnement de la tante Chazeron fut profond
de voir payer de ce prix une chose aussi simple\,: «\,Prêter sa maison
pour un bal\,? Pourquoi pas\,? Elle avait une chambre de quarante pieds
de long sur trente de largeur, bien lambrissée, plafonnée à l'antique et
parquetée de chêne. Pour peu qu'on cirât ce plancher, on pourrait au
besoin s'y mirer\ldots{} Mais\ldots\,» Il n'y eut pas de mais. La seule
objection qui eût pu se produire, la marquise l'avait détruite par
avance, en dénonçant son ferme propos de ne pas assister à cette fête.

--- Voyez-vous, ma bonne tante, je vieillis beaucoup. Tous ces bruits me
fatiguent. C'est à Bannes que je me sens vraiment dans mon assiette. La
paix des champs, la vie grasse\,! Se lever avec le soleil, se coucher
avec lui\,! Veiller n'est pas mon fort. Je hais la vie factice. Quelle
triste femme de cour j'aurais faite\,!

--- Ça, mon enfant, c'est affaire de naissance\ldots{} Ah\,!
pardon\,!\ldots{} Je\ldots{}

M\textsuperscript{me} de Chazeron, cette fois, regrettait sincèrement
ces mots. Ils lui étaient échappés\,: «\,Que je suis donc sotte, et où
ai-je la tête\,?\ldots{} Combien cette distraction va-t-elle me
coûter\,? Bien sûr, elle va me reprendre ces beaux écus dont je me
promettais merveille\,!\,» Ainsi songeait la bonne dame en serrant dans
ses mains l'argent de sa nièce la Drapière et en mirant son sac de damas
à la dérobée.

Comme si elle n'eût rien entendu, la marquise continuait\,:

--- Tout cela, c'est pour Florimond que je le demande. Je vous en prie,
ma chère tante, pensez à votre petit-neveu\,! Aidez-moi à le pousser
vers les plaisirs honnêtes\,! Grâce à vous, mon\ldots{} le fils du
marquis pourra fournir la preuve, devant la bonne société de Bourges, de
ces belles qualités qu'il prodigue bien inutilement à Paris.

--- Paris\,! --- fit M\textsuperscript{me} de Chazeron en levant vers le
plafond ses mains désormais libres, car elle avait adroitement glissé
l'or de sa nièce dans une poche, sous sa jupe, --- Paris\,! J'y suis
allée deux ou trois fois, jadis\ldots{} Quelle ville sale et puante,
avec ses boues\,! Personne me vous y connaît, et le monde vous regarde
en tordant le nez\,!\ldots{} Ne me parlez pas de Paris, mon
enfant\,!\ldots{} Enfin, pour te complaire, je prêterai ma pauvre maison
pour ce bal\ldots{} Mais ne compte pas sur moi pour autre chose\,!

--- Ma chère tante, songez à la merveilleuse jalousie dont souffrira
M\textsuperscript{me} de Saint-Aubin quand elle vous verra recevoir
ainsi la meilleure noblesse de Bourges. Cela lui portera un coup\ldots{}
Et le pain bénit qui suivra\,!\ldots{} Je vous gage que la dame s'en
mettra au lit avec une bonne fièvre.

--- Elle l'aura bien méritée\ldots{} Mais quelle affaire\,!\ldots{}
Aurai-je seulement le temps de remettre mon pauvre mobilier en
état\,?\ldots{} Cela te regarde, dis-tu\,?\ldots{} Allons, c'est
bien\,!\ldots{} Pour quel jour désires-tu que je sois prête\,?\ldots{}
Ah\,! mon Dieu, que d'embarras\,!

Julie, sentant la partie gagnée, donna son dernier effort avec le sac
noir. Une poignée d'écus d'or décida M\textsuperscript{me} de Chazeron.
Elle la prit, en murmurant pour la forme\,:

--- Ah\,! ma pauvre Julie, il n'en était pas besoin\,!

--- Et surtout, ma bonne tante, laissez entendre que je ne suis pour
rien dans tout cela\ldots{} A votre place, je répondrais, à qui m'en
parlerait, que tout cela est par la volonté du marquis.

--- Heureuse et charmante idée\,!\ldots{} Ainsi, c'est convenu, il ne
sera pas question de toi\,?

--- J'y compte\,!\ldots{} Ah\,! ma bonne tante, laissez-moi vous
embrasser, tant je vous aime\,!

La blonde Julie et la brune Madeleine s'accolèrent de bon cœur, parce
que chacune estimait qu'elle avait l'avantage dans cette affaire\,:
«\,Gageons, se disait M\textsuperscript{me} de Chazeron tout en comptant
ses écus, --- en tout il s'en trouvait trente et cinq, dont deux un peu
rognés, --- gageons que la coquine veut attirer ici quelque galante pour
son aimable fils. Gageons aussi que, maintenant qu'elle a trouvé le
toit, elle court chez la Saint-Agoulin, qui lui procurera la
demoiselle\,!\,»

Madeleine de Chazeron ne se trompait point. Julie, après cette première
victoire, n'eut rien de plus pressé que d'aller chez
M\textsuperscript{me} de Saint-Agoulin, pour la cueillir au nid, après
la sortie de la messe. Cette vieille dame, encore que parente éloignée
du marquis de Bannes, aimait à se donner pour sa tante. Pauvre et
décriée, Éléonore de Saint-Agoulin était cependant une vraie puissance.
Tout Bourges comptait avec elle. On l'avait instituée gardienne et
arbitre du bon ton\,: les jeunes femmes la consultaient pour la façon de
leurs robes, et les maris pour le choix d'une mignonne. Elle décidait en
dernier lieu, jugeait sans appel. Condamné à son tribunal, on n'avait
plus d'autre ressource que de s'enfuir, car on était, par la force des
choses, mis au ban de la société.

Il en allait toutefois des arrêts de la baronne Éléonore comme de ceux
des tribunaux. L'argent n'était pas étranger à ses verdicts. En n'y
regardant pas à cent écus, on pouvait obtenir de favorables sentences.
Comment cette vieille dame avait été intronisée souveraine maîtresse des
élégances, c'est ce dont personne ne gardait le souvenir précis. Elle
régnait de fait, et l'on ne discutait pas les origines de sa royauté. On
racontait que son mari le baron avait eu, à la cour du feu roi, cette
réputation de porter le mieux la cape à l'espagnole et de danser la
courante mieux que M. de Bellegarde en personne. Beaucoup de cette
solide gloire aurait rejailli sur la compagne du fameux danseur.
Celui-ci, devenu imbécile à force d'excès, vivait au fond de son petit
hôtel, sis rue du Four-Chaud, en face du cloaque, dans la vieille ville.
Il ne se montrait jamais. Pauvre d'esprit au moins autant que d'argent,
il subissait la tyrannie tracassière d'une épouse autrefois délaissée
qui prenait, à le confiner dans un infect taudis, le plaisir d'une
tardive revanche.

En se rendant chez M\textsuperscript{me} de Saint-Agoulin, la marquise
de Bannes croisa, dans son superbe carrosse de cuir de Russie monté en
bronze doré, la voiture antique et dédorée de M\textsuperscript{me} de
Rochefort, belle personne célèbre par sa morgue et son incroyable
méchanceté. Julie ordonna aussitôt à son écuyer Piccolomini, qui gardait
la portière gauche, de commander au cocher de céder le pas à
M\textsuperscript{me} de Rochefort. M. de Tourouvre, qui suivait à
cheval, en faillit avaler un des glands de son col, tant fut grand son
étonnement. Le cocher, à la livrée reluisante de Bannes, rangea ses
chevaux le long d'un mur\,; mais la rue était si étroite que le vieux
véhicule accrocha le carrosse neuf de Julie. Celle-ci aussitôt se
confondit en excuses, et la comtesse Isabelle de Rochefort, retenant les
brocards insultants dont elle s'apprêtait à charger la Drapière, lui
rendit son salut.

--- Ah\,! madame, dit Julie à demi sortie par la fenêtre encadrée de
cuivre, ah\,! madame, quelle rencontre heureuse\,! Je suis votre petite
servante\,!

Et, après quelques autres compliments pleins d'humilité, elle demanda à
M\textsuperscript{me} de Rochefort, qui souriait avec toute la grâce des
tenailles ouvertes par le bourreau pour déchirer un patient, la faveur
de la visiter. Il s'agissait de charité\,:

--- M. de Bannes m'a chargée, madame, de vous remettre une somme
d'argent pour les pauvres de M. Vincent dont vous êtes fille. Me
sera-t-il permis de verser cette pauvre aumône entre vos jolies mains\,?

--- Ne prenez pas cette peine, madame, répondit la voix pointue de la
comtesse Isabelle, qui, du fond de son carrosse capitonné de velours
puce, examinait la Drapière avec une impertinente curiosité. Mon
intendant passera chez vous.

--- C'est que, madame, j'aurais une grâce à implorer de votre bonté.

--- Dans ce cas, madame, exposez-moi votre demande.

Sans perdre courage ni patience, Julie, toujours à la fenêtre de son
carrosse, tenta d'expliquer qu'il s'agissait de choses d'importance
qu'on ne pouvait crier aux oreilles de tout venant. Mais, comme les
laquais avaient réussi à dégager les roues, M\textsuperscript{me} de
Rochefort, voyant la route libre, donna l'ordre de toucher, et Julie
parlait encore que la comtesse roulait, tout en riant comme une folle du
bon tour qu'elle venait de jouer à la Drapière.

--- Toi, ma fille, je te repincerai, dit tranquillement Julie en se
rasseyant à côté de Nicole, qui, flanquée de la fille d'atours Maroie
Lenatier, accompagnait la marquise. Nicole, dis à Jacques de s'arrêter
devant la porte d'Éléonore, où tu m'attendras.

Julie entra chez la baronne avec son sac noir dont elle eut soin de
faire sonner innocemment le contenu en prenant le fauteuil qu'on lui
avança.

Placée en face d'elle, M\textsuperscript{me} de Saint-Agoulin se tenait
un peu de côté, montrant son profil en lame de couteau, son nez pareil
au bec d'un aigle et son œil fureteur qui semblait percé à la vrille. Le
sourire qu'elle ébauchait, en s'efforçant de le rendre plaisant,
disloquait sa mine ingrate, où l'absence de menton complétait la
ressemblance avec une tête d'oiseau. On eût dit d'un vieux perroquet
blanchi par l'âge et qui aurait porté des frisons.

La tante Saint-Agoulin, Julie la savait par cœur. Endettée, toujours
besoigneuse, altérée de luxe et n'arrivant jamais à satisfaire ni ses
besoins ni ses goûts, M\textsuperscript{me} Éléonore de Saint-Agoulin,
née de Neuville de Saint-Thoret, tirait d'intrigues compliquées et
savamment conduites des ressources inavouées.

C'était une appareilleuse du grand monde. Elle possédait, en outre, le
merveilleux talent de se procurer des papiers compromettants. Bonne
personne au fond, M\textsuperscript{me} de Saint-Agoulin tirait les gens
d'embarras en leur cédant, contre une récompense honnête, ces mêmes
papiers que son esprit d'ordre lui commandait de ne pas laisser traîner.
Délivrant ainsi les uns de cuisants soucis, menaçant de façon ouatée les
autres, obligeant ceux-ci, inquiétant ceux-là, surveillant tout le
monde, n'exécutant personne, mais vendant à qui pouvait payer le repos
et la réputation des familles, elle jouissait de la position assez
enviable de conseillère à gages, sans que personne eût l'imprudence de
se vanter de l'avoir corrompue.

Avec cette dame, dont la rapprochait un goût commun des pratiques
sournoises, Julie n'était pas obligée d'y aller par quatre chemins. Elle
lui exposa son affaire, carrément. Le marquis son mari, ayant chargé
M\textsuperscript{me} de Chazeron de donner le bal en son hôtel, lui
avait confié à elle, Julie, l'argent nécessaire au payement des violons,
des lumières, de la collation, bref, de tous les frais. Les convenances
lui défendaient de paraître à cette fête.

M\textsuperscript{me} de Saint-Agoulin approuva d'un mouvement de tête.
Son nez s'abaissa sur sa maigre poitrine, pareil au bec d'un aigle
occupé à s'épouiller. Et elle dit en gémissant\,:

--- Eh\,! oui, ma pauvre fille, le monde est si méchant\,!

Julie, sans s'émouvoir, continua\,:

--- J'ai pensé, d'accord en cela avec le marquis, que vous seule pouviez
dignement organiser l'assemblée. Que vous l'approuviez seulement, et
tout le beau monde de Bourges y courra\ldots{} Surtout quand on saura
que vous avez daigné en assurer l'ordonnance. Je vais vous donner
l'argent.

M\textsuperscript{me} de Saint-Agoulin approuva encore. Mais son nez
s'allongea jusqu'à singer celui d'un vautour quand Julie, sans délier
les cordons de sa grosse bourse, ajouta\,:

--- Il faut que la petite Primelles vienne au bal. Florimond y tient
absolument.

--- Qu'elle y vienne si cela lui plaît, répondit sèchement
M\textsuperscript{me} de Saint-Agoulin. Pour moi, je ne me mêle pas de
ces choses.

--- En vérité\,?

L'exclamation posément étonnée de Julie pouvait se traduire ainsi\,:
«\,Et depuis quand\,?\,» M\textsuperscript{me} de Saint-Agoulin affirma,
la mine pincée, sa réponse première\,:

--- C'est la pure vérité, ma nièce. Cherche ailleurs\,!

Julie, qui maniait négligemment le sac aux écus, le laissa dormir sur
ses genoux et, regardant la Saint-Agoulin bien en face, la pria d'un ton
mielleux\,:

--- Même si l'on vous en priait à genoux\,?

--- Que veux-tu dire\,?\ldots{} Cela est impossible.

--- À vous peut-être, et encore\,! Mais vous savez certainement à quelle
porte frapper pour réussir\ldots{} Allons, chère tante, un bon
mouvement\,!

M\textsuperscript{me} de Saint-Agoulin secoua la tête et se renferma
dans un silence obstiné. Alors, très tranquillement, Julie tira de son
corsage des papiers pliés en long, et, sans les remettre à la vieille
dame qui regardait ce manège avec défiance, elle s'écria sur un ton de
surprise parfaitement jouée\,:

--- Oh\,! que je suis sotte\,!\ldots{} J'allais oublier\ldots{}
Figure-toi, ma pauvre Éléonore\ldots{}

A s'entendre ainsi tutoyer par la Drapière, M\textsuperscript{me} de
Saint-Agoulin rougit. Ses petits yeux s'élargirent, sa bouche édentée
bâilla, trembla. De ses deux mains appuyées aux bras de son fauteuil de
cuir flamand, elle se souleva à demi, puis retomba, pâle, anéantie, car
Julie poursuivait\,:

--- Figure-toi que Marcelin de Vaux, --- il est noble maintenant, tu
sais, et n'a point payé cher sa noblesse, --- enfin il m'a remis les
papiers de la succession de\ldots{} mon\ldots{} défunt Péréal. Tu dois
six mille écus, trente-six mille livres, ma pauvre Éléonore. Les
jugements sont exécutoires, tous les recours épuisés. Il te faudra
payer. Comment vas-tu faire\,?

--- Je\ldots{} croyais\ldots{} la chose réglée depuis longtemps\ldots{}
Ne m'avais-tu point promis\,?\ldots{}

--- Tu crois\,?\ldots{} Je t'en demande pardon\ldots{} Pourtant, à cause
de mon fils, je n'ai pas le droit d'abandonner tant d'argent\ldots{} À
la vérité, j'y ai mis de la négligence. Que veux-tu\,? À parler franc,
je suis un peu serrée en ce moment\ldots{} Alors, tu comprends, je
laisse Marcelin poursuivre.

«\,Poursuivre\,!\ldots{} Exécuter les jugements\,!\ldots{} Trente-six
mille livres au principal, sans compter les frais\,!\ldots\,»
M\textsuperscript{me} de Saint-Agoulin en pensa défaillir. Dans sa
chambre tapissée de nattes, tout dansa autour d'elle\,: les chaises
tournaient, et le lit drapé de serge et les tableaux accrochés aux murs
qui ondulaient, et le prie-Dieu, et la table de toilette semblaient se
saluer\ldots{} «\,Ruinée\,! Écrasée\,! Perdue\,!\,» Elle se vit déjà
dans la rue, avec son mari infirme\,: elle vit son vieil hôtel vendu,
ses meubles dispersés\ldots{} ses papiers\ldots{} ah, oui, ses
papiers\,!\ldots{} Mais, au fait, avec eux peut-être pouvait-elle parer
ce coup félon.

--- Trente-six mille livres\,! Sainte Vierge, dit-elle en se tordant les
mains, où veux-tu que je les prenne\,?

--- Ma foi, je n'en sais rien. Tu t'arrangeras. Tes amis Primelles
consentiront peut-être à t'aider\,?

--- Ah\,! oui, parlons-en, des Primelles\,!

--- Je ne demande qu'à en parler, ma chère tante.

--- Eh bien donc, puisque c'est là ce qui te touche\ldots{} écoute,
Julie, jure-moi de m'épargner, et je trouverai peut-être un
moyen\ldots{}

Silencieuse, Julie attendait. Alors M\textsuperscript{me} de
Saint-Agoulin prononça ces mots très bas\,:

--- Une seule personne peut décider Bouteiller à conduire sa nièce
Marguerite à ton assemblée. C'est Isabelle de Rochefort.

--- Voilà, Éléonore, ce qui s'appelle mal tomber\,! Je suis à couteaux
tirés avec elle\ldots{} Cherche autre chose\,!

--- Pourquoi\,?

--- Parce que de cette pimbêche je renonce à forcer la porte.

--- Et si je t'en donnais la clef\,?

La marquise rapprocha sournoisement son fauteuil. La conversation
devenait intéressante.

--- Écoute, Julie, remets-moi ma dette, et je te livre Rochefort pieds
et poings liés\,!

--- Trente-six mille livres, la peau de cette pécore\,! C'est trop cher.

Elles discutèrent âprement, telles deux fripières. Dans cette lutte, la
nature boutiquière de Julie lui assurait l'avantage. Quand elle sortit
de chez M\textsuperscript{me} de Saint-Agoulin, la Drapière avait trois
cents écus de moins en argent --- de mauvais écus blancs --- et douze
mille livres en un papier. Pour une de ces trois reconnaissances
trouvées dans son héritage après la mort de son premier mari, elle avait
acheté à la tante Éléonore une lettre que reçut son corsage de drap de
soie noir.

Le lendemain seulement, la marquise de Bannes se rendit chez
M\textsuperscript{me} Isabelle de Rochefort. Cette jolie personne,
attablée devant sa toilette avec Maceron, sa fille favorite de chambre,
consigna d'abord la visiteuse à sa porte. Julie ne réclama pas. Une
grande heure, elle posa dans son carrosse. Quand elle se fut donné le
plaisir d'humilier ainsi la marquise au dehors, Isabelle lui donna
licence de pénétrer dans la cour de son hôtel. Le suisse ouvrit les deux
battants. La lourde voiture rouge et or entra au pas de ses quatre
carrossiers normands, dont la robe de trois poils, rappelant celle des
vaches que l'on nourrit dans les mêmes pâturages, disparaissait sous les
harnais blancs à chasse-mouches houppés, avec des appliques de bronze
doré à tous les carrefours. Quatre cavaliers, six laquais à pied, quatre
pages accompagnaient la marquise. Cette fois, elle n'avait pas ses
femmes dans son équipage.

Quand Julie descendit, au ras du perron où l'on abaissa le lourd
marchepied à quatre échelons, la comtesse de Rochefort, qui la
surveillait par l'entre-bâillement des volets, rendit malgré elle
hommage au bon goût parfait de ses habits, à la souplesse aisée et fière
de sa démarche, à la distinction de sa personne\,: «\,Qu'elle est encore
belle, et combien est-il malheureux que de pareilles coquines s'emparent
du cœur des gentilshommes\,!\ldots{} Voyez-la, ne dirait-on pas d'une
reine, avec tout ce monde qui la suit\,?\ldots{} En somme, eu égard à
ses mérites, cette Péréal devrait être logée aux filles repenties.\,»

Et M\textsuperscript{me} de Rochefort, pour se consoler, laissa encore
la marquise de Bannes se morfondre trois longs quarts d'heure dans son
antichambre. Enfin elle la reçut, à dix heures du matin, toujours assise
devant sa toilette avec Aimée Maceron, qui lui soumettait des mouchoirs
de cou à choisir. Sans se déranger, sans même se retourner, Isabelle
cria d'une voix de tête\,:

--- Ah\,! vous voilà, madame\,! Depuis hier que je vous attends, le
temps m'a paru bien long\,! Ainsi, cet argent pour M. Vincent, vous me
l'apportez\,?

Et Isabelle, ainsi parlant, examinait la marquise, dont la sereine
beauté blonde voisinait avec la sienne dans le miroir posé sur la table,
au milieu des accessoires les plus variés, depuis les fers à friser
jusqu'aux curettes d'ivoire.

Julie, aussi à son aise que dans son salon doré de Bannes, dessina une
toute petite révérence et dit\,;

--- Madame, si vous avez besoin d'argent, on vous en donnera. Je viens
spécialement pour\ldots{}

Isabelle l'interrompit sans politesse, avec des exclamations qui
rappelaient le pépiement des oiseaux\,:

--- Regarde, Maceron, ce nœud-là\,! Oui, oui, ce nœud-là, avec son
effilé d'argent\,!\ldots{} Je le veux, je l'aime\,!\ldots{} Pose-le bien
vite sur ma tête\,!\ldots{}

Et, toujours tournant le dos à la marquise, elle reprit d'une voix
méprisante\,:

--- Vous disiez, madame\,?

Sans même demander à s'asseoir, Julie, debout au milieu de la chambre,
répondit avec suavité\,:

--- Je viens, madame, implorer une grâce de\ldots{}

--- Ah\,! Maceron, Maceron, est-il joli, ce nœud\,!\ldots{} Et bien
posé\,!\ldots{} J'en raffole\,!\ldots{} Tu es un ange,
Maceron\,!\ldots{} À propos, madame Péréal, auriez-vous la grande bonté
de me dire quelle est la qualité de ce drap que l'on veut me
vendre\,?\ldots{} Vous connaissez cela, vous\ldots{} Maceron, montre la
pièce.

Avec le plus beau sang-froid, la marquise regarda le drap, la lisière,
la marque, et donna son avis\,:

--- C'est du moncayar, madame. Et celui-là est d'une bonne étoffe. On ne
fait pas mieux en draps d'été.

--- Que vous êtes heureuse, madame, de garder une aussi fidèle mémoire à
votre âge\,!

--- De ma mémoire, madame, je n'ai jamais eu à me plaindre. Ainsi, il me
souvient à merveille de Saint-Sylvain-sur-Ablon\ldots{} Eh\,!
qu'avez-vous, madame\,? Vous seriez-vous piquée\,?

Isabelle, très rouge, les sourcils froncés, rappela Maceron\,:

--- Pose là cette pièce, et laisse-nous\,!\ldots{} Avance un fauteuil à
M\textsuperscript{me} de Bannes, et t'en va\,!

Maceron sortit. Alors Isabelle, qui de rouge s'était faite blanche comme
un linge, se leva. Ses grands yeux noirs luisants semblaient brûler de
fièvre, et tout son corps tremblait. Marchant sur Julie, qui s'éventait
dans son fauteuil avec un magnifique émouchoir d'autruche noire, elle la
toisa avec une arrogance empruntée et balbutia\,:

--- Que venez-vous ici chercher, vous\,?\ldots{} Vous allez filer, et
vivement\,! Ou bien j'appelle, et je vous fais bailler les étrivières
dans mon écurie\ldots{} Sortez\,!

La marquise s'éventait toujours, sans perdre la jeune comtesse des yeux.
Elle lui répondit, sans se presser\,:

--- A d'autres, ma petite\,!\ldots{} Il me suffirait d'un mot, d'un
seul, et, ici même, tu serais fouettée par mes pages. Regarde dans ta
cour\ldots{} Mes hommes sont plus nombreux que les tiens, et en armes.
Ton suisse est sous clef dans sa loge. Ta porte poussée et gardée.
Essaye d'appeler, et l'on t'empoigne\,!\ldots{} J'ai du monde sur toutes
les marches de ton escalier, jusqu'à la porte de ta chambre. En
doutes-tu\,?

Isabelle, grinçant des dents, poussa le battant. Le panache de M. de
Tourouvre se balançait aux mouvements de sa promenade. Conquérant et
superbe, il arpentait l'antichambre, où sonnaient ses éperons. Le
fourreau de son épée chatouillait de sa bouterolle noircie les déesses
des tapisseries. La marquise ne mentait pas\,: de ses écuyers, de ses
laquais, de ses pages, elle avait doublé le nombre avec les meilleurs
estafiers de son fils. De la chambre d'Isabelle à la rue de Mirebeau,
trente coquins déterminés, avec l'épée et la dague, sans compter les
pistolets dans les fontes, attendaient ses ordres.

La jeune femme serra les poings, retint ses larmes, et, haletante,
s'écria\,:

--- Vous me répondrez de cette violence\,! Mon mari reviendra de Paris
bientôt, et\ldots{}

--- Et il t'étranglera, pauvre sotte, quand il saura qu'avant de
l'épouser tu t'es laissée aller avec le chevalier de Jonzec\ldots{} Tu
sais bien, Jonzec, le beau Raymond\,? L'aurais-tu oublié, lui et la
petite maison de Saint-Sylvain-sur-Ablon où tu fis tes couches, soignée
par la mère de ta fidèle chambrière Maceron\,?

Isabelle du Doulçay, comtesse de Rochefort s'écroula dans une haute
chaise en ciseaux dont le cuir doré se tendit sous son poids. Béante,
sans force et sans voix, repoussant de ses paumes étendues le spectre de
son enfant noyé dans les fossés de Saint-Gilles, elle écoutait Julie la
Drapière, qui lui racontait ce crime brusquement tiré de sa nuit. Sur
ses traits charmants, à cette heure déformés par l'angoisse, se lisaient
l'épouvante et l'horreur, la peur qui coupe les jarrets et rend stupide,
le désespoir qui cloue sur place devant le danger qu'on ne peut plus
éviter. Puis, se voilant la face de ses mains crispées, elle fondit en
larmes. Sa pauvre nature d'oiselle apparut toute nue, dépouillée du
masque qu'y tenaient appliqué, en tous temps, la sécurité et l'orgueil.
Désormais sans défense, elle avoua son indignité, essaya de se forger
des excuses\,:

«\,Était-ce sa faute, après tout\,?\ldots{} Oui, elle aimait alors M. de
Jonzec\ldots{} Ses parents l'avaient mariée à un autre, malgré elle, à
seize ans\,! Était-ce sa faute, encore, si Raymond de Jonzec avait abusé
de son innocence et de sa faiblesse\,?\ldots{} Était-ce sa faute, enfin,
si son enfant était mort\,? Non, non, elle ne l'avait pas tué,
puisqu'elle ne l'avait même pas vu.\,»

Debout devant cette loque humaine qui se tordait en répandant l'eau de
ses pleurs, Julie jouit pleinement de son triomphe. Et elle songea\,:
«\,Oui, c'est ainsi que je veux voir Marguerite de Primelles\,!\ldots{}
Et moi, la Drapière, je la repousserai, en lui conseillant de s'aller
jeter à la rivière où se lavent les péchés.\,»

Isabelle, entre deux averses de larmes, se risqua enfin à parler\,:

--- Enfin, madame, que voulez-vous de moi\,?\ldots{} Et pourquoi me
torturer ainsi\,?\ldots{} De l'argent, peut-être\,? Hélas\,! vous savez
bien que je n'en ai pas\,!\ldots{} Et mon mari va revenir, et il me
tuera\,!\ldots{} Pensez, madame, que je n'ai pas vingt-deux
ans\,!\ldots{} Voulez-vous donc que je meure\,? Oh\,! parlez\,!\ldots{}
Ne me laissez pas ainsi\,!\ldots{} Parlez\,!

Et, brisée, anéantie, palpitante, à demi folle de terreur devant ce
silence obstiné, elle ajouta, très bas\,:

--- Pardonnez-moi, madame\,!\ldots{} J'ai été impertinente et
mauvaise\,!\ldots{} Hélas\,! je suis une enfant sotte et
malapprise\ldots{} Oh\,! si vous saviez comme j'ai été mal
élevée\,!\ldots{} Ma mère mourut quand je n'avais pas huit ans\ldots{}
Mais parlez-moi, parlez-moi donc\,!\ldots{} Vous me faites
peur\,!\ldots{}

--- Remettez-vous, madame de Rochefort, répondit froidement la marquise.
On se doit à son rang, que diable\,! Et que diraient vos gens s'ils vous
voyaient en cet état\,?

Isabelle reprenait peu à peu ses esprits. Une pointe de courage lui
revenait à cette heure. La bourrasque semblait passée. Maintenant il
fallait se tirer d'affaire\,: «\,Accuser les gens, rien de plus facile.
Des racontars, des babillages ne suffisent pas cependant pour perdre les
gens\ldots{} Après tout, cette femme n'avait pas de preuves à
produire.\,»

D'une voix qu'elle s'efforçait d'affermir, Isabelle demanda à Julie, qui
l'observait sans baisser ses yeux clairs\,:

--- Comment savez-vous cela\,?\ldots{} Ignorez-vous que, pour la bonne
moitié, cette histoire est un tissu de mensonges\,?

La marquise sourit très aimablement et tira de son corsage busqué à la
Flamande, véritable boîte de Pandore, un billet roulé pas plus gros que
le doigt\,:

--- Voici une lettre qui suffit à vous agenouiller, avec votre joli cou
si galamment ceinturé de perles, sous l'épée du bourreau\ldots{}
Regardez, mais sans toucher\,!\ldots{} La reconnaissez-vous\,?

Si elle la reconnaissait\,!\ldots{} Vivement, d'un élan subit, rapide
autant que sournois, Isabelle s'était jetée sur le papier accusateur,
sur la lettre où elle annonçait à Raymond son heureuse délivrance et la
mort subite de leur enfant. Plus vive encore, la marquise avait relevé
la main, et le billet se trouvait hors d'atteinte\,:

--- Recommence, mauvaise guenon, et j'appelle\,!\ldots{} Et moi qui me
laissais toucher par je ne sais quelle imbécile pitié\,!\ldots{} Écoute,
voici ce que je veux\,: à l'assemblée prochaine, chez
M\textsuperscript{me} de Chazeron, j'entends que tu ouvres le bal avec
mon fils Florimond.

Isabelle en demeura stupide d'étonnement\,: «\,Julie se moquait d'elle,
bien sûr\,!\ldots{} Comment\,! on ne lui demandait que cela, en échange
de ce terrible papier\,!\ldots\,» Elle s'écria avec une vivacité où
éclatait sa sottise\,:

--- Oh\,! ça, tant que vous voudrez, madame\,! Votre fils est de ceux
que l'on s'honore en distinguant. Tout le plaisir sera pour moi.

Insensible au compliment, Julie reprit\,:

--- Cela n'est rien, et pourtant j'y compte. Mais, si tu veux que je te
rende ton papier, tu conduiras à l'assemblée le baron de Mordicourt et
sa nièce Marguerite de Primelles\ldots{}

--- Hélas\,! madame, vous demandez l'impossible\,!\ldots{} Comment
voulez-vous\,?\ldots{} Jamais les Primelles ne consentiront à se
rencontrer avec vous\,!

--- Et, d'abord, je n'assisterai pas à ce bal. Je veux qu'ils y
viennent, voilà tout. Si tu ne réussis pas à les amener, ton mari aura
la lettre, j'en jure sur\ldots{} ton enfant\,!

Isabelle se taisait. La menace était sérieuse. D'autre part, si Julie ne
paraissait pas au bal, c'était une raison pour que le baron de
Mordicourt acceptât l'invitation. Elle réfléchit, tout en observant avec
défiance la marquise, blonde, claire et sereine dans ses habits
sombres\,: «\,Et dire que cette boutiquière a fait assassiner son
premier mari, ouvertement, aux portes de Bourges, et que c'est elle qui
me veut livrer comme meurtrière\,!\,» Timidement, elle répondit à Julie,
dont les yeux froids, à l'exemple de l'acier d'une arme,
l'interrogeaient\,:

--- Et si je les amenais au bal\,?

--- Tu auras la lettre, et moi, j'aurai oublié.

--- Eh bien, je vous assure sous serment que l'oncle et la nièce
viendront. Vous avez ma parole, rendez-moi ma lettre\,!

--- Non, ma belle\,: donnant, donnant\,! Le lendemain du bal, madame,
vous recevrez le papier, avec un beau bouquet, de mes mains.

--- Pourquoi me fierais-je à vous qui n'avez pas confiance en moi\,?

--- Parce que, petite linotte, je ne risque rien, tandis que toi tu
risques ta tête. Et puis, au fond, je ne te hais point\ldots{} Seulement
tu es fantasque et légère\ldots{} Voyons, madame, cessons de nous
harpiller comme deux poissardes, et jouons franc jeu\,! Favorisez mes
desseins\,: ils sont considérables. Aidez-moi un tant soit peu et sans
me mettre en cause, et je vous sauverai. Je vous couvrirai, de mon côté,
et vous ouvrirai ma bourse. Quelle jeune femme n'a pas de dettes ou le
désir d'en contracter\,?\ldots{} Je payerai la pièce de moncayar, et
aussi du velours, du satin. Vous avez bien voulu me rappeler que j'étais
assez connaisseuse\,: tout sera choisi par moi.

Isabelle, enrageant, désespérée, honteuse, se voila la face de ses deux
mains et dit en pleurnichant\,:

--- Pardonnez-moi, madame, j'étais folle, et votre bonté me confond\,!
Embrassez-moi, que je sache que vous m'avez pardonné\,!

Alors elles s'embrassèrent. Mais, dans cette étreinte commandée par
l'intérêt et la crainte, Isabelle promena en vain ses mains fureteuses
sur le superbe corsage de Julie\,: elle ne réussit pas à reprendre sa
lettre. La marquise avait glissé ce papier sans prix, loin de la garde
en velours de son haut gant, entre sa paume et le cuir d'Espagne. Dans
ce réduit doux, moite et parfumé, le billet d'Isabelle, comtesse de
Rochefort, défiait les entreprises des doigts les plus déliés.

La paix ainsi rétablie, les deux dames causèrent de leurs préoccupations
et de leurs espoirs. La marquise donna ses raisons, et Isabelle,
feignant de croire que celle-ci poursuivait une réconciliation loyale
avec les Primelles, annonça qu'elle partirait le surlendemain pour les
visiter.

«\,Surtout, que la baronne de Primelles ne sache rien de nos petits
complots, ma toute belle\,! Ce serait se condamner à échouer, et j'en
serais encore plus désespérée que vous. Il n'est pas encore temps de
prévenir cette pauvre dame qui vit entre sa douleur et son
prie-Dieu\,!\,»

Telle fut la recommandation dernière qu'envoya Julie la Drapière à la
comtesse Isabelle de Rochefort, née Latour-Champois de Broislin, qui
l'avait reconduite jusqu'à la tête du grand escalier, ainsi qu'il
convient d'honorer les marquises.

\hypertarget{chapitre-ix}{%
\chapter{CHAPITRE IX}\label{chapitre-ix}}

Ainsi la marquise Julie avait, en trois visites, remporté trois
victoires, et elles ne lui coûtaient pas trop cher, car des créances sur
M\textsuperscript{me} de Saint-Agoulin la valeur n'était que de
circonstance. De ces billets, elle n'en avait lâché qu'un. Deux autres
lui demeuraient, en réserve, car on ne saurait tout prévoir. Et ce
n'était pas tout\,: il fallait maintenant consolider cet édifice
ingénieusement élevé sur l'intérêt et la terreur, par l'adjonction
d'autres intérêts, et, à défaut de crainte, d'autres passions. Il
fallait se rendre favorables les reines de la mode. Et ces reines
étaient en révolte ouverte contre M\textsuperscript{me} de
Saint-Agoulin. Elle niaient son pouvoir. Que M\textsuperscript{mes} de
Saint-Aubin, d'Alloigny, de Puyferrand refusassent de paraître à
l'assemblée, et cette fête tombait dans l'eau, si l'on peut dire.

Julie s'en fut donc attaquer ces trois beautés, qui n'étaient ni à
vendre ni en condition de se laisser intimider. Décidée à ne reculer
devant aucun affront pour assurer le triomphe de Florimond, elle se
résolut à employer les armes coutumières des gens de cour, qui sont les
perfides insinuations, les prétéritions calculées, les calomnies
sournoises et les audacieuses affirmations. «\,Je prendrai, se disait
cette mère exaspérée par son amour pour son fils, l'une pour battre
l'autre. Je les ferai parler sans qu'elles me puissent contredire, et
les rendrai à ce point jalouses qu'elles marcheraient sur les mains dans
la crotte pourvu que leur rivale y soit enlisée jusqu'au nez.\,»

Se présenter seule chez ces dames, dont l'insolence valait l'orgueil,
c'eût été une faute grave. Julie n'y tomba pas. Elle se réfugia sous le
patronage de la marquise de Creulles, vieille amie du marquis de Bannes,
et qui avait, jadis, pallié avec un dévouement qui ne connut pas de
défaillances les frasques de ce seigneur. M\textsuperscript{me} Valérie
de Creulles aimait cette Julie, dont les fines flatteries avaient
toujours trouvé grand ouvert le chemin de son cœur un peu mou et
sensible aux délicates attentions. Assez simple pour prendre comme bon
argent les belles phrases de sa protégée sur les désirs de
réconciliation entre ennemis qui lui étaient également chers, la
marquise Valérie promit de s'entremettre et réussit à rendre possibles
les démarches de Julie auprès des trois belles de Bourges. Ce furent
huit jours de petites conspirations, d'intrigues à rendre jaloux des
Vénitiens ou des Génois. Auprès de l'aimable Saint-Aubin, dont la vanité
fastueuse, encore plus que la passion du jeu, était le péché mignon,
Julie réussit par les flagorneries les plus basses. Elle eut aussi ce
soin d'exciter la jalousie féroce de la dame. Elle lui voulut persuader
que M\textsuperscript{me} de Chazeron ne donnait ce bal que pour prendre
position en face de M\textsuperscript{me} de Saint-Aubin, qui l'avait
humiliée avec son pain bénit. Elle la supplia de ne pas aller à ce
bal\,: «\,Il sera ridicule, sans remède. Cette pauvre Chazeron est
incapable d'organiser une fête galante\ldots{} Comment le marquis ne
s'est-il pas adressé à vous\,? Pour moi, je m'intéresse à ce bal pour
mon fils\ldots{} Je ne vais plus aux danses, j'en ai passé l'âge. Si
j'étais jeune et belle comme vous\,!\ldots{} Mais, hélas\,!\ldots\,»

Et Julie eut cette idée de génie, en parlant négligemment des Primelles,
d'exciter chez M\textsuperscript{me} Antoinette de Saint-Aubin les
mauvais instincts de la joueuse\,:

--- On prétend que ces Primelles viendront\,!\ldots{} Quelle
sottise\,!\ldots{} Moi, ce parierais, contre qui voudra, mille écus
qu'on ne les verra pas.

--- Et si je vous les tenais\,? avait répondu vivement Antoinette. Vous
seriez bien attrapée\,!

--- Eh\,! madame, je ne voudrais pas vous voler votre argent\,!\ldots{}

--- Avouez, bonne Julie, que vous avez peur d'être prise au mot\,!

--- Non point, madame, mais\ldots{}

--- Il n'y a pas de mais. Eh bien, moi, je vous gage que j'amène l'oncle
et la petite nièce au bal Chazeron\,!

--- Puisque vous y tenez, madame, je tiens le pari\,; mais vous avez
perdu d'avance.

Pour M\textsuperscript{me} d'Alloigny, cette beauté huguenote, froide et
sévère se décida par des raisons de jalousie très mesquines. Quand Julie
lui eut confié, sous le sceau sacré du secret, que c'était
M\textsuperscript{me} de Puyferrand, pas une autre, qui donnait ce bal
sous le couvert de M\textsuperscript{me} de Chazeron, pour y danser avec
Florimond, dont elle était amoureuse plus qu'une bête, la comtesse
d'Alloigny se jura de souffler le galant à sa rivale. Et comme Julie
avait tenu les mêmes propos à Henriette de Puyferrand, qui détestait
Diane d'Alloigny au moins autant qu'elle aimait Florimond, --- sans en
être payée de retour, car cette jeune dame était un petit peu trop
rousse, --- chacune des deux belles se jura de paraître à son avantage
et d'écraser son amie. La présence de l'Incomparable Florimond,
l'absence de sa mère, furent si délicatement opposées que la marquise
repartit pour Bannes avec la joie et la sécurité dans le cœur. Elle
savait, depuis la veille, que Marguerite de Primelles viendrait à
l'assemblée de Bourges avec son oncle Le Bouteiller.

Cette dernière affaire n'était pas allée toute seule. Si
M\textsuperscript{me} Isabelle de Rochefort n'eût senti sur son cou de
cygne le fer affilé du bourreau, elle aurait battu en retraite devant la
mine renfrognée du vieux baron de Mordicourt. Les raisons de ce mauvais
accueil n'étaient, au vrai, que dans la préoccupation de ce gentilhomme
campagnard de dissimuler la pauvreté de sa maison, et l'impossibilité où
se trouvait Marguerite de se procurer des habits convenables.
L'astucieuse complicité de Françoise Colbert aplanit tous les obstacles.
Une nouvelle entrevue ménagée entre Florimond et Marguerite fut
décommandée, ajournée, remise sans cesse au lendemain, de manière à
exaspérer l'impatience de la jeune fille. Dès lors, elle vécut dans
l'attente du moment heureux où elle reverrait l'élu de son cœur. Car
Marguerite était blessée à mort par la flèche de l'enfant malin qui vole
au hasard, aveuglé par un bandeau. La chambrière Colbert, qui recevait
la bonne parole de Clément Malompret, ne cessa plus dès lors de répéter
que cet heureux moment se présenterait à l'assemblée de Bourges où M.
Florimond pourrait librement causer avec M\textsuperscript{lle}
Marguerite au sein de la plus honnête société. L'important était donc de
décider le baron à mener sa petite nièce au bal. M\textsuperscript{me}
de Rochefort jouissait auprès de M. Le Bouteiller d'une influence à
nulle autre seconde. Fille de M\textsuperscript{me} de Latour-Champois
de Broislin, le vieux compagnon d'armes du baron de Mordicourt et comme
lui ennemi juré du Bourbon, elle avait conservé, après la mort de son
père, cette amitié où entrait autant de dévouement et d'estime que de
tendresse. Le baron de Mordicourt tenait la jeune comtesse de Rochefort
pour un abrégé des merveilles des cieux et ne se privait pas de la
donner en exemple à sa nièce.

Aussi se défendit-il mal contre les attaques persévérantes d'Isabelle.
Elle semblait prévoir ses objections, lui ruinait ses arguments à peine
formés. Aux yeux de M\textsuperscript{me} de Rochefort, la question des
vêtements --- et Colbert avait déjà arrangé cela avec Marguerite ---
était des plus mesquines.

--- Des robes, des jupes, des rotondes et des rubans, eh\,! monsieur mon
ami, j'en ai de pleins coffres. Mon mari est très bon pour moi et me
gâte, à Bourges comme à Paris, en me donnant ce qui se fait de plus
beau. J'en suis excédée. Tenez, ce matin même, un messager m'apporta ici
deux bahuts bourrés de ces babioles. À cette heure, Marguerite les
dénombre avec sa fidèle Colbert. Et cette chambrière est si adroite
qu'entre ses mains toute cette friperie reprendra l'aspect du neuf.

Le baron grommela, mécontent, des paroles indistinctes. Cela lui
déplaisait. Sa fierté ombrageuse souffrait de ces aumônes déguisées.
Mais personne ne le soutenait. M\textsuperscript{me} de Primelles, qui
ne se montrait point, à son ordinaire, avait laissé sa fille libre
d'aller avec son oncle à Bourges. Toutes les femmes décidément prenaient
parti contre lui. Il consulta Catherine de Lépinière, rencontrée au
cours d'une promenade matinale. Depuis la battue aux loups où il s'était
entretenu avec elle, le baron avait gardé bon souvenir de la demoiselle.
Elle l'avait séduit par sa franchise, son naturel et sa fierté ingénue.
M\textsuperscript{me} de Lépinière répondit à M. Le Bouteiller qu'allant
elle-même au bal de Bourges elle ne pouvait trouver mauvais que sa nièce
y allât. «\,Je la reverrai avec plaisir. Depuis le couvent de Bourges où
nous fûmes élevées côte à côte, je ne lui ai plus parlé.\,» Et elle
s'appesantit sur les tristesses de ces haines de famille, ébaucha un
éloge de Florimond, se réjouit des dispositions pacifiques du baron et
insista sur le plaisir qu'elle aurait de danser avec lui\,: «\,Car je
veux vous avoir pour danseur. Ne me refusez pas\,!\,»

Cet aimable badinage prouva au vieux gentilhomme qu'il n'avait d'aide à
attendre de personne. Et, bien plus, M\textsuperscript{me} de
Saint-Aubin arriva prêter renfort à tous les jupons ligués contre
l'oncle.

Fermement décidée à gagner ses mille écus sur la Drapière, elle accusa
M. Le Bouteiller, en particulier, de se dérober à la réconciliation que
cherchait Florimond, et publiquement de s'affirmer tuteur morose et
ennemi des aimables passe-temps de la jeunesse. «\,Quant à Marguerite,
on l'emmènerait d'autorité, voilà tout\,! Il ferait beau voir que M. Le
Bouteiller refusât la conduite de sa petite-nièce à Antoinette de
Saint-Aubin\,! Des robes, des affiquets\,?\ldots{} Mais j'en ai à
revendre\,! Et pourquoi nous empêcher, s'il vous plaît,
M\textsuperscript{me} de Rochefort et moi, d'obliger notre chère
Marguerite d'un peu de notre superflu\,? Et, d'abord, c'étaient là
histoires de dames qui ne le regardaient pas\,!\,»

Mal armé contre ces jolies femmes qui ne lui ménageaient pas les
caresses et l'attaquaient obstinément et gentiment à grand renfort de
chambrières, craignant d'être pris pour un tyran domestique, secrètement
flatté par les attentions dont se montraient prodigues et les unes et
les autres, couvert par l'approbation de M\textsuperscript{me} de
Primelles, le baron se rendit. Il se rendit surtout parce que sa
petite-nièce Marguerite l'avait embrassé et prié si tendrement que son
cœur, sans force contre toute tentative affectueuse, avait battu la
chamade aux premières douces paroles de cette enfant qu'il aimait d'une
affection timide et inquiète\,:

--- Allons\,!\ldots{} Allons\,! Vous êtes des folles, cela ne fait pas
de doute, aussi vrai que je suis un vieux fou\,! Et j'ouvrirai sans
doute le bal avec vous, Antoinette de Saint-Aubin\,? A moins que cette
gloire ne vous soit disputée par Isabelle de Rochefort\,?\ldots{} Vous
aurez là un beau galant pour marquer le pas à la gaillarde\,!

Et le baron de Mordicourt, saisissant les deux fines tailles, esquissa
un pas à la vieille mode. Tout le monde battit des mains, les
chambrières comme les dames. M\textsuperscript{me} de Saint-Aubin sauta
au cou du baron.

--- Vous êtes le meilleur des hommes\,!

--- Et l'on veut vous embrasser\,!\ldots{} Non, Antoinette\ldots{} je
suis jalouse, moi d'abord\,!

Mais un qui ne voulait pas qu'on l'embrassât ni qu'on parlât de
l'emmener au bal, c'était Louis-Antoine de Primelles. Il se cacha, comme
à son ordinaire, dans la garenne de Tonlieu, où, le menton sur les
paumes, les coudes à terre, il écoutait, couché à plat ventre dans
l'herbe, les conseils ou les reproches de son amie Catherine. C'était
Catherine qui lui avait défendu d'aller à l'assemblée de Bourges, dont,
d'ailleurs, il ne se souciait pas. Catherine irait, elle, et lui
raconterait tout par le menu.

Si M\textsuperscript{me} de Chazeron ne mourut pas du chaud mal, c'est
qu'il y a des grâces d'état pour les dames dont l'organisation d'un bal
accapare et toutes les heures et tous les moyens pendant une semaine ou
deux. Soigneuse et regardante, elle fit passer le balai et la brosse
dans les moindres recoins, lessiver, cirer les parquets, épousseter
jusqu'aux corniches, battre les rideaux, sans compter le reste. Plus
d'une famille de souris et de rats qui vivait en paix depuis des années
derrière les lambris ou dans les retraites obscures des placards fut
pour longtemps troublée dans ses joies domestiques. Les mites
déménagèrent. On eût dit d'une nouvelle fuite des dieux quand ils durent
abandonner l'Olympe, empruntant les espèces des ratepenades, des fourmis
et des araignées. M\textsuperscript{me} de Chazeron, quand elle eut
place nette, emprunta des tapisseries flamandes chez l'un, des chaises
et des fauteuils chez l'autre. Elle se querella avec
M\textsuperscript{me} de Saint-Agoulin pour les violons. Celle-ci
trouvait que c'était assez de huit, avec les hautbois et la basse, quand
M\textsuperscript{me} de Chazeron en exigeait quinze, sans préjudice des
flûtes et des violes de gambe. Des vieilles femmes jalouses intriguèrent
pour empêcher le célèbre Taraux Lelicheux, le musicien le plus réputé de
Bourges, de diriger cet orchestre. Leur cabale fut déjouée, et
M\textsuperscript{me} de Chazeron, désormais sûre d'elle-même, requit de
l'argenterie dans cinq familles et des vieux serviteurs de tout repos
pour y avoir l'œil, tant il est rare que dans les plus brillantes
assemblées ne se glisse quelque fallacieux filou\,! Au bal offert par la
présidente Maloison, n'avait-on pas surpris un chevalier de Malte
glissant trois couverts d'argent dans ses chausses\,? Enfin
M\textsuperscript{me} de Chazeron se coucha exténuée la veille du bal,
auquel elle ne put assister, clouée dans son lit par une courbature
compliquée de fièvre. La marquise de Bannes, aux premières nouvelles
d'un mal si soudain, était accourue à Bourges. Elle s'assit au chevet de
sa tante et jura que ses mains seules lui présenteraient tisanes et
électuaires. Maroie Lenatier et Nicole Deleuze, courant parmi les gens
de service, entretenaient la malade et sa garde de renseignements,
minute par minute.

D'après les nouvelles ainsi reçues, M\textsuperscript{me} de Chazeron
jugea que sa fête méritait d'être comptée parmi les plus notoires et de
Bourges et de tout le Berry. M. le prince y avait paru en personne, et
M\textsuperscript{me} de la Châtre aussi. Leurs officiers avaient dansé.
Quand elle sut que le prince de Condé avait promené dans sa grande
chambre sa figure de hareng sauret, la bonne dame leva les yeux vers le
ciel de son lit, joignit les mains sur sa couverture soigneusement
ramenée par Julie, et murmura\,: «\,Non, c'est trop\,!\ldots{} Je ne
méritais pas autant\,!\ldots{} Non\,! Non\,!\,»

Florimond ne se nourrissait pas de réflexions aussi modestes quand il
parut dans le bal. Il entra comme un dieu qui s'abaisse jusqu'à
descendre parmi les mortelles, car il ne saluait que les femmes. Et, à
le voir s'avancer hautain, avantageux et superbe, on songeait, malgré
soi, --- autrement c'eût été jalousie, --- à un coq doré, crêté, ergoté,
qui parcourt despotiquement le poulailler, son domaine.

Depuis huit jours, il avait obtenu une commission de lieutenant aux
gardes, et le manteau d'écarlate galonné, brodé, pasquillé à rangées
d'or, ajoutait à la richesse de ses habits et les rehaussait du prestige
attaché à la profession des armes.

--- C'est étonnant, dit M. de Montenay à M. de Mauny d'Anrieux, comme le
gaillard ressemble à ce Buckingham qui vint naguère en ambassadeur à
Paris, où il écrasa les plus braves de son luxe et de sa prodigalité
quasi barbares\,! Et n'eut-il pas l'effronterie de lever les yeux sur la
jeune reine\,? La fin de ce méchant Anglais ne fut pas heureuse. Je
souhaite que ce Florimond arrogant et médiocre n'ait pas un meilleur
destin, dussé-je y aider pour ma part\,!

--- Ses manières, mon cher Montenay, ne me reviennent pas plus qu'à
vous. Mais les causes de son succès ne lui sont point personnelles. Les
femmes se décident pour des raisons tout à la fois vagues et précises
qui ne sont heureusement pas les nôtres, quoique, à vrai dire, la
sagesse n'y intervienne pas. Croyez-moi, la droiture et la franchise
dont nous abusons vous et moi, soit dit entre nous, ne sont pas des
marchandises qui plaisent aux dames. Elles s'attachent au succès sans en
rechercher les causes. Seuls les résultats sont pour les intéresser.
Plus un bellâtre, de l'espèce de cet odieux Florimond, se trouve chargé
de crimes amoureux, plus il plaît au sexe enchanteur. Qu'on sache que le
galant a causé le déshonneur et la ruine de quelque douze familles, les
dames et les demoiselles s'en émeuvent, toutes prêtes à s'offrir en
holocauste sur l'autel de l'amour. Le monde est ainsi fait. Ni vous ni
moi ne le changerons.

Ainsi les deux amoureux platoniques de Marguerite de Primelles
échangeaient leurs rancœurs et leurs regrets sans en dénoncer plus
explicitement les motifs. Et M\textsuperscript{me} de Primelles, qui ne
songeait guère à eux, n'avait d'yeux que pour l'Incomparable Florimond.
Le luxe extravagant de ses vêtements, les rangs de perles qui se
croisaient en sautoir sur le pourpoint et se tordaient autour des
taillades, les boutons en pierreries de ses manches déchiquetées, les
dentelles de son col et de ses manchettes, tout cela n'était, au regard
de la jeune fille fascinée, que la continuation logique de sa personne,
aussi bien que ses merveilleux cheveux blonds, avec leur moustache
enfilée dans une turquoise, aussi bien que ses yeux impérieux et doux,
aussi bien que son air à la fois léger, nonchalant et résolu.

Dans sa robe de taffetas colombin agrémentée de velours gris de rat,
doublée de damas fleur de seigle, Marguerite de Primelles, sans joyaux,
bracelets ni colliers, semblait une délicate fleur des champs oubliée
dans une plate-bande parmi les roses, les pivoines et les tulipes.
Autour d'elle, un essaim de femmes magnifiquement vêtues caquetaient,
livraient aux hommes non moins richement couverts les assauts de la
coquetterie permise, avec cette audace impudente et tranquille qui fait
le charme de ces assemblées, où seule la retenue du geste indique qu'on
ne s'est pas fourvoyé dans un mauvais lieu. Modeste et troublée, oubliée
sur son pliant, personne ne remarquait Marguerite. L'oncle Mordicourt,
vêtu à l'antique avec ses chausses à la suisse de lucquoise puce et son
pourpoint busqué en fleuret pain bis, à manches de velours isabelle
tracées d'argent, debout au coin de la haute cheminée, causait de
l'affaire des Ponts-de-Cé avec d'autres vieux gentilshommes, portant,
comme lui, des collerettes en façon de meules.

Marguerite de Primelles, bien que personne ne parût s'en occuper dans la
foule empanachée qui se pressait de plus en plus dense et laissait bien
juste la place utile aux danseurs, se trouvait cependant soumise à la
surveillance de quatre personnes, qui, indifférentes et distraites en
apparence, ne la perdaient point de vue. Tandis qu'éblouie par l'éclat
des deux cents bougies dont les feux valaient ceux du jour elle
regardait son Florimond de bien loin, lui, entouré autant que M. le
prince l'avait été pendant sa courte apparition, la couvait de ses yeux
caressants et joyeux, par-dessus les épaules nues et les têtes bouclées
par l'artifice du fer.

Du coin où ils s'étaient réfugiés, tels deux philosophes assistant à un
banquet de débauchés dans la Rome antique, M. de Montenay et son ami, M.
de Mauny d'Anrieux, examinaient avec un pareil intérêt cette Marguerite
pour laquelle chacun d'eux se fût lancé dans des entreprises
fantastiques et rares. Et, enfin, M\textsuperscript{lle} Catherine de
Lépinière, assise auprès de la marquise de Creulles, dont les vastes
jupes de brocatelle minime et de satin ondé couleur inde
s'épanouissaient de manière à cacher aux trois quarts la belle-fille du
marquis de Bannes, lorgnait du coin de l'œil son ancienne amie de
couvent. Si M\textsuperscript{me} de Creulles et ses jupes cachaient
Catherine aux trois quarts, le dernier quart disparaissait sous la robe
souci de M\textsuperscript{me} de Saint-Agoulin. Et
M\textsuperscript{lle} Catherine, habillée de velours pastel, ne
montrait guère que sa mine rieuse, éveillée, mal coiffée, mais attentive
et résolue comme toujours. Rien ne lui échappait des œillades amoureuses
dont Florimond foudroyait Marguerite fascinée.

Elle la vit pâlir affreusement et se rejeter en arrière, défaillante,
quand Florimond reçut avec une condescendante bonne grâce le bouquet
d'une dame qui l'engageait à danser. Au vrai, il y avait quatre
bouquets, et quatre dames qui s'offrirent d'un même temps. Mais, plus
ardente au succès que ses rivales, Isabelle de Rochefort réussit la
première à placer son piquet de fleurs. Elle le poussa aux mains de
Florimond d'une telle vivacité et d'une telle force que les trois autres
nymphes, écartées, distancées, indignées, laissèrent échapper un même
cri\,: «\,Oh\,! madame\,!\,» La belle comtesse d'Alloigny en perdit, de
dépit, cette sévère sérénité de ses traits qui faisait d'elle une autre
Minerve. Foudroyant d'un regard aussi noir que ses fins cheveux
descendant en cent boucles calamistrées sur le haut col en entonnoir qui
cachait ses épaules de marbre Rochefort la triomphante, elle regagna sa
place. Muette et tremblante, se mordant les lèvres, elle jeta son
bouquet à un jeune gentilhomme qui en faillit mourir de joie. Et, comme
Diane d'Alloigny avait son tabouret placé devant le pliant de Catherine
de Lépinière, celle-ci disparut complètement.

Moins maîtresse de soi que la plus belle entre les protestantes du
Berry, M\textsuperscript{me} de Saint-Aubin, secouant sa crinière
parfumée, couleur de tan, couronnée d'une aigrette, dit à mi-voix\,:
«\,Voyez l'effrontée\,!\,» et s'abandonna aux fades galanteries d'un
officier du prince, qui s'empara de son bouquet. Et Henriette de
Puyferrand, aux cheveux blond cendré, ébouriffés avec art, décolletée
assez bas pour qu'on ne perdit rien des perfections de sa gorge,
s'oublia jusqu'à tirer la langue quand la comtesse de Rochefort, plus
fière qu'une reine des Amazones, s'avança, au poing de Florimond, dans
le grand espace libre réservé au milieu de la pièce pour la panadelle.

Cette danse convenait bien à ce couple. L'orgueil, la vanité,
l'insolence du paon quand il étale à bon escient les émaux de sa roue au
grand soleil, appartenaient également à l'Incomparable Florimond et à
l'irréprochable Isabelle de Rochefort.

De celle-ci la face pâle, à peine avivée par une pointe de rouge aux
pommettes, les yeux brillants, le sourire contenu, les sourcils portés
hauts, la sveltesse de la taille, l'aisance souple et fière des gestes,
s'harmonisaient avec la robe sévère de satin noir, de velours noir,
ouverte sur la cotte de toile d'or noir, avec les bijoux de jais et les
plumes noires de la coiffure. Tout ce noir faisait valoir la blancheur
de la peau fine que le large décolletage carré découvrait mate ainsi que
les pétales des fleurs d'un arbre printanier. Isabelle de Rochefort,
malgré sa chevelure soyeuse et légère avec chacune de ses boucles
terminée par une pampille d'argent noir, malgré ses yeux veloutés,
malgré son charmant visage, avait quelque chose de vipérin\,; et cela
tenait autant à la flamme voilée de ses profondes prunelles qu'à la
qualité de son sourire, où les lèvres vermeilles, très minces, avaient
moins de part que les dents. Elle dansait à pas menus et traînés,
semblait glisser sur le parquet luisant, qui reflétait son image, ainsi
qu'une jolie poupée montée sur des roulettes, sans que ses traits
remuassent, sans que son buste accompagnât le mouvement de la danse que
scandait seul un imperceptible balancement de ses bras, au cours des
changements de main.

Aussi éclatant qu'un ostensoir, Florimond, doré, argenté, brodé des
larges bouffettes de ses souliers carrés jusqu'à son feutre emplumé plus
vaste qu'un parasol et dont les pennes d'un démesuré panache ondulaient
sous les bougies des lustres, se dressait comme une protestation vivante
contre les édits et règlements sur les superfluités des habits. Il
exécutait ses pas avec une telle noblesse que l'assemblée tout entière
ne s'intéressait plus qu'à lui. Henriette de Puyferrand en jaunissait de
dépit, malgré les compliments de M. Aimeri d'Olivier, attaché à sa
plaisante personne, ainsi qu'une brune sangsue l'est au baigneur
imprudent qui s'ébat dans un marécage. M. Aimeri récitait à la belle des
vers qu'il donnait comme de son cru, car le plagiat comptait parmi ses
plus habituelles ressources\,:

\begin{center}
\parbox{\mylen}{\textit{Si l’amour quelque part bâtit son paradis,                    \\
                        C’est où l’on fait ballet. On y voit face d'anges             \\
                        Au lieu d’astres  .   .   .   .   .   .   .   .   .   .   .}} \\
\end{center}

M\textsuperscript{me} de Puyferrand n'écoutait pas, car la poésie était
en tout incapable de combattre la jalousie qui habitait son cœur\,:

--- Taisez-vous\,! dit-elle à M. Aimeri, vous m'empêchez de suivre la
danse\ldots{} Il est vraiment malheureux qu'une maladroite aussi réputée
que cette Rochefort accapare cet unique danseur qu'est votre maître, le
baron de Chézal-Benoît.

--- Ah\,! madame, que vous parlez bien\,! Je n'ai pas mieux dit dans
cette petite pièce que je veux vous réciter sans tarder\,:

\begin{center}
\parbox{\mlena}{\textit{Aux sons des violons qui donnent la cadence,       \\
                        L'œil observe attentif celle qui le mieux danse…}} \\
\end{center}

--- Certes, monsieur Aimeri, celle qui danse le mieux n'est pas Isabelle
de Rochefort. Regardez-la\,!\ldots{} Quelle chipie, tout d'une pièce, et
avec un mauvais regard\,!\ldots{}

M. Aimeri, arrondissant son geste, la bouche en cul de poule,
continuait\,:

\begin{center}
\parbox{\mlenb}{\textit{Avecque plus de grâce, ou celle qui fait mal.}}
\end{center}

--- Si vous avez voulu parler de Rochefort, ce dernier trait la peint
mieux que tout autre. Monsieur Aimeri, vous êtes un maître homme, et je
veux que vous me traciez un portrait de cette sotte, et de votre
meilleure encre. Je saurai payer\ldots{}

Non loin de ce couple où la poésie se mettait à la solde de la
vengeance, M\textsuperscript{me} d'Alloigny épanchait sa bile dans le
sein de M\textsuperscript{me} de Creulles vers qui elle s'était
retournée, opposant avec insolence le dos de sa chaise à Florimond et à
sa danseuse.

Personne ne faisait attention à la pauvre Marguerite. Éblouie par les
grâces cavalières de Florimond, elle demeurait là, faible d'émotion, aux
côtés de Catherine de Lépinière, dont elle s'était rapprochée, pour se
sentir moins seule, au moment où l'on avait dégagé le milieu du salon
pour la panadelle.

--- Vraiment, madame, disait Diane d'Alloigny, l'on n'a pas idée d'une
pareille audace\,! Cette Rochefort mériterait d'être fouettée\ldots{}
S'afficher ainsi avec un homme qui, au su de tous, est son
amant\,!\ldots{}

Catherine vit Marguerite pâlir et tressaillir légèrement en portant sa
main à sa poitrine comme si elle s'était senti frapper.
M\textsuperscript{me} d'Alloigny, emportée par son dépit, n'arrêtait pas
d'accuser Isabelle\,:

--- Oui, madame, elle nous a repoussées pour l'engager à la danse, et
d'une telle force que l'envie me tenait de la battre\ldots{} Quelle
effrontée\,!\ldots{}

--- Mon Dieu, répondit M\textsuperscript{me} de Creulles, je ne prétends
pas, ma chère, défendre Rochefort. Elle me déplaît. Votre attaque
m'apparaît cependant injuste. La vie d'Isabelle est nette de faute.
Chacun ici connaît sa vertu. Et, si Florimond était son amant, je serais
des premières à le savoir, car on me met au courant de tout.
M\textsuperscript{me} de Saint-Agoulin m'en est garante, tout cela n'est
qu'inventions et mensonges.

Les couleurs revinrent aux joues de Marguerite. Elle jeta un coup d'œil
reconnaissant à M\textsuperscript{me} de Creulles, et Catherine commença
de soupçonner la vérité\,:

«\,Sainte Mère de Dieu, songea-t-elle, Marguerite est amoureuse de
Florimond\,! Celui-ci cherche à la séduire, sans doute. Non content de
préparer le meurtre du frère, poursuivrait-il le déshonneur de la
sœur\,? Hélas\,! une aussi méchante action est en tout digne de son
caractère. Mais je veillerai\ldots{} Eh bien, voilà tout, cela m'en fera
deux à protéger\,!\,»

Assise à l'autre bout de la salle, parmi les dames de la magistrature,
M\textsuperscript{me} Godefroy Harant se débattait sous les morsures de
la jalousie et de l'envie. Elle aussi avait tenté d'engager Florimond
pour la danse en lui présentant son bouquet. Coudoyée, foulée, bousculée
par les quatre belles de Bourges et leur suite, la blonde Jeanne de la
Pelice s'était vue repoussée, rejetée dans la foule, sans même avoir pu
attirer l'attention de son amant.

Et pourtant quelle persévérance dans l'intrigue\,! quelle patience à
supporter les avanies\,! quelle souplesse, quelle opiniâtreté, quelle
variété de moyens n'avait-elle pas déployées pour se faire inviter à ce
bal, où, derrière elle, les nobles de robe s'étaient glissées en
foule\,! Personne, d'abord, ne voulait d'elle. Décriée à cause de sa
liaison un peu trop notoire avec Florimond, elle supportait, tout comme
Julie la Drapière, le poids de la haine et du mépris de la noblesse de
Bourges. En dépit de sa naissance, on ne l'appelait jamais que
«\,Mademoiselle Harant\,». La marquise de Bannes, quoiqu'elle n'aimât
nullement la dame, qui ne faisait pas, à ses yeux, assez honneur à son
fils, la ménageait pour diverses raisons. La principale était dans la
nécessité de se conserver des appuis auprès des juges. Car, ignorant
tout des dispositions testamentaires du marquis, Julie craignait, si
celui-ci mourait à l'étranger, d'être dépouillée par la famille. Elle
connaissait trop bien Florimond pour attendre de lui une intervention
favorable. Et, d'ailleurs, son fils lui-même serait peut-être victime de
parents avides qui n'hésiteraient pas à attaquer une légitimation que
certains avaient jugée un peu bien précipitée.

M\textsuperscript{me} de Saint-Agoulin, qui cherchait à se concilier,
par tous les moyens, les magistrats, suivant en cela la pratique des
gens dont les affaires sont louches, avait soutenu Julie et Isabelle de
Rochefort, que l'impitoyable Drapière obligeait de marcher dans ses
voies. Il fut donc décidé que l'on convierait la noblesse de robe à
l'assemblée du 20 juin. On aurait ainsi un coin sombre qui rehausserait
les claires splendeurs de la belle noblesse. Les robes et les
soutanelles de satin noir donneraient à la fête un caractère sérieux et
honorable qui n'en pourrait qu'augmenter l'éclat.

Aussi, quand on sut que les parlementaires amis de ces dames viendraient
à l'assemblée, M\textsuperscript{me} de Saint-Aubin et sa coterie, pour
ne pas demeurer en reste, invitèrent de leur côté tout ce qu'elles
avaient de relations dans la robe. Les présidentes et les conseillères
se crurent obligées de paraître à leur avantage dans un bal où les
femmes de la grande noblesse devaient rivaliser de luxe. Narguant les
édits que leurs maris promulguaient sans sourire, toutes ces robines
étalèrent des traînes de brocart, de damas, de drap d'or, des rotondes
brodées en rosaces, des cols monumentaux et des manchettes en dentelles
plus ténues que toile d'araignée. Telle se pomponna d'aigrettes de prix,
telle autre de bijoux rares à faire pâlir d'envie les duchesses. C'est à
peine si Florimond, qui avait vidé pour la circonstance et les écrins de
sa mère et ceux de Nicole Deleuze, pouvait soutenir la comparaison avec
ses rangs de perles, ses boutons de pierreries, ses bagues et ses
chaînes de cou.

Quand il dansa avec la comtesse de Rochefort, M\textsuperscript{me}
Godefroy Harant se retint à quatre pour ne pas pleurer de colère. Mais,
redoutant la malveillance de l'entourage qui épiait avec une joie féroce
les moindres signes de son chagrin, elle se composa un visage riant et
badin et eut ce courage de chanter les éloges de la comtesse de
Rochefort et de critiquer la façon de danser de Florimond. Avec une
coquetterie savamment dosée qui enivra sa dupe, elle donna son bouquet à
un officier novice en lui confiant que leur prochaine danse vaudrait
bien celle de ces deux orgueilleux plus semblables à des figures de cire
qu'à de bons chrétiens. Elle abandonna négligemment sa main au jeune
homme, qui conçut aussitôt les plus flatteurs espoirs à la vue de cette
blonde encore si belle, tant élégante, et que jusque-là il n'avait
rencontrée qu'en carrosse et toujours masquée.

Et les voisines de Jeanne de la Pelice, épouse du conseiller Godefroy
Harant, conçurent une grande joie à l'idée que cette arrogante
conseillère, tout à la fois sucrée et pointue, ne tarderait pas à leur
fournir de nouveaux sujets de scandale.

Les violons cependant grinçaient, les hautbois pleuraient, la basse ne
cessait point de ronfler. Les danses succédaient aux danses, les
courantes aux gaillardes, les pavanes aux sarabandes, les figurées aux
branles. Et toujours de nouveaux invités envahissaient la salle. Dans
l'escalier c'était un va-et-vient continu entre ceux qui arrivaient et
ceux qui partaient\,; on montait, on descendait, on dansait jusque sur
les paliers, jusque sur les marches. Les degrés du perron servaient de
table de jeu à la valetaille. Les traînes des robes balayaient les
cartes et les enjeux. Tel drôle accroupi criait un «\,Quinola\,!\,» et
tombait aussitôt en avant poussé par un valet de pied criant plus haut
encore\,: «\,Place aux gens de M\textsuperscript{me} de Peyraffet\,!\,»
La cour de l'hôtel était une autre salle de bal en plein air, et la rue,
éclairée comme en plein jour par tous les flambeaux de poing passés aux
anneaux de la façade, continuait la fête. L'on menait des branles et des
bourrées au son des cornemuses et des vielles. Juchés sur les bornes ou
sur des tonneaux, les ménétriers s'évertuaient, puis quêtaient en
tendant leur chapeau où pleuvaient les liards. Pages, laquais, cochers,
porteurs de chaises, chambrières et petites servantes sautaient à
l'envi, disloquant tout juste leurs groupes serrés au passage d'un
cavalier ou d'un carrosse. Attachées à toutes les ferrures de la façade,
car depuis longtemps il n'y avait plus dans la cour place pour une mule,
les bêtes ruaient, s'ébrouaient au milieu de la foule affairée à ses
plaisirs, où s'empressaient les servantes des tavernes voisines avec des
brocs de vin, des pots d'hypocras, des tranches de jambon et des pains
en couronne passés au bras.

Depuis le pain bénit de M\textsuperscript{me} de Saint-Aubin, les rues
des Arènes et de Suez n'avaient vu pareille affluence de populaire. Et
Marion, la Hollandaise venue de la guerre, à califourchon sur sa jument
souris, attendait, suivant son habitude, M\textsuperscript{me} de Mauny
d'Anrieux, dont elle gardait le cheval enrêné derrière le carrosse de
M\textsuperscript{me} de Montenay. Et chacun admirait, sans oser
l'approcher, autant par crainte du maître que du fouet à la polonaise,
cette superbe femme blonde et fraîche, fière et paisible sous ses habits
d'homme galonnés, récamés des bordures aux tailles, or sur bleu, avec
l'épée wallonne au côté, les bottes à la croate, et le col de point
coupé, large et rond par derrière, droit par devant en façon de pelle,
et sur quoi roulaient les écheveaux épais de sa crinière couleur de
miel, rattachés chacun par une cocarde noire à ferrets d'acier bleu. Du
haut de sa selle en velours, Marion regardait le bal forain sans y
prendre part, et jouissait de l'admiration et de la considération
publiques.

Quand Florimond quitta le bal, son carrosse eut peine à se frayer un
chemin à travers ce peuple. La guirlande interminable d'un branle
s'égrenait encore dans la rue des Arènes alors que sa tête décrivait des
cercles sur la place des Jacobins. A l'horizon montait la pâle lumière
du jour, blafarde, couleur d'étain, avec des stries cuivrées, misérable
devant les feux rouges et orangés des flambeaux et des torches.

Florimond s'entretenait avec M. Aimeri d'Olivier sur la banquette du
fond. En avant, MM. de Tourouvre et de la Butière sommeillaient, de
telle sorte que ni les uns ni les autres ne virent la lingère Madelon
qui guettait son amant, perdue dans cette multitude en délire.
Enveloppée dans un long manteau, masquée comme une dame, appuyée au bras
de la Macette pareillement emmitouflée, la lingère s'était glissée là
sous la protection de deux laquais armés qui s'étaient faits forts de la
mener jusqu'à leur maître. Mais Madeleine Brossin, dite Madelon, en fut
pour ses frais\,; car la voiture passa avec ses pages porteurs de
flambeaux, sans que Florimond favorisât même d'un regard les deux
commères, dont la plus jeune, ayant mis son touret de nez à la main,
montrait son rose minois à découvert. Des préoccupations plus hautes
tenaient le fils de la Drapière. Il recevait des compliments de son
poète pour l'habileté supérieure dont il avait fourni des preuves
pendant le bal. Aucune occasion n'avait été perdue. De toutes, Florimond
avait profité avec une extraordinaire finesse. Par des galanteries
savamment espacées, il avait tour à tour désespéré et ravi les quatre
belles de Bourges, donné satisfaction à la jalouse Jeanne de la Pelice
en l'engageant une fois à danser\,: «\,Si je vous courtisais davantage,
ce serait déchaîner la calomnie, pour le plaisir.\,» Enfin, il avait
réussi à remettre entre les mains de M\textsuperscript{lle} de Primelles
un message amoureux, œuvre de M. Aimeri d'Olivier.

Florimond, après avoir flatté, empaumé le vieux baron de Mordicourt par
des compliments frappés au coin de la parfaite sincérité, s'était
évertué à le toucher. Avec des pleurs dans la voix, --- et son émotion
n'échappa point aux antiques amis qui se tenaient aux côtés de ce
représentant du passé, --- il avait ouvert son cœur\,: «\,Hélas\,!
monsieur, quelle position malheureuse est la mienne\,! Mon désir est de
rendre un public hommage à votre nièce et de lui témoigner toute ma
respectueuse bonne volonté. Mais, hélas\,! la fatalité, qui se complaît
à diviser nos deux familles, me défend de l'aborder et de la prier pour
danser\,!\,»

Longtemps, Florimond, encouragé par les murmures approbateurs des vieux
gentilshommes, avait continué sur ce ton. C'était à tirer les larmes des
yeux\,: «\,Qu'attendaient les Maréchaux pour arranger tout cela\,?
Bannes et Primelles ne pouvaient demeurer éternellement
ennemis\,!\ldots{} Que diable\,! On parlerait à M. le Prince, et on
s'adresserait même plus haut\,!\ldots{} Pourquoi ne pas aller jusqu'au
roi, en cas de besoin\,? Le marquis de Bannes, à tout prendre, ne
pouvait rester éternellement exilé.\,» Florimond craignit un instant
d'avoir dépassé le but. Personne ne désirait moins que lui le retour du
marquis son père. Heureusement que le baron de Mordicourt ramena
l'entretien sur un autre terrain. Se croyant obligé à consoler ce jeune
seigneur dont la délicatesse valait la franchise, il lui promit de
s'employer à cimenter une paix après laquelle il soupirait plus que
personne. Et Florimond l'avait laissé dans l'enchantement, pour
s'occuper de cette Marguerite que l'impitoyable destin lui défendait
d'engager pour la danse.

Il sut saisir le moment utile. Quand les dames passèrent de la salle du
bal dans la chambre où l'on servait la collation, il se précipita,
flanqué par La Butière et Tourouvre, de manière à couper la file juste à
hauteur de M\textsuperscript{lle} de Primelles, dont le chaperon était
M\textsuperscript{me} de Saint-Agoulin. Séparée des femmes qui suivaient
par les deux gentilshommes formant digue contre le courant, Marguerite
sentit Florimond à ses côtés. Tout en lui glissant avec une impérieuse
douceur le billet plié et replié jusqu'à ne pas dépasser les dimensions
d'un grain de sucre, il lui avait murmuré à l'oreille\,: «\,Prenez et
lisez pour l'amour de moi\,!\ldots{} Mon sort est lié au vôtre. Vous
êtes reine et maîtresse de mes destinées. Un mot de vous, et j'abandonne
tout\,!\ldots\,» Il ne put ou plutôt n'en voulut pas dire davantage\,;
la foule, lâchée par ses deux braves qui s'écartèrent en se doublant,
sur un signe de lui, sépara Florimond de la jeune fille. Serrant
vivement dans son gant ce papier qui la caressait et la brûlait,
Marguerite eut bien juste la force de gagner la table de la collation.
Une chaise vide se présenta de fortune\,; elle s'y assit, tremblante\,;
ses jambes refusaient de la porter, et elle ne se douta pas que
Catherine de Lépinière lui avait poussé ce siège pour qu'elle ne tombât
point de son haut sur le plancher.

Car, lorsqu'il racontait à Aimeri le merveilleux succès de son
entreprise, Florimond était loin de compte avec la belle-fille de son
père.

--- Oui, mon cher Aimeri, tu peux m'en croire\,! Je lui ai filé mon
billet avec une telle adresse --- on a l'habitude ou on ne l'a pas, que
diable\,! --- que tous ces gens n'y ont vu que du feu. J'observais la
demoiselle tout en lui chuchotant à l'oreille les galanteries
nécessaires. Je la vis pâlir, palpiter\,; encore un peu, elle se pâmait
en plein couloir\ldots{} Va, Aimeri, elle est à moi, la belle
Marguerite, elle est mon bien, ma chose\,!\ldots{} Ou, si tu préfères,
la reine de mon cœur\,!\ldots{} Au reste, je me sens du goût pour cette
beauté rustique, et, à tout considérer, comme maîtresse elle en vaudra
bien une autre. Le difficile maintenant sera de l'attirer hors de son
nid à rats\ldots{} Je l'enlèverai, Aimeri, nous l'enlèverons, homme de
génie\,!\ldots{} Rien ne nous est impossible\,!\ldots{} Et
après\,?\ldots{} Après\,? Vogue la galère\,!

M. Aimeri secoua la tête, et Florimond put croire que c'était
d'admiration. Ce mot de galère sonnait toujours mal aux oreilles du
poète entretenu\,: «\,Tu en feras tant, pensait-il, qu'elle voguera, en
effet, la galère. Et tu seras dedans, occupé à la vogue pour le service
du roi\,!\,» Et il demanda à Florimond, tout en le louant pour son
audace et sa prudence\,:

--- Êtes-vous bien sûr que personne ne vous a vu remettre votre lettre à
la demoiselle\,?

--- Sûr, Aimeri, absolument sûr\,!\ldots{} Voyons, réfléchis un peu\,!
Comment veux-tu que dans ce couloir, assez étroit pour que deux
personnes puissent bien juste y passer de front, l'on ait pu me voir
par-dessus les épaules de Tourouvre et de la Butière, qui sont au moins
aussi grands que moi\,?

--- Mais, mon cher enfant, si l'on n'a pu voir par-dessus les épaules de
ces braves, l'on a pu très bien voir par-dessous. N'y avait-il personne
derrière eux\,?

--- Aimeri, tu broies toujours du noir\,! Le couloir était d'ailleurs un
peu obscur, et l'on s'y bousculait si gentiment qu'on ne savait qui vous
coudoyait, vous précédait ou vous suivait. Finis-en donc avec ces
craintes chimériques et ne pense plus qu'à notre grand dessein. Je ne
veux pas que deux semaines passent avant que je n'aie enlevé la belle
Marguerite à la barbe de son oncle\ldots{} Et ce sera bien joué. Je la
prendrai comme gage de réconciliation\,!

Les soupçons de M. Aimeri d'Olivier n'avaient rien de chimérique.
Catherine de Lépinière, attentive à surveiller Marguerite de Primelles,
avait vu Florimond lui passer son billet parmi les vastes plis de la
robe en damas orange de M\textsuperscript{me} de Saint-Agoulin.

\hypertarget{chapitre-x}{%
\chapter{CHAPITRE X}\label{chapitre-x}}

Comme tous les enfants qui ne prennent conseil que d'eux-mêmes,
Marguerite de Primelles devait se laisser porter aux plus folles
résolutions. Cette jeune fille de dix-sept ans, aveuglée par un amour de
roman à l'âge où la plupart des filles de son état avaient depuis trois
ou quatre années contracté mariage, se livra corps et biens au séducteur
pervers qui prenait son honneur pour jouet. Quand elle eut commencé
d'entretenir avec Florimond une correspondance où M. Aimeri d'Olivier
dépensait sa meilleure encre sous la signature pastorale de Tircis,
Marguerite, qui s'y appelait Astrée, ne respira plus que pour son
berger. Tout disparut\,: famille, dignité, devoirs. Elle ne pensa plus
qu'à ce Florimond par qui elle avait commencé de vivre. Elle ne lui
cacha pas son ferme propos de s'unir à lui pour jamais. Puisque les
préjugés brutaux et grossiers les séparaient, puisqu'il était impossible
que la fille du baron de Primelles épousât publiquement le fils du
marquis de Bannes, elle se marierait avec lui secrètement.

Était-ce leur faute, après tout, si le père de Florimond avait tué le
père de Marguerite\,? Et faudrait-il que leur vie fût à jamais gâtée par
les crimes ou les malheurs de leurs proches\,? Mille fois non\,!
Marguerite de Primelles n'était point de ces âmes vulgaires promptes à
se plier aux règles imposées par la prudence humaine. Elle se tenait
pour une de ces natures d'exception qui ne sont point justiciables du
tribunal de la médiocrité. Aux grandes pensées doivent succéder les
grandes actions\,: elle avouerait son amour pour Florimond aux quatre
vents du ciel\,; elle suivrait l'élu de son cœur où il la voudrait
conduire. Elle s'en remettrait à lui de la protéger, et elle régnerait
en souveraine tendrement et despotiquement sur son cœur.

C'est pourquoi Marguerite de Primelles entra dans les projets de
Florimond avec une facilité qui ne laissa pas d'étonner un peu celui-ci.
Mais cet étonnement fut éphémère, tant l'incomparable fils de Julie
nourrissait de confiance dans le pouvoir de sa personne. Incapable de
comprendre que son succès était moins dû à sa propre qualité qu'aux
circonstances, il se fit honneur d'une aventure misérable où tout galant
de quelque expérience eût été capable d'entraîner la fille intraitable
et bornée, à l'esprit faussé par des lectures prises au pied de la
lettre, qu'était M\textsuperscript{lle} de Primelles. L'isolement,
l'oisiveté, la vanité, l'activité sans objet d'un esprit condamné à
tourner ainsi qu'une meule à vide, avaient à tout jamais faussé cette
intelligence médiocre dont les barrières étaient irrévocablement tracées
par une obstination sauvage.

Florimond, d'ailleurs, en s'attachant à perdre M\textsuperscript{lle} de
Primelles, suivait moins les projets de vengeance de sa mère que les
inspirations d'un caprice. Quand il se décida à enlever Marguerite, il
en était aux trois quarts amoureux. Et c'est de quoi se réjouissait M.
Aimeri d'Olivier, qui, suivant avec son génie discipliné et étroit les
instructions de la marquise, trouvait dans cette ardeur une garantie de
la réussite de ses desseins. Chargé de composer les épîtres galantes de
Florimond à sa bergère, il y consumait plus d'art qu'un agent politique
du cardinal ministre n'en dépensait pour rendre compte des menées d'un
commissaire impérial. Et le poète entretenu, à écrire de si jolies
choses, se surprenait à penser, avec amertume, que cette prose ou ces
vers, signés tout bonnement par lui, n'obtiendraient jamais le prix
qu'en attendait le berger Tircis, autrement dit le jeune baron de
Chézal-Benoît.

M. Clément Malompret était le messager de choix qui assurait le
fonctionnement régulier et secret de cette correspondance amoureuse.
Chaque billet passait de ses mains dans celles de Françoise Colbert, qui
le remettait à sa jeune maîtresse et renvoyait la réponse par les mêmes
moyens. Et tout cela avec tant de précautions et de prudence que
personne ne se doutait à Primelles de ces intrigues dont Catherine de
Lépinière elle-même ne parvenait pas à connaître la nature exacte.

Pendant des jours, elle courut par les bois et par les champs, avec son
oiseau ou ses chiens, pour surveiller Florimond et surprendre quelqu'une
de ses entrevues avec Marguerite. Elle perdit son temps. Depuis qu'ils
communiquaient par lettres, leurs rendez-vous avaient pris fin. Et
M\textsuperscript{lle} de Lépinière ne put rien connaître des projets de
Florimond. Elle remarqua seulement que des corvées ne cessèrent de
travailler à remettre en état le chemin carrossable qui, ayant sa tête à
la forêt des Usages, rejoignait entre Chérigny et Poireuil la route qui
mène à Issoudun par la région boisée de Chézal-Benoît et de Cheurs. Ce
chemin, depuis longtemps sans emploi puisqu'il commençait au château de
Primelles pour entrer aussitôt dans les terres de Bannes, était retourné
à la nature. La végétation l'avait recouvert. Aujourd'hui, sur une
longueur de plus de deux lieues, les tenanciers du marquis s'occupaient
à le rendre praticable.

Au départ de la Drapière, qui se rendit à Paris avec Nicole Deleuze, le
7 juillet, succéda bientôt celui de Florimond. Accompagné par Aimeri
d'Olivier et ses fidèles La Butière et Tourouvre, sans préjudice des
grands laquais, il s'était mis en route avec ses meilleurs équipages,
annonçant à qui voulait l'entendre qu'il s'en allait à Paris pour y
remplir les devoirs de son grade de lieutenant aux gardes. L'écuyer
Piccolomini avait pris place dans la suite de la marquise.
M\textsuperscript{lle} de Lépinière demeura donc seule à Bannes avec sa
maison, qui s'augmenta de Maroie Lenatier. La belle fille d'atours était
tombée en disgrâce pour avoir roussi plus que de raison une papillote de
Julie.

Demeurée presque seule au château, elle offrit ses services à
M\textsuperscript{lle} de Lépinière. Celle-ci les accepta parce qu'elle
voyait en Maroie une alliée sûre. La fille d'atours n'était-elle point
l'amie de Robert de Rustigny, l'écuyer de M\textsuperscript{lle} de
Primelles\,? Ainsi, Catherine pourrait, au besoin, se tenir au courant
des actions de Marguerite, qu'elle jurait de surveiller sans relâche.
Sourde aux prières de M. de Montenay, son tuteur, qui lui conseillait de
se retirer à Bourges chez la marquise de Creulles, Catherine continua
donc d'habiter Bannes.

--- Je suis obligée d'y résider, lui dit-elle, parce que je sens se
resserrer autour de mes amis Primelles les mailles d'un sinistre
complot\ldots{} Ne me taxez pas d'injustice\,! Ceci n'est pas le fruit
d'une imagination maladive, et je ne me trompe pas, hélas\,! Bientôt,
sans doute, vous appellerai-je à l'aide.

--- Vous me trouverez à vos côtés et aux leurs. Mais chassez ce souci\,!
Peut-on vivre ainsi avec des idées noires, tristes vapeurs qui
obscurcissent votre jeunesse qui ne demande qu'à se dépenser
utilement\,?\ldots{} Croyez-moi, Catherine, redevenez ce que vous fûtes
toujours, avant que ces singuliers soupçons vous enlèvent la paix de
l'âme. N'êtes-vous donc plus l'Atalante, la Diane des anciens jours\,?
Quand vous souhaiterez une bonne journée de chasse, prévenez-moi, et
vous me verrez arriver avec Mauny d'Anrieux\ldots{} À propos, vous ne
savez peut-être pas que Marion a été la reine du bal de Bourges\,? Du
bal de la rue, s'entend\,! Toute la ville s'est émue et l'a voulu
admirer\ldots{} Beaucoup d'honneur en a rejailli sur Mauny, heureux
possesseur de cette beauté du Nord.

Et, heureux de saper son rival en présentant l'éloge de sa gouvernante,
M. de Montenay avait demandé assez timidement à Catherine ce qu'elle
pensait de M\textsuperscript{lle} de Primelles. Car cet amoureux
platonique de la nièce de Mordicourt tenait un peu, quand il
s'aventurait sur le pays de Tendre, la conduite du lièvre, animal que
ronge une crainte qui n'est jamais sans objet.

--- Avez-vous remarqué son air distrait à l'assemblée\,?\ldots{} Quand
je la priai pour danser, elle me refusa sans paraître ni entendre ni
voir. On eût dit qu'elle dormait éveillée.

Catherine jugea inutile de mettre Montenay au courant des causes qui
rendaient M\textsuperscript{lle} de Primelles insensible à ses
hommages\,:

--- Il est bien malheureux, ami Montenay, que Marguerite soit sujette à
des visions où se complaît son humeur bizarre. A vous donner franchement
mon avis, la pauvre fille est plus digne de compassion que de blâme.
Depuis quelques jours, j'observe chez elle des changements qui
m'inquiètent. Mais, quelque chagrin qu'elle éprouve, Marguerite est si
fière qu'elle ne veut jamais rien confier. Laissez au temps le soin de
la rendre plus sage, et comptez sur moi pour l'y aider\,!\ldots{}
Aujourd'hui, je renonce à lui parler raison.

--- Je crains, répondit M. de Montenay, que cette charmante fille ne
soit victime des rêvasseries poétiques auxquelles incitent tant de
méchants auteurs avec leurs bergeries, leurs discussions amoureuses
reposant sur des pointes d'aiguilles et autres billevesées de même
farine. C'est un grand malheur que de semblables livres, tenus pour
moraux par les prudes inoccupées, courent librement le monde et gagnent
la place sur nos vieux conteurs libres, voire licencieux, mais qui
n'empoisonnaient pas les cœurs par un perfide venin enrobé dans le miel
de leurs phrases.

Ainsi causant, M. de Montenay et Catherine traversaient la petite
clairière du bois de Toux, qui est le cœur de la vaste étoile dont
chacun des rais est une allée forestière. Par celle des ventes Michaud,
l'éclaircie se continuait jusqu'à ce chemin auquel les corvées de
Saint-Baudel et de Corquoy ne cessaient de travailler depuis plus de
huit jours. La jeune fille proposa à son tuteur de longer sous bois ce
chemin pour rentrer au château de Bannes par la forêt des Usages. Et
elle envoya son écuyer André d'Archelet, dans la direction contraire,
côtoyer le même chemin jusqu'à Chézal-Benoît, sous le prétexte
d'interroger les gardes des Rauches sur l'état des sangliers. À la
vérité, elle donna à André d'autres instructions. Puis, suivie par les
deux laquais armés dont elle se faisait accompagner toujours depuis
cette battue aux loups où elle avait failli tomber victime des sicaires
de Julie, Catherine rejoignit son tuteur.

C'était par la fin d'une belle après-midi de juillet, lourde et chaude.
Les chevaux que cessaient de tourmenter les mouches depuis que le soleil
descendait sur l'horizon, s'ébrouaient en martelant lourdement le tapis
moussu de l'allée. Des hêtres la bordaient, et aussi des chênes et des
ormes dont le feuillage sombre s'éclairait de place en place par les
frondaisons vert tendre des peupliers blancs. À cette heure du jour,
tout reposait. Les paysans étaient rentrés chez eux pour souper. Seuls
les gens de corvée peinaient à combler les ornières, à aplanir le
terrain du chemin qui aboutissait à la forêt des Usages.

--- Savez-vous. Montenay, fit Catherine, pourquoi l'on travaille si
activement de ce côté\,?\ldots{} Vous l'ignorez, me dites-vous\ldots{}
Qui nous empêche de mettre pied à terre, et, remisant dans les taillis
nos gens et nos chevaux, d'écouter, à l'abri des arbres, ce que se
racontent ces hommes\,?\ldots{} Si nous les interrogeons, nous ne
tirerons rien d'eux, j'en ai peur. Tandis que, s'ils ne nous voient pas,
peut-être pourrons-nous saisir quelques propos utiles\ldots{} Que
voulez-vous\,? Je ne vois partout qu'embûches et conspirations contre
mes amis Primelles. Le départ subit de la mère et du fils, je parle de
Florimond et de Julie la Drapière, ne me plaît qu'à moitié. Et d'abord
Florimond a-t-il vraiment quitté le pays\,? C'est ce dont je ne jurerais
pas\ldots{} Allons, venez\,!\ldots{}

Obéissant comme toujours aux caprices de sa fantasque pupille, M. de
Montenay avançait maintenant à pied, avec elle, dans le lacis de
végétations parasites qui formait un épais rideau à la lisière orientale
du bois de Toux. Se glissant à travers les tiges entre-croisées avec la
souplesse d'une bête des bois, Catherine ouvrait la marche, et son
tuteur la suivait avec assez de peine dans cet enchevêtrement forestier
où s'accrochaient, à chacun de ses pas, son chapeau, ses éperons ou son
épée. Ils atteignirent enfin le pied d'un gros châtaignier dont les
branches, vierges de tout élagage, commençaient à cinq pieds du sol pour
retomber à quinze pas autour et former ainsi une toiture circulaire
composant le plus impénétrable des abris. Quand ils furent entrés dans
cette sombre retraite, ils se trouvèrent surplomber la route à hauteur
d'homme, et ils pouvaient, invisibles, surveiller et écouter les gens
occupés à remuer la terre au-dessous d'eux.

Tous travaillaient sans entrain, et, même, la plupart se reposaient sur
les talus, ou, assis dans le fossé qu'avaient accentué les besoins du
remblai, échangeaient leurs réflexions, --- leurs plaintes surtout. Le
mécontentement de ces rustiques, toujours habiles à éluder les
prestations, se traduisait par des malédictions contre Florimond
d'abord, et ensuite contre le valet de chambre, Clément Malompret. Sans
cesse sur leur dos, cet étranger ne craignait pas de stimuler à coups de
bâton le zèle des moins empressés. Et puis, quel besoin de refaire cette
route dont on ne se servait pas depuis vingt-cinq ans, et qui
aboutissait au bois des Usages, propriété des Primelles, où ceux de
Bannes n'avaient point droit d'accès\,? Seuls les anciens du pays
l'avaient vue en bel état et nette, cette route, à ces époques où les
gens du roi donnaient la chasse aux huguenots et obligeaient les voyers
de Dampierre, de Condé et de Mareuil à entretenir les chemins pour le
passage des troupes et de leurs charrois.

--- Et tout cela, mes enfants, pour que ce petit seigneur de quatre sous
puisse se promener en carrosse de son château de Chézal-Benoît jusqu'à
Primelles, où il n'a que faire\,!

--- C'est la vérité, répondit un second paysan au premier qui exprimait
ses doléances, la pure vérité\,! Pendant que le sieur Florimond, que
j'ai connu haut comme ma bêche et s'appelant Pontaillan, se prélasse à
Chézal-Benoît en compagnie d'une lingère de la ville, nous devons peiner
sous le soleil, au grand dommage de nos récoltes\ldots{}

Catherine avait appuyé la main sur le bras de M. de Montenay\,:

--- Vous avez entendu. Mes pressentiments ne m'avaient pas
trompée\ldots{} Florimond est bien ici\ldots{} Chut\,!\ldots{}
Écoutons\,!\ldots{} Que se passe-t-il\,?

Il se passait une scène des plus ordinaires depuis qu'on avait entrepris
de réparer le chemin. M. Clément Malompret, fondant à l'improviste sur
les bavards, les chargeait à coups de bâton. Et, tel un chef d'armée, il
les objurguait du haut de son cheval\,:

--- Ah\,! ah\,! canailles\,! C'est ainsi que vous obéissez aux ordres de
M. le baron\,! Eh bien, tenez-vous pour assurés que si le travail n'est
pas fini dans deux jours, après-demain, treizième de juillet, et avant
le coucher du soleil, on vous mettra à la raison\,! Cottebleue vous
baillera de nos nouvelles\ldots{}

Un vieux paysan, tenant office de conducteur des travaux, s'interposa\,:
«\,Point n'était besoin de mener tant de bruit ni de bousculer les gens.
Que ce valet prit garde à lui. Le bâton pourrait bien répondre au bâton.
Les choses seraient en état, ainsi qu'on en avait convenu, dans la
matinée du surlendemain. Maintenant, c'était l'heure de rentrer. Que
chacun prit ses outils et s'en fût chez soi\,!\,» Rappelé au sens de la
réalité par ces sages paroles, M. Clément Malompret, piquant des deux,
partit dans la direction de Cheurs.

M\textsuperscript{lle} de Lépinière, une fois de retour à Bannes, n'eut
rien de plus pressé que de mander André d'Archelet. Mais celui-ci ne
rentra que tard dans la nuit, car la lumière de la lune, voilée par les
nuages gros de pluie, lui avait fait défaut dès l'entrée des bois. Il
n'avait pu courir qu'à petite allure, avait failli vingt fois se rompre
le cou, sans compter que son cheval était revenu boiteux. André insista
sur les précautions qu'il avait dû prendre pour dérouter les espions
dont il croyait être poursuivi. Mais il rapportait des renseignements
utiles\,: Florimond était à Chézal-Benoît avec ses équipages et son
monde, et aussi sa mignonne de Bourges. André avait vu Madelon se
promenant dans le jardin entre M. de la Butière et une vieille femme qui
portait un enfant entre ses bras. Il avait même entendu ces gens causer
de leurs projets et de leurs affaires. Le lendemain au soir, tous
devaient retourner à Bourges, où Florimond les rejoindrait sans tarder.
M\textsuperscript{lle} Madelon ne cachait pas son mécontentement. Le
beau carrosse qu'elle avait vu dans la cour du château et dont elle
avait compté se servir pour ce voyage lui avait été refusé. Comme pour
l'aller, le retour se ferait à dos de mule\,; et les deux commères se
demandaient à qui cette voiture était destinée.

Aux explications embarrassées de La Butière, elles répondaient par des
remarques aigres et pointues. Pleine de jalousie et de défiance, Madelon
prétendait que Florimond voulait emmener une femme avec lui dans ce
carrosse\,; et la vieille Macette grommelait de vagues menaces tout en
berçant le marmot, qui criait. Puis Florimond était arrivé, et André
d'Archelet les avait perdus de vue parce qu'ils s'étaient enfoncés sous
une charmille.

Quant à la voiture, il l'avait examinée par-dessus le mur. Elle était
neuve, de cuir noir à ferrures bleuies, massive, solide, à la façon de
celles dont on se sert pour les longs voyages. Briand Perrasset, le
premier cocher de la marquise, se pavanait autour et en expliquait les
beautés aux hommes d'écurie qui l'escortaient\,:

--- Avec un pareil carrosse, mes enfants, on passerait par les plus
mauvais chemins. Voyez comme les roues en sont solides et légères, et
quelles courbes de rais\,! Les bandages des jantes ont là plus de jeu
qu'à l'ordinaire. Cet écartement est prévu pour éviter l'échauffement,
et les têtes des clous sont forgées en table de diamant, ce qui présente
ses avantages\ldots{}

Et, comme on lui demandait s'il savait dans quel but ce remarquable
véhicule avait été amené de la Châtre à Chézal-Benoît, le cocher trancha
de l'important\,:

--- On sait ce que l'on sait\ldots{} Il suffit\ldots{} Cela n'est pas
pour votre museau, mes drôles\,! Ce que logera là dedans M. le baron est
de telle qualité\,!\ldots{} Enfin, je m'entends\,!\ldots{} Allez,
occupez-vous de fourbir les harnais. Ce sont vos affaires, laissez-moi
aux miennes\,!

Et Catherine de Lépinière, réfléchissant sur tout ce que lui rapportait
l'écuyer André d'Archelet, se demandait si Florimond ne méditait pas
quelqu'un de ces rapts audacieux qui assurent à leur auteur une place
honorable parmi les héros de la galanterie. Quelle pouvait être la
victime de ce criminel projet\,?\ldots{} Marguerite de Primelles,
peut-être\,?\ldots{} Mais, dans ce cas, pourquoi Florimond, au lieu de
demeurer au château de Bannes, qui touchait à celui de Primelles,
disposait-il tout à Chézal-Benoît, maison séparée du logis de Marguerite
par six lieues de pays et de mauvais chemins\,?\ldots{} Voulait-il
assurer le secret\,?\ldots{} Oui, sans doute. En passant par la route
qu'il faisait réparer, il amenait son carrosse à deux ou trois cents pas
de la pièce d'eau qui marquait la place du vieux donjon de Primelles.
Maintenant, s'agissait-il d'un enlèvement par la force ou par la ruse\,?
Ou bien Marguerite était-elle de connivence avec Florimond\,?

A cette dernière question que se posait Catherine, la réponse fut
fournie le lendemain, par hasard, ou plutôt grâce à la patiente sagacité
de la jeune fille. Comme elle causait, suivant son habitude, avec
Louis-Antoine, parmi les herbes de la garenne de Tonlieu, elle vit
Françoise Colbert qui marchait avec précaution dans le chemin creux,
théâtre de la déconvenue de M. Clément Malompret. La chambrière de
Marguerite s'arrêta près d'un vieil arbre crevassé, puis, regardant
autour d'elle, s'approcha de ce tronc ruiné, introduisit sa main dans un
trou et en retira un papier plié qu'elle cacha furtivement dans son
corsage. Quand Colbert eut disparu, Catherine, qui n'avait pas cessé de
la suivre des yeux, demanda à Louis-Antoine s'il n'avait pas remarqué la
présence, aux premières heures du matin, de Clément Malompret ou de
quelque autre valet de Bannes dans le chemin.

--- Si fait, répondit le jeune garçon. Clément a encore passé par là il
n'y a pas deux heures. Je ne sais ce qu'il cherchait dans ce gros orme
où, l'année dernière, des frelons avaient installé leur nid, mais il
fouillait dans la crevasse et semblait y chercher quelque chose.
Peut-être y cache-t-il son argent\,?\ldots{} Je me proposais d'aller le
reconnaître lorsque tu es survenue.

--- Tu aurais mal agi, Louis-Antoine. Laissons ce drôle thésauriser à sa
guise, et ne nous abaissons pas jusqu'à surveiller ses gestes\ldots{} Je
te défends de fouiller dans le terreau de cet arbre. Cela ne te regarde
pas. C'est une action en tout indigne d'un gentilhomme que d'épier un
valet\ldots{} Allons, l'heure s'avance, il faut nous séparer. Va-t'en
donc prendre ta leçon d'escrime\,! Demain, nous chasserons à
l'oiseau\ldots{} Aujourd'hui, j'ai affaire\ldots{} Adieu\,!

Elle reconduisit Louis-Antoine jusqu'à la chaumière des bergers, où elle
entra, pendant que son ami regagnait sans hâte le château où
l'attendaient son grand-oncle Le Bouteiller, l'écuyer Robert de Rustigny
et les épées d'exercice.

M\textsuperscript{lle} de Lépinière s'entretint quelque temps avec la
vieille Jeannette Labrande et lui donna rendez-vous pour le lendemain
matin, dans la garenne de Tonlieu\,:

--- Tu viendras avec Symphorien et tes fils, et les garçons bergers. Sur
toutes choses, prends bien garde que l'on ne nous voie pas ensemble. Il
s'agit d'une question de vie ou de mort pour les Primelles\,!\ldots{} Je
ne puis t'en dire davantage pour l'heure\ldots{} Jeannette, je retourne
à Tonlieu. Arrange-toi de manière que Marin m'y rejoigne avant la
nuit\,!

--- Allez en paix\,! Vous serez aussi en sûreté, seule que vous êtes,
mademoiselle, que si vous étiez gardée par tous vos gens\,! Symphorien
veillera sur vous, et Marin viendra. Avant que vous ne soyez rendue à
Tonlieu, ils seront avertis\ldots{} Allez, chère enfant du bon Dieu, ma
future maîtresse, et que les anges du ciel vous conduisent\,!

Catherine passa trois heures, étendue dans les broussailles, au-dessus
du chemin creux, sans rien voir venir. Elle ne se découragea pas et fit
bien, car, vers la cinquième heure du soir, Françoise Colbert reparut.
Avec la même circonspection que la première fois, la fille de chambre
s'approcha de l'arbre. Elle tira une lettre de son corsage, la glissa
dans le creux du bois, et se sauva, légère, dans la direction de
Primelles, sûre de ne pas avoir été observée.

M\textsuperscript{lle} de Lépinière n'hésita pas. Elle descendit à son
tour dans le chemin, et, un instant après, elle tenait dans sa main le
billet écrit à Florimond par Marguerite de Primelles. L'adresse
s'étalait en toutes lettres sur le dos du papier plié, scellé d'un grain
de cire, et Catherine reconnut l'écriture de M\textsuperscript{lle} de
Primelles. La délicatesse, qu'elle avait gravement prêchée à
Louis-Antoine, commandait à Catherine de ne pas violer le secret d'une
correspondance qui ne lui était pas destinée. Mais le sentiment d'un
devoir supérieur lui ordonna de ne pas s'arrêter aux mondaines
convenances. Était-ce bien le moment de s'attarder là-dessus, lorsque
l'honneur d'une famille qu'elle chérissait et la tête plus chère encore
de son Louis-Antoine dépendaient de ce qu'elle allait apprendre et de ce
qu'elle pourrait peut-être empêcher\,?\ldots{}

Catherine se glissa jusqu'à sa coutumière cachette. Elle se déganta,
rompit avec mille soins la cire après l'avoir ramollie sous son doigt,
et lut le billet amoureux de Marguerite de Primelles. Ce n'était pas
l'épître tarabiscotée d'une précieuse, mais une lettre écrite d'un seul
jet, sans art, encore qu'on y retrouvât la fâcheuse tendance de la
liseuse de \emph{l'Astrée} au pathos et au langage des ruelles\,:

«\,Oui, mon cœur, par ses battements réglés sur ceux du vôtre, est le
balancier qui m'avertit de la marche boiteuse des heures et de la fuite
trop lente du temps\,! Aurai-je jamais ce courage d'attendre
jusque-là\,? Oui, demain, treizième jour de juillet, à minuit, sonnera
cette heure de la délivrance où je tomberai dans vos bras. À minuit, à
votre premier appel, je serai à la porte d'eau du sud-est. Colbert, qui
sait ramer, nous mènera du fossé à la rive. Et là je remettrai ma vie
entre vos mains. Que la mort ne me prenne pas avant cette heure que
j'appelle de mes vœux, c'est là tout ce que je demande. Vivez pour moi,
comme ne veut vivre que pour vous celle qui vous aime et qui ne signe
plus Astrée pour son berger Tircis, mais Marguerite de Primelles.\,»

A lire cela, M\textsuperscript{lle} de Lépinière n'éprouva ni douleur ni
surprise. Elle ne songea pas davantage à s'enorgueillir de sa
pénétration. Courageuse et sage, pareille au chirurgien qui va tenter de
sauver un blessé par une opération cruelle, que peut faire manquer la
mollesse de la pitié mal entendue, elle prit son parti sans faiblir\,:
«\,Oui, je la sauverai malgré elle\,!\,»

La cire, soigneusement refoulée avec la pointe d'un petit couteau qui ne
quittait jamais Catherine, reprit sa place. Et, comme aucun cachet n'y
était imprimé, les choses se trouvèrent remises en état. Impossible de
voir que la lettre avait été ouverte. Elle fut cachée dans le creux de
l'arbre. Et, quelques minutes plus tard, un homme tout vêtu de gris,
masqué, les bords du chapeau rabattus sur le nez, arrivait à cheval.
Malgré son déguisement et son masque, Catherine reconnut le valet de
chambre de Florimond. En même temps, le son aigre du cornet de
Symphorien déchira l'air derrière elle. Ainsi le vieux berger
annonçait-il à M\textsuperscript{lle} de Lépinière la présence d'un
danger et la présence aussi de ses dévoués protecteurs.

M. Clément Malompret, sans quitter sa selle, cueillit au fond de son
trou le précieux papier, qu'il enfouit dans sa fonte droite, et repartit
à vive allure, tirant sur Lunery, puisqu'il laissa sur sa gauche le
château de Bannes.

Catherine, toujours immobile dans ses broussailles, entendit alors un
sifflement doux et prolongé. Les branchages s'écartèrent tout auprès
d'elle, et apparut la tête embroussaillée de Marin.

--- M'est avis que le méchant coquin est venu ici pour mal faire. Il est
parvenu jusqu'ici par le chemin neuf et le bois des Usages\,! Faut-il
courir après lui, notre demoiselle\,?

--- Te doutes-tu, Marin, de ce qu'est ce coquin\,?

--- Non, par ma foi\,!

--- Eh bien, c'est Clément Malompret, le valet de M. Florimond qui
maintenant te protège\,!\ldots{} Puisses-tu ne pas apprendre trop tôt le
mal que le seigneur de Bannes veut à tes maîtres et à toi\,!\ldots{}
Mais ce n'est pas le temps de se répandre en vaines paroles. Cours sur
les traces de Clément\,! Et, si tu tiens à la vie, qu'il ne soupçonne
pas qu'on le suive. Ne le quitte pas avant qu'il ne soit de retour à
Chézal-Benoît. A quelque heure de la nuit que tu reviennes, demande-moi
au château. Un homme de garde t'attendra à la porte et te mènera jusqu'à
moi.

Aussitôt rentrée au château de Bannes, Catherine ordonna à André
d'Archelet de se tenir en avant de la porte d'entrée, de guetter Marin,
toute la nuit, s'il le fallait, et de le conduire dans le parc où elle
demeurerait au fond du cabinet de verdure qui s'ouvrait sur la grande
allée, du côté de Léchalusse. Attentive et bienveillante, s'intéressant
toujours aux besoins des petits, qu'elle ne croyait pas formés d'une
autre pâte que les riches de la terre, M\textsuperscript{lle} Lépinière
fit dresser dans le bosquet un repas solide\,; et elle se livra, seule,
à ses réflexions, dans l'ombreuse retraite où elle se plaisait à passer
les heures les plus chaudes du jour.

Ainsi c'était vrai\,! Florimond allait enlever Marguerite de Primelles,
et celle-ci se prêtait au rapt\,! Catherine frémit en songeant au sort
de la malheureuse fille livrée sans défense à cette brute malfaisante
qu'était le fils de Julie. Mais M\textsuperscript{lle} de Lépinière
était trop courageuse pour s'arrêter longtemps sur des regrets et de
stériles récriminations. Elle s'était juré de sauver Louis-Antoine des
sinistres projets de Florimond, de sauver aussi Marguerite. Sans se
tracer une ligne de conduite immuable, elle se décida à s'inspirer des
circonstances. Avant toutes choses, il convenait de ménager l'honneur de
M\textsuperscript{lle} de Primelles. Prévenir sa mère eût été inutile et
dangereux. Catherine connaissait, sans lui avoir pour ainsi dire jamais
parlé, cette femme exaltée et silencieuse. Elle avait deviné que la
veuve du baron de Primelles ne vivait plus, depuis le meurtre de son
mari, que pour la vengeance. Cette mère, déchirée à la fois par sa
passion vindicative et par ses remords religieux, incapable d'obéir à
ceux-ci quand ils lui ordonnaient de répudier celle-là, ne voyait dans
Louis-Antoine que le vengeur de son père assassiné.

Et c'est pourquoi M\textsuperscript{lle} de Primelles ne s'occupait que
de pousser son jeune fils vers les exercices de l'escrime, comptant,
dans son aveugle désespoir, sur la seule épée pour détruire l'héritier
du marquis de Bannes. Si Catherine dénonçait à cette veuve implacable
l'infamie de sa fille, la baronne ne verrait dans Marguerite que
l'abominable complice de Florimond, c'est-à-dire de cet homme qui était
le fils du meurtrier de son père. M\textsuperscript{me} de Primelles,
dans sa colère sans frein, irait au-devant du scandale. Peut-être
chasserait-elle Marguerite, l'enverrait-elle dans quelque couvent\,; et
le monde n'ignorerait rien d'une histoire dont la chronique amusante
s'emparerait. Le beau rôle serait pour Florimond. Et encore
Louis-Antoine serait victime dans cette affaire. La baronne le lancerait
contre Florimond.

Du combat le dénouement n'apparaissait pas douteux à Catherine. Plus âgé
de quatre ans au moins que Louis-Antoine, ayant l'âge d'homme alors que
celui-ci n'était encore qu'un enfant, Florimond, entraîné à la pratique
de l'épée, riche de l'expérience acquise dans dix duels, assassinerait
le malheureux garçon, en satisfaisant aux lois de l'honneur.

Ainsi, c'en serait fait de tous les espoirs de Catherine, qui aimait ce
petit Louis-Antoine de toute l'affection d'une mère et de la tendresse
délicate d'une épouse. Depuis plus de deux années, elle étudiait ce
caractère timide et sauvage, s'appliquait à redresser cet arbrisseau de
plein vent. Elle avait fait de Louis-Antoine l'unique objet de ses
soins, et il faudrait qu'une bête farouche, riche de ses seuls mauvais
instincts, que le bâtard d'une boutiquière anobli par l'incurable
faiblesse d'un gentilhomme dévoyé, fauchât dans sa fleur cette plante
qu'elle, Catherine, s'attachait à défendre\ldots{} Elle la défendrait au
prix de sa vie\,!

Et aussi Marguerite de Primelles\,!\ldots{} Non, mille fois non\,!
Jamais M\textsuperscript{lle} de Lépinière ne confierait à la baronne de
Primelles les audacieux projets de Florimond et la coupable connivence
de Marguerite\,!\ldots{} Seul M. Le Bouteiller lui parut capable de
recevoir une aussi lourde confidence et de l'aider à sauver
Marguerite\ldots{} et à punir Florimond.

Car Catherine entendait que Florimond trouvât, pris dans la pleine
exécution de son crime, un châtiment assez terrible pour qu'on fût à
tout jamais débarrassé et de lui et de Julie\ldots{} Le marquis
apprendrait, dans son exil, d'un même temps, et le forfait et la juste
peine. Au risque de se brouiller avec son beau-père, Catherine se
chargeait de cela. D'ailleurs, M. de Montenay, son tuteur désigné,
saurait la protéger contre la colère du marquis\ldots{} Maintenant,
était-il prudent d'avertir M. de Montenay de ce que Florimond tramait
contre Marguerite\,? N'était-il pas à craindre que, dans son aveugle
colère, il ne provoquât sur l'heure Florimond, qui, montrant les lettres
de Marguerite, prouverait le déshonneur de celle-ci avant que de mettre
l'épée à la main\,?

Voilà ce que la jeune fille se demandait avec anxiété lorsque Marin
parut.

Faible de fatigue et de besoin, l'alerte braconnier n'avançait à cette
heure qu'avec peine\,; et il traînait la jambe. André d'Archelet le
précédait, porteur d'une lanterne, et Marin avait certainement peine à
le suivre.

--- André, pose ici ta lumière et laisse-nous\,! Mène bonne garde à
l'entrée de l'allée, et que personne ne nous dérange\,! Quelle heure
est-il\,?\ldots{} J'ai négligé de remonter ma montre.

--- Près de minuit, mademoiselle\ldots{} Quand vous aurez besoin de moi,
il vous suffira de siffler.

Catherine avait fait asseoir Marin sur un banc de pierre. Poussant
jusqu'à lui la table, elle le servit de ses mains\,:

--- Mange et bois, mon pauvre garçon\,!\ldots{} Va, tu me remercieras
plus tard\,!\ldots{} Et tu parleras après\ldots{} Quels chemins as-tu
donc choisis\,? Tes pauvres habits bâillent, déchirés en pièces. Et
as-tu marché à quatre pattes, que tes mains ne sont qu'une plaie\,?

Quels chemins il avait suivis, lui seul le savait\,! Depuis plus de six
heures, il n'avait pas cessé de courir, de sauter, de ramper. Attaché
aux pas de Clément, tel un limier sur les traces d'un cerf, les oiseaux
du ciel auraient seuls pu le voir se hâtant le long des vignes, des
emblavures, ou parmi les ronces qui hérissent les coteaux de la Rille au
Breuil de Lunery. Pendant une lieue de pays, Marin déboula comme une
bête noire, tout en gardant le contact avec Clément et en le surveillant
du haut des plis de terrain, mais toujours caché, sûr d'échapper aux
regards.

--- Oui da, ma belle demoiselle, je n'ai pas plaint ma peine, mais j'ai
été payé de retour. J'en sais tant et plus, j'en sais trop\ldots{} Il me
semble que ma tête va éclater\ldots{} Vive Dieu, je veux que ce méchant
bâtard, Pontaillan ou Florimond, ainsi qu'il vous plaira de l'appeler,
paye demain toutes ses méchantes actions et aussi celles de sa bonne
mère la Drapière\,!\ldots{} Et quand je pense que je me laissai prendre
à leurs paroles d'or et de sucre\,!\ldots{} Ah\,! mademoiselle, si vous
saviez, si vous saviez\,!

Patiemment, Catherine lui passait les meilleurs morceaux, lui versait à
boire\,:

--- Ne te presse pas, mon bon Marin\,!\ldots{} A parler ainsi la bouche
pleine tu vas t'étouffer, certainement\ldots{} Mange, bois\,!\ldots{}
Chaque chose en son temps\,!\ldots{} Tu me raconteras après ça toutes
tes histoires. Et je crois comprendre qu'elles sont d'importance.

--- Vous pouvez le dire, chère et bonne demoiselle Catherine\,!\ldots{}
Là, c'est fini\ldots{} Permettez-moi de vous rendre grâces\,!

Marin but encore un grand coup de vin trempé d'eau, s'étira, soupira, et
commença son récit\,:

--- Pour lors, je suivais donc notre vilain valet, qui n'allait pas trop
vite, grâce au mauvais état des chemins, lorsque, près du Breuil de
Lunery, voilà qu'un cavalier se place en travers du sentier. Si Clément
n'eût pas arrêté justement son cheval d'une saccade à la bouche qui
obligea la bête à faire un pont-levis de première qualité, il culbutait
le nouveau venu. Et c'eût été grand dommage, car ce nouveau venu était
M. Florimond, comme je vous le dis, mademoiselle\ldots{} Je sus choisir
mon moment, me hisser dans le gros peuplier de droite, à toucher
l'écluse. Et je voyais les deux mauvais chrétiens ainsi que je vous
vois, et je les entendais encore mieux.

«\,As-tu la lettre\,? demanda Florimond. --- Bien sûr que je l'ai\,!
répondit Clément. --- Alors, donne\,!\,»

«\,Florimond prit un papier que lui tendait son valet, lut et s'écria\,:
«\,Victoire\,!\ldots{} La petite est à nous\,!\ldots{} Oh\,! là,
Tourouvre\,!\,»

«\,Le Tourouvre entra en scène, avec son grand chapeau et son plumet\,:
«\,Écoute, Tourouvre, que dit Florimond, l'affaire est dans le sac\,! La
jeune Marguerite nous attend demain à minuit\,! C'est vraiment trop
facile, et je suis honteux de réussir à si peu de frais\,! --- Avec
vous, monsieur, n'en est-il pas toujours de même\,?\,» que fait l'autre.
Et Florimond de répondre en se rengorgeant\,: «\,Ce n'est pas tout ça\,!
Il nous faut maintenant régler le compte de Marin.\,»

«\,Vous pouvez penser si je tendais l'oreille dans mon arbre. Que me
voulaient-ils donc, à moi, ces trois méchants drôles\,?\ldots{} Je ne
l'appris que trop tôt\,! «\,Cela, répliqua Tourouvre, concerne
Cottebleue. Il fait le pied de grue là, derrière le moulin. Si vous
voulez, je vais l'appeler.\,» Et voilà Cottebleue qui s'avance, avec son
brassard et son bâton, accompagné par La Butière, plus sec qu'un hareng
sauret, et tout de noir vêtu.

«\,Alors, mademoiselle, les cinq compères tinrent une conversation que
je ne saurais vous répéter sans rougir\ldots{} Souffrez que je la résume
honnêtement. Pendant que Florimond, escorté par La Butière, Tourouvre,
Clément et une douzaine d'estafiers enlèveraient M\textsuperscript{lle}
Marguerite, qui doit se remettre à eux avec sa chambrière Françoise
Colbert, qui est de mèche, Cottebleue, assisté d'une vingtaine de bons
compagnons, travaillerait de son côté. Oui, Cottebleue irait me cueillir
dans la maison de mon père. Ils y mèneraient assez de bruit pour attirer
à la bergerie tous les gens de Primelles. D'autant qu'à la faveur du
désordre ils bouteraient le feu à notre demeure. Tout le monde courrait
à l'incendie, et Florimond, cependant, entraînerait les deux femmes, à
travers le bois des Usages, jusqu'au carrosse qui stationnerait sur la
nouvelle route et recevrait les fugitives, Florimond et ses complices.
Et fouette cocher\,!\ldots{} Quant à moi, l'on m'aurait vivement bouclé,
transporté pieds et poings liés à Lunery, où l'on m'accuserait non
seulement d'avoir allumé le feu, mais encore d'avoir aidé au rapt de
M\textsuperscript{lle} Marguerite par des inconnus, de manière à bien
établir le scandale, à le répandre et à donner de quoi rire au monde
jusqu'à plus ample informé. Enfin, pour couronner ces exploits,
Florimond recommanda à Tourouvre et à La Butière de guetter
particulièrement M. Louis-Antoine et de lui donner un mauvais coup,
comme par hasard, tout en criant que c'était moi qui avais frappé le
jeune monsieur. Je vous fais grâce des détails. Je ne crois pas que les
pires garçons parmi les soldats ou les valets d'armée aient jamais conçu
pareil projet ni exprimé de désirs plus déshonnêtes\ldots{} Suffit\,!

«\,Voilà tout ce que je sais. Après un quart d'heure de conversation,
nos cinq amis se séparèrent. Cottebleue s'en fut du côté de Lunery,
Florimond reprit le chemin de Chézal-Benoît avec La Butière et
Tourouvre. Pour Clément, il s'en alla étudier les abords de Primelles,
où sans doute il avait donné rendez-vous à Françoise Colbert. Moi, je
coupai au plus court et arrivai à Chézal-Benoît un bon quart d'heure
avant M. Florimond et ses compères. Grimpé sur le mur du jardin,
j'assistai à une autre comédie.

Et Marin raconta au prix de quelles difficultés Florimond avait renvoyé
à Bourges sa maîtresse Madelon, qui ne voulait point partir. Enfin la
lingère et sa gouvernante Macette étaient montées sur des mules où leur
faisait contrepoids un panier. L'enfant avait été couché dans l'un, sur
des hardes. Le convoi s'était éloigné sous l'escorte de La Butière et de
quatre valets armés\ldots{}

À cet endroit de son récit, Marin, autant recru de fatigue qu'alourdi
par son souper, s'endormit. Catherine respecta son sommeil. Elle ne
voulait pas renvoyer le fils de Symphorien avant d'en avoir tiré
d'autres renseignements, si possible.

Au bout d'une heure, Marin se réveilla en sursaut, criant\,: «\,Au
voleur\,! Au meurtre\,!\ldots{} Gredins, vous n'échapperez
pas\,!\ldots{} A mort, à mort\,!\,» Il rêvait que Cottebleue
l'assaillait et le chargeait de liens. Son premier mouvement fut de se
frotter les yeux, puis d'implorer le pardon de M\textsuperscript{lle} de
Lépinière et de se désespérer d'un pareil manque de respect. Mais
Catherine l'interrompit avec bonté\,:

--- Mon pauvre garçon, c'est grande cruauté de ma part de te retenir
ainsi quand tu es brisé par tant de courses dans les halliers\,!\ldots{}
Ne t'excuse donc pas, et t'en va coucher, car tu as bien mérité ton
repos\ldots{} Mais n'as-tu rien de plus particulier à me dire\,? N'as-tu
rien retenu des intentions de Florimond à l'égard de
M\textsuperscript{lle} de Primelles\,?

--- Ma foi, chère et bonne demoiselle, il ne m'en souvient pas\ldots{}
Hem\,!\ldots{} Hem\,!\ldots{} Si, toutefois\,!\ldots{} Ne vous ai-je pas
touché un mot de l'ancien bedeau de Lunery\,?

--- Non, Marin. Quelle est cette nouvelle histoire\,?

--- Ah\,! mademoiselle, une abomination\,! Écoutez plutôt comment
Florimond prétend arranger ses affaires et celles de
M\textsuperscript{lle} Marguerite. Vous avez ouï parler de Nicolas
Craquelin, ce bedeau, ou pour mieux dire ce sacristain de Lunery qui fut
chassé par l'archiprêtre pour un vol de chandeliers d'argent\,? Ce
méchant homme vit en ermite du côté de Castelnau d'Entrevins, à l'orée
du bois de la Nisse, où il vend des charmes, des médailles et des
reliques. Il bénit les unions à la belle étoile et est réputé pour les
soins qu'il sait donner au bétail. Eh bien, Florimond a contracté marché
avec ce mendiant hypocrite qui doit, la nuit prochaine, aussitôt
M\textsuperscript{lle} Marguerite arrivée à Chézal-Benoît, l'unir par
sacrement de mariage à M. Florimond\,! Vous voyez d'ici quel mariage et
quelles en peuvent être les suites. Évitez-moi de narrer à une pure
demoiselle telle que vous ce que ce vilain monde manigance\,!\ldots{}
Songez que le sieur Tourouvre escompte déjà la succession de Florimond
et se flatte de consoler la fausse mariée aux vendanges\,!\ldots{}

--- Écoute, Marin, dit Catherine, le temps est venu de nous séparer.
Retourne chez toi, et jure-moi que tout ceci demeurera secret entre
nous. Demain, avant midi, je te manderai par Robert de Rustigny ce que
j'attends de toi et des tiens. Annonce à ta mère que, par prudence, je
m'abstiendrai de la voir ce matin ainsi que je le lui avais promis. Que
personne sur la terre ne se doute de notre entrevue\,! Adieu\,! Demain
soir tu seras entouré d'amis, et, à moins que je ne périsse moi-même,
nul à Primelles n'encourra ni peine ni dommage. Adieu\,!\,»

Et, comme Marin lui baisait respectueusement les mains, elle le
souffleta joyeusement, appela André d'Archelet et remonta dans sa
chambre. Le sommeil fut long à visiter la courageuse fille, qui, décidée
à ne prendre conseil que d'elle-même, s'endormit seulement aux premières
heures du jour. Mais son parti était arrêté.

Aussitôt levée, elle s'enferma avec Maroie Lenatier. En cette brune
fille d'atours, d'une admirable beauté, dont la froideur se tempérait
par des yeux de l'éclat le plus vif et le plus doux, Catherine savait
qu'elle trouverait la plus dévouée et la plus avisée des alliées.
Maroie, élevée par la marquise Julie dans l'espoir que la chambrière se
montrerait complaisante pour Florimond, avait résolu ce problème
d'échapper aux brutales entreprises du jeune homme sans perdre la faveur
de sa maîtresse. Au vrai, cette faveur n'était qu'apparente. Julie
détestait Maroie autant que celle-ci haïssait et méprisait Florimond.
Mais elle redoutait la servante, qui possédait trop de secrets pour
qu'on ne fût pas obligé à la ménager. Si la marquise n'avait pas emmené
Maroie à Paris, c'est qu'elle craignait que celle-ci ne surprit des
conversations dangereuses et n'écrivit à Bourges ou à Bannes. L'amour
que ressentait Maroie pour l'écuyer de M\textsuperscript{lle} de
Primelles, Robert de Rustigny, n'était pas ignoré de Julie. Il ne
fallait donc pas que Maroie pût avertir l'écuyer de ce qui se tramait
contre les Primelles.

La marquise, en laissant Maroie seule à Bannes, commit une lourde faute,
car la haine commune contre Florimond unit la fille d'atours à
M\textsuperscript{lle} de Lépinière, au moins autant que les espoirs
dont Catherine sut susciter le mirage aux yeux de la belle chambrière.

--- Maroie, lui dit-elle, si tu veux m'assister dans l'affaire que je
vais t'exposer, je jure un grand serment que je rendrai possible ton
mariage avec Robert de Rustigny. Tu me connais respectueuse de ma
parole. Je n'y ai jamais manqué. Tu peux donc compter sur moi, aussi
veux-je compter sur toi. Tu es fille d'honneur. Fais-moi le serment de
garder le secret que je te vais confier.

La fille d'atours ne demandait pas mieux que de marcher dans les voies
de Catherine, envers qui elle se sentait coupable. N'était-ce point par
sa lâcheté que M\textsuperscript{lle} de Lépinière était demeurée
ignorante de ce que la marquise avait tramé contre elle avec Aimeri
d'Olivier, La Butière et Tourouvre\,? Elle n'avait pas osé divulguer à
Catherine ce qu'elle avait entendu du petit cabinet où elle couchait\,;
et, quand elle apprit par la suite les bruits vagues qui couraient d'un
attentat contre la belle-fille du marquis, Maroie avait pleuré sur sa
pusillanimité.

Sa nature ardente, en tous temps sévèrement contenue, la poussa aux
confidences que sa tendresse, une fois lâchée, rendit excessives. Et,
tout en s'engageant à ne pas trahir le secret de Catherine, elle se jeta
à ses pieds et lui avoua sa faiblesse, lui raconta tout ce qu'elle
savait et la supplia de lui pardonner. M\textsuperscript{lle} de
Lépinière comprit que cette belle jeune fille, aussi sage que modeste,
était à sa complète dévotion. Elle lui confia donc l'histoire des
projets de Florimond et la chargea de transmettre à Robert de Rustigny
les instructions les plus minutieuses pour la sûreté de Marin et des
siens, pour d'autres choses encore. Et Maroie écrivait de peur d'oublier
les détails. Puis la fille d'atours partit, prenant son chemin par le
parc, de manière à se poster à la brèche du mur regardant Léchalusse.
Par là Robert passait chaque matin, sous couleur de promener son cheval,
mais, en vérité, pour échanger quelques mots avec sa très belle amie.

La journée tout entière se passa pour Catherine dans l'isolement.
Verrouillée dans sa chambre où la retenait une indisposition cruelle et
contre quoi était impuissante la science toujours sûre d'elle-même des
médecins, elle envoya ses ordres dans tout le château, avec défense de
la déranger sous n'importe quel prétexte. Elle ne voulait voir âme qui
vive. La seule Gillette Léchanson demeura près d'elle, assise dans
l'antichambre, dont la porte demeura impitoyablement poussée. Quant à la
chambre de Catherine, elle était depuis longtemps vide, car, dès la
première heure après midi, la belle-fille du marquis de Bannes avait
quitté le château sous des habits de servante et enveloppée dans une
longue mante à capuchon.

Par Tonlieu, elle gagna le château de Primelles, où elle entra en
traversant le potager, un panier d'herbes à la main. Louis-Antoine, qui
passait alors avec ses lignes sur l'épaule, ne la reconnut pas. Entre
les carrés d'oignons, M. Le Bouteiller se promenait, solitaire, le nez
chaussé de besicles et penché sur un livre d'agriculture qu'il lisait
attentivement. Cette lecture l'attachait assez pour qu'il méprisât le
soleil obstiné à l'éblouir de ses rayons.

Quand Catherine s'arrêta devant lui, le bonhomme esquissa un pas de côté
pour l'éviter. Puis, étonné de son insistance à lui présenter un panier,
il lui ordonna d'un ton bourru de débarrasser le chemin\,: «\,Que venait
chercher ici cette vagabonde avec ses herbes\,?\,»

Alors, rabattant son capuchon, Catherine de Lépinière répondit au baron
de Mordicourt stupéfait\,:

--- C'est vous que je cherche, monsieur. Il s'agit de choses
d'importance et qui doivent demeurer cachées. Et voilà pourquoi je me
voile la tête sous ce capuchon et le corps sous ces méchants vêtements.

Ainsi parlant, Catherine recouvrit son visage. L'oncle Le Bouteiller,
ayant débarrassé son nez de ses lunettes, salua poliment et demanda à la
jeune fille si elle ne voulait pas le suivre dans une chambre du
château. Tout prêt à l'entendre, il voulait seulement que l'entretien ne
se donnât pas en plein vent, de peur des indiscrets.

--- Monsieur, répondit Catherine, ce que j'ai à vous apprendre est
tellement grave, et les dangers qui nous entourent sont tels que
personne ici ne doit soupçonner ma présence. Si cela vous agrée,
traitez-moi ainsi qu'une fille de la campagne qui vous apporterait un
message. Rudoyez-moi selon votre ordinaire, conduisez-moi dans quelque
resserre isolée, et là je vous parlerai.

--- Je suis à vos ordres, mademoiselle. Une personne de votre mérite et
de votre condition a droit à tous les égards. Je vais y manquer,
uniquement pour vous obéir. Par avance, je vous prie de me pardonner.

Brusquement, il haussa le ton, et, tirant légèrement Catherine par un
bras comme s'il la secouait avec force, il s'écria\,:

--- Peste de toi, pécore\,!\ldots{} Et tu ne pouvais pas me dire,
maîtresse sotte, que dans tes herbes est une lettre de mon vieil ami
Lantaume de Lavaufranche\,?\ldots{} Viens çà, que je te baille la
réponse, et, une autre fois, montre-toi moins avare de paroles, pimbêche
plus digne de pâturer avec les oies que de les conduire aux prés\,!

Et, poussant devant lui Catherine qui soupirait, piaillait, imitant en
tout les allures d'une volaille apeurée, il la dirigea vers le pont
volant du potager, entra avec elle dans la loge à demi ruinée du château
de la porte, puis ouvrit une petite chambre où, sur deux tables
boiteuses, voisinaient des graines, des plantes sèches et autres
curiosités de jardins.

--- Personne ici ne nous dérangera. Tenez, mon enfant, voici une chaise,
rustique certes, mais qui a ses quatre pieds. C'est une qualité rare
dans le mobilier du lieu.

Il poussa la porte, assura la barre, s'assit sur une table, sans souci
des sachets de semences, et dit gravement\,:

--- Maintenant, mademoiselle Catherine de Lépinière, me ferez-vous cet
honneur de me dire ce que signifient vos mystérieuses façons\,?

--- Monsieur de Mordicourt, ce n'est pas sans raisons qu'une fille de
mon âge se permet de déranger un gentilhomme du vôtre\ldots{} Je
m'expliquerai donc\ldots{} Mais que me répondriez-vous, s'il vous plaît,
si je requérais votre parole de gentilhomme de m'accorder la faveur que
je viens implorer de vous\,?

Ainsi pris de court, séduit par les grâces ingénues et fières de cette
jeune fille qui le regardait avec ses yeux francs et honnêtes, trop
homme du siècle dernier pour dépouiller une galanterie chevaleresque
passée à l'état d'habitude, le baron repartit\,:

--- Mon Dieu, mademoiselle, à toute autre que vous j'opposerais des
réserves. Ce que je connais de votre courage et de votre droiture
m'ordonne d'y renoncer. Comme vous êtes incapable de rien demander qui
soit contraire à l'honneur\ldots{} je vous donne ma parole et je vous
écoute.

Le combat fut rude, dans le cœur du vieux baron, entre la voix de
l'honneur familial et celle de l'honneur personnel, quand il eut entendu
le récit de Catherine. S'ingéniant à excuser Marguerite, son amie ne se
crut pas obligée de raconter dans ses détails l'acquiescement de
M\textsuperscript{lle} de Primelles à l'entreprise de Florimond. Par
contre, elle ne chercha pas à pallier la noirceur des projets de
celui-ci. Et, très habilement, elle réussit à détourner la colère de M.
Le Bouteiller sur le traître impudent et pervers qui s'était joué
indignement de sa bonne foi et de sa générosité. Enfin, elle somma le
vieux gentilhomme d'exécuter son engagement\,:

--- Vous ne toucherez mot de cette déplorable affaire ni à Marguerite,
ni à sa mère, ni à Louis-Antoine\,!\ldots{} A personne\,! Vous êtes lié,
monsieur, demeurez-le pour l'amour des vôtres, sinon pour l'amour de
moi\,!

Secouant tristement la tête, M. Le Bouteiller avoua que sa bonne foi
avait été surprise\,: «\,C'était bien, il ne dirait rien\,! Fallait-il
pourtant qu'un pareil forfait ne restât point impuni\,?

--- Vous vous tromperiez du tout au tout, monsieur, si vous pouvez
croire que la juste vengeance n'atteindra pas le coupable. Pour moi, je
désire que Florimond soit châtié avec la plus grande sévérité. Quelques
mesures que vous preniez contre lui, je les approuverai avec toute
l'amitié et le respect que je ressens pour vous\,!

Longtemps ils s'entretinrent dans le réduit aux graines. Quand ils se
quittèrent, ils avaient arrêté leur plan. M\textsuperscript{lle} de
Lépinière, avec son capuchon rabattu, fut conduite par Margot
Larçonnière chez Marguerite de Primelles. À Françoise Colbert, qui
prétendait lui interdire l'entrée de la chambre, Catherine chuchota à
l'oreille\,:

--- De la part du baron de Chézal-Benoît.

Colbert redit ces mots à mi-voix. Aussitôt. M\textsuperscript{lle} de
Primelles, qui se tenait aux aguets contre l'huis entrebâillé,
apparut\,:

--- As-tu une lettre de lui\,?\ldots{} Ah\,! donne, donne-la sans
tarder\,!

Catherine laissa les deux servantes sur le palier de l'escalier et
entra. Elle poussa la porte, donna deux tours de clef, et, découvrant
son visage, dit à Marguerite qui la regardait effarée, béante, surprise
au milieu de ses préparatifs de départ\,:

--- Tu ne m'attendais pas, malheureuse\,!\ldots{} Et cependant, dans ton
égarement, m'as-tu donc oubliée, moi ton amie, moi qui me suis résolue à
te sauver\ldots{} malgré toi, s'il le faut\,?

Marguerite avait repris son sang-froid. D'un ton glacial, elle répondit
à Catherine\,:

--- Que voulez-vous dire\,? Je ne vous comprends pas\ldots{}

--- Ah\,! tu ne me comprends que trop\,!\ldots{} Espères-tu que je sois
devenue aveugle au point de rester ta dupe\,? Va, pauvre folle, je
n'ignore rien de ce que tu trames avec Florimond\,!

--- C'est vous qui êtes folle, Catherine, et cela ne vous change
pas\,!\ldots{} Laissez là ces visions, et expliquez-moi ce qui vous
amène\ldots{} Voyez, je suis souffrante, et je me préparais à me mettre
au lit lorsque vous êtes arrivée\,!\ldots{} Si vous avez quelque chose à
me demander, parlez\,!\ldots{} Sinon, laissez-moi en repos\,!

Marchant sur elle, Catherine lui saisit les deux poignets, et dardant
ses regards au fond des yeux bleus qui se détournaient en vain\,:

--- Malheureuse, malheureuse Marguerite\,!\ldots{} Je viens te sauver.
Devrais-je te le répéter dix fois, vingt fois, cent fois, je te
sauverai\,! Oui, je te sauverai moins de Florimond que de
toi-même\,!\ldots{} Si, sourde à la voix de ce qu'une fille a de plus
précieux, de l'honneur, si, sourde à la voix de ton père assassiné, tu
veux te jeter dans les bras de ce séducteur exécrable qui se rit de toi
et médite déjà de te livrer à ses compagnons de débauche et de
crime\ldots{}

Marguerite l'interrompit d'une voix que la colère rendait tremblante et
profonde\,:

--- Tais-toi\,!\ldots{} Tais-toi, méchante et fausse amie, vilaine
fille\,!\ldots{} Oui, je devine tes artifices\,!\ldots{} Jalouse de mon
bonheur, tu t'es imaginée que je te laisserais le détruire\,!\ldots{} De
quel droit viens-tu te jeter entre moi et celui que j'aime\,?\ldots{}
Crois-tu donc que tes tristes calomnies auront prise sur un cœur qui ne
bat que pour lui\,! Va-t'en\,! Fuis-t'en\,! Laisse-moi\,!\ldots{} Ah\,!
orgueilleuse et fausse femelle, gonflée de tes richesses, tu veux me
prendre mon Florimond, et tu me le peins sous des couleurs si noires
dans l'espoir que je le tiendrai pour le scélérat qu'il te plaît de me
présenter\,!\ldots{} Bannis ces soucis, ma belle\,!\ldots{} Veux-tu que
je te dise la vérité\,? Eh bien, tu es amoureuse de Florimond, je le
sais, et tu enrages\,! Tu veux me le disputer\ldots{} Quelle
sottise\,!\ldots{} Avec tes habits d'homme et tes allures cavalières,
dévergondée, maniaque\,!\ldots{} C'est moi, moi, entends-tu, qu'il
chérit\,! Moi seule qu'il aime\,!\ldots{} Tandis que toi\,!\ldots{}

Catherine, sans se laisser troubler par cette attaque insensée, saisit à
nouveau Marguerite, qui s'était échappée, et lui dit sans impatience\,:

--- Louis-Antoine est celui que j'aime. N'as-tu point pensé à
lui\,?\ldots{} Oui, as-tu pensé à ton frère\,?\ldots{} À Louis-Antoine,
que Florimond a juré de tuer\,? Ne te suffit-il pas d'avoir perdu ton
père par l'épée du marquis de Bannes, et souhaites-tu que le fils du
marquis assassine aussi ton frère\,?\ldots{} Allons, parle\,!

Alors Marguerite de Primelles parla. Toutes ses colères éclatèrent.
Toutes ses rancunes, tous ses désirs, vagues, obscurs et violents
l'emportèrent. Et ses paroles roulaient, tel un torrent furieux qui
charrie les rochers, les troncs d'arbres rompus et les cadavres
démembrés des hommes et des bêtes, dans un chaos tourbillonnant et
bourbeux.

Repoussant M\textsuperscript{lle} de Lépinière avec une force dont on
l'eût crue incapable, elle s'adossa au mur, et, la figure convulsée par
la haine, les sourcils froncés, les yeux étincelants, elle cria\,:

--- Mon père, mon frère\,!\ldots{} Que m'importe\,?\ldots{} Je les hais,
entends-tu\,?\ldots{} Je les hais, et tous ceux qui de près ou de loin
prétendent se dresser entre moi et ce Florimond que j'aime, au nom de je
ne sais quels principes de morale bons tout au plus pour le commun des
humains\,!\ldots{} Je méprise la foule\,! Périssent famille,
amis\,!\ldots{} Que ce toit exécré s'abatte sur ma tête, mais que je
tombe entre les bras de l'aimé\,!\ldots{} Entends-tu, Catherine, fille
forte et sage au regard du monde, mais plus hypocrite et menteuse qu'une
sibylle de carrefour\,! Je me moque de tous et de toi, de tes
objurgations, de tes insinuations, de tes reproches\,!\ldots{} Tout ce
que tu racontes est faux, et tu plaides ce faux pour savoir le
vrai\,!\ldots{} Pauvre sotte\,!\ldots{} Oui, je suivrai Florimond,
quelque jour\ldots{} Quand il m'appellera\ldots{} Mais les temps ne sont
pas venus\ldots{} Oui, je le suivrai, quand je devrais, alors, pour le
rejoindre, sauter en chemise dans le fossé et nager parmi les crapauds,
les herbes pourries et les bêtes gluantes\,!\ldots{} Honneur,
vertu\,!\ldots{} Des mots, des mots, te dis-je\,!\ldots{} Je veux vivre
pour moi, vivre pour lui\,! Foin du reste\,! Que le monde entier
s'abîme, pourvu qu'il en reste un coin où je puisse me retirer avec
Florimond et couler mes jours près de lui\,!

Sans perdre courage, Catherine s'attacha à cette désespérée qui ne
voulait rien entendre. A demi dévêtue, courant parmi ses hardes qu'elle
réunissait en un paquet à ce moment même où M\textsuperscript{lle} de
Lépinière était entrée, Marguerite passait par toutes les transes de la
terreur, les transports du désespoir, les abandons du découragement. Ce
qui la terrassait, c'était surtout le doute. Elle voyait bien que sa
fuite serait contrariée, mais elle ne la croyait pas impossible.

Rappelant à soi toute son énergie, M\textsuperscript{lle} de Primelles
s'appliqua à ruser. Elle regrettait d'ailleurs son explosion de franche
colère. Comme elle ne pouvait reprendre ses paroles, elle louvoya et
cessa de tenir tête à Catherine. Elle se mit au lit, ainsi qu'elle en
avait exprimé l'intention, promit à son amie de lui obéir en tout.

«\,Pourvu qu'elle s'en aille, songeait-elle, il me restera bien une
chance. Florimond est tant habile et Colbert tant dévouée qu'ils sauront
certainement trouver quelque moyen de me tirer de là. Si j'ai bien
compris Catherine, il ressort de tous ses sermons que ma famille n'est
pas prévenue. Voilà le principal. C'est sous son bonnet que cette
moderne amazone a pris de me prêcher ainsi\ldots{} Vienne le signal, et
nous nous enfuirons vivement, Colbert et moi, sans crier gare\,!\ldots{}
Ah\,! enfin, voici Catherine qui sort de la chambre\,; feignons de
dormir à poings fermés\ldots{} Quand le diable y serait, je pourrai
toujours descendre par la fenêtre, en nouant ensemble mes draps\,;
Colbert m'a parlé de ce moyen\,: je Le crois bon\,!\ldots{} D'ici là je
me tiens coite, car ma Catherine est bien capable de m'espionner par
quelque trou\ldots\,»

Avec une exactitude qu'il n'avait jamais apportée dans l'accomplissement
d'un devoir, Florimond se trouva, au premier coup de minuit qui
s'entendait du château de Bannes jusqu'à celui de Primelles, à cet
endroit du fossé situé presque en face de la porte d'eau qui s'ouvrait
vers la chambre de M\textsuperscript{lle} Marguerite. Il siffla, appela
doucement, en progressant avec précautions de manière que les murailles
de la basse-cour et du chenil, qui s'avançaient dans la douve en façon
de presqu'île, le cachassent aux yeux des habitants du château. Un
sifflement légèrement modulé lui répondit par deux fois, et aussi une
voix dont il ne distingua pas les mots qu'elle prononçait, tant elle
était étouffée. Une barque, sortant de l'ombre, commença de glisser sur
l'eau, conduite avec un tel art qu'on ne percevait même pas le
clapotement des avirons.

Éclairé par la lune, l'esquif traversa la pièce d'eau marquant la place
de l'ancien donjon et toucha le bord opposé. Deux femmes prirent terre,
masquées, enveloppées dans des manteaux sombres. Florimond, triomphant,
saisit Marguerite de Primelles, qui s'était jetée éperdue dans ses bras,
en même temps que M. Clément Malompret, qui s'était brusquement
découvert, ravissait Françoise Colbert. Et les deux hommes
s'éloignaient, ainsi chargés, à grands pas, quand Florimond, sentant sa
proie se débattre entre ses bras, poussa une malédiction sauvage\,:
«\,Ne me serre pas ainsi, Pontaillan, lui soufflait à l'oreille une voix
exécrée, ou je dirai à ta mère la Drapière combien tu es peu
galant\,!\,»

Et, se dégageant lestement, riant de son plus mauvais rire, Catherine de
Lépinière sauta sur l'herbe du pré riverain, tandis que Clément
s'enfuyait à toutes jambes sans lâcher Françoise Colbert suspendue à son
cou.

Sous le coup d'un accès de rage, Florimond, perdant toute prudence,
appelait hautement à l'aide. Il avait tiré son épée et taillait,
estocadait au hasard contre des ennemis invisibles dont il se croyait
entouré. La lumière de la pleine lune, un instant voilée par les nuages,
éclaira alors en plein Catherine, sans masque, et qui, le capuchon
rabattu sur les épaules, continuait de braver Florimond en riant\,:

--- Fuis-t'en, Pontaillan, il n'en est que temps\,!\ldots{} Sans quoi,
tu ne reverras pas ta tendre mère et son aune\,!

Il ne connut plus rien. Pareil au taureau qui fonce, il courut sur la
jeune fille\,:

--- Ah\,! vipère, te trouverai-je donc toujours en train de baver contre
moi\,?

Et, lui détachant un taillant, il l'atteignit de plein fouet. Catherine
chut, en retenant ses plaintes. Sur son col de linge plat s'élargit une
tache de sang. Au même moment Florimond, tiré violemment en arrière,
laissa son manteau aux mains de son agresseur inconnu et tomba sur les
genoux. Ce que voyant, M. de la Butière, qui faisait le guet derrière
lui, sauta par-dessus une haie et gagna au pied en criant\,:
«\,Trahison\,!\,» M. de Tourouvre se crut dès lors obligé de le suivre,
tout en encourageant sa bande d'estafiers attachée à ses talons\,:

--- Au large, au large, mes garçons\,! Voici les rustres qui accourent à
grand renfort de flambeaux et de fourches\,!\ldots{} Le maître est en
avant avec sa belle\,!\ldots{} Au carrosse, au carrosse\,!

Pendant que ces braves jouaient ainsi de l'épée à deux jambes sous
l'ombre propice des bois, un cercle de gens campagnards allait se
rétrécissant autour de Florimond. Le jeune homme se releva vivement.
Comme il n'avait point lâché son épée, il poussa de l'avant, sans
s'occuper du nombre des assaillants que son aveugle courage lui
défendait d'évaluer. Reconnaissant le baron de Mordicourt, qui venait à
lui, il le chargea avec des cris d'aigle où se mêlaient des ordres à ses
donneurs d'étrivières, qu'il croyait toujours derrière lui\,:

--- Allons, vous autres, débarrassez-moi de cette canaille, pendant que
j'expédie la vieille carcasse\,!\ldots{} Tiens, sinistre imbécile,
prends toujours cela pour toi\,!

M. Le Bouteiller était sur ses gardes. Il para le coup avec sa main
gauche enroulée dans son manteau et riposta par une botte poussée avec
tant d'à-propos que Florimond dut reculer pour n'être pas atteint au
défaut des côtes. Un bâton siffla à ses oreilles, s'abattit sur son bras
droit. Son épée, échappée de sa main engourdie, se perdit dans l'herbe,
et Florimond chut sur les genoux pour la seconde fois, à moitié assommé
par un poing qui lui martela la tempe. Appuyé sur sa senestre, il tenta
de se relever. Alors il se sentit saisir au collet, enlever de terre,
puis écraser le nez dans l'herbe piquante. Plus d'armes\,; de sa dague,
arrachée de sa ceinture, il ne lui restait que le fourreau. Maintenant
son dos gémissait sous la grêle des horions qui pleuvaient\,: coups de
pied, coups de poing, coups de bâton.

--- Misérables, vous verrez ce qu'il en coûte d'attirer un gentilhomme
dans un guet-apens\,! Et toi, vieillard sans honneur ni courage, me
laisseras-tu traiter ainsi par tes valets\,?\ldots{} Mon épée, mon épée,
que je te découse\,!

M. de Mordicourt, sans s'émouvoir, laissait Florimond hurler à
s'enrouer. L'épée nue sous le bras, il se rapprocha de son ennemi, tenu
à plat ventre sous l'effort de dix bras, et parla\,:

--- Demeure collé au sol, méchant bâtard\,! Et attends ce que je
déciderai de toi\,!\ldots{} Tu es venu ici comme un larron\,; comme un
larron tu seras châtié\ldots{} Tais-toi\,!

Florimond, étouffant de fureur impuissante, mordait la glèbe desséchée.
En vain essayait-il de lutter contre ceux qui le clouaient à terre. Il
eut beau se tordre, user de ces artifices sournois de la lutte professés
par les prévôts d'académie, ses efforts furent vains. Et voilà que Marin
apparut à ses yeux, Marin, que Cottebleue avait dû enlever pourtant et
conduire à Lunery\,!\ldots{} Pourquoi, aussi, Florimond ne voyait-il pas
l'horizon en feu, là, du côté de Tonlieu, en face\,?\ldots{} On avait
donc négligé d'exécuter ses ordres, d'incendier la tanière des
Labrande\,? Pourquoi tout ce monde ne courait-il pas à l'incendie\,? Il
se crut le jouet d'un cauchemar, au milieu de toutes ces ombres muettes
qui l'entouraient, avec, au-dessus, les dominant de sa haute taille, la
ridicule personne de l'oncle Bouteiller, son bonnet à l'antique et son
épée à la mode du siècle passé.

S'adressant au baron de Mordicourt, Marin dit alors\,:

--- C'est à n'y rien comprendre, monsieur. Voici M\textsuperscript{lle}
Catherine, car on l'a reconnue, qui gît là, navrée par ce scélérat. Ma
mère et mes sœurs s'occupent d'elle\ldots{} À Dieu plaise qu'elle n'ait
point été frappée à mort\,!\ldots{} Que faut-il faire\,?

--- Mon brave Marin, que l'on emporte la pauvre enfant dans ta maison,
et que Jacques Lorquin galope et nous ramène un ou deux médecins de
Bourges pour assister celui de Lunery que Médard s'en ira chercher\,! En
attendant, que M\textsuperscript{me} ma nièce prenne soin de
M\textsuperscript{lle} Catherine\,!\ldots{} Et surtout que
Louis-Antoine, enfermé dans sa chambre suivant mes ordres, ne sache rien
de tout cela\,!\ldots{} Tu y veilleras, Robert.

A ces dernières paroles destinées à l'écuyer Rustigny, Florimond
répondit avec une audacieuse impudence\,:

--- Vieil animal, il le saura toujours, et ce sera moi qui lui
apprendrai le sujet de ma visite, et je lui en rendrai raison\,!\ldots{}
Puisque ton sang se fige à l'idée de te mesurer avec moi, je saurai
obliger ton benêt de neveu à tirer l'épée rouillée de son noble père et
à voir, dans le même instant, son premier combat et son dernier\,!

--- Tais-toi, bâtard\,! Ou, j'en jure le saint nom de Dieu, je te fais
pendre à cet arbre\,!

--- Voici une bonne corde, monsieur, dit le vieux Roquelin Saboureau en
sortant du cercle\,; je la réservais à Cottebleue, mais je crois que le
jeune Pontaillan figurera mieux au bout. Donnez-nous-le, nous prendrons
l'affaire sur nous.

Florimond sentit une sueur froide perler sur sa face. Ces gens
allaient-ils vraiment le pendre\,? La menace n'était point vaine.
L'arbre était là, tout auprès, un vilain poirier décharné, allongeant
une maîtresse branche, tel un bras, et la corde se nouait en coulant,
aux mains sèches du vieil escogriffe armé d'une longue épée qui
traînait. Et Florimond se vit étranglé, se balançant, la langue sur le
menton, tout raide, dansant au gré du vent, lui, l'incomparable, le
coureur de ruelles, l'arbitre du bon ton, le galant chéri des belles\,!
Tenant l'honneur et la vie des autres pour rien, il s'attendrissait à
toute bonne occasion sur lui-même. Et il se lamentait intérieurement de
finir ainsi, quand il avait encore tant de belles années à vivre, de
mourir obscurément, loin de tout secours, et d'une mort infâme qu'ont le
droit de refuser les gens nés\,!

Il protesta donc à sa manière, en insultant et en bravant. Mais de ses
accents insolents et cyniques les sons mal posés laissaient percer son
inquiétude\,:

--- Allons, je vois ce qu'il vous faut\,!\ldots{} De l'argent\,? Rien de
plus aisé\,! Donnez-moi un papier, je le signe, et, foi de Bannes, je
payerai à l'échéance, fidèlement.

--- Tu te vantes, paraît-il, de posséder des lettres de ma petite-nièce,
M\textsuperscript{lle} de Primelles\,?\ldots{} Est-ce vrai\,?

A cette question de M. Le Bouteiller, qui répondait ainsi à Florimond,
dont il s'était encore rapproché jusqu'à ce que son soulier fût à trois
pouces de son front, le jeune baron de Chézal-Benoît sentit renaître
l'espoir de se tirer de ce mauvais pas. Il suffisait de gagner du temps
et d'amuser le bonhomme.

--- Des lettres d'elle\,? Certes j'en ai. Mais elles me sont à ce point
précieuses que pour rien au monde je ne m'en séparerai\,!

--- Sont-elles donc sur toi\,?

--- Prends ma vie\,! Tu n'auras pas mes lettres\,! cria Florimond,
furieux de s'être ainsi sottement laissé deviner.

--- Holà\,! vous autres, fouillez-le, et donnez- moi tout ce qu'il porte
sur lui de papiers\,!

Ce fut une lutte sauvage. Étendu sur l'herbe, tiré aux quatre membres,
écartelé, Florimond, sanglotant, écumant de colère, se vit arracher son
pourpoint. On visita jusqu'à ses chaussures. Un petit paquet de lettres
fut trouvé dans le haut-de-chausses. Tranquillement, M. de Mordicourt
garnit son nez de besicles, lut tout à la clarté d'une lanterne que
tenait à bonne hauteur le portier Saboureau. Puis, ouvrant la petite
porte de corne, il communiqua à chacun des billets le feu de la
chandelle et les réduisit en cendres. Après quoi, il dit, sans hausser
la voix\,:

--- Tu en as menti, bâtard, ces petits vers ridicules ne sont pas des
épîtres. Tu n'as pas de lettres de ma petite-nièce.

Bien qu'à moitié étouffé par un coup de sang, Florimond comprit et
maudit son imprudence. Par fatuité, le galant n'allait jamais sans les
lettres de sa nouvelle conquête, afin de les montrer à tout venant.
Toutes les lettres de Marguerite de Primelles à lui adressées venaient
d'êtres détruites. Évidemment les paroles de M. Le Bouteiller étaient
l'expression de la vérité, quoi qu'elles continssent de mensonges. Ces
lettres, Florimond ne les avait pas, c'est-à-dire qu'il ne les avait
plus.

M. Le Bouteiller reprit\,:

--- Enfin, me diras-tu ce que tu étais venu chercher ici\,?\ldots{} Tu
ne veux pas répondre\,? Eh bien\,! je répondrai pour toi. Tu t'es
introduit chez moi comme un voleur et un vil meurtrier pour assassiner
M\textsuperscript{lle} de Lépinière, dont ta mère la Drapière et toi
convoitez le bien\ldots{} Fâché d'avoir manqué ton coup lorsque tu
essayas de faire tuer cette héritière par tes laquais Tourouvre et La
Butière, tu as voulu frapper toi- même\,!\ldots{} Allons, avoue que
c'est vrai\,!

--- Tu mens\,!\ldots{} D'ailleurs, je ne prendrai plus la peine de
répondre à un vieux fou de ton espèce, dont le cerveau est gros de
chimères\,!

--- Si tu te rends insolent, bâtard, fils d'une dévergondée et d'un
assassin\ldots{}

Florimond ne put entendre cela sans rugir. Secouant, par un effort
désespéré, la grappe humaine accrochée après lui, il se redressa à demi.
Les éclats de sa colère remplirent la prairie d'un tel bruit que les
gens de Primelles en frémirent, et Marguerite, qui suivait la scène,
cachée derrière ses volets, à demi-morte de honte et de terreur,
tremblait comme si la bise de décembre eût baisé ses épaules nues.

--- Ah\,! vieux mendiant, n'insulte pas les miens, ou rends-moi mon
épée, que je les défende\,!\ldots{} Sang de rustre, seigneur de la soupe
aux choux, pauvre fesse-lièvre, sache que les femmes de ta maisonnée de
gueux ne seraient point même bonnes à divertir mes valets\,!\ldots{} Va,
j'avais disposé de ta nièce\,! Dès demain, mon valet de chambre avait
licence d'en user\,!

Marin, à entendre cela, se rua sur Florimond. On eut grand'peine à
l'arracher de ses mains. M. Le Bouteiller imposa silence à tous et donna
ses ordres\,:

--- En voilà assez. Qu'on mène cet injurieux brutal où vous savez, et
qu'il soit traité suivant ce que Symphorien commandera\,!

Et, tournant le dos à Florimond toujours étendu sur son herbe avec deux
hommes à chaque bras, deux à chaque jambe, sans préjudice d'un autre qui
le tenait par les oreilles, le baron de Mordicourt rentra au château de
Primelles.

Le vieux Symphorien Labrande, accompagné par quelques paysans munis de
torches, donna le signal du départ. Florimond, emporté comme un paquet,
vit qu'on se dirigeait vers l'allée principale du bois des Usages où
aboutissait cette route même qu'il avait fait réparer\,: «\,C'est fini,
pensait-il, les drôles n'osent rien entreprendre contre moi. Ils vont me
remettre sur mes terres\ldots{} À moins qu'ils n'espèrent, peut-être,
surprendre mes gens et mon carrosse\,?\ldots{} Mais ceux-ci doivent être
déjà loin\,! J'ai été bien abandonné, royalement, si je puis
dire\ldots{} Après tout, pouvaient-ils me défendre\,? Mieux vaut qu'ils
aient gagné le large\ldots{} Une bataille, un rapt, un incendie venant
d'un même temps, l'affaire eût été grosse\,!\ldots{} Bast\,! demain il
n'en sera plus question\,!\ldots{} Quelle vengeance à suivre\,! Je jure
que pas un de ces hobereaux, pas un de ces rustiques, ne s'en tirera
sans une dégelée de bois vert\,!\ldots{} Patience, mes enfants\,! Vous
verrez ce qu'il en coûte de porter la main sur moi\,!\ldots{} Ah\,!
ah\,! nous voici arrivés\,!\,»

Oui, l'on était arrivé à la clairière de la Table. Et cette clairière
était ainsi appelée parce qu'en son milieu un gros rocher plat,
grossièrement travaillé par les antiques Gaulois, y formait une sorte de
table qu'entouraient d'informes blocs de pierre. Les gros chênes
disposés en cercle retinrent l'attention de Florimond, car au tronc de
chacun d'eux était attaché, le ventre contre l'écorce, un homme nu
jusqu'à la ceinture.

Et du jeune baron de Chézal-Benoît la crinière blonde se hérissa, parce
qu'il connaissait chacun de ces hommes\,: Cottebleue, les six garçons de
Landry Vaillard, le fermier des Aubroys, ceux-là même qui avaient déjà
arrêté Marin en trahison, dix autres encore, tous gens choisis par
Cottebleue pour l'aider à enlever encore Marin cette nuit même et à
brûler la chaumière des Labrande.

Florimond se sentit pareillement dépouillé et attaché par les poignets à
un chêne. Muet de terreur, ne comprenant rien, redoutant tout, il
entendit Symphorien, assis à la table de pierre, réciter une sorte
d'acte de procédure\,: «\,Cottebleue et consorts, coupables d'avoir
envahi violemment et nuitamment les bergeries de Primelles et d'avoir
essayé de les incendier, seraient livrés demain à la justice d'Issoudun,
ainsi que leur seigneur, ou se disant tel, le bâtard Pontaillan, dont
ils avaient suivi les ordres. S'ils voulaient échapper à la justice du
roi, force leur était d'accepter celle des bergers. Ils avaient le
choix\,; qu'ils parlassent\,! S'ils se taisaient, c'est qu'ils
acceptaient cette dernière justice.\,»

A ce moment, Florimond crut distinguer à quelque cent pas de lui, sur sa
route neuve, les lumières du carrosse et de ses laquais. On venait à son
secours, c'était certain\,!

--- À moi, mes amis\,! cria-t-il d'une voix tonnante. Par ici\,! On veut
assassiner votre maître\,!

Aucune voix ne répondit à la sienne, sinon celle du vieux Labrande\,:
«\,Puisque personne ne se soucie d'aller prendre à Issoudun son
passeport pour les galères de Sa Majesté, chacun sera payé suivant son
mérite.\,»

Florimond n'avait pas pensé à cela. Plus d'un gentilhomme s'était vu
rouer pour une semblable histoire. S'il échappait, par grand hasard, aux
malveillantes rigueurs des magistrats, son père, le marquis, averti par
Catherine\ldots{} Ah\,! oui\,!\ldots{} Il y avait Catherine, et il
serait accusé de l'avoir assassinée par surcroît\,!\ldots{}

Lentement, le vieux berger Symphorien dénombrait les coups de corde ou
d'étrivière qui revenaient aux coupables\,: «\,Cent coups à Pontaillan,
soixante-quinze à Cottebleue, cinquante aux six garçons de Landry
Vaillard\ldots\,» et ainsi des autres.

Se détournant rageusement de la rude écorce moussue et suintante où se
souillaient ses cheveux et sa moustache toujours enfilée dans la
turquoise, Florimond vit les bergers de Primelles rangés dans la
clairière. Jamais il ne les aurait crus aussi nombreux. Marin, la figure
éclairée par un rire sinistre, s'approchait de lui, une courroie double
de sellerie au poing. Et sous la cinglante morsure l'échine de
l'Incomparable Florimond céda et frémit. Au vingtième coup, les longs
hurlements du misérable dominaient ceux des autres fustigés. L'assassin
de Catherine de Lépinière appelait sa mère. Mais le vieux Labrande
continuait de compter, et, régulièrement, chaque coup s'abattait sur son
homme.

Au trente et unième, Florimond perdit la connaissance de lui-même. Il
demeura suspendu au tronc, éclaboussé de l'écume vermeille de son sang,
par ses poignets que déchirait la corde, la tête lourde de sa chevelure
fauve rejetée en arrière. Le vieux Labrande et son fils Marin ne
s'interrompaient point de frapper ni de compter.

\hypertarget{chapitre-xi}{%
\chapter{CHAPITRE XI}\label{chapitre-xi}}

M\textsuperscript{lle} de Lépinière fut transportée dans une chambre du
château de Primelles, où la nourrice Ursule s'occupa de la déshabiller
et de la mettre au lit, avec l'aide de la vieille Jeannette Labrande et
de ses filles Francine et Marion. Stupide d'épouvante, Margot
Larçonnière était tombée en pâmoison dès qu'elle eut vu tout le sang qui
inondait la jeune fille et avait collé sa chemise au corps. Quant à
M\textsuperscript{lle} de Primelles, le mauvais état de sa santé
l'obligea à ne pas quitter sa couche. On dut se contenter de cette
excuse, et sur les quatre paysannes retomba toute la charge de cette
blessée qu'on avait craint de voir passer de vie à trépas au premier
moment. Les soins intelligents de ces simples femmes des champs
réussirent à ramener la vie dans ce corps frêle que l'on avait cru tout
d'abord inanimé pour jamais.

Quand le vieux médecin de Lunery arriva sur sa mule Toineau à deux
heures du matin, Catherine n'avait pas repris complètement connaissance,
mais la vie luttait contre la mort, ce qui se sentait à l'irrégularité
du pouls. Deux autres médecins, accourus de Bourges, ne purent que
constater, au lever du jour, l'état de la blessée, sans oser donner de
l'espoir. L'apothicaire qu'ils avaient amené maniait ses drogues, et le
curé de Primelles, à genoux dans un coin avec son sacristain, priait, en
attendant d'administrer les derniers sacrements à la mourante.

C'est alors que, derrière la porte soigneusement close, on entendit des
coups précipités, des cris et des menaces. Louis-Antoine, qui s'était
sauvé en sautant par une fenêtre de la chambre où Robert de Rustigny se
flattait de le tenir enfermé, menait un tapage infernal et menaçait de
se tuer si on ne lui ouvrait pas. Aux éclats de cette voix tour à tour
suppliante et furieuse, M\textsuperscript{lle} de Lépinière remua
légèrement. La pâleur affreuse de ses joues disparut pour faire place à
une teinte rosée. Elle entr'ouvrit les yeux, se guinda sur ses coudes,
releva la tête et murmura\,:

--- Est-ce toi, mon Louis-Antoine\,?\ldots{} Ah\,! viens, viens ici près
de moi\,!

Un débat s'éleva alors entre les trois médecins. Si le plus vieux était
d'avis qu'on obéit au désir de la blessée, puisque, seul, le son de
cette voix avait pu réveiller la jeune fille de la léthargie fatale où
elle paraissait s'enfoncer, les deux autres prétendaient que l'émotion
pouvait tuer tout net la malade. Et, à l'appui de leurs dires, ils
montraient l'appareil de bandes disposé autour du cou et du buste et sur
qui, à chaque effort de Catherine et en dépit de la charpie et du cérat,
s'élargissaient des taches pourprées. «\,Les blessures se rouvriraient
et la patiente s'éteindrait, privée de sang, ainsi qu'une lampe sans
huile.\,»

Le curé de Primelles se rallia pourtant à l'opinion du médecin de
Lunery. Il obtint que Louis-Antoine serait introduit, à cette condition
qu'il garderait le silence et se retirerait aussitôt qu'on le lui
ordonnerait. Le jeune baron, ainsi prêché à travers la porte, promit
tout ce qu'on voulut. S'échappant des mains de son oncle, qui, mal aidé
par Robert de Rustigny, s'obstinait à le vouloir retenir, il entra sur
la pointe de ses pieds déchaussés, encore tout mouillé par l'eau du
fossé où il avait barboté pour s'évader. Ses pauvres vêtements
ruisselaient moins d'eau que son visage de larmes, et il sanglotait si
fort qu'aucune parole ne pouvait sortir de sa bouche. Il tomba à genoux
près du lit, tendit les bras vers celle qui était tout pour lui en ce
monde, et sa face bouleversée criait dans son mutisme un tel désespoir
que tous les assistants courbèrent la tête, frémissants d'émotion.

Catherine alors parla\,:

--- Louis-Antoine, dit-elle d'une voix si faible qu'on en pouvait à
peine percevoir les accents à deux pas de la couche, mon cher
Louis-Antoine, ne te désole pas ainsi\,!\ldots{} Vois, je te reconnais,
et même je me sens déjà mieux depuis que tu es là, près de moi\ldots{}
Tu le vois, je suis entourée de soins\ldots{} et en train de
guérir\ldots{} Il y a si longtemps que je suis couchée ici\ldots{} Cela
ne serait rien si une bête ne me mordait la tête\ldots{} Non, non, tu
m'attendras, et je volerai encore à l'oiseau\ldots{} avec toi,
Louis-Antoine\ldots{} Oui\ldots{} à l'oiseau\ldots{} Pourquoi me
frappe-t-on toujours sur le cou\,?\ldots{} Ah\,!\ldots{} Louis\ldots{}

Un hoquet rompit sa voix qui s'élevait, chantante, avec le délire.
Catherine devint encore plus blanche qu'elle n'était quand on la releva
près du fossé. Ses mains jouèrent de l'épinette sur la couverture\,; sa
tête roula entre deux oreillers\,: elle s'évanouit. Les femmes
s'élancèrent avec leur vinaigre, leurs plumes flambantes, et le curé
emmena Louis-Antoine. Il le portait presque, car l'enfant se pouvait à
peine soutenir. L'oncle Bouteiller l'entraîna dans sa chambre et
commanda qu'on lui changeât ses vêtements, car ses dents claquaient de
fièvre.

Louis-Antoine avait appris d'un petit berger courant le long du fossé le
malheur de M\textsuperscript{lle} de Lépinière, sans pouvoir savoir qui
l'on accusait de cet assassinat. Le pâtour parlait d'une chute de
cheval. D'autres gens étaient venus qui l'avaient emmené sans répondre
aux questions de Louis-Antoine. Cela se passait aux premières heures du
matin, car le beau sommeil de la jeunesse avait empêché Louis-Antoine
d'entendre quoi que ce fût des scènes de la nuit. Quand il voulut sortir
de sa chambre, il s'aperçut qu'elle était fermée. Il cogna en vain,
personne ne lui ouvrit. Alors il sauta par la fenêtre dans la douve et
rentra par la cuisine. En vain son oncle et l'écuyer Robert
s'étaient-ils mis à ses trousses. Courant derrière Marion, qu'il voyait
chargée de charpie et de bandes, se diriger vers la chambre,
ordinairement inhabitée, du rez-de-chaussée, il avait failli entrer
derrière elle. Et c'est contre cette porte brusquement poussée qu'il
s'était évertué à frapper.

Pendant que la pauvre maison de Primelles s'agitait dans l'inquiétude et
la crainte, Marguerite, obstinément barricadée chez elle, roulait dans
sa tête malade les projets les plus contradictoires et les plus
insensés. La prudence réussit cependant à se faire écouter de ce cœur,
meurtri encore plus par l'orgueil que par l'écroulement de ses rêves
amoureux. Convaincue de l'indignité de Florimond, puisqu'elle n'avait
rien perdu du drame furieux qui s'était déroulé sous sa croisée, elle ne
s'apitoyait pas sur son bonheur ruiné, mais s'exaspérait d'avoir été
ainsi trompée par celui qu'elle avait institué son maître et son roi.
Marguerite de Primelles avait compris que son oncle, en brûlant ses
lettres, anéantissait les preuves matérielles de sa faute. D'autres
preuves existaient cependant\,: les lettres de Florimond. Elle les prit
dans son corsage, car elle n'avait pas voulu s'en séparer pour fuir, et
les consuma à la flamme d'une chandelle. Une à une elle les détruisit,
sans se donner la joie amère et stérile de relire cette correspondance
passionnée où M. Aimeri d'Olivier avait jeté, sans compter, les lieux
communs et le pathos de la galanterie des ruelles.

Marguerite piétina les cendres de sa passion sur l'âtre de la cheminée,
aussi froid que son cœur de liseuse pédante, et se jura de n'avoir
désormais plus commerce avec homme sur terre, de ne plus croire à
l'amour, voire à l'amitié, et de se retirer dans un monde tout
intérieur, éclairé par ses seules réflexions. Dans son égoïsme farouche
de vierge précocement vieillie par l'expérience des désillusions, elle
avait trouvé du plaisir à voir Florimond houspillé par ses gens\,: elle
en avait trouvé plus encore à voir Catherine de Lépinière tomber sous
l'épée de Florimond. Elle avait reconnu dans ce crime abominable
l'inexorable vengeance d'un Dieu de justice qui protège les amours des
créatures supérieures et préserve celles-ci des écarts déshonnêtes. Car
l'indignité manifeste de Florimond fortifiait M\textsuperscript{lle} de
Primelles dans cette idée qu'elle seule possédait la noblesse et la
délicatesse des sentiments qui élèvent l'humanité au-dessus des vains
préjugés et des lois hypocrites des convenances.

Son orgueil monstrueux, peut-être comparable à celui d'un poète, n'avait
pas, néanmoins, tué en elle cet esprit de finesse qui fait rarement
défaut aux personnes de son sexe. Aucune inquiétude pour sa sûreté ne
vint troubler ses réflexions morales. Elle savait l'oncle Bouteiller
trop brutalement chevaleresque pour la dénoncer à sa mère\,; elle savait
aussi que le secret serait gardé par tous les témoins de la nuit, parce
que ces témoins, suivant à la vérité le châtiment d'autres offenses,
ignoraient la nature autant que les détails de son intrigue. Seule,
Catherine\ldots{}

Ah\,! oui, la Catherine détestée\,!\ldots{} Mais, d'abord, elle devait
être morte à cette heure, ou, tout au moins, elle n'en valait guère
mieux. Et puis Catherine n'avait pas d'intérêt à parler. Restait
Françoise Colbert\,?\ldots{} Sans doute\,! Mais cette servante,
irrévocablement compromise dans une affaire où le beau rôle n'était
point pour elle, et qui ne détenait aucune preuve de ses bons offices,
n'irait pas chanter aux quatre coins du royaume cette aventure, demeurée
à l'état d'ébauche, et dont demain personne ne se soucierait.

En tout cas, M\textsuperscript{lle} de Primelles regrettait sa
chambrière Colbert, pour cette raison principale que, si cette fine
mouche fût demeurée à la maison, elle, Marguerite ne serait point
demeurée sans nouvelles. Car Marguerite ne se préoccupait que de
Marguerite, et le monde commençait et finissait avec sa discrète
personne.

Pas un seul instant M\textsuperscript{lle} de Primelles ne fut
tourmentée par cette idée qu'elle avait trahi ses devoirs de fille
noble. --- Qu'elle se fût laissé abuser par un homme méprisable, à cela
aucune négation n'était opposable. Toutefois elle objectait qu'erreur ne
fait pas compte et que la dupe est toujours supérieure à l'imposteur,
quoi qu'en édicte le tribunal de la belle société, et quand cette dupe
est Marguerite de Primelles en personne. Quant à nourrir un sentiment de
gratitude pour Catherine de Lépinière, qui s'était impudemment
substituée à elle pour la sauver, Marguerite s'en serait gardée comme
d'avaler du poison\,: «\,De quoi s'est mêlée cette évaporée, cette
virago champêtre\,?\ldots{} Et qui m'assure qu'elle ne cherchait pas à
obliger Florimond à l'épouser en rendant ainsi son enlèvement
manifeste\,?\ldots{} Qu'elle vive ou qu'elle crève, je ne m'en soucie
pas, non plus que de Florimond, d'ailleurs. Si mon oncle l'a tué, c'est
un malheur dont je suis déjà consolée.\,»

M\textsuperscript{lle} de Lépinière vécut. Quatre jours après la nuit
terrible où l'épée de Florimond, heureusement amortie par les boucles de
sa chevelure, avait balafré son cou, sans offenser les grosses veines,
pour entamer profondément la clavicule et labourer le sein droit, on put
la transporter à la maison de M. de Montenay sur des brancards. Elle
trouva à la Vergne la marquise de Creulles avec ses femmes de service et
aussi Gilette Léchanson et Maroie Lenatier. Les hôtes de Primelles
l'avaient accompagnée dans ce voyage d'une petite lieue qui dura deux
grandes heures, tant on tenait à éviter les cahots à la blessée. L'oncle
Bouteiller gardait la droite avec M. de Montenay, et Louis-Antoine
n'arrêtait pas d'aller de l'avant, pour revenir en arrière, essoufflant
son bidet à tous crins et trompant son impatience par un perpétuel
mouvement. Les rideaux de cuir de l'appareil porté par deux mulets
étaient en effet tirés pour que la poussière n'incommodât point
Catherine. Et la pire douleur de Louis-Antoine était de savoir son amie
si près de lui sans pouvoir ni lui parler ni la voir.

Trois mois passèrent avant que M\textsuperscript{lle} de Lépinière fût
capable de sortir. Pendant ce temps, elle ne reçut à la Vergne aucune
visite, ni de Bourges ni d'ailleurs. Seuls ses amis de Primelles, M. de
Montenay et sa parente M\textsuperscript{me} de Creulles eurent accès
auprès d'elle, et Louis-Antoine, qui passait désormais à la Vergne les
journées consacrées naguère à marauder du côté de Tonlieu. M. de Mauny
d'Anrieux n'était point de son côté demeuré inactif. On le vit arriver
un jour porteur d'une petite dame-jeanne artistement clissée d'osier.
C'était un baume qu'avait fabriqué sa gouvernante Marion, baume de
Hollande, entre tous souverain pour la cicatrisation des blessures et
bien supérieur à cette eau d'arquebuse dont on a certainement exagéré
les mérites. Le vieux médecin de Lunery, homme bizarre, très épris de
nouveautés, n'hésita pas à appliquer ce remède. Les résultats furent
merveilleux. Alors Marion chargea M. de Mauny d'une seconde bouteille.
Celle-là renfermait un cordial distillé par la dame avec on ne savait
quel élixir de noyaux de fruits et aussi des jus d'herbes champêtres
odorantes et dont les esprits subtils auraient rappelé un mort à la vie.

Ainsi soignée et droguée, Catherine ne tarda pas à se sentir plus
vaillante. Elle se levait, marchait par les chambres, appuyée sur Maroie
ou sur Louis-Antoine. Et le cuisinier de la marquise de Creulles, qui
avait transporté ses casseroles et ses épices dans la petite maison de
M. de Montenay, abondait en ingénieuses inventions. Ses coulis, ses
bisques et autres potages embaumaient la maison. Il consentit à donner
quelques recettes que Catherine adressa à Marion avec une mallette
pleine de cols et de manchettes en point coupé.

Quant à l'attaque dont M\textsuperscript{lle} de Lépinière avait failli
demeurer victime, il n'en fut jamais question. Une singulière complicité
liait les langues. Jamais secret ne fut mieux gardé. D'ailleurs,
personne ne savait au juste ce dont il retournait. L'on disait\,:
«\,l'accident de la demoiselle\,». Et l'on évitait d'en parler. De
Florimond, de sa mère, l'on n'avait pas de nouvelles, non plus que de
Cottebleue du reste, ni de ses complices. La chronique scandaleuse et
amusante de Bourges ne trouva pas de quoi s'enrichir dans le drame de la
nuit du 13 juillet. L'année n'était pas écoulée que l'histoire de
Florimond tombait dans l'oubli. Au surplus, on ne s'en était que peu
inquiété. Personne ne songea à établir des rapports entre l'absence du
jeune homme et la maladie de M\textsuperscript{lle} de Lépinière. La
marquise Julie vivait à Paris avec son fils, à ce que l'on disait. Et
encore l'on n'en était pas certain. Et enfin aucune plainte ne fut
portée contre les gens de Primelles.

Et pourtant ni M. de Montenay ni Louis-Antoine n'ignoraient de qui
venait le coup. À force d'interroger Jeannette Labrande, de la câliner,
de la corrompre par de petits cadeaux, et surtout de la tenir au courant
de la santé de Catherine, il finit par savoir une partie de la vérité.
Mais cette partie était pour lui la principale\,: Catherine avait été
frappée par Florimond.

Dès lors, l'assiduité dont il donna des preuves en ses exercices
d'escrime, les progrès vraiment extraordinaires qu'il fit, étonnèrent
Robert de Rustigny et charmèrent M. Le Bouteiller\,: «\,Il comprend
enfin son devoir, disait celui-ci. Le gentilhomme a été long à
s'éveiller en lui. En fin de compte il s'est éveillé, c'est le
principal\,!\,» Et Robert de Rustigny de se glorifier\,: «\,Je puis être
fier d'un tel élève\ldots{} Le jeune homme mord, monsieur\ldots{} Il
mord à la noble science des armes\ldots{} Si vous aviez vu, ce matin,
--- car maintenant il prend leçons matin et soir, --- l'estocade de pied
ferme dont il m'a arrêté quand je portais l'épée haute, les larmes,
monsieur, vous en auraient jailli des yeux\ldots{} J'en porte encore les
marques\ldots{} Quelle botte\,!\ldots{} N'importe qui serait tombé sous
ce coup, tant il fut fourni avec dissimulation et souplesse\,!\ldots{}
Coup de spadassin, indigne de la doctrine académique\,?\ldots{}
Possible\ldots{} Mais pour mettre un homme par terre je n'en connais pas
un pareil\ldots{} Rappelez-vous que c'est ainsi que Thémines tua jadis
le marquis, frère aîné du cardinal ministre\ldots{} Ah\,! oui, monsieur,
il ira loin, le jeune homme, il ira loin\,!\,»

La nouvelle de cette éclatante conversion parvint jusqu'à la baronne de
Primelles. Vivant toujours en recluse dans sa chambre, loin de tous et
de tout, on la vit, presque avec effroi, arriver un jour dans la salle
des exercices, avec ses guimpes monastiques et ses longs habits de
deuil. Silencieuse et grave, elle ordonna par un signe à l'écuyer de
continuer d'enseigner son fils, et, assise sur un pliant qu'avait
apporté Margot Larçonnière qui tremblait de peur au choc des épées, elle
assista à la leçon tout entière. Les approbations fréquentes du vieux
baron étaient reçues par elle avec des regards où se lisait une
fiévreuse reconnaissance. Quand la séance eut pris sa fin,
M\textsuperscript{me} de Primelles, saisissant Louis-Antoine par les
épaules, l'embrassa sur le front avec une tendresse dont l'enfant se
sentit étrangement pénétré. Les pleurs de sa mère mouillèrent son
visage, se mêlant à la sueur qui en découlait. Et la baronne, toujours
muette, froide et sereine, ainsi qu'une image taillée dans le marbre,
regagna sa chambre.

Elle y reçut M\textsuperscript{lle} Marguerite de Primelles, qui lui
demandait audience par la bouche de la petite Francine, attachée à sa
personne comme servante depuis que Françoise Colbert s'était envolée aux
bras de M. Clément Malompret. L'entretien de la mère et de la fille,
tenu sur un ton glacial, fut bref et compendieux, comme si, aux yeux de
ces deux femmes pour ainsi dire étrangères l'une à l'autre, les paroles
eussent une valeur telle qu'on ne dût les prodiguer sans péché.
Marguerite exprima à sa mère son ferme propos d'entrer en religion.
Avant de faire profession, elle passerait un an ou deux au couvent des
Augustines de Bourges, où l'on pourrait l'utiliser pour l'instruction
des jeunes pensionnaires.

M\textsuperscript{me} de Primelles n'éleva point d'objections. Elle
répondit à Marguerite que le projet lui apparaissait raisonnable et
qu'on s'en occuperait.

--- Du moment que votre humeur ne vous porte pas vers le mariage, rien
ne peut être meilleur que ce parti auquel vous vous arrêtez. J'eusse
préféré vous voir contracter union avec M. de Montenay, qu'on m'a dit
avoir du goût pour vous. Mais je ne veux point vous contraindre. Ayant
le choix entre ce gentilhomme, qui est riche et de bonne maison, et M.
de Mauny d'Anrieux, qui, je le sais, vous recherche, vous auriez pu vous
décider pour le premier, car de M. de Mauny l'on jase un peu beaucoup à
propos de sa gouvernante. Encore n'est-ce point là un prétexte
suffisant. Toute femme qui se respecte doit savoir en supporter. N'en
parlons plus\,! Pour observer les convenances, j'en toucherai un mot à
mon oncle Bouteiller. En tous cas, aujourd'hui comme demain, vous avez
mon approbation. Allez, je vous donne permission de vous retirer.

Et Marguerite de Primelles, après avoir baisé la main de sa mère, était
rentrée chez elle, cependant que la baronne, agenouillée sur le carreau,
--- car, par mortification, elle ne se servait point de son prie-Dieu, à
moins qu'elle ne fût malade, --- priait, attendant que le ciel
l'assistât de ses lumières\ldots{} A grand'peine se retenait-elle de
supplier le Souverain Juge d'exaucer des vœux qu'elle n'osait point
formuler. Car ces vœux n'étaient que d'homicide et de vengeance.

Et c'est pourquoi cette femme, encore jeune, se consumait lentement,
pareille à ces braises dont on ne distingue pas l'incandescent éclat,
mais seulement le revêtement gris et velouté sous quoi elles ne
s'arrêtent de brûler qu'à l'heure où toute leur substance tombe, réduite
en cendres.

L'automne finit, puis ce fut l'hiver qui ensevelit tout sous la neige.
Chacun vécut claquemuré. Seul, dédaigneux des frimas, Louis-Antoine
continuait de courir entre Primelles et la Vergne, où Catherine, sur
pied et tout à fait brave, s'amusait à chasser avec celui que les gens
du lieu, dans leur langage innocent et bienveillant, n'appelaient pas
autrement que «\,son petit mari\,».

Mais quand les souffles capricieux du printemps eurent fondu cette neige
que méprisait Louis-Antoine, M\textsuperscript{lle} Marguerite de
Primelles dénonça sa ferme intention de se rendre au couvent de Bourges.
Et Catherine offrit de l'y mener, avec sa mère, dans le carrosse qu'elle
fit tirer de sa remise au château de Bannes. M\textsuperscript{me} de
Primelles accepta cette offre que Marguerite aurait volontiers éludée.
D'ailleurs tout était devenu indifférent à la jeune fille. Le voisinage
de Catherine pendant quelques heures ne lui semblait pas bien pénible,
puisque la présence de sa mère empêcherait cette étrangère de se livrer
à des réflexions déplacées. Au fond, la seule chose que Marguerite de
Primelles ne pardonnait pas à Catherine de Lépinière, c'était de ne pas
avoir péri sous l'épée de Florimond et, grâce à ce contre-temps, d'être
demeurée maîtresse d'un secret qu'elle n'eût pas dû pénétrer.

Les trois femmes partirent, après la semaine de Pâques, avec quatre
chevaux, un cocher, deux petits laquais placés derrière le coffre, et
Margot Larçonnière, qui prit place dans la voiture avec les maîtresses.
Comme l'on voyageait en plein jour et que jamais le pays n'avait été
plus tranquille, l'on ne prit point d'escorte. André d'Archelet et
Robert de Rustigny devaient se rendre à Bourges, à cinq heures du soir,
avec quelque huit laquais armés, pour le retour. Louis-Antoine
travaillait l'escrime à Primelles, l'oncle Bouteiller souffrait de la
goutte, M. de Montenay était parti pour Blois, appelé par des affaires,
et M. de Mauny d'Anrieux chassait du côté de Vatan. Quant au château de
Bannes, il montrait tous ses volets joints, ses portes fermées, et la
rumeur courait que Florimond et sa mère étaient partis pour l'Autriche
sur l'ordre du marquis, pour lors en résidence à Vienne.

Or il advint, en traversant le bois de la Corne, que le carrosse,
empêché par le mauvais état de la route, fut entouré par une bande de
cavaliers dont les mauvais desseins apparurent évidents. Ces hommes,
masqués, se coiffaient de grands chapeaux dont les vastes bords rabattus
eussent suffi, à défaut de leurs perruques de toutes couleurs, à leur
cacher la mine. Deux se placèrent en travers du chemin, arrêtèrent les
chevaux en les prenant au mors. Un troisième menaça le cocher de son
pistolet, cependant que d'autres, demeurés derrière, effrayaient les
petits laquais, qui, sautant à terre, s'enfuirent vers la couture de
Vernillier et crièrent si fort que le reste des agresseurs, hésitant,
accomplirent mal la besogne dont on les avait chargés. Chacun d'eux
brandissait, en effet, un flacon de verre rempli d'un liquide noir\,; et
ils cherchaient à en frapper au visage les quatre femmes serrées dans la
voiture.

Sans se laisser troubler par cette audacieuse attaque et au risque de se
faire défigurer, M\textsuperscript{lle} de Lépinière se jeta devant
M\textsuperscript{lle} de Primelles et la couvrit de son corps. Margot
Larçonnière, tombée en pâmoison, embarrassait M\textsuperscript{lle}
Marguerite, qui, machinalement, se couvrit les yeux de ses bras croisés,
opposant l'épais tissu de ses manches renflées aux coups dont on lui
menaçait la tête. Les bouteilles, mal dirigées, parce que leurs
porteurs, pris entre le remblai et les moyeux des roues, étaient obligés
de travailler leurs chevaux pour les garder des atteintes, se brisèrent
contre les portières et les cadres des fenêtres, mêlant leurs débris
coupants aux éclats des glaces qui volaient de tous côtés. Les jets
d'encre retombèrent en pluie sur les quatre femmes, et les malandrins se
bousculèrent pour gagner au pied.

La scène ne dura que quelques minutes.

Grâce au sang-froid et au courage du cocher Jacquin Navelier, le
carrosse s'était remis à rouler. Dressé sur son siège, son long fouet au
poing, Jacquin, qui avait servi dans les piquiers, méprisa le canon de
l'arme braquée sur lui. La portée de cuir et de corde tressés tournoya,
dessina un huit, atteignit l'homme, qui hurla, aveuglé par le coup qui
l'éborgna à demi, et tira au hasard, en même temps que sa monture,
exaspérée par le fouet, l'emportait après deux écarts qui faillirent le
désarçonner. Cramponné à la crinière, démasqué, pleurant du sang, le
drôle ne disparut pas si vite que Jacquin ne pût mettre un nom sur le
museau ainsi endommagé. Et ce nom fut celui de Briand Perrasset, premier
cocher de la marquise de Bannes.

A la détonation rendue ainsi innocente, répondit une autre, aussi forte.
M\textsuperscript{lle} de Lépinière, saisissant vivement un des deux
grands pistolets de carrosse qui dormaient dans les bottes à la portière
de gauche, tira dans le cheval d'un des bandits aux bouteilles, et si
heureusement que la bête s'abattit et que l'homme, atteint à la cuisse,
demeura engagé sous son barbe, la tête portant contre la roue droite de
derrière. Jacquin, continuant de fouailler, sangla alors à toute volée
les braves qui prétendaient immobiliser son attelage. Sous la force des
coups, masques, chapeaux, perruques s'éparpillèrent, les chevaux
pointèrent fous de douleur, les hommes se laissèrent emmener, tournant
le dos au cinglant orage, et le coche recommença de rouler après une
violente secousse.

Les cris de rage des bandits se doublèrent par les exclamations
d'horreur des femmes, lorsque la tête du cavalier abattu fut broyée par
la roue. Du crâne qui péta avec un bruit sec giclèrent des fragments de
cervelle. M\textsuperscript{lle} de Primelles en fut couverte et se
renversa évanouie.

Du côté de Vernillier, une huée montait. Bientôt un gros de paysans
brandissant des volants, des fourches et des faux, dévala sur la route,
avec les petits laquais maniant chacun un grand bâton. Redoutant que
cette canaille ne leur fit un mauvais parti, les cavaliers, qui l'épée à
la main s'apprêtaient à piquer dans la voiture, tournèrent bride et
s'enfuirent à toute vitesse vers la Chapelle-Saint-Ursin, pour gagner
sans doute la route de Tours. Mais ils ne s'enfuirent pas si vite qu'un
coup de faux ne tranchât le jarret d'un barbe. On entendit des
malédictions, des ordres\,: «\,Sauvez M. de Tourouvre\,! Briand, Rénier,
holà, vous autres\,! Que Briand le prenne en croupe, c'est encore lui le
mieux monté\,!\,»

Catherine reconnut la voix, celle de Clément Malompret. Elle ordonna à
Jacquin de s'arrêter et descendit. Le cheval blessé par elle et l'homme
tué gisaient à cinquante pas derrière. M\textsuperscript{lle} de
Lépinière s'en approcha. L'un des petits laquais, écartant la chevelure
postiche empouacrée de sang, découvrit la face disloquée qui semblait
rire hideusement au soleil. Et le garçon ne se trompa ni sur la personne
ni sur la condition du défunt\,:

--- C'est Cottebleue, le porteur d'exploits, qu'on disait parti pour
Paris\,!

Quant au cheval, il portait sur la hanche la marque des écuries de
Bannes, un double fer à moulin. Les paysans dirent que le barbe, en
train de crever derrière la haie, en présentait une semblable. Et ces
braves gens regrettaient de n'avoir pu assommer quelqu'un de ces bandits
de grands chemins\,: «\,Ils avaient si vertueusement joué de l'éperon,
après avoir relevé leur compagnon, qu'un tourbillon de poussière chassé
par le vent ne les eût pas gagnés en vitesse.\,» Et le cocher Jacquin,
qui visitait ses bêtes et les trouvait heureusement sans blessure,
affirmait que les perruques et les masques dispersés sous son fouet lui
avaient montré la mine de Richard de Mallenay et de Gracien Larotte,
gredins réputés parmi les donneurs d'étrivières de Florimond et anciens
soldats aux gardes.

On s'empressait autour de Marguerite de Primelles, toujours évanouie.
Margot Larçonnière risquait un œil et soupirait à fendre l'âme.
M\textsuperscript{me} de Primelles dégrafait sa fille, on jetait de
l'eau à la chambrière, on lui bassinait les tempes, et l'encre ainsi
étendue d'eau noircissait les visages de ces femmes. On eût dit d'autant
de négresses. Catherine, dont le front laissait suinter le sang sous cet
enduit sombre, déclara n'avoir besoin de rien. Enfin l'on se remit en
route. Les quatre voyageuses arrivèrent à Bourges dans un état à faire
peur. Personne cependant ne vit le dégât dans les rues de la ville. Car
du coche les rideaux ne furent tirés que lorsque les portes du couvent
se furent refermées sur lui. Les religieuses se réjouirent à voir
qu'aucune de ces dames n'était blessée. On les lava, les changea de
vêtements, sans du reste déployer une curiosité déplacée au sujet de
leur aventure, «\,la prudence commandant toujours de ne pas chercher à
atteindre le fond des choses\,». Et la supérieure chercha si peu à
l'atteindre qu'elle ne vit pas que M\textsuperscript{lle} de Lépinière
avait ramené bien en avant ses boucles sur son front, que coupait, d'une
façon désormais indélébile par la vertu de l'encre dont elle était
imprégnée, une balafre allant du sourcil jusqu'à la racine des cheveux.
Dirigée par la haine et plus ferme que celle de ses complices, la main
de M. Acresin de Tourouvre avait frappé assez juste. S'il n'avait pas
été, au moment utile, froissé par le cheval de Clément Malompret en
danger lui-même d'être écharpé par la roue de l'avant-train, il eût
encore mieux fourni son coup, et M\textsuperscript{lle} de Lépinière eût
été marquée atrocement pour la vie.

Telle fut la manière dont la marquise Julie annonça à ceux de Primelles
son retour et celui de l'Incomparable Florimond en ce Berry où l'on
allait, du moins dans certaines maisons, jusqu'à se bercer de l'espoir
qu'on ne la reverrait jamais. Depuis plus de huit mois, il n'était plus
question de M\textsuperscript{me} de Bannes, non plus que de son fils.
On avait oublié les commérages du mois de juillet, et personne ne se
doutait à Bourges que Florimond avait passé deux mois à Chézal-Benoît
entre la vie et la mort et que la marquise, revenue de Paris au bout de
trois jours, l'avait soigné de ses mains, déguisée en ursuline tout
comme Nicole Deleuze et deux servantes engagées à Paris. Ces filles,
qu'on ne laissa pas sortir un seul jour du petit château, repartirent
avec Julie et Nicole quand Florimond put être transporté sans danger. On
régla leurs gages à l'entrée du faubourg Saint-Marcel, avec licence de
chercher fortune à leur goût.

Lorsque Florimond avait été détaché de son arbre, dans cette clairière
de la Table où il faillit périr sous le fouet, son corps inerte n'était
plus, de la nuque aux reins, qu'une plaie sanglante. Cottebleue ne
valait guère mieux. Des autres fustigés les blessures s'annonçaient
moins graves. Aussi, après le départ des bergers qui rendirent à chacun
la liberté de ses membres, les plus dispos se rhabillèrent tant bien que
mal et accomplirent cette dure besogne de transporter Florimond et
Cottebleue sur leurs bras jusqu'à Chézal-Benoît.

Là se passa une scène singulière. Autour du maître et du porteur
d'exploits toujours inanimés et allongés sur la table de la cuisine, un
débat s'engagea où l'un accusait invariablement l'autre de lâcheté,
voire de trahison, sans jamais s'attribuer l'apparence d'un tort. M.
Clément Malompret mit fin à ces invectives, tout à la fois prudentes et
passionnées et par-dessus tout contradictoires, en donnant d'autorité
l'ordre de mettre le baron sur son lit. Et là, aidé par Françoise
Colbert, que les événements de la nuit avaient remplie d'une salutaire
terreur, il commença par bassiner, avec du vin aromatique et de l'huile,
le dos excorié de son maître. Le bon Samaritain ne dut pas, en son
temps, oindre avec plus de précautions le voyageur qu'il rencontra
dépouillé et navré par les voleurs. À des voleurs, pareillement, il
décida de faire honneur des blessures qui déshonoraient le baron. Mais
encore cet honneur ne leur reviendrait que si la justice se mêlait
d'informer.

Lorsqu'un premier pansement et l'ingestion de précieux et généreux
cordiaux eurent rappelé Florimond à la vie, sans qu'il eût pour cela
retrouvé sa connaissance, Clément, le laissant à la garde de Colbert,
réunit toutes les victimes et tous les acteurs plus heureux du drame de
la nuit et leur soumit son idée. Ceux qui, à l'exemple de MM. de
Tourouvre et de la Butière, s'étaient courageusement enfuis au premier
acte et avaient, de ce fait, conservé intactes et leurs lombes et leurs
épaules approuvèrent la sagesse des dispositions de M. Clément.
Cottebleue, à moitié fou de souffrance, puis calmé à force d'eau-de-vie,
murmura, toujours allongé le nez sur la table de la cuisine, qu'il était
en effet plus expédient de se taire. Il avait gardé, dans son ivresse,
assez de lucidité pour nommer un barbier de Lunery capable de panser
tout le monde à bas prix, capable aussi de garder à bas prix le silence.
On l'envoya aussitôt chercher, et M. Clément, avec l'aide de MM. de la
Butière et de Tourouvre, écrivit une belle lettre à la marquise. Il
chargea Gracien Larotte, brave, alerte et déterminé, ancien soldat aux
gardes, de la porter à Paris. Gracien partit après avoir appris par cœur
une relation de l'affaire, œuvre de Clément, qui mettait chacun à l'abri
de l'accusation de couardise.

«\,Il sera toujours temps, avait dit Clément, de dire la vérité à la
patronne quand elle sera ici. Chacun en racontera ce qu'il en sait. À
elle d'établir la vérité. Et, quand le diable y serait, d'ici là nous
aurons bien trouvé quelque chose\,!\ldots\,»

Ces messieurs ne trouvèrent rien du tout, lorsque, trois jours après,
arriva la marquise, escortée par Nicole Deleuze, Ottavio Piccolomini, M.
Aimeri d'Olivier et les femmes de service. Une vraie lionne en fureur se
rua sur Tourouvre, La Butière, comme sur Clément Malompret, quand elle
connut l'état où se débattait Florimond malgré les emplâtres, le baume
et le cérat du barbier de Lunery, renommé pour sa discrétion. Aux
explications confuses de La Butière et de Tourouvre, occupés seulement
de se disculper et de rejeter l'un sur l'autre les mauvais résultats de
l'expédition de Primelles, se mêlaient les insinuations perfides de
Clément et d'Aimeri d'Olivier. En somme, ces gens ne cherchaient qu'à
s'entre-détruire, et Julie ne comprenait qu'une chose, c'est que, par la
faute de tous, son fils se trouvait en danger de mort. Et nul ne voulait
lui raconter exactement comment Florimond avait été mis en cet état. La
gangrène était à craindre, et le délire n'abandonnait pas le blessé.
Julie, épouvantée par ces sinistres symptômes, arrêta qu'il fallait
trouver coûte que coûte un médecin, et que celui-ci ne fût pas du pays.
À M. de Tourouvre échut la délicate mission de partir pour Paris et de
ramener un chirurgien également capable de panser les plaies et de se
taire.

Nicole Deleuze était trop experte en fait d'aventures rares et secrètes
pour ne pas tirer d'embarras le piteux cavalier qui ne savait à quelle
porte frapper. Elle lui indiqua la bonne adresse. Mais cela demanda huit
jours, et la fièvre maligne ne laissait pas à Florimond un seul instant
de répit. Les éclats de sa voix, tour à tour plaintive et furieuse,
remplissaient la petite maison, où l'on vivait les uns sur les autres
dans la crainte des gens de justice. L'on redoutait également les
visites, la curiosité des voisins, des chasseurs, des marchands
ambulants. Il était cependant nécessaire de s'approvisionner. Mainte
chose utile faisait défaut, notamment les cordiaux et les drogues. Des
hommes de confiance établirent un va-et-vient entre Chézal-Benoît et le
château de Bannes. De celui-ci partaient des domestiques pour Bourges.
Ils en rapportaient leurs emplettes à Bannes, et d'autres domestiques
les y prenaient pour les porter à Chézal-Benoît. On ne marchait que la
nuit\,; à la maison on tenait tous les volets clos\,: les vivres étaient
reçus à la porte du pont, sans qu'on laissât entrer personne.

M. Aimeri d'Olivier, déguisé en vieux suisse, s'abaissa jusqu'à remplir
l'office de portier. Sa prudence était si grande qu'on se sentait en
sûreté dans un logis gardé par un homme de cette qualité. A l'exemple du
lièvre, et comme lui rongé par la crainte, le poète de Florimond ne
dormait que d'une oreille, et son sommeil était empoisonné par des rêves
où la police le tourmentait sans merci. Sous leurs habits et leurs
guimpes de religieuses, Julie et Nicole n'avaient pas une condition
meilleure, et elles se désolaient en veillant le misérable Florimond,
craignant à toute heure de le voir expirer entre leurs bras. Quant à
Françoise Colbert, maudissant ses folles déterminations, elle soupirait
sous la crainte de Clément Malompret et de tous.

Les hôtes de Chézal-Benoît diminuèrent bientôt de nombre. Car, à mesure
qu'un homme était guéri de ses coups de fouet, il repartait pour son
village. Là, il racontait n'importe quoi. Les valets de Landry Vaillard,
dûment payés, allèrent jusqu'à s'accuser d'intempérance. Ils
prétendirent que des racoleurs les avaient circonvenus à Bourges,
enivrés, enfermés, bref, qu'ils avaient couru les pires dangers. La
femme de Cottebleue reçut, par les soins de Clément, une somme d'argent
et une lettre de son mari lui annonçant que le baron de Chézal-Benoît
l'avait emmené à Paris, où il s'occupait de lui acheter une charge
d'huissier.

On emmena en effet Cottebleue à Paris, mais seulement à la fin de
l'automne, quand on put transporter Florimond en carrosse. Durant trois
mois, le jeune baron demeura pour sa mère un sujet d'inquiétudes
constantes. Sa jeunesse et sa vigueur triomphèrent à la fin du mal qui
le clouait sur son lit de douleurs depuis l'été passé. Et, en tout
semblable aux serpents, qui, aux premiers rayons du soleil printanier,
sortent de leurs retraites souterraines, gonflés de venin et ayant fait
peau neuve, il ne rêva plus que sang et massacre. Il dénonça sa ferme
intention de retourner au château de son père et d'exercer une prompte
et rigoureuse justice contre ses ennemis.

La marquise Julie n'eut garde de traverser ces desseins. Altérée de
vengeance au moins autant que son enfant chéri, à peu près sûre de
l'impunité puisque l'aventure de Primelles paraissait tombée dans
l'oubli, elle activa le départ pour Bannes. Mais, bien conseillée par
Aimeri d'Olivier, elle se résolut à n'y rentrer que de nuit, et à tenir,
aussi longtemps que\,possible, son arrivée secrète, de manière à pouvoir
ourdir ses trames en pleine sécurité. Elle réussit au delà de ses
désirs. Par un espionnage actif, elle connut les projets de Catherine et
de M\textsuperscript{lle} de Primelles. Ce fut elle qui arma les laquais
et Tourouvre de bouteilles d'encre et les paya généreusement, et
d'avance, pour cette belle besogne.

Quand elle eut appris de Tourouvre et de Clément l'heureux succès d'une
attaque où le seul Cottebleue avait succombé, sans compter deux chevaux,
Julie, sans se préoccuper des suites possibles du guet-apens, courut
annoncer à son fils cette première victoire. Aussitôt Florimond s'écria,
en se frottant les mains, que, pour terminer tout cela à la plus grande
gloire de la maison de Bannes, l'heure était venue de provoquer le jeune
Louis-Antoine et d'en finir avec lui.

--- Gardez-vous-en bien\,! fit M. Aimeri d'Olivier en se bourrant le nez
de tabac. Gardez-vous-en bien\,! Ce serait mal jouer. À peine revenu,
voulez-vous reprendre la poste avec tous les sergents du Berry, puis
ceux du royaume sur les talons\,?

La marquise pria Aimeri de s'expliquer. Et celui-ci, qui ne détesta
jamais se faire valoir, développa un raisonnement en trois points sur
l'inopportunité d'un défi.

--- Attendez plutôt que le cartel vienne de l'ennemi. Vous aurez alors
pour vous les apparences du bon droit\ldots{} Patience, écoutez jusqu'au
bout\,! Le tour plaisant des bouteilles d'encre eut cela de fâcheux que
vous le signâtes d'un trop lisible paraphe. Tout le monde sait, soyez-en
certains, que le malencontreux Cottebleue était à votre service. Des
chevaux laissés morts sur le chemin les marques ont dû se déchiffrer
facilement\,; et chacun peut, en commentant la plaisanterie qui n'est
--- soit dit sans vous offenser --- pas nouvelle, assurer que la
cavalerie sortait de chez vous. Que les pécores aient été défigurées, le
mal n'est sans doute pas très grand, mais ces pécores ne sont pas à ce
point abandonnées que des parents ou des amis ne se lèvent pour défendre
leur cause.

Florimond sourit avec une hautaine affectation\,:

--- Jolis parents\,! Le vieux Bouteiller, pareil à un héron sans plumes,
et le jeune Louis-Antoine, oiseau qui n'a pas encore mué\,!

--- Il y a aussi, continua M. Aimeri d'Olivier, M. de Montenay, qui est
le tuteur de M\textsuperscript{lle} de Lépinière.

--- C'est ma foi vrai, je n'avais pas pensé à celui-là\,! dit Florimond.
Mais je me moque de lui\ldots{} et des autres\,!

--- Vous croyez-vous, mon cher Florimond, à ce point sûr de votre bras,
affaibli comme vous l'êtes par la maladie, pour affronter cet homme de
guerre encore jeune, et qu'on m'a souvent dépeint comme courageux et
solide\,? Y avez-vous songé\,?

Et, rentrant sa tête dans son rabat à glands, M. Aimeri d'Olivier
s'offrit une prise, puis se caressa le menton.

Florimond recommença de sourire d'un air avantageux et superbe. La
marquise se taisait. Au fond, elle était désolée. Malgré l'indulgence
quasi imbécile de ses yeux de mère, la vérité les venait impitoyablement
dessiller. Aimeri d'Olivier avait raison\,!\ldots{} Oui, à l'instar des
serpents printaniers, Florimond avait changé de peau. Et Julie savait
dans quel état était cette peau. Le dos de Florimond à peine cicatrisé,
mais couturé, labouré dans tous les sens, semblait habillé de baudruche
rougeâtre. Florimond avait donc changé de peau\ldots{} Mais des
serpents, sortis de leur vieille enveloppe, avait-il la force, l'énergie
et les dents cruelles\,? Hélas\,! A regarder ce corps amaigri, ces
épaules déjà voûtées, cette mine pâle et atone, Julie se demandait s'il
aurait la vigueur de mordre pour tuer par son venin. Elle se demandait
encore si son fils était toujours l'Incomparable Florimond des anciens
jours, et l'impitoyable bon sens lui répondait que non.

Un silence gênant régna alors sur ces trois êtres, qui, avec des
sentiments à coup sûr bien différents, se défiaient également les uns
des autres. Des grattements sous quoi cria la porte le rompirent.
C'était M. Clément Malompret qui portait sur un plateau une belle lettre
scellée de cire bleue. Et, avant que de l'ouvrir, Florimond reconnut les
armes, d'azur à trois têtes de loup arrachées de gueules, qui sont de
Montenay. Il rompit le cachet, lut et apprit que Jean Ludovic de
Lastour, comte de Montenay, l'assignait à trois jours de là, avec l'épée
et la dague, à la corne du bois de la Borne, où il l'attendrait à huit
heures du matin, assisté de M. Lucien-Timoléon-Hannibal de Mauny
d'Anrieux et de M. Pamphile-André d'Archelet. Et M. de Montenay
terminait sa lettre par divers souhaits et formules de politesse, avec
le désir d'éviter à M. Florimond Pontaillan la peine de s'en retourner.

\hypertarget{chapitre-xii}{%
\chapter{CHAPITRE XII}\label{chapitre-xii}}

--- Si vous m'en croyez, monsieur, nous mettrons ici pied à terre et
ferons, en marchant, ces quelque cent pas qui nous séparent de l'endroit
où l'on se battra. Dérouiller ses jarrets est une bonne chose. Rien de
plus mauvais que d'arriver, les jambes raides, sur un terrain que l'on
n'a pas reconnu. Et puis, nous trouvant en avance, nous pouvons choisir
nos places, tourner le dos au soleil pour attaquer, en un mot prendre
nos avantages.

Pour déterminé que se montrât toujours M. de Tourouvre à rompre en
visière à M. de la Butière, il dut reconnaître que celui-ci parlait le
langage du bon sens, et il engagea Florimond à suivre ses conseils. Le
baron de Chézal-Benoît eût désiré étonner ses adversaires par la pompe
de ses chevaux et de ses laquais. Mais, comme il s'agissait d'un duel
qui serait bien probablement sévère, le jeune homme, tout en ne doutant
pas un seul instant de sa victoire, résolut de multiplier les chances
heureuses\,; et il se rallia à l'opinion des deux braves dont
l'expérience, en semblable matière, n'était assurément pas douteuse.

On descendit donc de cheval, et, laissant bêtes et gens de livrée sous
le commandement de M. Clément Malompret, Florimond et ses seconds se
dirigèrent à petits pas vers l'espace découvert qui précédait l'avenue
principale du bois de la Borne, où devait se donner le combat contre M.
de Montenay. M. de Tourouvre, toujours prudent, s'assura que chacun
avait conservé sa dague\,; car c'est un malheur fréquent que de perdre
cette arme au mouvement du cheval, surtout quand la lame en est aisée
dans son fourreau, comme il convient pour les poignards usités dans les
duels.

--- Nous sommes beaucoup trop en avance, fit Florimond en consultant sa
montre, et par ta faute, Tourouvre\,! Pourquoi nous as-tu obligés à
partir aussi tôt\,? Attendre ici trois mortels quarts d'heure ne me
plaît point. J'ai envie d'envoyer un laquais à la Vergne, pour annoncer
à ces messieurs que nous sommes à leur disposition.

M. de Tourouvre n'était pas de cet avis. C'est pourquoi M. de la Butière
se rangea du côté de Florimond\,: «\,On ne peut ainsi faire le pied de
grue. Et, avec votre congé, j'irai moi-même les chercher.\,»

--- Inutile de se déranger\,: les voici\,! répondit Tourouvre.

Trois cavaliers, en effet, apparurent au bout de l'avenue. Mais, outre
qu'ils venaient d'une direction opposée à celle de la Vergne, qui était
sur la droite, c'est-à-dire au levant, leur aspect n'avait rien de
commun avec MM. de Montenay et de Mauny d'Anrieux. Assez pauvrement
montés, on eût dit que ces inconnus étaient là pour figurer les trois
âges principaux de la vie. Celui qui tenait le milieu avait la barbe
blanche d'un vieillard. À sa droite chevauchait un jeune homme, presque
un enfant, et à sa gauche un homme dans la pleine force de l'âge.

--- La peste soit des fâcheux\,! grommela Florimond. Est-ce que par
hasard on nous mettrait sur le dos des sergents de justice\,?\ldots{}

M. de Tourouvre, qui, depuis l'affaire des bouteilles d'encre et la mort
malheureuse de Cottebleue, ne dormait pas du sommeil du juste, essaya de
chasser ces sinistres préoccupations\,:

--- Ne croyez-vous pas plutôt que ce Montenay, redoutant les suites
funestes d'un impertinent défi, ait adopté ce misérable expédient pour
nous éloigner d'ici et préserver, par conséquent, sa peau à défaut de sa
réputation\,?

--- C'est à n'y rien comprendre, dit alors La Butière, qui, à cause des
bords étroits de son chapeau en pot à beurre, tenait la main au-dessus
de ses yeux, regardant avec attention ces nouveaux venus. Les voilà qui
laissent leurs chevaux aux gens de leur suite et se dirigent vers
nous\ldots{} Ah çà\,! mais\ldots{} je ne me trompe pas\,!\ldots{} C'est
l'antique Bouteiller avec son oison de neveu Primelles, et un quidam
sans importance.

M. de Tourouvre s'écria alors\,:

--- Le quidam n'est autre que Robert de Rustigny, l'écuyer meneur de
M\textsuperscript{me} de Primelles\ldots{} Que cherchent ici ces
imbéciles\,?

--- J'ai bien envie, répondit Florimond, de leur faire administrer par
les valets une volée de bois vert, pour leur apprendre à me déranger et
à traverser mes terres sans ma permission.

--- Prenez garde\,! murmura Tourouvre. Si j'en crois ce que je vois, les
trois gentilshommes sont mieux accompagnés que nous\ldots{} Et c'est
bien à nous qu'ils en veulent.

--- Attendons-le sans faiblir, mes amis, et montrons-leur, au besoin
avec le tranchant de l'épée, si le plat n'y suffit point, combien peu
nous faisons de cas de leurs personnes. Foi de Florimond, à la première
parole malsonnante, je coupe les oreilles du vieux, et vous ferez autant
aux autres\,! Quel malheur que ce duel avec Montenay m'empêche de
déconfire ici-même ce nigaud de Louis-Antoine\,!\ldots{} Il ne perdra
rien pour attendre\ldots{} Eh quoi\,?\ldots{} Ils osent s'adresser à
moi\,?\ldots{} Que disent-ils\,?

--- Ils disent, répliqua une voix jeune, claire et vibrante, ou plutôt
je dis que je vais te clouer à un de ces arbres comme un oiseau de nuit.
Entends-tu, Pontaillan, méchant bâtard\,?\ldots{} Insulteur de femmes,
mets-toi en défense, que je ne te frappe pas désarmé ainsi que ton père
assassina jadis le mien\,!

Frémissant encore plus de stupéfaction que de colère, Florimond se
retourna pour ordonner à M. de Tourouvre d'appeler les valets. Celui-ci
n'en eut pas le loisir, car M. Le Bouteiller marchait sur lui, l'épée et
la dague dégainées. Quant à M. de la Butière, pressé par Robert de
Rustigny, il avait tiré ses armes du fourreau et commencé de parer les
coups d'un ennemi qui le poussait avec méthode de manière à l'éloigner
de Florimond. Celui-ci cria alors, sans cesser de ricaner\,:

--- Ah ça\,! mais\,!\ldots{} C'est un guet-apens\,!\ldots{} Holà\,!
Clément et les autres, à l'aide\,!\ldots{} Et qu'on m'assomme cette
canaille, sans plus tarder\,!

--- Si ta valetaille remue, drôle, les miens la mettront en charpie. Tu
parles de guet-apens\,? Qu'il te souvienne du 13 juillet passé, impudent
bâtard, baron de boutique\,!\ldots{} Allons, qu'attends-tu\,? Devrai-je
te faire bailler des coups de bâton, comme naguère, au carrefour de la
Table\,? Défends-toi, ou bien je frappe\,!

Et Louis-Antoine, ayant ainsi parlé, menaça Florimond de son épée et de
sa dague tendues. A ces mots «\,le carrefour de la Table\,», le fils de
Julie sentit le sang affluer à son cœur. Un moment, il craignit
d'étouffer. Malgré le brouillard qui obscurcissait ses yeux, malgré les
pulsations qui martelaient ses tempes et lui donnaient envie de hurler,
d'écumer, de mordre, il se raffermit sur ses jambes tremblantes,
empoigna son épée et sa dague, et, adoptant une garde purement
défensive, essaya sur Louis-Antoine la force de son terrible regard.
Mais il ne put longtemps se contenir et accabla le jeune baron des plus
grossières insultes, prodiguant de préférence la boue de ses outrages à
M\textsuperscript{me} de Primelles et à sa fille.

Ce fut en pure perte. Muet et attentif, l'enfant le chargeait si
résolument, et sans commettre de faute, que Florimond dut rompre la
mesure. Le désavantage des armes était d'ailleurs pour lui. Obéissant à
cette mode qui voulait qu'on portât l'épée française large et assez
courte, en mépris de la longue et raide épée espagnole mesurant quatre
pieds et demi du pommeau à la pointe, le baron de Chézal-Benoît ne
pouvait pas gagner sur Louis-Antoine, parce que celui-ci, s'en tenant
aux coups de temps, l'arrêtait à chaque attaque, sans croiser. Et la
dague ne valait rien contre cette lame démesurée qui se remettait
toujours, en dégageant, dans la ligne. La haute taille de Florimond lui
devenait aussi une cause d'infériorité dans la lutte, et surtout la
roideur de son échine souffrant encore de l'affreuse morsure des
étrivières maniées par Marin. Trop occupé de défendre sa vie pour songer
à autre chose, il se taisait à cette heure, mais il voyait derrière
Louis-Antoine, à quinze pas peut-être, l'odieuse figure de ce Marin qui
le dévisageait avec impudence. Reculer devant ce drôle, quelle honte\,!
Plutôt mourir, mais pas seul\,! Et Florimond se décida à tuer
Louis-Antoine, dût-il rouler en même temps transpercé. Donc, à un coup
d'allonge que lui détacha celui-ci, il essaya de l'enferrer par une
estocade de pied ferme, en se rasant à terre, la jambe gauche
complètement développée en arrière. Florimond, pour son malheur, ne put
assez ployer les reins, et il tomba si lourdement de côté, sur sa main
gauche tenant toujours sa dague, qu'il se foula à demi le poignet.

Atteint au front, aveuglé par son sang, il se couvrit de ses armes et se
releva furieux pour se ruer à corps perdu sur Louis-Antoine, taillant
presque au hasard, tant la rage, le désespoir et par-dessus tout
l'humiliation l'emportaient. Reculant juste assez pour garder la mesure,
Louis-Antoine lui opposa une contre-attaque, la main haute, en portant
le pied gauche en avant et en recevant sur sa dague le revers qui
l'aurait certainement mis hors de combat s'il ne l'eût ainsi paré. En
même temps, il détacha avec son épée un fendant sur le genou de
Florimond, qui baissa la pointe de sa dague pour se couvrir.
Louis-Antoine saisit le moment avec un merveilleux à-propos. Il cava la
main, passa par-dessus le bras gauche de son adversaire, et lui donna de
côté dans la gorge. La pointe entra, suivie de deux pieds de fer,
coupant sur son passage et la voix et les vaisseaux qui charrient la
vie.

Le baron de Chézal-Benoît ouvrit la bouche sans qu'un seul son en
sortit. Ses bras lâchèrent rapière et poignard, battirent à vide, et
Louis-Antoine, ayant retiré son arme en tournant trois fois la lame dans
le cou du moribond pour bien indiquer que son désir n'était pas de
l'épargner, l'Incomparable Florimond chut à terre, de côté, tel un bœuf
égorgé. De son col en point coupé à rosaces jaillissait une masse de
sang vermeil qui vint se mêler à celui de M. de Tourouvre. Celui-là
gisait, décoiffé de son feutre à panache, le crâne fendu par un maître
estramaçon fourni par le vieux baron de Mordicourt, qui essuyait
gravement sa lame ébréchée et ruisselante aux basques du pourpoint de sa
victime. Ce que voyant, Bérenger de la Butière, qui ferraillait sans
ardeur avec Robert de Rustigny, lui demanda poliment s'il n'en avait pas
assez\,: «\,Pour lui, il s'estimait grandement satisfait d'avoir tenu
tête, sans trop de désavantage, à un maître de cette qualité.\,»

L`écuyer de M\textsuperscript{me} de Primelles n'avait pas à ce point
soif du sang de ce gentilhomme entretenu qu'il prétendit l'obliger à
continuer le combat. Il laissa donc partir M. de la Butière, qui, sans
demander son reste, rejoignit Clément et les valets de Bannes, témoins
indifférents ou découragés d'un drame dont ils avaient pu suivre les
moindres détails. M. de Montenay arrivait alors dans le bois de la Borne
avec M. de Mauny d'Anrieux, André d'Archelet et une suite de laquais
armés assez nombreuse, tant il nourrissait peu de confiance dans la
gentilhommerie de Florimond. Il vit le corps de celui-ci allongé dans
l'allée près de M. de Tourouvre, dont les dernières convulsions
secouaient encore les membres.

--- Qu'est-ce ceci, dit-il, et qui s'est substitué à moi pour punir ce
méchant bâtard de ses vilaines actions\,?\ldots{} Aurait-il été
assassiné, d'occasion\,? La perte serait petite.

--- Eh\,! mon ami, fit M. Le Bouteiller, qui embrassait Louis-Antoine,
voilà monsieur mon neveu, cet autre David qui abattit le petit Goliath
de Bannes. Il vous répondra mieux que nous là-dessus.

--- Serait-il possible\,?\ldots{} En vérité, Louis-Antoine, as-tu mis
par terre ce fendeur de naseaux\,?\ldots{} Çà, parleras-tu\,?

Mais, repris par sa timidité sauvage, le jeune baron de Primelles
rengainait son épée et sa dague. On ne put ni le retenir ni lui arracher
un mot, et il courut à son cheval, sauta en selle, et tous le perdirent
des yeux tant il poussait sa monture vigoureusement du côté de
Primelles.

--- Holà\,! mes enfants, cria le baron de Mordicourt, courez après lui
et me le rattrapez, sans tarder\,!\ldots{} Que les mieux montés
galopent, en prenant les raccourcis, et atteignent avant lui le
château\,!\ldots{} Je crains que sa pauvre mère, en le voyant apparaître
ainsi brusquement, ne soit tuée par un coup de sang, car la joie est
souvent plus lourde à supporter que la peine. Et c'est assez de deux
morts aujourd'hui\,! Montenay, je vous défie à la course\,! Avec ma
vieille bique qui me porta aux Ponts-de-Cé, je prétends entrer le
premier dans Primelles\,!

Bientôt il ne resta plus dans l'allée que les gens de Florimond,
empressés autour des deux cadavres. On les guinda chacun sur une selle
couverte d'un manteau\,: on les y attacha, jambes et bras ballants, et
le funèbre cortège regagna au petit pas le château de Bannes, où la
marquise et M\textsuperscript{lle} Deleuze, en compagnie de M. Aimeri,
attendaient le retour de leur enfant chéri. Elles purent assister à ce
retour du haut de la principale terrasse, reconnaître les deux corps
jetés en travers des chevaux. Julie poussa un cri si horrible que M.
d'Olivier en faillit perdre le sentiment. Il trouva cependant le courage
de recevoir la marquise dans ses bras et de la porter, avec l'aide de
Nicole Deleuze, jusqu'à la porte de l'escalier, où les femmes la
reçurent.

Fuyant le concert de gémissements, de soupirs et de pleurs qui s'élevait
maintenant de tous les coins du château, M. Aimeri d'Olivier courut à sa
chambre, appela son valet Sylvestre et lui commanda de tout préparer
pour le départ. S'étant assuré la protection du prince de Carancy lors
de son dernier séjour à Paris, M. d'Olivier l'allait rejoindre. Tels les
rats qui s'enfuient vivement de la maison qui brûle, le poète quittait
le château, où la ruine semblait près de s'abattre. Il partit si vite
qu'il ne fit même pas ses adieux à la marquise, par pure discrétion,
sans doute. Et M. Bérenger de la Butière, montrant une pareille
diligence, disparut de son côté sans demander audience ni congé.

Julie n'eût pas été, certes, en état de recevoir leurs consolations.
Serrant entre ses bras, avec une opiniâtreté de folle, le corps inanimé
de son fils que l'on avait posé, tout botté et éperonné, sur son lit,
elle le défendait, avec une obstination muette et farouche, contre
Nicole et les femmes de service, qui prétendaient le laver et le changer
de vêtements. Insensible au sang coagulé, à la fange, qui souillaient
cette face, ce cou, ces cheveux blonds dont les pareils n'existèrent
point sur la terre, elle demeurait la bouche collée sur ce visage, dont
un ricanement arrogant et sinistre altérait les traits et dont les yeux
fixes la regardaient avec une expression furieuse où se lisaient aussi
et l'angoisse et la stupéfaction. Telle se présenta sans doute la tête
de la gorgone Méduse quand elle se sépara du tronc pantelant sous le
coutelas de Persée.

La marquise Julie passa plusieurs heures ainsi, sans que prières ni
remontrances pussent l'arracher de cette couche où dormaient de
l'éternel sommeil l'orgueil et l'amour de sa vie. À grand'peine put-on
obtenir qu'elle s'allongeât sur un lit de repos pendant que l'on
procédait à la dernière toilette du mort. Vers le milieu de la nuit,
elle tomba dans une sorte de léthargie. Puis les mouvements désordonnés,
l'oppression de sa poitrine, le désordre de sa figure prouvèrent que si
Julie avait retrouvé la vie elle n'avait pas gagné le repos. Ni les
calmants ni les saignées ne purent apaiser ses transports. Le délire
l'étreignait avec ses cauchemars plus effrayants encore que la triste
réalité. Sans doute le Souverain Juge voulut-il que la femme adultère et
homicide vit repasser sous ses yeux béants d'horreur sa vie d'impudente
audace et ses fautes jusque-là impunies. Elle vit son premier mari
assassiné par ses soins\,; elle vit son fils tué en duel grâce à ses
machinations haineuses.

Qu'elle fût la cause de cette mort, en douter ne lui était pas permis.
N'avait-elle pas, croyant avoir rangé de son côté toutes les bonnes
chances, et cette fois ne prenant conseil que d'elle-même, ménagé cette
rencontre entre Florimond et Louis-Antoine\,? Que M. de Montenay
demeurât vainqueur de Florimond, et la vengeance contre les Primelles
tombait dans l'eau\,! Julie n'était pas allée jusqu'à croire que M. de
Montenay pût tuer Florimond. Tourouvre et La Butière, Clément et les
laquais sauraient, au besoin, l'en empêcher. Mais l'état de faiblesse de
Florimond était à ce point avéré que la victoire pouvait très bien se
décider en faveur de son ennemi.

Julie s'ingénia donc à mettre Florimond dans l'obligation de se battre
avec Louis-Antoine avant que M. de Montenay n'arrivât sur le terrain.
«\,La mort de ce petit drôle --- car pour elle la défaite finale de
Louis-Antoine ne faisait pas question --- retardera certainement le duel
avec M. de Montenay, s'il ne l'empêche pas définitivement en obligeant
Florimond à s'éloigner du royaume. Venant en forces au bois de la Borne,
Florimond pourra dicter sa volonté, crier au guet-apens, que
sais-je\,?\ldots{} Devant Louis-Antoine mort, on prendra d'autres
décisions\ldots{} Et, au diable, Tourouvre trouvera bien quelque
faux-fuyant\,! Sans plus m'expliquer qu'il ne conviendra, je lui en
toucherai deux mots\ldots{} Quant au moyen d'amener le frère de
Marguerite en présence de Florimond une heure avant Montenay, rien de
plus simple\,: Tourouvre saura bien presser mon fils en insinuant que La
Butière ne souhaite rien tant que d'arriver en retard, pour éviter le
combat, si possible.\,»

L'espionnage auquel se livrait la valetaille qui entourait Florimond
permettait à la marquise de presque tout suivre, jour par jour, de ce
qui se passait à Primelles. Et elle suppléait pour le reste, grâce à sa
finesse toujours aiguisée de frais par la haine. Julie savait bien et
depuis longtemps quelle femme était la baronne de Primelles et combien
son impassible et dévote attitude réussissait à masquer une nature à la
vérité plus altérée de vengeance que ne le furent de sang, dans
l'héroïque antiquité, ces âmes plaintives qu'Ulysse dut repousser de son
épée quand il sacrifia aux mânes du fameux devin. M\textsuperscript{me}
de Primelles ne songeait qu'à venger le meurtre de son mari. Pour cela,
elle préparait son fils par les exercices de l'escrime, n'attendant que
le jour où ce jeune homme --- qui n'était encore, au vrai, qu'un enfant
--- pourrait combattre avec quelques chances de succès contre Florimond.
Bien informée par Clément Malompret, qui tenait ses renseignements de
Françoise Colbert, Julie savait encore combien Louis-Antoine était peu
versé dans la science des armes, ou, du moins, elle s'enfonçait dans
cette opinion qui flattait ses désirs, sans tenir compte de ce que lui
avait raconté récemment Aimeri d'Olivier, toujours au courant des
diverses rumeurs qui couraient le pays.

«\,On répète partout, lui avait dit Aimeri, que le jeune baron de
Primelles fait, depuis peu, d'extraordinaires progrès dans l'escrime.
Son écuyer Robert de Rustigny en raconte des choses étonnantes\,: que ce
Louis-Antoine atteigne à l'âge d'homme, et il sera rangé parmi les plus
dangereux.\,»

Julie avait relégué ce propos dans l'oubli avec les autres commérages
dont la favorisait ce gazetier scandaleux qu'était M. d'Olivier. Cela ne
lui revint à la mémoire que pour l'exciter à persévérer dans son
projet\,: «\,Raison de plus pour attirer Louis-Antoine dans ce piège où
son orgueil ou celui des siens le poussera, puisqu'on le considère comme
un champion de telle valeur. Écrasons le serpenteau dans l'œuf, avant
que le dard ne lui ait poussé\,!\,»

Ainsi déterminée, Julie écrivit un billet qu'elle chargea Nicole de
recopier. Cette personne avisée fit mieux. Elle calqua sur une vitre
cette écriture avec un crayon, puis la traça par-dessus à l'encre,
détruisit avec la mie de pain les traces de la mine de plomb et obtint
ainsi une calligraphie cursive d'un caractère singulier\,:

~

«\,Madame, l'amour profond que nous avons tous ici de votre race et que
nous héritâmes de nos parents, qui furent, en leur temps, les fidèles
amis des vôtres, nous commande de vous aviser de diverses choses
touchant cette maison de Bannes pour laquelle nous ne saurions nourrir
les mêmes sentiments. L'on tient pour certain, madame, que dans trois
jours, ce prochain samedi d'avril, le jeune Florimond Pontaillan, se
disant baron de Chézal-Benoît, rencontrera à huit heures du matin, dans
la grande avenue du bois de la Borne, le comte de Montenay, qui l'a
assigné. L'on a encore pour chose certaine que ledit Pontaillan, malade
et endolori des suites d'une aventure remontant à l'été passé, ne pourra
guère résister avantageusement à M. de Montenay, dont la vigueur et la
vertu ne sont ignorées de personne. L'on est sûr enfin de cette
particularité que Pontaillan, dans la crainte que son adversaire ne
doute de son courage, arrivera une bonne heure d'avance dans le bois de
la Borne.

«\,Si le jeune baron de Primelles, votre fils, l'y désire joindre, rien
ne lui sera plus aisé. Qu'il vienne à sept heures et demie du matin, et
il aura le temps de tirer l'épée à la mémoire de son père. Pontaillan
sera assisté par ses gentilshommes domestiques Tourouvre et La Butière.
Votre fils trouvera aisément qui leur opposer.

«\,Un guet-apens n'est pas à craindre. Qui peut, d'ailleurs, empêcher
votre fils de marcher bien accompagné\,? Ceux de Primelles ont déjà
prouvé qu'ils ne redoutent personne. Toutefois la prudence ne se doit
point confondre avec la lâcheté, et un gentilhomme qui se respecte ne
doit point aller sans un grand nombre de gens.

«\,Les conditions où nous vivons nous défendent de signer cette lettre
de notre nom. Mais nous pouvons vous assurer, madame, de notre entier
dévouement et du grand désir qui nous tient de voir votre maison
reprendre le lustre que lui méritent et votre fermeté et vos vertus.\,»

~

Et, comme si la fortune se plaisait à favoriser la réussite de ses
desseins, Julie reçut à point nommé un courrier de M\textsuperscript{me}
d'Alloigny, qui lui apportait une lettre indifférente. Nicole paya
généreusement cet homme de livrée, qui accepta de porter le billet
anonyme, secrètement, à Primelles. Il le jeta vivement, sans descendre
de cheval, dans le guichet du portier, après avoir heurté à l'huis. Le
vieux Roquelin Saboureau reçut la missive, mais le courrier avait déjà
repris son chemin\,; et le portier, sans reconnaître exactement les
couleurs de la livrée des Alloigny, put dire en toute sincérité à
M\textsuperscript{me} de Primelles qu'il connaissait cela pour
appartenir à la bonne noblesse de Bourges.

La baronne de Primelles lut avec soin ce billet. Puis, selon sa coutume,
elle s'abîma dans la prière, pensant y trouver la fin de ses
irrésolutions comme aussi la suprême consolation de son incurable
tristesse. Ne sachant pourtant se décider, elle allait demander à son
oncle Le Bouteiller de lui donner son avis, quand, à ce moment,
Louis-Antoine passa dans le couloir. Voyant dans cette rencontre un
ordre venu d'en haut, la baronne prit son fils par la main, l'emmena
dans sa chambre et, sans autre explication, lui tendit la lettre. Quand
Louis-Antoine eut fini de lire, il saisit le poignet de sa mère, baisa
doucement ses longs doigts fluets et dit\,:

--- Madame, sauf votre bon plaisir, je veux aller au bois de la Borne et
mettre par terre ce Pontaillan. Il m'est odieux comme à vous.
Bénissez-moi, s'il vous plaît, Dieu m'assistera sans doute en cette
affaire.

M\textsuperscript{me} de Primelles eut cette force de demeurer debout,
quoique les murs de sa chambre tapissée de nattes parussent danser
autour d'elle. Elle attira sur son cœur l'enfant qu'elle envoyait
peut-être à la mort, sut retenir ses larmes, et l'embrassa avec autant
de froideur qu'elle put opposer à l'amour dont son cœur meurtri
débordait. Elle murmura, très vite, pour que son fils ne pût remarquer
l'altération de sa voix\,:

--- C'est bien, Louis-Antoine\,! Vous êtes un Primelles\ldots{} Je
n'attendais pas moins de vous\,!

Et, comme le jeune baron se retirait, elle le rappela, prit sur le
coffre de chêne sombre, où elles dormaient depuis près de dix ans dans
leur bourse de chamois, l'épée et la dague que portait le baron de
Primelles le jour où il fut tué par le marquis de Bannes, et les mit aux
mains de son fils, qui les emporta, frémissant. Si grands furent ses
soins à porter cette relique, encore qu'il ne cessât de penser un seul
instant à Catherine de Lépinière, blessée au cou, puis marquée au front
par la bouteille d'encre, Louis-Antoine, déjà engagé dans l'escalier,
n'entendit pas un bruit sourd dans la chambre de sa mère. Demeurée seule
chez elle, Marie-Célestine-Françoise de Saudres, baronne veuve de
Primelles, tomba de son haut et tout de son long sur le carreau. Elle ne
devait plus se relever que pour languir, vérifiant ainsi les dires de
son oncle le baron de Mordicourt\,: «\,Ma nièce Françoise est un
tabernacle de vertus, mais ce n'est malheureusement pas une femme
énergique.\,»

Julie l'était-elle davantage à cette heure\,?

Grand besoin lui en eût été, pourtant, car l'aveugle hasard se
complaisait maintenant à l'accabler de ses coups. Tous portaient.
Florimond, marqué au front, saigné au cou, n'était pas enterré dans la
chapelle de Bannes, que la marquise recevait une mauvaise nouvelle\,: le
marquis de Bannes avait été tué du côté de Bude, par un coup de canon,
croyait-on. Et ce coup ne fut pas le dernier. Les Caumont La Force
faisaient saisir le château et tous les autres biens du défunt. Julie la
Drapière n'avait plus ni parents ni amis.

Seule contre tous, qui la protégerait\,? Aucun homme ne répondait
d'elle, et l'on parlait d'instruction criminelle pour des affaires
passées. Le testament du marquis, dont on eut connaissance six mois
après ces catastrophes, ne mentionnait la marquise que pour bien établir
la séparation de ses propres d'avec les biens de Bannes sur quoi on lui
déniait tout droit. Le château lui-même, sa demeure, était, par le fait
même du trépas de Florimond, attribué à M\textsuperscript{lle} de
Lépinière. Et celle-ci, avec une méprisante condescendance, donnait à
savoir que la marquise-drapière pourrait y séjourner tant qu'elle
voudrait jusqu'au prochain mariage de la propriétaire légitime.

En attendant, les propres de Julie étaient saisis, menacés par des
créanciers brusquement sortis de l'ombre. Un vol d'ailes noires, une
rumeur de croassements se resserraient autour d'elle. M. Marcelin de
Vaux, qui venait d'acheter un titre de baron et de liquider, du moins
ostensiblement, son officine de procureur, intervint avec chevalerie et
prit le parti de la Drapière, dont il n'avait jamais désespéré de faire
une baronne de robe, car il parlait d'acheter une charge dans la
magistrature, à Blois. Nicole Deleuze poussa fortement à la roue pour
que cette belle alliance vint justifier celle que l'écuyer Piccolomini
brûlait de contracter avec elle.

Quant aux affaires de Primelles, le cardinal ministre entendit les
régler lui-même, car il ne se souciait pas de voir les Caumont acquérir
ainsi une pareille puissance territoriale dans le Berry. Il procéda
d'après sa méthode ordinaire, qui consistait à humilier les grandes
maisons et à élever les petites à condition qu'elles tinssent et
gardassent tout de sa main. Non content de délivrer à Louis-Antoine des
lettres d'abolition pour le meurtre de l'Incomparable Florimond, il
commanda qu'on lui amenât «\,ce héros de la campagne\,» et sa Catherine,
dont il écouta l'histoire sans rire autrement que des yeux.

--- Je veux, dit il, tout en caressant une nichée de chats qui
s'ébattaient parmi ses papiers, je veux que ces enfants soient mariés. À
Dieu plaise, qui nous débarrassa de ces Bannes, que la souche des
Primelles reverdisse et couvre tout le Berry de ses rameaux\,!
Arrangez-vous, messieurs de la Force, pour que ces cases d'échiquier,
qui composent leurs biens et les vôtres, disparaissent et laissent la
place à des domaines mieux ordonnancés\,! C'est affaire à vos notaires
et à vos hommes de loi, et l'on rachètera les justices. Mettez-y de la
bonne volonté\,! Vous pouvez compter sur la mienne. Que Primelles joigne
Lépinière et que Bannes rentre chez soi\,! Et, encore, que l'on m'envoie
pour un an ce Louis-Antoine porter le mousquet dans une bonne
compagnie\,! Il en reviendra lieutenant, je suppose, et on le mariera
alors\ldots{} En tous cas, qu'il ne tire plus l'épée hors du service du
Roy\,!\,»

M. Le Bouteiller, si l'on eût suivi son avis, aurait plutôt poussé son
petit-neveu vers un évêché, car la mollesse des temps lui semblait peu
favorable au développement des caractères. La guerre qui incendiait
l'Allemagne, sans que la France s'y pût utilement mêler, ne lui disait
rien. N'ayant connu que la guerre civile, il la jugeait seule bonne, ne
croyait pas une autre possible, et tenait les temps pour révolus.

--- À porter le mousquet, ce garçon n'ira pas loin\ldots{} Toutefois,
avec une pareille femme\,?\ldots{} C'est Thalestris ou Penthésilée, à
moins que ce ne soit les deux ensemble\,!\ldots{} Mais je vous le
demande, ami Mauny d'Anrieux, ne marierons-nous pas aussi cette sotte
Marguerite, ma petite-nièce, avec notre Montenay\,?\ldots{} Il l'a bien
méritée par sa patience.

M. de Mordicourt ne vit point ce vœu exaucé. M\textsuperscript{lle} de
Primelles s'enferma dans son couvent de Bourges, où elle vécut avec ses
romans, peut-être aussi avec le souvenir de l'Incomparable Florimond,
dont ses lectures réussirent sans doute à effacer les mauvaises qualités
à ses yeux. Cette jeune personne vécut ainsi jusqu'à ce que la grâce la
touchât. Alors elle prit l'habit. Et cela n'arriva qu'en l'an 1649. M.
de Montenay avait été tué depuis longtemps au milieu des Weymariens\,;
M. de Mauny d'Anrieux avait épousé sa gouvernante Marion, la Hollandaise
venue de la guerre\,; et Louis-Antoine de Primelles, toujours protégé
par sa chère Catherine, qui l'accompagnait fidèlement dans ses campagnes
et ses garnisons du Piémont, était capitaine de gens de pied et en passe
de se faire nommer mestre de camp. Tant il est vrai qu'il n'y a point,
en ce monde, de bien plus précieux qu'une bonne femme.

~

\begin{flushright}\textit{Paris, 3 Mai 1911}\end{flushright}

\end{document}
